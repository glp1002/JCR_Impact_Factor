\documentclass[a4paper,12pt,twoside]{memoir}

% Castellano
\usepackage[spanish,es-tabla]{babel}
\selectlanguage{spanish}
\usepackage[utf8]{inputenc}
\usepackage[T1]{fontenc}
\usepackage{lmodern} % Scalable font
\usepackage{microtype}
\usepackage{placeins}

\RequirePackage{booktabs}
\RequirePackage[table]{xcolor}
\RequirePackage{xtab}
\RequirePackage{multirow}

% Links
\PassOptionsToPackage{hyphens}{url}\usepackage[colorlinks]{hyperref}
\hypersetup{
	allcolors = {red}
}

% Ecuaciones
\usepackage{amsmath}

% Rutas de fichero / paquete
\newcommand{\ruta}[1]{{\sffamily #1}}

% Párrafos
\nonzeroparskip

% Huérfanas y viudas
\widowpenalty100000
\clubpenalty100000

% Cuadros de notas (comando propuesto por la alumna)
% Los parámetros son:
% 1 --> Texto de la nota
\newcommand{\nota}[1]{
    \begin{center}
        \fcolorbox{blue}{white}{\textcolor{black}{
            \begin{minipage}{0.915\linewidth}
        	\textbf{\color{blue}NOTA}
                \linebreak\linebreak
                #1
                \linebreak
            \end{minipage}
        }}
    \end{center}
}

% Cuadros de WARNINGS (comando propuesto por la alumna)
% Los parámetros son:
% 1 --> Texto del aviso
\newcommand{\aviso}[1]{
    \begin{center}
        \fcolorbox{red}{white}{\textcolor{black}{
            \begin{minipage}{0.915\linewidth}
        	\textbf{\color{red}AVISO}
                \linebreak\linebreak
                #1
                \linebreak
            \end{minipage}
        }}
    \end{center}
}

% Imágenes

% Comando para insertar una imagen en un lugar concreto.
% Los parámetros son:
% 1 --> Ruta absoluta/relativa de la figura
% 2 --> Texto a pie de figura
% 3 --> Tamaño en tanto por uno relativo al ancho de página
\usepackage{graphicx}
\newcommand{\imagen}[3]{
	\begin{figure}[!h]
		\centering
		\includegraphics[width=#3\textwidth]{#1}
		\caption{#2}\label{fig:#1}
	\end{figure}
	\FloatBarrier
}

% Comando para insertar una imagen sin posición.
% Los parámetros son:
% 1 --> Ruta absoluta/relativa de la figura
% 2 --> Texto a pie de figura
% 3 --> Tamaño en tanto por uno relativo al ancho de página
\newcommand{\imagenflotante}[3]{
	\begin{figure}
		\centering
		\includegraphics[width=#3\textwidth]{#1}
		\caption{#2}\label{fig:#1}
	\end{figure}
}

% El comando \figura nos permite insertar figuras comodamente, y utilizando
% siempre el mismo formato. Los parametros son:
% 1 --> Porcentaje del ancho de página que ocupará la figura (de 0 a 1)
% 2 --> Fichero de la imagen
% 3 --> Texto a pie de imagen
% 4 --> Etiqueta (label) para referencias
% 5 --> Opciones que queramos pasarle al \includegraphics
% 6 --> Opciones de posicionamiento a pasarle a \begin{figure}
\newcommand{\figuraConPosicion}[6]{%
  \setlength{\anchoFloat}{#1\textwidth}%
  \addtolength{\anchoFloat}{-4\fboxsep}%
  \setlength{\anchoFigura}{\anchoFloat}%
  \begin{figure}[#6]
    \begin{center}%
      \Ovalbox{%
        \begin{minipage}{\anchoFloat}%
          \begin{center}%
            \includegraphics[width=\anchoFigura,#5]{#2}%
            \caption{#3}%
            \label{#4}%
          \end{center}%
        \end{minipage}
      }%
    \end{center}%
  \end{figure}%
}

%
% Comando para incluir imágenes en formato apaisado (sin marco).
\newcommand{\figuraApaisadaSinMarco}[5]{%
  \begin{figure}%
    \begin{center}%
    \includegraphics[angle=90,height=#1\textheight,#5]{#2}%
    \caption{#3}%
    \label{#4}%
    \end{center}%
  \end{figure}%
}
% Para las tablas
\newcommand{\otoprule}{\midrule [\heavyrulewidth]}
%
% Nuevo comando para tablas pequeñas (menos de una página).
\newcommand{\tablaSmall}[5]{%
 \begin{table}
  \begin{center}
   \rowcolors {2}{gray!35}{}
   \begin{tabular}{#2}
    \toprule
    #4
    \otoprule
    #5
    \bottomrule
   \end{tabular}
   \caption{#1}
   \label{tabla:#3}
  \end{center}
 \end{table}
}

%
% Nuevo comando para tablas pequeñas (menos de una página).
\newcommand{\tablaSmallSinColores}[5]{%
 \begin{table}[H]
  \begin{center}
   \begin{tabular}{#2}
    \toprule
    #4
    \otoprule
    #5
    \bottomrule
   \end{tabular}
   \caption{#1}
   \label{tabla:#3}
  \end{center}
 \end{table}
}

\newcommand{\tablaApaisadaSmall}[5]{%
\begin{landscape}
  \begin{table}
   \begin{center}
    \rowcolors {2}{gray!35}{}
    \begin{tabular}{#2}
     \toprule
     #4
     \otoprule
     #5
     \bottomrule
    \end{tabular}
    \caption{#1}
    \label{tabla:#3}
   \end{center}
  \end{table}
\end{landscape}
}

%
% Nuevo comando para tablas grandes con cabecera y filas alternas coloreadas en gris.
\newcommand{\tabla}[6]{%
  \begin{center}
    \tablefirsthead{
      \toprule
      #5
      \otoprule
    }
    \tablehead{
      \multicolumn{#3}{l}{\small\sl continúa desde la página anterior}\\
      \toprule
      #5
      \otoprule
    }
    \tabletail{
      \hline
      \multicolumn{#3}{r}{\small\sl continúa en la página siguiente}\\
    }
    \tablelasttail{
      \hline
    }
    \bottomcaption{#1}
    \rowcolors {2}{gray!35}{}
    \begin{xtabular}{#2}
      #6
      \bottomrule
    \end{xtabular}
    \label{tabla:#4}
  \end{center}
}

%
% Nuevo comando para tablas grandes con cabecera.
\newcommand{\tablaSinColores}[6]{%
  \begin{center}
    \tablefirsthead{
      \toprule
      #5
      \otoprule
    }
    \tablehead{
      \multicolumn{#3}{l}{\small\sl continúa desde la página anterior}\\
      \toprule
      #5
      \otoprule
    }
    \tabletail{
      \hline
      \multicolumn{#3}{r}{\small\sl continúa en la página siguiente}\\
    }
    \tablelasttail{
      \hline
    }
    \bottomcaption{#1}
    \begin{xtabular}{#2}
      #6
      \bottomrule
    \end{xtabular}
    \label{tabla:#4}
  \end{center}
}

%
% Nuevo comando para tablas grandes sin cabecera.
\newcommand{\tablaSinCabecera}[5]{%
  \begin{center}
    \tablefirsthead{
      \toprule
    }
    \tablehead{
      \multicolumn{#3}{l}{\small\sl continúa desde la página anterior}\\
      \hline
    }
    \tabletail{
      \hline
      \multicolumn{#3}{r}{\small\sl continúa en la página siguiente}\\
    }
    \tablelasttail{
      \hline
    }
    \bottomcaption{#1}
  \begin{xtabular}{#2}
    #5
   \bottomrule
  \end{xtabular}
  \label{tabla:#4}
  \end{center}
}



\definecolor{cgoLight}{HTML}{EEEEEE}
\definecolor{cgoExtralight}{HTML}{FFFFFF}

%
% Nuevo comando para tablas grandes sin cabecera.
\newcommand{\tablaSinCabeceraConBandas}[5]{%
  \begin{center}
    \tablefirsthead{
      \toprule
    }
    \tablehead{
      \multicolumn{#3}{l}{\small\sl continúa desde la página anterior}\\
      \hline
    }
    \tabletail{
      \hline
      \multicolumn{#3}{r}{\small\sl continúa en la página siguiente}\\
    }
    \tablelasttail{
      \hline
    }
    \bottomcaption{#1}
    \rowcolors[]{1}{cgoExtralight}{cgoLight}

  \begin{xtabular}{#2}
    #5
   \bottomrule
  \end{xtabular}
  \label{tabla:#4}
  \end{center}
}



\graphicspath{ {./img/} }

% Capítulos
\chapterstyle{bianchi}
\newcommand{\capitulo}[2]{
	\setcounter{chapter}{#1}
	\setcounter{section}{0}
        \setcounter{subsection}{0}
        \setcounter{subsubsection}{0}
	\setcounter{figure}{0}
	\setcounter{table}{0}
	\chapter*{#2}
	\addcontentsline{toc}{chapter}{#2}
	\markboth{#2}{#2}
}

% Apéndices
\renewcommand{\appendixname}{Apéndice}
\renewcommand*\cftappendixname{\appendixname}

\newcommand{\apendice}[1]{
	%\renewcommand{\thechapter}{A}
	\chapter{#1}
}

\renewcommand*\cftappendixname{\appendixname\ }

% Formato de portada
\makeatletter
\usepackage{xcolor}
\newcommand{\tutor}[1]{\def\@tutor{#1}}
\newcommand{\course}[1]{\def\@course{#1}}
\definecolor{cpardoBox}{HTML}{E6E6FF}
\def\maketitle{
  \null
  \thispagestyle{empty}
  % Cabecera ----------------
\noindent\includegraphics[width=\textwidth]{cabecera}\vspace{1cm}%
  \vfill
  % Título proyecto y escudo informática ----------------
  \colorbox{cpardoBox}{%
    \begin{minipage}{.8\textwidth}
      \vspace{.5cm}\Large
      \begin{center}
      \textbf{TFG del Grado en Ingeniería Informática}\vspace{.6cm}\\
      \textbf{\LARGE\@title{}}
      \end{center}
      \vspace{.2cm}
    \end{minipage}

  }%
  \hfill\begin{minipage}{.20\textwidth}
    \includegraphics[width=\textwidth]{escudoInfor}
  \end{minipage}
  \vfill
  % Datos de alumno, curso y tutores ------------------
  \begin{center}%
  {%
    \noindent\LARGE
    Presentado por \@author{}\\ 
    en Universidad de Burgos\\
    a \@date{}\\
    Tutores: \@tutor{}\\
  }%
  \end{center}%
  \null
  \cleardoublepage
  }
\makeatother

\newcommand{\nombre}{Gadea Lucas Pérez} %%% cambio de comando

% Datos de portada
\title{Impact Factor Oracle}
\author{\nombre}
\tutor{Virginia Ahedo y Álvar Arnaiz}
\date{\today}

\begin{document}

\maketitle


\newpage\null\thispagestyle{empty}\newpage


%%%%%%%%%%%%%%%%%%%%%%%%%%%%%%%%%%%%%%%%%%%%%%%%%%%%%%%%%%%%%%%%%%%%%%%%%%%%%%%%%%%%%%%%
\thispagestyle{empty}


\noindent\includegraphics[width=\textwidth]{cabecera}\vspace{1cm}

\noindent Dña. Virginia Ahedo, profesora del departamento de Ingeniería de Organización, área de Organización de Empresas.

\noindent Expone:

\noindent Que el alumno Dña. \nombre, con DNI 71483074V, ha realizado el Trabajo final de Grado en Ingeniería Informática titulado ``Impact Factor Oracle''. 

\noindent Y que dicho trabajo ha sido realizado por el alumno bajo la dirección del que suscribe, en virtud de lo cual se autoriza su presentación y defensa.

\begin{center} %\large
En Burgos, {\large \today}
\end{center}

\vfill\vfill\vfill

% Author and supervisor
\begin{minipage}{0.45\textwidth}
\begin{flushleft} %\large
Vº. Bº. del Tutor:\\[2cm]
Dña. Virginia Ahedo
\end{flushleft}
\end{minipage}
\hfill
\begin{minipage}{0.45\textwidth}
\begin{flushleft} %\large
Vº. Bº. del co-tutor:\\[2cm]
D. Álvar Arnaiz
\end{flushleft}
\end{minipage}
\hfill

\vfill


\newpage\null\thispagestyle{empty}\newpage




\frontmatter

% Abstract en castellano
\renewcommand*\abstractname{Resumen}
\begin{abstract}
El presente proyecto se centra en el desarrollo de una aplicación web que utiliza técnicas de aprendizaje automático para predecir el factor de impacto de las revistas científicas. Esta métrica se utiliza para evaluar la importancia de una revista en un campo científico determinado. Se mide a través de la frecuencia con la que los artículos de la misma han sido citados en un año específico.

El proyecto utilizará los datos históricos disponibles en Google Scholar como inputs para los algoritmos de aprendizaje automático. Estos algoritmos serán supervisados y se utilizarán para estimar el valor del índice de impacto de las revistas indexadas en el JCR (Journal Citation Reports). El objetivo final es desarrollar una aplicación web accesible y de fácil uso para la comunidad científica, que permita predecir la importancia de las revistas científicas en tiempo real.

Este proyecto es relevante para la comunidad científica ya que el factor de impacto es un criterio importante en la evaluación de la calidad del trabajo científico y puede ser de gran ayuda en la selección de la revista adecuada para publicar un nuevo trabajo. Además, la aplicación web será de acceso abierto y se encontrará en un repositorio público para garantizar su disponibilidad y uso por parte de la comunidad científica. 

\end{abstract}

\renewcommand*\abstractname{Descriptores}
\begin{abstract}
Bibliometría, publicación de artículos, revistas científicas, índice de impacto.
\end{abstract}

\clearpage

% Abstract en inglés
\renewcommand*\abstractname{Abstract}
\begin{abstract}
This project focuses on the development of a web application that uses machine learning techniques to predict the impact factor of scientific journals. The impact factor is a metric used to assess the importance of a journal in a given scientific field. It is measured through the frequency with which the articles of a journal have been cited in a specific year.

The project will use the historical data available in Google Scholar as inputs for the machine learning algorithms. These algorithms will be supervised and used to estimate the impact index value of the journals indexed in the JCR (Journal Citation Reports). The ultimate goal is to develop an accessible and easy-to-use web application for the scientific community, which allows predicting the importance of scientific journals in real time.

This project is relevant for the scientific community since the impact factor is an important criterion in the evaluation of the quality of scientific work and can be of great help in the selection of the appropriate journal to publish a new work. In addition, the web application will be open access and will be in a public repository to guarantee its availability and use by the scientific community.
\end{abstract}

\renewcommand*\abstractname{Keywords}
\begin{abstract}
Bibliometrics, articles publication, scientific journals, impact index.
\end{abstract}

\clearpage

% Indices
\tableofcontents

\clearpage

\listoffigures

\clearpage

\listoftables
\clearpage

\mainmatter
\capitulo{1}{Introducción}

El factor de impacto de una publicación científica mide la frecuencia con la cual ha sido citado el artículo promedio de una revista en un año en particular. Específicamente, sirve para evaluar la importancia de una revista dentro de un determinado campo científico. Existen múltiples metodologías de cálculo y sus correspondientes métricas, siendo el JCR (Journal Citation Reports) y el SJR (Scimago Journal Rank) los dos índices de impacto más utilizados.
Además, cabe destacar que el factor de impacto es uno de los principales criterios empleados en los procesos de acreditación y promoción interna para evaluar la calidad del trabajo científico de millones de académicos en todo el mundo.

Por su propia naturaleza, el factor de impacto se calcula con carácter retrospectivo, i.e., sobre datos de años anteriores. Así pues, si a la hora de seleccionar la revista a la que mandarán un nuevo trabajo, los académicos quieren tener en cuenta el posible factor de impacto que tendrá la revista en el año de publicación del artículo, lo único que pueden hacer es fijarse en su índice de impacto de los años anteriores y hacer sus propias hipótesis/predicciones de futuro.

Dado que a día de hoy la herramienta Google Scholar recoge información extremadamente actualizada sobre la publicación y citación de artículos científicos (podría decirse que se actualiza prácticamente en tiempo real), creemos que puede ser muy útil para la comunidad científica utilizar los datos de Google Scholar como \textit{inputs} de modelos de aprendizaje automático para estimar el índice de impacto que tendrán las distintas revistas científicas en el año en curso.

El \textit{output} esperado del proyecto será una aplicación web de tipo open-access, la cual implementará algoritmos de aprendizaje supervisado que utilizarán los datos históricos disponibles en Google Scholar para predecir el valor del índice de impacto de todas las revistas científicas indexadas en el JCR. Dicha aplicación se dejará en un repositorio público, para así garantizar que pueda ser utilizada por toda la comunidad científica. 

\capitulo{2}{Objetivos del proyecto}

A continuación, se enumeran los principales objetivos de este proyecto:

\begin{enumerate}
\item Lectura de literatura científica sobre bibliometría para comprender bien el marco conceptual en el que se encuadra este proyecto.
\item Estudio de la metodología de cálculo de los distintos índices de impacto en general, y del JCR (Journal Citation Report) en particular. Asimismo, se estudiarán las sucesivas modificaciones/excepciones que se han ido introduciendo en el cálculo del JCR a lo largo de los años.
\item Estudio de la API de Google Scholar y otras APIs.
\item Diseño y creación de una base de datos en la que se almacenará la información bibliográfica descargada.
\item Implementación de las funciones de cálculo del índice JCR para aplicarlas sobre los datos extraídos. 
\item Experimentación y evaluación de distintos modelos de regresión (aprendizaje supervisado) para predecir el índice de impacto JCR a partir de las series temporales históricas disponibles. Selección del mejor algoritmo.
\item Diseño y creación de una aplicación web donde se incluirá el mejor modelo de regresión para ponerlo a disposición de la comunidad científica.
\end{enumerate}
\capitulo{3}{Conceptos teóricos}

Cuando un investigador termina su trabajo, genera un artículo científico en el que plasma el resultado de su investigación. Esta documentación sirve de precedente para aquellos que posteriormente investiguen sobre temáticas relacionadas. Así pues, el artículo científico es el elemento principal en torno al que giran los estudios bibliométricos.

Antes de adentrarnos en más detalles, se ilustrarán algunos conceptos esenciales sobre este proyecto, para dotarlo de mayor comprensión y claridad. Así pues, en esta sección se abarcarán los conceptos de bibliometría, cienciometría e índice de impacto; conceptos relativos al avance de la ciencia y la producción de conocimientos a partir de la actividad de la investigación.


\section{Bibliometría}

Biblio (libros) - metría (medición)

Ciencia que aplica métodos matemáticos para encontrar comportamientos estadísticos en la literatura científica. Estos estudios y análisis pretenden cuantificar toda la actividad científica escrita con el objetivo final de orientar sobre el impacto de una investigación.

Los estudios bibliométricos tienen origen en 1960 con la aparición del Science Citation Index (Eugene Garfield) y el análisis de redes de citas (Derek John de Solla Price), que sentaron las bases fundamentales sobre bibliometría.

El objetivo de los estudios bibliométricos pueden limitase solamente al análisis de la envergadura, el crecimiento y la distribución de la literatura científica, pero también son útiles a la hora de encontrar las revistas donde es más conveniente publicar un artículo o descubrir a los autores más importantes en cada ámbito, así como las nuevas tendencias.

Sin embargo, la información organizada de esta forma también tiene sus limitaciones. Ejemplo de ello es el patrón de citas usado en cada área de investigación; es mucho más común incluir citas en investigaciones tecnológicas y científicas. Es decir, cada temática deberá ser tratada de forma distinta. Para ello se normalizará el índice de impacto.
Por otro lado, también influye la base de datos utilizada, ya que cada una tiene un método de indexación distinta.
El idioma es otro ejemplo de limitación. Ya que el inglés es el idioma predominante, será más complicado encontrar citas de documentos 	escritos en otros idiomas.
También es preciso tener en cuenta problemas de dispersión debidos a perfiles duplicados o nombres similares. Para poder solucionar este tipo de confusiones, es recomendable crear identificadores (ORCID) para cada autor.
De esta forma, todos los comportamientos irregulares (como las citas a uno mismo) deben tenerse en consideración.
Por otro lado, el momento en que se realizan los estudios es también fundamental, ya que los datos pueden cambiar de un momento a otro.

\subsection{Ejemplo}

Un ejemplo de análisis bibliométrico podría consistir en averiguar los autores que, en cierta revista, traten un tema en concreto. Para ello se deberían anotar los siguientes datos:
\begin{enumerate}
    \item Lista de identificadores de los autores (ordenados alfabéticamente) aparecidos en la revista en cuestión.
    \item Cantidad de artículos de cada uno de estos autores en la revista.
    \item Revistas en las que aparecen los artículos citados.
    \item Los artículos (cantidad) que aparecen en cada una de las revistas listadas.
\end{enumerate}


\section{Cienciometría}

Ciencio (ciencia) - metría (medición)

Estudio de los metadatos e indicadores sobre la bibliografía científica con el fin de medir y analizar toda la producción científica.
La cienciometría es útil para ordenar conocimientos científicos mediante la creación de sistemas de conceptos. También para transferir conocimientos a través de la enseñanza y la formación. A su vez, también busca comunicar conocimientos de un lenguaje a otro a través de símbolos lingüísticos. Logra resumir y sintetizar la información científica y, por último, recupera y almacena información mediante la indexación y lenguajes documentales.

La cienciometría, puesta en práctica, trabaja relacionada con otras ciencias y disciplinas como las que se muestran en la figura 1 (insertar imagen en la plantilla del TFG).

(Idea aproximada de la imagen)
Figura 1: Ambos campos se superponen en gran medida


Es decir, se trata de un concepto más amplio que engloba al anterior (bibliometría). Se podría decir que la bibliometría surge como resultado del contacto interdisciplinar de entre el conjunto de disciplinas que integran lo que se conoce como ``ciencia de la ciencia'', cuya fuente es la propia cienciometría.
 

\section{Índice de impacto}

Existe un gran abanico de índices distintos englobados dentro de la bibliometría. Sin embargo, en este caso, nos centraremos especialmente en los índices de citas o, en inglés, ``Citation Index''.

Se trata de índices de autor con características especiales, pues no solamente citan junto a cada autor la lista de los documentos por él publicados, sino
que añade, en cada referencia, la lista de los documentos que ha citado esta referencia en su propia bibliografía. Permiten localizar otros autores que han tratado las mismas materias y buscar también documentos más recientes que éste ya conocido.
Contienen literatura científica de medicina, psicología, agricultura, tecnología y documentación científica en general. 


\subsection{Ejemplo}

En un índice de citas, aparecen los artículos de un mismo autor agrupados por orden cronológico. Justo debajo, aparece la lista de autores que han citado este artículo y,
a su lado, la referencia del documento (ver figura 2).

	(Insertar imágen en plantilla del TFG)

Es deir, bastará con saber el nombre de un solo autor que trabaje en el campo en el que se esté interesado, para obtener de forma inmediata (consultando simplemente el índice) el conjunto de documentos que traten del mismo tema.

Generalmente se divide en dos subíndices: uno de autores (con referencias de documentos citados), y otro de fuentes (acompañadas de los títulos de los documentos).

Otros intereses que se deducen de este índice de autores consiste en el valor que puden  tener los documentos citados debido a que el autor de un documento, especialista en un campo, a tenido interés en dar a conocer a dichos autores.


\subsection{SJR}

SJR (SCImago Journal Rank) es un indicador de calidad de revistas científicas basado en la idea de que "más citadas son mejores." El indicador se calcula utilizando un algoritmo que también tiene en cuenta la importancia de la revista que hace la cita, de tal manera que una cita de una revista de alto impacto tendrá más peso que una cita de una revista de bajo impacto. Este indicador se utiliza para medir la calidad y el impacto de una revista en la comunidad científica.

\subsection{Hindex}
Hindex es un indicador de productividad y visibilidad de un investigador. El Hindex se calcula a partir de la cantidad de artículos publicados por un investigador y la cantidad de veces que estos artículos han sido citados. El Hindex se determina buscando el número de publicaciones con al menos ese número de citas. Por ejemplo, un investigador con un Hindex de 10 tiene al menos 10 publicaciones que han sido citadas al menos 10 veces cada una. Este indicador se utiliza para medir la importancia y el impacto de un investigador en la comunidad científica.

\subsection{CiteScore}
Es un indicador desarrollado por Scopus que tiene en cuenta el número de citas recibidas por una revista en un período de tres años, el número de documentos en la base de datos de Scopus y el número de documentos publicados en esa revista en ese período.

\subsection{g-index}
Es un indicador que se utiliza para medir la productividad y la impacto de un investigador. El g-index se calcula ordenando los artículos de un investigador por el número de citas y luego identificando el número más grande "g" tal que al menos "g" de los artículos han sido citados al menos "g" veces.

\section{JCR}
El JCR (\textit{Journal Citation Reports}) es una base de datos desarrollada por Clarivate Analytics que proporciona información sobre revistas científicas. 

Consiste en el análisis biométrico de las revistas del banco de datos ISI, con número de citas al año de cada una, revistas que citan a otras revistas, listado de abreviaturas de títulos y su desarrollo, etc. JCR proporciona información estadística sobre la frecuencia de citas de las revistas, incluyendo el \textit{Impact Factor} (IF) y el \textit{Impact Factor of a Year} (IFY) entre otros indicadores. 

De entre los datos estadísticos que se obtienen, nos importa especialmente el ya mencionado \textbf{Factor de Impacto}, que permiten determinar de una manera sistemática y objetiva la importancia relativa de las principales revistas de investigación internacionales dentro de sus categorías temáticas. 

JCR cubre más de 12,000 revistas de más de 80 disciplinas diferentes, lo que permite comparar el impacto y la calidad de las revistas en diferentes campos. Los datos de JCR se utilizan a menudo como un indicador de la calidad y el impacto de una revista en la comunidad científica.

La última actualización del JCR ofrece los datos del Factor de Impacto del 2022.

 

\capitulo{4}{Técnicas y herramientas}

En esta sección se presentarán las distintas herramientas y recursos que se han utilizado para la realización del proyecto. 
% aquí debería venir una pequeña enumeración de lenguajes, bibliotecas, APIs, BBDD...

\section{Lenguaje de programación}
En primer lugar, nos planteamos la cuestión del lenguaje o lenguajes de programación más eficaces para nuestro objetivo. Las listas de popularidad actuales nos muestran dos entornos ganadores para proyectos de \textit{scraping}: Python y Javascript. Aunque ambos lenguajes son altamente capaces para nuestro proyecto, la enorme base de conocimientos y la diversidad de herramientas creadas en el universo \textbf{Python} decanta la balanza hacia ese lado. Quizás JavaScript permita mejores resultados usando la gestión de memoria en \textit{requests} simultáneas, pero a costa de un código más oscuro y difícil de mantener. Aunque JavaScript cuenta con un gran repertorio de paquetes Node.JS como utilidades de \textit{scraping}, en el entorno Python es difícil imaginar una tarea para la que no se haya escrito una (o más) herramientas que resuelvan eficazmente nuestro problema. Por otro lado, la comunidad de programadores de Python es inmensa y su creciente popularidad facilita el hallazgo de soluciones rápidamente, tanto en los foros como en la extensa documentación con la que cuenta. Aunque encontramos un rendimiento ligeramente inferior a otros lenguajes en ciertas búsquedas, es el precio a pagar por el tipado dinámico. Como valor añadido, Python es fácil de mantener cuando necesitamos adaptar nuestro código a las cambiantes estructuras de las páginas web. Además, sus reconocidas herramientas de análisis de datos nos permiten continuar en el mismo entorno, sin necesidad de buscar alternativas para afrontar tareas relacionadas con la \textit{data science}.

\section{Bibliotecas}
A lo largo del proyecto se ha recurrido a diversas bibliotecas de \textbf{Python}. A continuación, se presenta brevemente cada una de ellas.

\subsection{Scholarly}
Se trata de una librería de Python que permite acceder a los datos de Google Scholar de manera fácil y rápida. La librería proporciona una interfaz sencilla para buscar y recuperar información sobre artículos, autores y revistas en Google Scholar, incluyendo metadatos, citas y otra información relacionada.

Sin embargo, esta biblioteca ha terminado siendo descartada para este proyecto. A diferencia de otras técnicas de \textit{web scraping}, Scholarly no está diseñada para extraer grandes cantidades de datos de Google Scholar. La biblioteca tiene una serie de limitaciones en cuanto a la cantidad de datos que se pueden recolectar, ya que está diseñada para ser utilizada en investigaciones científicas y no para la extracción masiva de datos. Además, Scholarly está diseñada para respetar los términos de servicio de Google Scholar y no violar la política de uso de la plataforma, por lo que no se recomienda su uso para recolectar grandes cantidades de datos.

En resumen Scholarly es una herramienta útil para acceder a información científica y académica de manera rápida y sencilla, pero no esta diseñada para extraer grandes cantidades de datos.

\subsection{Beautiful Soup}
Beautiful Soup es una biblioteca de Python extremadamente útil para la extracción de datos de páginas HTML o XML. Actualmente se encuentra en su versión 4.8.1. En el siguiente \href{https://beautiful-soup-4.readthedocs.io/en/latest/}{enlace} se puede acceder a la documentación de su página oficial. 

Esta biblioteca nos proporciona numerosos módulos para navegar a través de las páginas web y para extraer fácilmente su contenido. Puesto que gran parte del proyecto se basa en el uso de \textit{web scrapping}, se ha hecho uso intensivo de la misma para extraer los principales datos de los artículos científicos que posteriormente conformarán la BBDD del proyecto.

De Beautiful Soup hay que destacar su facilidad de uso y la amplia documentación que aporta. En su página web nos aconseja utilizar el analizador \textit{lxml}, que proporciona al entorno Python la disponibilidad de las bibliotecas \textit{libxml2} y \textit{libxslt}. Allí mismo, se anima incluso a utilizar aisladamente este parser cuando el tiempo de respuesta sea una cuestión crítica. En nuestro caso, las facilidades que proporciona Beautiful Soup justifican ampliamente su uso, aunque la  rapidez de resultados no iguale la de la utilización aislada de los analizadores sobre los que trabaja.
 

\subsection{Habanero}
Para poder hacer uso de la \textbf{API de Crossref}, se han explorado diversas alternativas. De entre todas las bibliotecas de Python se ha seleccionado Habanero, ya que es una librería muy fácil de usar y esta en constante actualización.

Esta biblioteca está diseñada para facilitar el acceso a las bases de datos de revistas científicas y a otras relacionadas con el ámbito académico. Ofrece una interfaz simple para recuperar información utilizando los protocolos y las API de diferentes bases de datos, incluyendo JSTOR, Unpaywall, Crossref, DataCite, etc. Además, Habanero es compatible con las normas de Open Access, lo que permite a los usuarios acceder a contenido científico gratuito y de libre acceso.


\section{Bases de datos}

Antes de su compra por Oracle Corporation (2010), MySQL era la aplicación de base de datos más popular de código abierto para la programación web. En el momento actual. el software libre nos ofrece dos soluciones ampliamente contrastadas en gestores de bases de datos relacionales: MariaDB y PostgreSQL. Necesitaremos una herramienta de este tipo para organizar los datos recolectados. La estructura tabular de la información nos permite aplicar sobre ella el lenguaje de interrogación SQL.

En realidad, MariaDB es una bifurcación de MySQL nacida para garantizar la supervivencia del proyecto como código abierto. Hoy en día, MariaDB es altamente compatible con MySQL y superior en sus últimas versiones (10.1.1), ya que la comunidad ha ido añadiendo nuevas características al proyecto original. Por otro lado, aunque no está tan extendido como MySQL, PostgreSQL es posiblemente el gestor de bases de datos de código abierto más solido y potente a día de hoy.


\tablaSmall{Comparativa entre MariaDB y PostgreSQL}{r l}{comparativabbdd}
{ MariaDB & PostgreSQL\\}{ 
No totalmente compatible con SQL & Compatible con SQL estándar\\
Soporte para tipos de datos estándar SQL & Soporta además tipos avanzados\\
Tipos de datos flexibles & Tipos de datos estrictos \\
Tamaño pequeño de base de datos & Tamaño grande de base de datos \\
Sin soporte directo para JSON  & Soporte directo de JSON \\
No índices parciales & Índices parciales \\
No soporte para web dinámica & Soporta sitios web dinámicos\\
} 


\imagen{MariaDB_icon}{Icono de MariaDB}{.40}
\imagen{Postgresql_icon}{Icono de Postgres}{.20}


\subsection{PostgreSQL}
Finalmente se opta por \href{https://www.postgresql.org/}{PostgreSQL}. Pese a que ambas opciones son muy similares, se ha elegido esta última debido a que ya se ha trabajado con ella anteriormente. Además, PostgreSQL posee una sólida reputación por su arquitectura comprobada, confiabilidad, integridad de datos, conjunto sólido de características, extensibilidad y la dedicación de la comunidad de código abierto detrás del software para ofrecer soluciones innovadoras y de rendimiento constante \cite{Postgresql.org}.


También se denomina Postgres. Al igual que MariaDB, es un sistema de gestión de bases de datos relacional de código abierto. Así pues, está dirigido y desarrollado por una comunidad altruista de desarrolladores (PostgreSQL Global Development Group).
PostgreSQL utiliza y amplía el lenguaje SQL combinado con muchas características que almacenan y escalan de forma segura las cargas de trabajo de datos más complicadas \cite{Postgresql.org}. 
Su última versión y la usada para el proyecto es la versión 15.1 lanzada el 10 de noviembre de este año.



\section{APIs}
A lo largo del proyecto se ha recurrido a varias APIs diferentes, todas ellas de acceso gratuito.

\subsection{Google Scholar}
Se trata de una de las principales herramientas que ofrece Google a los investigadores. \href{https://scholar.google.com/}{Google Scholar} es, fundamentalmente, un buscador de contenido y bibliografía científica que permite localizar artículos de revistas especializadas ordenados por relevancia en función de las palabras clave introducidas en el buscador. También se puede  filtrar la información en función de su fecha de publicación, idioma o número de citas.
\imagen{GoogleScholar_icon}{Icono de Google Scholar}{.20}


\subsection{Crossref}
\href{https://www.crossref.org/}{Crossref} es una herramienta que facilita el  acceso a la información de los artículos científicos a partir de su DOI.

\nota{El DOI (\textit{Digital Object Identifier}) es el acrónimo con el que se conoce al identificador inequívoco de un artículo científico. Se trata de un enlace permanente al contenido electrónico de dicho artículo. Por lo general, un DOI tiene forma de código alfanumérico.}

Crossref es una organización sin fines de lucro que pretende facilitar las conexiones académicas.

\imagen{crossref_icon}{Icono de Crossref}{.20}

Como ya se ha mencionado previamente, en base a la información de los DOIs la agencia de registro CrossRef es capaz de proporcionarnos aplicaciones útiles para hacer nuestro flujo de investigación más sencillo.
Se trata de una asociación sin ánimo de lucro de editoriales científicas que no solo facilita el registro de DOIs a las editoriales, sino que también ofrece servicios y aplicaciones para el personal investigador que tienen como base estos códigos.

Los DOIs que CrossRef almacena van acompañados de información que refleja las cualidades básicas de una publicación científica. Me estoy refiriendo a datos como títulos, abstracts, palabras clave, autores…

CrossRef Metadata Search hace posible obtener toda esta información al instante con tan solo proporcionar el DOI asociado a una publicación, o al contrario, obtener el DOI de la publicación con tan solo introducir algunos de estos datos en su buscador.









\capitulo{5}{Aspectos relevantes del desarrollo del proyecto}

%Este apartado pretende recoger los aspectos más interesantes del desarrollo del proyecto, comentados por los autores del mismo.
%Debe incluir desde la exposición del ciclo de vida utilizado, hasta los detalles de mayor relevancia de las fases de análisis, diseño e implementación.
%Se busca que no sea una mera operación de copiar y pegar diagramas y extractos del código fuente, sino que realmente se justifiquen los caminos de solución que se han tomado, especialmente aquellos que no sean triviales.
%Puede ser el lugar más adecuado para documentar los aspectos más interesantes del diseño y de la implementación, con un mayor hincapié en aspectos tales como el tipo de arquitectura elegido, los índices de las tablas de la base de datos, normalización y desnormalización, distribución en ficheros3, reglas de negocio dentro de las bases de datos (EDVHV GH GDWRV DFWLYDV), aspectos de desarrollo relacionados con el WWW...
%Este apartado, debe convertirse en el resumen de la experiencia práctica del proyecto, y por sí mismo justifica que la memoria se convierta en un documento útil, fuente de referencia para los autores, los tutores y futuros alumnos.

En esta sección se presenta un resumen de los hallazgos más relevantes obtenidos a lo largo del proyecto, así como una descripción detallada de las distintas fases por las que se ha atravesado hasta lograr alcanzar una solución satisfactoria al problema planteado. A saber, la fase de extracción de datos, la fase de aprendizaje automático y la fase de creación de la aplicación web.

\imagen{ciclodevidaproyecto}{Fases del proyecto}{0.7}

\section{Extracción de la información}
La primera fase de este proyecto es la más extensa y consiste en la extracción de datos para alimentar la base de datos con la que entrenar los modelos. Para esto, es necesario tener en cuenta varias bases de datos bibliográficas que ofrecen diversos recursos para los investigadores. Entre las más importantes se encuentran Google Scholar, Scopus, WoS y Crossref.

\subsection{Prototipos para Google Scholar}

De todas las bases de datos bibliográficas mencionadas, la más destacada es Google Scholar, puesto que crece más rápido que cualquier base de datos tradicional en todos los campos científicos~\cite{harzing2010, lopez2017}.

Así pues, el primer prototipo diseñado consiste en la extracción de datos de Google Scholar. Sin embargo, si bien es cierto que Google Scholar es un motor de búsqueda gratuito, universal y rápido con una amplia cobertura, tiene muchas limitaciones~\cite{lopez2017}. 


Por ejemplo, no hay funcionalidades de exportación de datos o API disponibles debido a restricciones comerciales. Además, solo se pueden mostrar los primeros 1\,000 resultados de cada consulta. Aunque algunas técnicas como los retrasos temporales o el uso de \textit{proxies} pueden ayudar, no resuelven completamente estas limitaciones. Además, la extracción automatizada de estos datos va en contra de las políticas del archivo \texttt{robots.txt} de Google Scholar, lo que hace que los usuarios que realizan demasiadas consultas automatizadas sean bloqueados cada 200 solicitudes detectadas~\cite{lopez2017}.

Durante el avance del proyecto, nos dimos cuenta de muchas de las limitaciones mencionadas.

\subsubsection{Prototipo inicial}

Tras comprender que las complicaciones eran numerosas, se decidió crear un prototipo sencillo, consistente en un \textit{script} en Python, que trataría de lanzar mil peticiones de búsqueda. Esto nos permitiría establecer los límites de realizar \textit{web scrapping} sobre Google Scholar.

Así pues, manos a la obra, se desarrolló un \textit{script} sencillo, que solicita acceso a la página principal de Google Scholar y, mediante métodos HTTP, realiza una búsqueda (a partir de parámetros solicitados por pantalla). Finalmente, extrae el título de la página resultante tras hacer la búsqueda. Todo esto se logra haciendo uso de la biblioteca de Python BeautifulSoup.

% Filtro por revistas

El primer obstáculo encontrado es que Google Scholar no permite hacer búsquedas en función de la revista. Si se trata de escribir el nombre de una revista en el buscador, aparecerán artículos relacionados que mencionan esa revista, pero no siempre artículos de la revista en cuestión. En cambio, permite hacer búsquedas a partir de palabras clave.
Sin embargo, aunque no se puede buscar directamente por revista, sí se puede extraer de los resultados encontrados tras buscar una temática concreta.

\imagen{ejemploResultadoGS}{Ejemplo de búsqueda por palabra clave}{0.8}

Sin embargo, existe una problemática: en cada artículo se hace referencia a la revista que lo publica de forma distinta. Por ejemplo, como se puede apreciar en la figura~\ref{fig:etiquetaDeRevista}, la revista se encuentra ubicada en una <<etiqueta>> HTML diferente.

\imagen{etiquetaDeRevista}{Localización de la revista en Google Scholar}{0.8}

Por ello se buscó una solución alternativa.
Existe una etiqueta (\texttt{<b> … </b>}) que incluye aquellas palabras resaltadas en negrita en la página web. Google Scholar resalta en una búsqueda aquellas palabras que coinciden con alguno de los parámetros buscados. Así pues, si se realiza una búsqueda en Google Scholar pasando como parámetro el nombre de la revista que nos interesa, bastará con buscar en la clase  \texttt{gs\_a} (donde se almacenan los detalles del artículo) alguna etiqueta \texttt{<b>}.
Se ha escrito un pequeño código de prueba y, en principio, parece que funciona sin problemas. Se adjunta dicho código (\texttt{extraerRevistas.py}) en el repositorio de GitHub. %TODO: meter enñace
Este código hace dos búsquedas: una a la revista <<Alas Peruanas>> y otra a la revista <<The Lancet>> (elegidas de forma aleatoria).

%La salida obtenida (solo de la primera página de resultados) es la siguiente:

%\imagen{salidaObtenida}{Salida por consola}{1}

%La salida por pantalla mostrada en la imagen anterior sigue la siguiente estructura: [\texttt{‘nombre artículo’, ‘nombre de la revista’, número\_citas, fecha, id}]
La salida que se mostrará por pantalla sigue la siguiente estructura: [\texttt{‘nombre artículo’, ‘nombre de la revista’, número\_citas, fecha, id}]

De igual forma se recogen los resultados en los CSVs (codificados en UTF-8) con nombre: \texttt{BBDD1.csv} y \texttt{BBDD2.csv}.


% Paginación

Otra de las mejoras que se implementa en el prototipo en la capacidad de navegar a través de las distintas páginas de resultados de Google Scholar. Para ello, se incluye un nuevo campo en la consulta de búsqueda que irá aumentando de 10 en 10 por cada página de resultados nueva que se consulte.
Ahora ya se obtiene un número de resultados considerable (aproximadamente 100). Estos resultados se guardarán también en los CSVs (codificados en UTF-8) con los siguientes nombres: \texttt{BBDD3.csv} y \texttt{BBDD4.csv}.

% CAPTCHA y proxies

Tras diseñar y programar el código mencionado, se procede a su prueba. La primera ejecución del \textit{script} resultó desaletadora, ya que, a partir de la solicitud número 726, Google Scholar detecta que un \textit{bot} está realizando búsquedas. A partir de ese momento, nuestra dirección IP queda bloqueada y las solicitudes fallan sin excepción. Google Schoolar nos redirecciona a una página (figura \ref{fig:error}) donde se solicita al usuario resolver un \textit{captcha}. 

\imagen{error}{Captura de reCAPTCHA de Google}{0.8}

El número de solicitudes exitosas es demasiado bajo para cumplir su función en nuestro proyecto, por lo que se procede a buscar una solución alternativa. 

Tras distintas pruebas, se encontró una forma de superar la barrera del \textit{captcha}. A saber: añadiendo a la \textit{url} una sección de texto extra que permite suprimir esta excepción durante un periodo concreto de tiempo. Así pues, logramos ejecutar con éxito el \textit{script} tantas veces como fuese necesario. Sin embargo, esta solución tampoco es válida a largo plazo. Si se intenta ejecutar de nuevo el programa, en otra sesión, la dirección vuelve a ser inválida. Además, es un remedio poco práctico, ya que se debe concatenar distintos parámetros y cadenas de texto que, dependiendo del momento y el contexto en que se ejecute, pueden no servir.

Como la propuesta anterior no fue satisfactoria, se siguieron buscando opciones. La siguiente propuesta consistía en usar un agente de usuario distinto para cada solicitud\footnote{Un agente de usuario es cualquier software, que actúa en nombre de un usuario, que <<recupera, presenta y facilita la interacción del usuario final con el contenido web>>. Algunos ejemplos destacados de agentes de usuario son los navegadores web. La cadena User-Agent es uno de los criterios por los cuales los rastreadores web pueden ser excluidos del acceso a ciertas partes de un sitio web utilizando el Estándar de exclusión de robots (\texttt{archivo robots.txt}).}. De esta forma <<enmascaramos>> nuestra dirección IP. La biblioteca Request de Python ofrece métodos para lograrlo. Dicho esto, se implementó un método que extrae \textit{proxies} de listas públicas y gratuitas de Internet (e.g.: \href{https://www.proxyscrape.com/free-proxy-list}{proxyscrape.com}). Se prueba la ejecución del prototipo una vez más y, finalmente, funciona sin inconvenientes. Ahora ya se puede decir que el proyecto es \textbf{viable}.


% DOI y Crossref
\subsubsection{Prototipo con extracción del DOI}

La siguiente meta de nuestro prototipo es la obtención del DOI, que es un campo fundamental que funcionará a modo de clave primaria en la base de datos. De esta forma se busca evitar futuros errores a la hora de identificar un artículo o a la hora de comparar artículos para eliminar duplicados. Sin embargo, se plantea una problemática al respecto: Google Scholar no comparte el DOI de los artículos en ninguna división de su página web (si bien es cierto que algunos artículos lo incluyen al final de su URL, pero no siempre ocurre esto). Puesto que por el momento no existe una forma de obtenerlo directamente, se han propuesto varias soluciones, a saber:
\begin{itemize}
	\item Comprobar el porcentaje de éxito de extraer el DOI de la URL de los artículos (en aquellos casos en los que aparezca).
	\item Acceder a la página del artículo en cuestión y extraerlo de dicha página.
         \item Introducir el nombre del artículo en la página de \href{https://www.crossref.org/}{Crossref} o similares, que te permiten obtener el DOI del artículo cuyo nombre pases como parámetro.
\end{itemize}
Como conclusión, se descarta la primera opción tras hacer una breve prueba, ya que falla en seguida. Se descarta también la segunda opción, ya que cada página sitúa el DOI en un sitio haciendo muy difícil la búsqueda del mismo a través de un algoritmo. Por lo tanto, la opción más adecuada en este caso es la tercera. Aunque supone una búsqueda extra en la complejidad algorítmica del prototipo, es la única forma segura de calcular el DOI sin equivocaciones. Por lo tanto, se incluye en el bucle del prototipo un nivel extra de profundidad de búsqueda.

Para hacer la búsqueda en Crossref es necesario indicar tanto el título como el año de publicación, a fin de asegurarnos de que se obtiene exactamente el artículo que deseamos y no otros similares. Además, es preciso tener en cuenta que Crossref no permite a los programas hacer búsquedas desde la misma interfaz que los usuarios (para evitar que los programas bloqueen las búsquedas de los usuarios). Por lo tanto, se realizarán las búsquedas en la dirección donde se ubica la API creada por Crossref especialmente para la extracción de datos por parte de programas.

% Multihilo
\subsubsection{Prototipo multihilo}

Con el prototipo listo, se prueba a extraer la información de los artículos de las primeras 20 páginas de resultados de Google Scholar de dos revistas. Los resultados no son muy esperanzadores: se obtienen 180 artículos de cada revista en 15 minutos. Estos resultados no son eficientes, por lo que se tratará de optimizar el prototipo siguiendo dos pautas. 

\begin{itemize}
    \item La primera pauta es tratar de extraer las llamadas a Crossref de forma que solo se tenga que realizar una única llamada una vez que se han extraído el resto de detalles sobre los artículos.
    \item La segunda pauta es emplear programación concurrente para ejecutar varios hilos al mismo tiempo.
\end{itemize}

Así, cada revista será procesada por un hilo distinto. Para ello, será necesario recurrir a la librería de Python \textit{multiprocessing}. Tras actualizar estos cambios, el prototipo obtiene resultados mucho más rápidos: obtiene los 180 artículos de ambas revistas en tan solo 8 minutos. El próximo objetivo es determinar el número máximo de hilos que se pueden lanzar sin impactar negativamente el rendimiento del prototipo.

Para ello, se solicitó permiso para acceder a los servidores de la universidad. A través de SSH, nos conectamos a una de las máquinas y, utilizando un entorno virtualizado de Miniconda, comenzamos a lanzar hilos. Sin embargo, el resultado fue desastroso, ya que al realizar 200 llamadas a Google Scholar, nos bloqueaban y teníamos que resolver un CAPTCHA. Intentamos insertar esperas de tiempo aleatorias y cambiar el user-agent en cada una de las llamadas. Incluso se probó a cambiar el término de búsqueda. Sin embargo, nada de esto funcionó. Matemáticamente, después de aproximadamente 200-250 llamadas, aparecía el CAPTCHA.


\subsubsection{Prototipo con Selenium}
La única solución posible frente a las restricciones de Google Scholar es tratar de <<humanizar>> las búsquedas para que parezca que las realiza un ser humano. Por lo tanto, se decidió emplear Selenium, que nos permite emular los pasos que realizaría un usuario normal al hacer una búsqueda. El nuevo prototipo entra directamente en el navegador con la ruta de la búsqueda, lo que ahorra la necesidad de realizar llamadas adicionales. Luego, hace clic en la sección citar de cada uno de los resultados de la búsqueda y aparece un cuadro emergente del cual se puede extraer la información completa de la cita. Esto resuelve otro de los problemas que teníamos con el prototipo anterior: Google Scholar acorta los nombres largos añadiendo puntos suspensivos, mientras que en la sección citar aparecen los nombres completos. 

Después de extraer esta información, se cierra el cuadro y se procede a realizar la misma acción en el siguiente artículo. Cuando se llega al décimo resultado (ya que solo hay 10 artículos por página de resultados), se emula un clic en el botón <<Siguiente>> y se procede a realizar lo mismo en la siguiente página de resultados, todo con pausas aleatorias de tiempo entre cada acción. Posteriormente, se procesan los datos de las citas usando expresiones regulares y transformando los diccionarios de datos en un CSV.

Como nota adicional, cabe mencionar que, si se incluye la etiqueta \texttt{source: <nombre de la revista>} en la búsqueda, es posible hacer una búsqueda solo por el nombre de la revista, algo que desconocíamos hasta ahora.

Aunque este método ha probado ser mejor que el anterior, ya que se obtuvieron casi 300 resultados, Google Scholar detecta que son consultas automatizadas y salta el CAPTCHA al final.

Para intentar resolver esta situación, se ha intentado superar el CAPTCHA emulando un clic sobre el botón correspondiente. No obstante, para lograrlo es necesario superar varias capas ocultas que se encuentran sobre dicho botón, lo cual resulta complejo. Una vez se logra acceder al botón, aparecen las imágenes que se deben reconocer y esto no se puede automatizar de manera sencilla. Se llega a la conclusión de que no es posible forzar más llamadas de las permitidas en Google Scholar, tal y como se ha mencionado en otros trabajos, como en el caso de \textit{Publish or Perish}~\cite{harzing2010}.

\subsection{Web of Science y Scopus}

Se consideraron también estas dos potentes fuentes bibliográficas. Sin embargo, para acceder a los datos a través de sus APIs correspondientes, se requiere una licencia que permita su uso. A pesar de haber solicitado dichas licencias, las limitaciones y restricciones de ambas APIs han impedido la realización de prototipos que cumplan con los objetivos requeridos para este proyecto. Finalmente se terminan descartando ambas opciones.

\subsection{Prototipos para Crossref}

Vista la imposibilidad de obtener una cantidad aceptable de resultados de Google Scholar, Scopus y WoS, se decidió cambiar la fuente de datos. En este caso, Crossref (aunque no es una base de datos tan potente, tiene una API gratuita y sin tantas restricciones).

Este modelo nos permite la extracción de una gran cantidad de datos en un tiempo reducido. Para ilustrar la eficiencia de este modelo, se ha generado una gráfica donde se muestra el tiempo que se ha tardado en extraer los datos (de los últimos 20 años) de cada revista de la categoría de \textit{Computer Science}.

\imagen{grafica_tiempos}{Tiempo de extracción de datos de Crossref}{1}

Sin embargo, puesto que los datos que se extraen son menos completos y exactos, posteriormente se deberá tratar el margen de error al calcular el Índice de Impacto. Además, es preciso mencionar que, según retrocedemos en el tiempo, la escasez de datos disponibles en Crossref aumenta. Así, resulta evidente que la fiabilidad del cálculo del Índice de Impacto disminuye como consecuencia.

\section{Aprendizaje automático}
La segunda fase del proyecto consiste en predecir y estimar, a partir de los datos obtenidos, el valor del Índice de Impacto de cada revista seleccionada.

\subsection{Cálculo del JCR}
Antes de comenzar a predecir el JCR, se ha desarrollado un algoritmo para calcular el Índice de Impacto de las revistas científicas a partir de los datos extraídos en la fase anterior. Para comprobar la exactitud de los resultados obtenidos, se ha generado un gráfico de cajas en el que se contrastan los valores obtenidos con los datos reales del JCR. 

%%% Creator: Matplotlib, PGF backend
%%
%% To include the figure in your LaTeX document, write
%%   \input{<filename>.pgf}
%%
%% Make sure the required packages are loaded in your preamble
%%   \usepackage{pgf}
%%
%% Also ensure that all the required font packages are loaded; for instance,
%% the lmodern package is sometimes necessary when using math font.
%%   \usepackage{lmodern}
%%
%% Figures using additional raster images can only be included by \input if
%% they are in the same directory as the main LaTeX file. For loading figures
%% from other directories you can use the `import` package
%%   \usepackage{import}
%%
%% and then include the figures with
%%   \import{<path to file>}{<filename>.pgf}
%%
%% Matplotlib used the following preamble
%%   
%%   \usepackage{fontspec}
%%   \setmainfont{DejaVuSerif.ttf}[Path=\detokenize{C:/Users/Gadea/AppData/Local/Programs/Python/Python311/Lib/site-packages/matplotlib/mpl-data/fonts/ttf/}]
%%   \setsansfont{DejaVuSans.ttf}[Path=\detokenize{C:/Users/Gadea/AppData/Local/Programs/Python/Python311/Lib/site-packages/matplotlib/mpl-data/fonts/ttf/}]
%%   \setmonofont{DejaVuSansMono.ttf}[Path=\detokenize{C:/Users/Gadea/AppData/Local/Programs/Python/Python311/Lib/site-packages/matplotlib/mpl-data/fonts/ttf/}]
%%   \makeatletter\@ifpackageloaded{underscore}{}{\usepackage[strings]{underscore}}\makeatother
%%
\begingroup%
\makeatletter%
\begin{pgfpicture}%
\pgfpathrectangle{\pgfpointorigin}{\pgfqpoint{15.360000in}{8.024000in}}%
\pgfusepath{use as bounding box, clip}%
\begin{pgfscope}%
\pgfsetbuttcap%
\pgfsetmiterjoin%
\definecolor{currentfill}{rgb}{1.000000,1.000000,1.000000}%
\pgfsetfillcolor{currentfill}%
\pgfsetlinewidth{0.000000pt}%
\definecolor{currentstroke}{rgb}{1.000000,1.000000,1.000000}%
\pgfsetstrokecolor{currentstroke}%
\pgfsetdash{}{0pt}%
\pgfpathmoveto{\pgfqpoint{0.000000in}{0.000000in}}%
\pgfpathlineto{\pgfqpoint{15.360000in}{0.000000in}}%
\pgfpathlineto{\pgfqpoint{15.360000in}{8.024000in}}%
\pgfpathlineto{\pgfqpoint{0.000000in}{8.024000in}}%
\pgfpathlineto{\pgfqpoint{0.000000in}{0.000000in}}%
\pgfpathclose%
\pgfusepath{fill}%
\end{pgfscope}%
\begin{pgfscope}%
\pgfsetbuttcap%
\pgfsetmiterjoin%
\definecolor{currentfill}{rgb}{1.000000,1.000000,1.000000}%
\pgfsetfillcolor{currentfill}%
\pgfsetlinewidth{0.000000pt}%
\definecolor{currentstroke}{rgb}{0.000000,0.000000,0.000000}%
\pgfsetstrokecolor{currentstroke}%
\pgfsetstrokeopacity{0.000000}%
\pgfsetdash{}{0pt}%
\pgfpathmoveto{\pgfqpoint{1.920000in}{0.882640in}}%
\pgfpathlineto{\pgfqpoint{13.824000in}{0.882640in}}%
\pgfpathlineto{\pgfqpoint{13.824000in}{7.061120in}}%
\pgfpathlineto{\pgfqpoint{1.920000in}{7.061120in}}%
\pgfpathlineto{\pgfqpoint{1.920000in}{0.882640in}}%
\pgfpathclose%
\pgfusepath{fill}%
\end{pgfscope}%
\begin{pgfscope}%
\pgfsetbuttcap%
\pgfsetroundjoin%
\definecolor{currentfill}{rgb}{0.000000,0.000000,0.000000}%
\pgfsetfillcolor{currentfill}%
\pgfsetlinewidth{0.803000pt}%
\definecolor{currentstroke}{rgb}{0.000000,0.000000,0.000000}%
\pgfsetstrokecolor{currentstroke}%
\pgfsetdash{}{0pt}%
\pgfsys@defobject{currentmarker}{\pgfqpoint{0.000000in}{-0.048611in}}{\pgfqpoint{0.000000in}{0.000000in}}{%
\pgfpathmoveto{\pgfqpoint{0.000000in}{0.000000in}}%
\pgfpathlineto{\pgfqpoint{0.000000in}{-0.048611in}}%
\pgfusepath{stroke,fill}%
}%
\begin{pgfscope}%
\pgfsys@transformshift{2.256753in}{0.882640in}%
\pgfsys@useobject{currentmarker}{}%
\end{pgfscope}%
\end{pgfscope}%
\begin{pgfscope}%
\definecolor{textcolor}{rgb}{0.000000,0.000000,0.000000}%
\pgfsetstrokecolor{textcolor}%
\pgfsetfillcolor{textcolor}%
\pgftext[x=2.295069in, y=0.146583in, left, base,rotate=90.000000]{\color{textcolor}\sffamily\fontsize{10.000000}{12.000000}\selectfont Diff 2007}%
\end{pgfscope}%
\begin{pgfscope}%
\pgfsetbuttcap%
\pgfsetroundjoin%
\definecolor{currentfill}{rgb}{0.000000,0.000000,0.000000}%
\pgfsetfillcolor{currentfill}%
\pgfsetlinewidth{0.803000pt}%
\definecolor{currentstroke}{rgb}{0.000000,0.000000,0.000000}%
\pgfsetstrokecolor{currentstroke}%
\pgfsetdash{}{0pt}%
\pgfsys@defobject{currentmarker}{\pgfqpoint{0.000000in}{-0.048611in}}{\pgfqpoint{0.000000in}{0.000000in}}{%
\pgfpathmoveto{\pgfqpoint{0.000000in}{0.000000in}}%
\pgfpathlineto{\pgfqpoint{0.000000in}{-0.048611in}}%
\pgfusepath{stroke,fill}%
}%
\begin{pgfscope}%
\pgfsys@transformshift{3.011403in}{0.882640in}%
\pgfsys@useobject{currentmarker}{}%
\end{pgfscope}%
\end{pgfscope}%
\begin{pgfscope}%
\definecolor{textcolor}{rgb}{0.000000,0.000000,0.000000}%
\pgfsetstrokecolor{textcolor}%
\pgfsetfillcolor{textcolor}%
\pgftext[x=3.049720in, y=0.146583in, left, base,rotate=90.000000]{\color{textcolor}\sffamily\fontsize{10.000000}{12.000000}\selectfont Diff 2008}%
\end{pgfscope}%
\begin{pgfscope}%
\pgfsetbuttcap%
\pgfsetroundjoin%
\definecolor{currentfill}{rgb}{0.000000,0.000000,0.000000}%
\pgfsetfillcolor{currentfill}%
\pgfsetlinewidth{0.803000pt}%
\definecolor{currentstroke}{rgb}{0.000000,0.000000,0.000000}%
\pgfsetstrokecolor{currentstroke}%
\pgfsetdash{}{0pt}%
\pgfsys@defobject{currentmarker}{\pgfqpoint{0.000000in}{-0.048611in}}{\pgfqpoint{0.000000in}{0.000000in}}{%
\pgfpathmoveto{\pgfqpoint{0.000000in}{0.000000in}}%
\pgfpathlineto{\pgfqpoint{0.000000in}{-0.048611in}}%
\pgfusepath{stroke,fill}%
}%
\begin{pgfscope}%
\pgfsys@transformshift{3.766053in}{0.882640in}%
\pgfsys@useobject{currentmarker}{}%
\end{pgfscope}%
\end{pgfscope}%
\begin{pgfscope}%
\definecolor{textcolor}{rgb}{0.000000,0.000000,0.000000}%
\pgfsetstrokecolor{textcolor}%
\pgfsetfillcolor{textcolor}%
\pgftext[x=3.804370in, y=0.146583in, left, base,rotate=90.000000]{\color{textcolor}\sffamily\fontsize{10.000000}{12.000000}\selectfont Diff 2009}%
\end{pgfscope}%
\begin{pgfscope}%
\pgfsetbuttcap%
\pgfsetroundjoin%
\definecolor{currentfill}{rgb}{0.000000,0.000000,0.000000}%
\pgfsetfillcolor{currentfill}%
\pgfsetlinewidth{0.803000pt}%
\definecolor{currentstroke}{rgb}{0.000000,0.000000,0.000000}%
\pgfsetstrokecolor{currentstroke}%
\pgfsetdash{}{0pt}%
\pgfsys@defobject{currentmarker}{\pgfqpoint{0.000000in}{-0.048611in}}{\pgfqpoint{0.000000in}{0.000000in}}{%
\pgfpathmoveto{\pgfqpoint{0.000000in}{0.000000in}}%
\pgfpathlineto{\pgfqpoint{0.000000in}{-0.048611in}}%
\pgfusepath{stroke,fill}%
}%
\begin{pgfscope}%
\pgfsys@transformshift{4.520703in}{0.882640in}%
\pgfsys@useobject{currentmarker}{}%
\end{pgfscope}%
\end{pgfscope}%
\begin{pgfscope}%
\definecolor{textcolor}{rgb}{0.000000,0.000000,0.000000}%
\pgfsetstrokecolor{textcolor}%
\pgfsetfillcolor{textcolor}%
\pgftext[x=4.559020in, y=0.146583in, left, base,rotate=90.000000]{\color{textcolor}\sffamily\fontsize{10.000000}{12.000000}\selectfont Diff 2010}%
\end{pgfscope}%
\begin{pgfscope}%
\pgfsetbuttcap%
\pgfsetroundjoin%
\definecolor{currentfill}{rgb}{0.000000,0.000000,0.000000}%
\pgfsetfillcolor{currentfill}%
\pgfsetlinewidth{0.803000pt}%
\definecolor{currentstroke}{rgb}{0.000000,0.000000,0.000000}%
\pgfsetstrokecolor{currentstroke}%
\pgfsetdash{}{0pt}%
\pgfsys@defobject{currentmarker}{\pgfqpoint{0.000000in}{-0.048611in}}{\pgfqpoint{0.000000in}{0.000000in}}{%
\pgfpathmoveto{\pgfqpoint{0.000000in}{0.000000in}}%
\pgfpathlineto{\pgfqpoint{0.000000in}{-0.048611in}}%
\pgfusepath{stroke,fill}%
}%
\begin{pgfscope}%
\pgfsys@transformshift{5.275354in}{0.882640in}%
\pgfsys@useobject{currentmarker}{}%
\end{pgfscope}%
\end{pgfscope}%
\begin{pgfscope}%
\definecolor{textcolor}{rgb}{0.000000,0.000000,0.000000}%
\pgfsetstrokecolor{textcolor}%
\pgfsetfillcolor{textcolor}%
\pgftext[x=5.313670in, y=0.146583in, left, base,rotate=90.000000]{\color{textcolor}\sffamily\fontsize{10.000000}{12.000000}\selectfont Diff 2011}%
\end{pgfscope}%
\begin{pgfscope}%
\pgfsetbuttcap%
\pgfsetroundjoin%
\definecolor{currentfill}{rgb}{0.000000,0.000000,0.000000}%
\pgfsetfillcolor{currentfill}%
\pgfsetlinewidth{0.803000pt}%
\definecolor{currentstroke}{rgb}{0.000000,0.000000,0.000000}%
\pgfsetstrokecolor{currentstroke}%
\pgfsetdash{}{0pt}%
\pgfsys@defobject{currentmarker}{\pgfqpoint{0.000000in}{-0.048611in}}{\pgfqpoint{0.000000in}{0.000000in}}{%
\pgfpathmoveto{\pgfqpoint{0.000000in}{0.000000in}}%
\pgfpathlineto{\pgfqpoint{0.000000in}{-0.048611in}}%
\pgfusepath{stroke,fill}%
}%
\begin{pgfscope}%
\pgfsys@transformshift{6.030004in}{0.882640in}%
\pgfsys@useobject{currentmarker}{}%
\end{pgfscope}%
\end{pgfscope}%
\begin{pgfscope}%
\definecolor{textcolor}{rgb}{0.000000,0.000000,0.000000}%
\pgfsetstrokecolor{textcolor}%
\pgfsetfillcolor{textcolor}%
\pgftext[x=6.068321in, y=0.146583in, left, base,rotate=90.000000]{\color{textcolor}\sffamily\fontsize{10.000000}{12.000000}\selectfont Diff 2012}%
\end{pgfscope}%
\begin{pgfscope}%
\pgfsetbuttcap%
\pgfsetroundjoin%
\definecolor{currentfill}{rgb}{0.000000,0.000000,0.000000}%
\pgfsetfillcolor{currentfill}%
\pgfsetlinewidth{0.803000pt}%
\definecolor{currentstroke}{rgb}{0.000000,0.000000,0.000000}%
\pgfsetstrokecolor{currentstroke}%
\pgfsetdash{}{0pt}%
\pgfsys@defobject{currentmarker}{\pgfqpoint{0.000000in}{-0.048611in}}{\pgfqpoint{0.000000in}{0.000000in}}{%
\pgfpathmoveto{\pgfqpoint{0.000000in}{0.000000in}}%
\pgfpathlineto{\pgfqpoint{0.000000in}{-0.048611in}}%
\pgfusepath{stroke,fill}%
}%
\begin{pgfscope}%
\pgfsys@transformshift{6.784654in}{0.882640in}%
\pgfsys@useobject{currentmarker}{}%
\end{pgfscope}%
\end{pgfscope}%
\begin{pgfscope}%
\definecolor{textcolor}{rgb}{0.000000,0.000000,0.000000}%
\pgfsetstrokecolor{textcolor}%
\pgfsetfillcolor{textcolor}%
\pgftext[x=6.822971in, y=0.146583in, left, base,rotate=90.000000]{\color{textcolor}\sffamily\fontsize{10.000000}{12.000000}\selectfont Diff 2013}%
\end{pgfscope}%
\begin{pgfscope}%
\pgfsetbuttcap%
\pgfsetroundjoin%
\definecolor{currentfill}{rgb}{0.000000,0.000000,0.000000}%
\pgfsetfillcolor{currentfill}%
\pgfsetlinewidth{0.803000pt}%
\definecolor{currentstroke}{rgb}{0.000000,0.000000,0.000000}%
\pgfsetstrokecolor{currentstroke}%
\pgfsetdash{}{0pt}%
\pgfsys@defobject{currentmarker}{\pgfqpoint{0.000000in}{-0.048611in}}{\pgfqpoint{0.000000in}{0.000000in}}{%
\pgfpathmoveto{\pgfqpoint{0.000000in}{0.000000in}}%
\pgfpathlineto{\pgfqpoint{0.000000in}{-0.048611in}}%
\pgfusepath{stroke,fill}%
}%
\begin{pgfscope}%
\pgfsys@transformshift{7.539305in}{0.882640in}%
\pgfsys@useobject{currentmarker}{}%
\end{pgfscope}%
\end{pgfscope}%
\begin{pgfscope}%
\definecolor{textcolor}{rgb}{0.000000,0.000000,0.000000}%
\pgfsetstrokecolor{textcolor}%
\pgfsetfillcolor{textcolor}%
\pgftext[x=7.577621in, y=0.146583in, left, base,rotate=90.000000]{\color{textcolor}\sffamily\fontsize{10.000000}{12.000000}\selectfont Diff 2014}%
\end{pgfscope}%
\begin{pgfscope}%
\pgfsetbuttcap%
\pgfsetroundjoin%
\definecolor{currentfill}{rgb}{0.000000,0.000000,0.000000}%
\pgfsetfillcolor{currentfill}%
\pgfsetlinewidth{0.803000pt}%
\definecolor{currentstroke}{rgb}{0.000000,0.000000,0.000000}%
\pgfsetstrokecolor{currentstroke}%
\pgfsetdash{}{0pt}%
\pgfsys@defobject{currentmarker}{\pgfqpoint{0.000000in}{-0.048611in}}{\pgfqpoint{0.000000in}{0.000000in}}{%
\pgfpathmoveto{\pgfqpoint{0.000000in}{0.000000in}}%
\pgfpathlineto{\pgfqpoint{0.000000in}{-0.048611in}}%
\pgfusepath{stroke,fill}%
}%
\begin{pgfscope}%
\pgfsys@transformshift{8.293955in}{0.882640in}%
\pgfsys@useobject{currentmarker}{}%
\end{pgfscope}%
\end{pgfscope}%
\begin{pgfscope}%
\definecolor{textcolor}{rgb}{0.000000,0.000000,0.000000}%
\pgfsetstrokecolor{textcolor}%
\pgfsetfillcolor{textcolor}%
\pgftext[x=8.332272in, y=0.146583in, left, base,rotate=90.000000]{\color{textcolor}\sffamily\fontsize{10.000000}{12.000000}\selectfont Diff 2015}%
\end{pgfscope}%
\begin{pgfscope}%
\pgfsetbuttcap%
\pgfsetroundjoin%
\definecolor{currentfill}{rgb}{0.000000,0.000000,0.000000}%
\pgfsetfillcolor{currentfill}%
\pgfsetlinewidth{0.803000pt}%
\definecolor{currentstroke}{rgb}{0.000000,0.000000,0.000000}%
\pgfsetstrokecolor{currentstroke}%
\pgfsetdash{}{0pt}%
\pgfsys@defobject{currentmarker}{\pgfqpoint{0.000000in}{-0.048611in}}{\pgfqpoint{0.000000in}{0.000000in}}{%
\pgfpathmoveto{\pgfqpoint{0.000000in}{0.000000in}}%
\pgfpathlineto{\pgfqpoint{0.000000in}{-0.048611in}}%
\pgfusepath{stroke,fill}%
}%
\begin{pgfscope}%
\pgfsys@transformshift{9.048605in}{0.882640in}%
\pgfsys@useobject{currentmarker}{}%
\end{pgfscope}%
\end{pgfscope}%
\begin{pgfscope}%
\definecolor{textcolor}{rgb}{0.000000,0.000000,0.000000}%
\pgfsetstrokecolor{textcolor}%
\pgfsetfillcolor{textcolor}%
\pgftext[x=9.086922in, y=0.146583in, left, base,rotate=90.000000]{\color{textcolor}\sffamily\fontsize{10.000000}{12.000000}\selectfont Diff 2016}%
\end{pgfscope}%
\begin{pgfscope}%
\pgfsetbuttcap%
\pgfsetroundjoin%
\definecolor{currentfill}{rgb}{0.000000,0.000000,0.000000}%
\pgfsetfillcolor{currentfill}%
\pgfsetlinewidth{0.803000pt}%
\definecolor{currentstroke}{rgb}{0.000000,0.000000,0.000000}%
\pgfsetstrokecolor{currentstroke}%
\pgfsetdash{}{0pt}%
\pgfsys@defobject{currentmarker}{\pgfqpoint{0.000000in}{-0.048611in}}{\pgfqpoint{0.000000in}{0.000000in}}{%
\pgfpathmoveto{\pgfqpoint{0.000000in}{0.000000in}}%
\pgfpathlineto{\pgfqpoint{0.000000in}{-0.048611in}}%
\pgfusepath{stroke,fill}%
}%
\begin{pgfscope}%
\pgfsys@transformshift{9.803256in}{0.882640in}%
\pgfsys@useobject{currentmarker}{}%
\end{pgfscope}%
\end{pgfscope}%
\begin{pgfscope}%
\definecolor{textcolor}{rgb}{0.000000,0.000000,0.000000}%
\pgfsetstrokecolor{textcolor}%
\pgfsetfillcolor{textcolor}%
\pgftext[x=9.841572in, y=0.146583in, left, base,rotate=90.000000]{\color{textcolor}\sffamily\fontsize{10.000000}{12.000000}\selectfont Diff 2017}%
\end{pgfscope}%
\begin{pgfscope}%
\pgfsetbuttcap%
\pgfsetroundjoin%
\definecolor{currentfill}{rgb}{0.000000,0.000000,0.000000}%
\pgfsetfillcolor{currentfill}%
\pgfsetlinewidth{0.803000pt}%
\definecolor{currentstroke}{rgb}{0.000000,0.000000,0.000000}%
\pgfsetstrokecolor{currentstroke}%
\pgfsetdash{}{0pt}%
\pgfsys@defobject{currentmarker}{\pgfqpoint{0.000000in}{-0.048611in}}{\pgfqpoint{0.000000in}{0.000000in}}{%
\pgfpathmoveto{\pgfqpoint{0.000000in}{0.000000in}}%
\pgfpathlineto{\pgfqpoint{0.000000in}{-0.048611in}}%
\pgfusepath{stroke,fill}%
}%
\begin{pgfscope}%
\pgfsys@transformshift{10.557906in}{0.882640in}%
\pgfsys@useobject{currentmarker}{}%
\end{pgfscope}%
\end{pgfscope}%
\begin{pgfscope}%
\definecolor{textcolor}{rgb}{0.000000,0.000000,0.000000}%
\pgfsetstrokecolor{textcolor}%
\pgfsetfillcolor{textcolor}%
\pgftext[x=10.596223in, y=0.146583in, left, base,rotate=90.000000]{\color{textcolor}\sffamily\fontsize{10.000000}{12.000000}\selectfont Diff 2018}%
\end{pgfscope}%
\begin{pgfscope}%
\pgfsetbuttcap%
\pgfsetroundjoin%
\definecolor{currentfill}{rgb}{0.000000,0.000000,0.000000}%
\pgfsetfillcolor{currentfill}%
\pgfsetlinewidth{0.803000pt}%
\definecolor{currentstroke}{rgb}{0.000000,0.000000,0.000000}%
\pgfsetstrokecolor{currentstroke}%
\pgfsetdash{}{0pt}%
\pgfsys@defobject{currentmarker}{\pgfqpoint{0.000000in}{-0.048611in}}{\pgfqpoint{0.000000in}{0.000000in}}{%
\pgfpathmoveto{\pgfqpoint{0.000000in}{0.000000in}}%
\pgfpathlineto{\pgfqpoint{0.000000in}{-0.048611in}}%
\pgfusepath{stroke,fill}%
}%
\begin{pgfscope}%
\pgfsys@transformshift{11.312556in}{0.882640in}%
\pgfsys@useobject{currentmarker}{}%
\end{pgfscope}%
\end{pgfscope}%
\begin{pgfscope}%
\definecolor{textcolor}{rgb}{0.000000,0.000000,0.000000}%
\pgfsetstrokecolor{textcolor}%
\pgfsetfillcolor{textcolor}%
\pgftext[x=11.350873in, y=0.146583in, left, base,rotate=90.000000]{\color{textcolor}\sffamily\fontsize{10.000000}{12.000000}\selectfont Diff 2019}%
\end{pgfscope}%
\begin{pgfscope}%
\pgfsetbuttcap%
\pgfsetroundjoin%
\definecolor{currentfill}{rgb}{0.000000,0.000000,0.000000}%
\pgfsetfillcolor{currentfill}%
\pgfsetlinewidth{0.803000pt}%
\definecolor{currentstroke}{rgb}{0.000000,0.000000,0.000000}%
\pgfsetstrokecolor{currentstroke}%
\pgfsetdash{}{0pt}%
\pgfsys@defobject{currentmarker}{\pgfqpoint{0.000000in}{-0.048611in}}{\pgfqpoint{0.000000in}{0.000000in}}{%
\pgfpathmoveto{\pgfqpoint{0.000000in}{0.000000in}}%
\pgfpathlineto{\pgfqpoint{0.000000in}{-0.048611in}}%
\pgfusepath{stroke,fill}%
}%
\begin{pgfscope}%
\pgfsys@transformshift{12.067207in}{0.882640in}%
\pgfsys@useobject{currentmarker}{}%
\end{pgfscope}%
\end{pgfscope}%
\begin{pgfscope}%
\definecolor{textcolor}{rgb}{0.000000,0.000000,0.000000}%
\pgfsetstrokecolor{textcolor}%
\pgfsetfillcolor{textcolor}%
\pgftext[x=12.105523in, y=0.146583in, left, base,rotate=90.000000]{\color{textcolor}\sffamily\fontsize{10.000000}{12.000000}\selectfont Diff 2020}%
\end{pgfscope}%
\begin{pgfscope}%
\pgfsetbuttcap%
\pgfsetroundjoin%
\definecolor{currentfill}{rgb}{0.000000,0.000000,0.000000}%
\pgfsetfillcolor{currentfill}%
\pgfsetlinewidth{0.803000pt}%
\definecolor{currentstroke}{rgb}{0.000000,0.000000,0.000000}%
\pgfsetstrokecolor{currentstroke}%
\pgfsetdash{}{0pt}%
\pgfsys@defobject{currentmarker}{\pgfqpoint{0.000000in}{-0.048611in}}{\pgfqpoint{0.000000in}{0.000000in}}{%
\pgfpathmoveto{\pgfqpoint{0.000000in}{0.000000in}}%
\pgfpathlineto{\pgfqpoint{0.000000in}{-0.048611in}}%
\pgfusepath{stroke,fill}%
}%
\begin{pgfscope}%
\pgfsys@transformshift{12.821857in}{0.882640in}%
\pgfsys@useobject{currentmarker}{}%
\end{pgfscope}%
\end{pgfscope}%
\begin{pgfscope}%
\definecolor{textcolor}{rgb}{0.000000,0.000000,0.000000}%
\pgfsetstrokecolor{textcolor}%
\pgfsetfillcolor{textcolor}%
\pgftext[x=12.860174in, y=0.146583in, left, base,rotate=90.000000]{\color{textcolor}\sffamily\fontsize{10.000000}{12.000000}\selectfont Diff 2021}%
\end{pgfscope}%
\begin{pgfscope}%
\pgfsetbuttcap%
\pgfsetroundjoin%
\definecolor{currentfill}{rgb}{0.000000,0.000000,0.000000}%
\pgfsetfillcolor{currentfill}%
\pgfsetlinewidth{0.803000pt}%
\definecolor{currentstroke}{rgb}{0.000000,0.000000,0.000000}%
\pgfsetstrokecolor{currentstroke}%
\pgfsetdash{}{0pt}%
\pgfsys@defobject{currentmarker}{\pgfqpoint{0.000000in}{-0.048611in}}{\pgfqpoint{0.000000in}{0.000000in}}{%
\pgfpathmoveto{\pgfqpoint{0.000000in}{0.000000in}}%
\pgfpathlineto{\pgfqpoint{0.000000in}{-0.048611in}}%
\pgfusepath{stroke,fill}%
}%
\begin{pgfscope}%
\pgfsys@transformshift{13.576507in}{0.882640in}%
\pgfsys@useobject{currentmarker}{}%
\end{pgfscope}%
\end{pgfscope}%
\begin{pgfscope}%
\definecolor{textcolor}{rgb}{0.000000,0.000000,0.000000}%
\pgfsetstrokecolor{textcolor}%
\pgfsetfillcolor{textcolor}%
\pgftext[x=13.614824in, y=0.146583in, left, base,rotate=90.000000]{\color{textcolor}\sffamily\fontsize{10.000000}{12.000000}\selectfont Diff 2022}%
\end{pgfscope}%
\begin{pgfscope}%
\definecolor{textcolor}{rgb}{0.000000,0.000000,0.000000}%
\pgfsetstrokecolor{textcolor}%
\pgfsetfillcolor{textcolor}%
\pgftext[x=7.872000in,y=0.091027in,,top]{\color{textcolor}\sffamily\fontsize{10.000000}{12.000000}\selectfont Año}%
\end{pgfscope}%
\begin{pgfscope}%
\pgfsetbuttcap%
\pgfsetroundjoin%
\definecolor{currentfill}{rgb}{0.000000,0.000000,0.000000}%
\pgfsetfillcolor{currentfill}%
\pgfsetlinewidth{0.803000pt}%
\definecolor{currentstroke}{rgb}{0.000000,0.000000,0.000000}%
\pgfsetstrokecolor{currentstroke}%
\pgfsetdash{}{0pt}%
\pgfsys@defobject{currentmarker}{\pgfqpoint{-0.048611in}{0.000000in}}{\pgfqpoint{-0.000000in}{0.000000in}}{%
\pgfpathmoveto{\pgfqpoint{-0.000000in}{0.000000in}}%
\pgfpathlineto{\pgfqpoint{-0.048611in}{0.000000in}}%
\pgfusepath{stroke,fill}%
}%
\begin{pgfscope}%
\pgfsys@transformshift{1.920000in}{1.018757in}%
\pgfsys@useobject{currentmarker}{}%
\end{pgfscope}%
\end{pgfscope}%
\begin{pgfscope}%
\definecolor{textcolor}{rgb}{0.000000,0.000000,0.000000}%
\pgfsetstrokecolor{textcolor}%
\pgfsetfillcolor{textcolor}%
\pgftext[x=1.734412in, y=0.965996in, left, base]{\color{textcolor}\sffamily\fontsize{10.000000}{12.000000}\selectfont 0}%
\end{pgfscope}%
\begin{pgfscope}%
\pgfsetbuttcap%
\pgfsetroundjoin%
\definecolor{currentfill}{rgb}{0.000000,0.000000,0.000000}%
\pgfsetfillcolor{currentfill}%
\pgfsetlinewidth{0.803000pt}%
\definecolor{currentstroke}{rgb}{0.000000,0.000000,0.000000}%
\pgfsetstrokecolor{currentstroke}%
\pgfsetdash{}{0pt}%
\pgfsys@defobject{currentmarker}{\pgfqpoint{-0.048611in}{0.000000in}}{\pgfqpoint{-0.000000in}{0.000000in}}{%
\pgfpathmoveto{\pgfqpoint{-0.000000in}{0.000000in}}%
\pgfpathlineto{\pgfqpoint{-0.048611in}{0.000000in}}%
\pgfusepath{stroke,fill}%
}%
\begin{pgfscope}%
\pgfsys@transformshift{1.920000in}{2.304316in}%
\pgfsys@useobject{currentmarker}{}%
\end{pgfscope}%
\end{pgfscope}%
\begin{pgfscope}%
\definecolor{textcolor}{rgb}{0.000000,0.000000,0.000000}%
\pgfsetstrokecolor{textcolor}%
\pgfsetfillcolor{textcolor}%
\pgftext[x=1.646047in, y=2.251555in, left, base]{\color{textcolor}\sffamily\fontsize{10.000000}{12.000000}\selectfont 50}%
\end{pgfscope}%
\begin{pgfscope}%
\pgfsetbuttcap%
\pgfsetroundjoin%
\definecolor{currentfill}{rgb}{0.000000,0.000000,0.000000}%
\pgfsetfillcolor{currentfill}%
\pgfsetlinewidth{0.803000pt}%
\definecolor{currentstroke}{rgb}{0.000000,0.000000,0.000000}%
\pgfsetstrokecolor{currentstroke}%
\pgfsetdash{}{0pt}%
\pgfsys@defobject{currentmarker}{\pgfqpoint{-0.048611in}{0.000000in}}{\pgfqpoint{-0.000000in}{0.000000in}}{%
\pgfpathmoveto{\pgfqpoint{-0.000000in}{0.000000in}}%
\pgfpathlineto{\pgfqpoint{-0.048611in}{0.000000in}}%
\pgfusepath{stroke,fill}%
}%
\begin{pgfscope}%
\pgfsys@transformshift{1.920000in}{3.589876in}%
\pgfsys@useobject{currentmarker}{}%
\end{pgfscope}%
\end{pgfscope}%
\begin{pgfscope}%
\definecolor{textcolor}{rgb}{0.000000,0.000000,0.000000}%
\pgfsetstrokecolor{textcolor}%
\pgfsetfillcolor{textcolor}%
\pgftext[x=1.557682in, y=3.537114in, left, base]{\color{textcolor}\sffamily\fontsize{10.000000}{12.000000}\selectfont 100}%
\end{pgfscope}%
\begin{pgfscope}%
\pgfsetbuttcap%
\pgfsetroundjoin%
\definecolor{currentfill}{rgb}{0.000000,0.000000,0.000000}%
\pgfsetfillcolor{currentfill}%
\pgfsetlinewidth{0.803000pt}%
\definecolor{currentstroke}{rgb}{0.000000,0.000000,0.000000}%
\pgfsetstrokecolor{currentstroke}%
\pgfsetdash{}{0pt}%
\pgfsys@defobject{currentmarker}{\pgfqpoint{-0.048611in}{0.000000in}}{\pgfqpoint{-0.000000in}{0.000000in}}{%
\pgfpathmoveto{\pgfqpoint{-0.000000in}{0.000000in}}%
\pgfpathlineto{\pgfqpoint{-0.048611in}{0.000000in}}%
\pgfusepath{stroke,fill}%
}%
\begin{pgfscope}%
\pgfsys@transformshift{1.920000in}{4.875435in}%
\pgfsys@useobject{currentmarker}{}%
\end{pgfscope}%
\end{pgfscope}%
\begin{pgfscope}%
\definecolor{textcolor}{rgb}{0.000000,0.000000,0.000000}%
\pgfsetstrokecolor{textcolor}%
\pgfsetfillcolor{textcolor}%
\pgftext[x=1.557682in, y=4.822674in, left, base]{\color{textcolor}\sffamily\fontsize{10.000000}{12.000000}\selectfont 150}%
\end{pgfscope}%
\begin{pgfscope}%
\pgfsetbuttcap%
\pgfsetroundjoin%
\definecolor{currentfill}{rgb}{0.000000,0.000000,0.000000}%
\pgfsetfillcolor{currentfill}%
\pgfsetlinewidth{0.803000pt}%
\definecolor{currentstroke}{rgb}{0.000000,0.000000,0.000000}%
\pgfsetstrokecolor{currentstroke}%
\pgfsetdash{}{0pt}%
\pgfsys@defobject{currentmarker}{\pgfqpoint{-0.048611in}{0.000000in}}{\pgfqpoint{-0.000000in}{0.000000in}}{%
\pgfpathmoveto{\pgfqpoint{-0.000000in}{0.000000in}}%
\pgfpathlineto{\pgfqpoint{-0.048611in}{0.000000in}}%
\pgfusepath{stroke,fill}%
}%
\begin{pgfscope}%
\pgfsys@transformshift{1.920000in}{6.160995in}%
\pgfsys@useobject{currentmarker}{}%
\end{pgfscope}%
\end{pgfscope}%
\begin{pgfscope}%
\definecolor{textcolor}{rgb}{0.000000,0.000000,0.000000}%
\pgfsetstrokecolor{textcolor}%
\pgfsetfillcolor{textcolor}%
\pgftext[x=1.557682in, y=6.108233in, left, base]{\color{textcolor}\sffamily\fontsize{10.000000}{12.000000}\selectfont 200}%
\end{pgfscope}%
\begin{pgfscope}%
\definecolor{textcolor}{rgb}{0.000000,0.000000,0.000000}%
\pgfsetstrokecolor{textcolor}%
\pgfsetfillcolor{textcolor}%
\pgftext[x=1.502126in,y=3.971880in,,bottom,rotate=90.000000]{\color{textcolor}\sffamily\fontsize{10.000000}{12.000000}\selectfont Valor de diferencias}%
\end{pgfscope}%
\begin{pgfscope}%
\pgfpathrectangle{\pgfqpoint{1.920000in}{0.882640in}}{\pgfqpoint{11.904000in}{6.178480in}}%
\pgfusepath{clip}%
\pgfsetrectcap%
\pgfsetroundjoin%
\pgfsetlinewidth{1.003750pt}%
\definecolor{currentstroke}{rgb}{0.000000,0.000000,0.000000}%
\pgfsetstrokecolor{currentstroke}%
\pgfsetdash{}{0pt}%
\pgfpathmoveto{\pgfqpoint{2.068090in}{1.018757in}}%
\pgfpathlineto{\pgfqpoint{2.445415in}{1.018757in}}%
\pgfpathlineto{\pgfqpoint{2.445415in}{1.803334in}}%
\pgfpathlineto{\pgfqpoint{2.068090in}{1.803334in}}%
\pgfpathlineto{\pgfqpoint{2.068090in}{1.018757in}}%
\pgfusepath{stroke}%
\end{pgfscope}%
\begin{pgfscope}%
\pgfpathrectangle{\pgfqpoint{1.920000in}{0.882640in}}{\pgfqpoint{11.904000in}{6.178480in}}%
\pgfusepath{clip}%
\pgfsetrectcap%
\pgfsetroundjoin%
\pgfsetlinewidth{1.003750pt}%
\definecolor{currentstroke}{rgb}{0.000000,0.000000,0.000000}%
\pgfsetstrokecolor{currentstroke}%
\pgfsetdash{}{0pt}%
\pgfpathmoveto{\pgfqpoint{2.256753in}{1.018757in}}%
\pgfpathlineto{\pgfqpoint{2.256753in}{1.018757in}}%
\pgfusepath{stroke}%
\end{pgfscope}%
\begin{pgfscope}%
\pgfpathrectangle{\pgfqpoint{1.920000in}{0.882640in}}{\pgfqpoint{11.904000in}{6.178480in}}%
\pgfusepath{clip}%
\pgfsetrectcap%
\pgfsetroundjoin%
\pgfsetlinewidth{1.003750pt}%
\definecolor{currentstroke}{rgb}{0.000000,0.000000,0.000000}%
\pgfsetstrokecolor{currentstroke}%
\pgfsetdash{}{0pt}%
\pgfpathmoveto{\pgfqpoint{2.256753in}{1.803334in}}%
\pgfpathlineto{\pgfqpoint{2.256753in}{2.746935in}}%
\pgfusepath{stroke}%
\end{pgfscope}%
\begin{pgfscope}%
\pgfpathrectangle{\pgfqpoint{1.920000in}{0.882640in}}{\pgfqpoint{11.904000in}{6.178480in}}%
\pgfusepath{clip}%
\pgfsetrectcap%
\pgfsetroundjoin%
\pgfsetlinewidth{1.003750pt}%
\definecolor{currentstroke}{rgb}{0.000000,0.000000,0.000000}%
\pgfsetstrokecolor{currentstroke}%
\pgfsetdash{}{0pt}%
\pgfpathmoveto{\pgfqpoint{2.162421in}{1.018757in}}%
\pgfpathlineto{\pgfqpoint{2.351084in}{1.018757in}}%
\pgfusepath{stroke}%
\end{pgfscope}%
\begin{pgfscope}%
\pgfpathrectangle{\pgfqpoint{1.920000in}{0.882640in}}{\pgfqpoint{11.904000in}{6.178480in}}%
\pgfusepath{clip}%
\pgfsetrectcap%
\pgfsetroundjoin%
\pgfsetlinewidth{1.003750pt}%
\definecolor{currentstroke}{rgb}{0.000000,0.000000,0.000000}%
\pgfsetstrokecolor{currentstroke}%
\pgfsetdash{}{0pt}%
\pgfpathmoveto{\pgfqpoint{2.162421in}{2.746935in}}%
\pgfpathlineto{\pgfqpoint{2.351084in}{2.746935in}}%
\pgfusepath{stroke}%
\end{pgfscope}%
\begin{pgfscope}%
\pgfpathrectangle{\pgfqpoint{1.920000in}{0.882640in}}{\pgfqpoint{11.904000in}{6.178480in}}%
\pgfusepath{clip}%
\pgfsetbuttcap%
\pgfsetroundjoin%
\definecolor{currentfill}{rgb}{0.000000,0.000000,0.000000}%
\pgfsetfillcolor{currentfill}%
\pgfsetfillopacity{0.000000}%
\pgfsetlinewidth{1.003750pt}%
\definecolor{currentstroke}{rgb}{0.000000,0.000000,0.000000}%
\pgfsetstrokecolor{currentstroke}%
\pgfsetdash{}{0pt}%
\pgfsys@defobject{currentmarker}{\pgfqpoint{-0.041667in}{-0.041667in}}{\pgfqpoint{0.041667in}{0.041667in}}{%
\pgfpathmoveto{\pgfqpoint{0.000000in}{-0.041667in}}%
\pgfpathcurveto{\pgfqpoint{0.011050in}{-0.041667in}}{\pgfqpoint{0.021649in}{-0.037276in}}{\pgfqpoint{0.029463in}{-0.029463in}}%
\pgfpathcurveto{\pgfqpoint{0.037276in}{-0.021649in}}{\pgfqpoint{0.041667in}{-0.011050in}}{\pgfqpoint{0.041667in}{0.000000in}}%
\pgfpathcurveto{\pgfqpoint{0.041667in}{0.011050in}}{\pgfqpoint{0.037276in}{0.021649in}}{\pgfqpoint{0.029463in}{0.029463in}}%
\pgfpathcurveto{\pgfqpoint{0.021649in}{0.037276in}}{\pgfqpoint{0.011050in}{0.041667in}}{\pgfqpoint{0.000000in}{0.041667in}}%
\pgfpathcurveto{\pgfqpoint{-0.011050in}{0.041667in}}{\pgfqpoint{-0.021649in}{0.037276in}}{\pgfqpoint{-0.029463in}{0.029463in}}%
\pgfpathcurveto{\pgfqpoint{-0.037276in}{0.021649in}}{\pgfqpoint{-0.041667in}{0.011050in}}{\pgfqpoint{-0.041667in}{0.000000in}}%
\pgfpathcurveto{\pgfqpoint{-0.041667in}{-0.011050in}}{\pgfqpoint{-0.037276in}{-0.021649in}}{\pgfqpoint{-0.029463in}{-0.029463in}}%
\pgfpathcurveto{\pgfqpoint{-0.021649in}{-0.037276in}}{\pgfqpoint{-0.011050in}{-0.041667in}}{\pgfqpoint{0.000000in}{-0.041667in}}%
\pgfpathlineto{\pgfqpoint{0.000000in}{-0.041667in}}%
\pgfpathclose%
\pgfusepath{stroke,fill}%
}%
\begin{pgfscope}%
\pgfsys@transformshift{2.256753in}{4.307038in}%
\pgfsys@useobject{currentmarker}{}%
\end{pgfscope}%
\begin{pgfscope}%
\pgfsys@transformshift{2.256753in}{3.852824in}%
\pgfsys@useobject{currentmarker}{}%
\end{pgfscope}%
\begin{pgfscope}%
\pgfsys@transformshift{2.256753in}{3.196572in}%
\pgfsys@useobject{currentmarker}{}%
\end{pgfscope}%
\begin{pgfscope}%
\pgfsys@transformshift{2.256753in}{4.497378in}%
\pgfsys@useobject{currentmarker}{}%
\end{pgfscope}%
\begin{pgfscope}%
\pgfsys@transformshift{2.256753in}{3.401156in}%
\pgfsys@useobject{currentmarker}{}%
\end{pgfscope}%
\end{pgfscope}%
\begin{pgfscope}%
\pgfpathrectangle{\pgfqpoint{1.920000in}{0.882640in}}{\pgfqpoint{11.904000in}{6.178480in}}%
\pgfusepath{clip}%
\pgfsetrectcap%
\pgfsetroundjoin%
\pgfsetlinewidth{1.003750pt}%
\definecolor{currentstroke}{rgb}{0.000000,0.000000,0.000000}%
\pgfsetstrokecolor{currentstroke}%
\pgfsetdash{}{0pt}%
\pgfpathmoveto{\pgfqpoint{2.822740in}{1.018757in}}%
\pgfpathlineto{\pgfqpoint{3.200065in}{1.018757in}}%
\pgfpathlineto{\pgfqpoint{3.200065in}{1.811967in}}%
\pgfpathlineto{\pgfqpoint{2.822740in}{1.811967in}}%
\pgfpathlineto{\pgfqpoint{2.822740in}{1.018757in}}%
\pgfusepath{stroke}%
\end{pgfscope}%
\begin{pgfscope}%
\pgfpathrectangle{\pgfqpoint{1.920000in}{0.882640in}}{\pgfqpoint{11.904000in}{6.178480in}}%
\pgfusepath{clip}%
\pgfsetrectcap%
\pgfsetroundjoin%
\pgfsetlinewidth{1.003750pt}%
\definecolor{currentstroke}{rgb}{0.000000,0.000000,0.000000}%
\pgfsetstrokecolor{currentstroke}%
\pgfsetdash{}{0pt}%
\pgfpathmoveto{\pgfqpoint{3.011403in}{1.018757in}}%
\pgfpathlineto{\pgfqpoint{3.011403in}{1.018757in}}%
\pgfusepath{stroke}%
\end{pgfscope}%
\begin{pgfscope}%
\pgfpathrectangle{\pgfqpoint{1.920000in}{0.882640in}}{\pgfqpoint{11.904000in}{6.178480in}}%
\pgfusepath{clip}%
\pgfsetrectcap%
\pgfsetroundjoin%
\pgfsetlinewidth{1.003750pt}%
\definecolor{currentstroke}{rgb}{0.000000,0.000000,0.000000}%
\pgfsetstrokecolor{currentstroke}%
\pgfsetdash{}{0pt}%
\pgfpathmoveto{\pgfqpoint{3.011403in}{1.811967in}}%
\pgfpathlineto{\pgfqpoint{3.011403in}{2.820520in}}%
\pgfusepath{stroke}%
\end{pgfscope}%
\begin{pgfscope}%
\pgfpathrectangle{\pgfqpoint{1.920000in}{0.882640in}}{\pgfqpoint{11.904000in}{6.178480in}}%
\pgfusepath{clip}%
\pgfsetrectcap%
\pgfsetroundjoin%
\pgfsetlinewidth{1.003750pt}%
\definecolor{currentstroke}{rgb}{0.000000,0.000000,0.000000}%
\pgfsetstrokecolor{currentstroke}%
\pgfsetdash{}{0pt}%
\pgfpathmoveto{\pgfqpoint{2.917072in}{1.018757in}}%
\pgfpathlineto{\pgfqpoint{3.105734in}{1.018757in}}%
\pgfusepath{stroke}%
\end{pgfscope}%
\begin{pgfscope}%
\pgfpathrectangle{\pgfqpoint{1.920000in}{0.882640in}}{\pgfqpoint{11.904000in}{6.178480in}}%
\pgfusepath{clip}%
\pgfsetrectcap%
\pgfsetroundjoin%
\pgfsetlinewidth{1.003750pt}%
\definecolor{currentstroke}{rgb}{0.000000,0.000000,0.000000}%
\pgfsetstrokecolor{currentstroke}%
\pgfsetdash{}{0pt}%
\pgfpathmoveto{\pgfqpoint{2.917072in}{2.820520in}}%
\pgfpathlineto{\pgfqpoint{3.105734in}{2.820520in}}%
\pgfusepath{stroke}%
\end{pgfscope}%
\begin{pgfscope}%
\pgfpathrectangle{\pgfqpoint{1.920000in}{0.882640in}}{\pgfqpoint{11.904000in}{6.178480in}}%
\pgfusepath{clip}%
\pgfsetbuttcap%
\pgfsetroundjoin%
\definecolor{currentfill}{rgb}{0.000000,0.000000,0.000000}%
\pgfsetfillcolor{currentfill}%
\pgfsetfillopacity{0.000000}%
\pgfsetlinewidth{1.003750pt}%
\definecolor{currentstroke}{rgb}{0.000000,0.000000,0.000000}%
\pgfsetstrokecolor{currentstroke}%
\pgfsetdash{}{0pt}%
\pgfsys@defobject{currentmarker}{\pgfqpoint{-0.041667in}{-0.041667in}}{\pgfqpoint{0.041667in}{0.041667in}}{%
\pgfpathmoveto{\pgfqpoint{0.000000in}{-0.041667in}}%
\pgfpathcurveto{\pgfqpoint{0.011050in}{-0.041667in}}{\pgfqpoint{0.021649in}{-0.037276in}}{\pgfqpoint{0.029463in}{-0.029463in}}%
\pgfpathcurveto{\pgfqpoint{0.037276in}{-0.021649in}}{\pgfqpoint{0.041667in}{-0.011050in}}{\pgfqpoint{0.041667in}{0.000000in}}%
\pgfpathcurveto{\pgfqpoint{0.041667in}{0.011050in}}{\pgfqpoint{0.037276in}{0.021649in}}{\pgfqpoint{0.029463in}{0.029463in}}%
\pgfpathcurveto{\pgfqpoint{0.021649in}{0.037276in}}{\pgfqpoint{0.011050in}{0.041667in}}{\pgfqpoint{0.000000in}{0.041667in}}%
\pgfpathcurveto{\pgfqpoint{-0.011050in}{0.041667in}}{\pgfqpoint{-0.021649in}{0.037276in}}{\pgfqpoint{-0.029463in}{0.029463in}}%
\pgfpathcurveto{\pgfqpoint{-0.037276in}{0.021649in}}{\pgfqpoint{-0.041667in}{0.011050in}}{\pgfqpoint{-0.041667in}{0.000000in}}%
\pgfpathcurveto{\pgfqpoint{-0.041667in}{-0.011050in}}{\pgfqpoint{-0.037276in}{-0.021649in}}{\pgfqpoint{-0.029463in}{-0.029463in}}%
\pgfpathcurveto{\pgfqpoint{-0.021649in}{-0.037276in}}{\pgfqpoint{-0.011050in}{-0.041667in}}{\pgfqpoint{0.000000in}{-0.041667in}}%
\pgfpathlineto{\pgfqpoint{0.000000in}{-0.041667in}}%
\pgfpathclose%
\pgfusepath{stroke,fill}%
}%
\begin{pgfscope}%
\pgfsys@transformshift{3.011403in}{4.295597in}%
\pgfsys@useobject{currentmarker}{}%
\end{pgfscope}%
\begin{pgfscope}%
\pgfsys@transformshift{3.011403in}{3.363823in}%
\pgfsys@useobject{currentmarker}{}%
\end{pgfscope}%
\begin{pgfscope}%
\pgfsys@transformshift{3.011403in}{3.202691in}%
\pgfsys@useobject{currentmarker}{}%
\end{pgfscope}%
\begin{pgfscope}%
\pgfsys@transformshift{3.011403in}{3.756690in}%
\pgfsys@useobject{currentmarker}{}%
\end{pgfscope}%
\begin{pgfscope}%
\pgfsys@transformshift{3.011403in}{3.817986in}%
\pgfsys@useobject{currentmarker}{}%
\end{pgfscope}%
\begin{pgfscope}%
\pgfsys@transformshift{3.011403in}{3.284941in}%
\pgfsys@useobject{currentmarker}{}%
\end{pgfscope}%
\begin{pgfscope}%
\pgfsys@transformshift{3.011403in}{3.131831in}%
\pgfsys@useobject{currentmarker}{}%
\end{pgfscope}%
\end{pgfscope}%
\begin{pgfscope}%
\pgfpathrectangle{\pgfqpoint{1.920000in}{0.882640in}}{\pgfqpoint{11.904000in}{6.178480in}}%
\pgfusepath{clip}%
\pgfsetrectcap%
\pgfsetroundjoin%
\pgfsetlinewidth{1.003750pt}%
\definecolor{currentstroke}{rgb}{0.000000,0.000000,0.000000}%
\pgfsetstrokecolor{currentstroke}%
\pgfsetdash{}{0pt}%
\pgfpathmoveto{\pgfqpoint{3.577391in}{1.018757in}}%
\pgfpathlineto{\pgfqpoint{3.954716in}{1.018757in}}%
\pgfpathlineto{\pgfqpoint{3.954716in}{1.720345in}}%
\pgfpathlineto{\pgfqpoint{3.577391in}{1.720345in}}%
\pgfpathlineto{\pgfqpoint{3.577391in}{1.018757in}}%
\pgfusepath{stroke}%
\end{pgfscope}%
\begin{pgfscope}%
\pgfpathrectangle{\pgfqpoint{1.920000in}{0.882640in}}{\pgfqpoint{11.904000in}{6.178480in}}%
\pgfusepath{clip}%
\pgfsetrectcap%
\pgfsetroundjoin%
\pgfsetlinewidth{1.003750pt}%
\definecolor{currentstroke}{rgb}{0.000000,0.000000,0.000000}%
\pgfsetstrokecolor{currentstroke}%
\pgfsetdash{}{0pt}%
\pgfpathmoveto{\pgfqpoint{3.766053in}{1.018757in}}%
\pgfpathlineto{\pgfqpoint{3.766053in}{1.018757in}}%
\pgfusepath{stroke}%
\end{pgfscope}%
\begin{pgfscope}%
\pgfpathrectangle{\pgfqpoint{1.920000in}{0.882640in}}{\pgfqpoint{11.904000in}{6.178480in}}%
\pgfusepath{clip}%
\pgfsetrectcap%
\pgfsetroundjoin%
\pgfsetlinewidth{1.003750pt}%
\definecolor{currentstroke}{rgb}{0.000000,0.000000,0.000000}%
\pgfsetstrokecolor{currentstroke}%
\pgfsetdash{}{0pt}%
\pgfpathmoveto{\pgfqpoint{3.766053in}{1.720345in}}%
\pgfpathlineto{\pgfqpoint{3.766053in}{2.714950in}}%
\pgfusepath{stroke}%
\end{pgfscope}%
\begin{pgfscope}%
\pgfpathrectangle{\pgfqpoint{1.920000in}{0.882640in}}{\pgfqpoint{11.904000in}{6.178480in}}%
\pgfusepath{clip}%
\pgfsetrectcap%
\pgfsetroundjoin%
\pgfsetlinewidth{1.003750pt}%
\definecolor{currentstroke}{rgb}{0.000000,0.000000,0.000000}%
\pgfsetstrokecolor{currentstroke}%
\pgfsetdash{}{0pt}%
\pgfpathmoveto{\pgfqpoint{3.671722in}{1.018757in}}%
\pgfpathlineto{\pgfqpoint{3.860384in}{1.018757in}}%
\pgfusepath{stroke}%
\end{pgfscope}%
\begin{pgfscope}%
\pgfpathrectangle{\pgfqpoint{1.920000in}{0.882640in}}{\pgfqpoint{11.904000in}{6.178480in}}%
\pgfusepath{clip}%
\pgfsetrectcap%
\pgfsetroundjoin%
\pgfsetlinewidth{1.003750pt}%
\definecolor{currentstroke}{rgb}{0.000000,0.000000,0.000000}%
\pgfsetstrokecolor{currentstroke}%
\pgfsetdash{}{0pt}%
\pgfpathmoveto{\pgfqpoint{3.671722in}{2.714950in}}%
\pgfpathlineto{\pgfqpoint{3.860384in}{2.714950in}}%
\pgfusepath{stroke}%
\end{pgfscope}%
\begin{pgfscope}%
\pgfpathrectangle{\pgfqpoint{1.920000in}{0.882640in}}{\pgfqpoint{11.904000in}{6.178480in}}%
\pgfusepath{clip}%
\pgfsetbuttcap%
\pgfsetroundjoin%
\definecolor{currentfill}{rgb}{0.000000,0.000000,0.000000}%
\pgfsetfillcolor{currentfill}%
\pgfsetfillopacity{0.000000}%
\pgfsetlinewidth{1.003750pt}%
\definecolor{currentstroke}{rgb}{0.000000,0.000000,0.000000}%
\pgfsetstrokecolor{currentstroke}%
\pgfsetdash{}{0pt}%
\pgfsys@defobject{currentmarker}{\pgfqpoint{-0.041667in}{-0.041667in}}{\pgfqpoint{0.041667in}{0.041667in}}{%
\pgfpathmoveto{\pgfqpoint{0.000000in}{-0.041667in}}%
\pgfpathcurveto{\pgfqpoint{0.011050in}{-0.041667in}}{\pgfqpoint{0.021649in}{-0.037276in}}{\pgfqpoint{0.029463in}{-0.029463in}}%
\pgfpathcurveto{\pgfqpoint{0.037276in}{-0.021649in}}{\pgfqpoint{0.041667in}{-0.011050in}}{\pgfqpoint{0.041667in}{0.000000in}}%
\pgfpathcurveto{\pgfqpoint{0.041667in}{0.011050in}}{\pgfqpoint{0.037276in}{0.021649in}}{\pgfqpoint{0.029463in}{0.029463in}}%
\pgfpathcurveto{\pgfqpoint{0.021649in}{0.037276in}}{\pgfqpoint{0.011050in}{0.041667in}}{\pgfqpoint{0.000000in}{0.041667in}}%
\pgfpathcurveto{\pgfqpoint{-0.011050in}{0.041667in}}{\pgfqpoint{-0.021649in}{0.037276in}}{\pgfqpoint{-0.029463in}{0.029463in}}%
\pgfpathcurveto{\pgfqpoint{-0.037276in}{0.021649in}}{\pgfqpoint{-0.041667in}{0.011050in}}{\pgfqpoint{-0.041667in}{0.000000in}}%
\pgfpathcurveto{\pgfqpoint{-0.041667in}{-0.011050in}}{\pgfqpoint{-0.037276in}{-0.021649in}}{\pgfqpoint{-0.029463in}{-0.029463in}}%
\pgfpathcurveto{\pgfqpoint{-0.021649in}{-0.037276in}}{\pgfqpoint{-0.011050in}{-0.041667in}}{\pgfqpoint{0.000000in}{-0.041667in}}%
\pgfpathlineto{\pgfqpoint{0.000000in}{-0.041667in}}%
\pgfpathclose%
\pgfusepath{stroke,fill}%
}%
\begin{pgfscope}%
\pgfsys@transformshift{3.766053in}{3.972536in}%
\pgfsys@useobject{currentmarker}{}%
\end{pgfscope}%
\begin{pgfscope}%
\pgfsys@transformshift{3.766053in}{4.434180in}%
\pgfsys@useobject{currentmarker}{}%
\end{pgfscope}%
\begin{pgfscope}%
\pgfsys@transformshift{3.766053in}{3.530380in}%
\pgfsys@useobject{currentmarker}{}%
\end{pgfscope}%
\begin{pgfscope}%
\pgfsys@transformshift{3.766053in}{3.318160in}%
\pgfsys@useobject{currentmarker}{}%
\end{pgfscope}%
\begin{pgfscope}%
\pgfsys@transformshift{3.766053in}{3.021402in}%
\pgfsys@useobject{currentmarker}{}%
\end{pgfscope}%
\begin{pgfscope}%
\pgfsys@transformshift{3.766053in}{2.829930in}%
\pgfsys@useobject{currentmarker}{}%
\end{pgfscope}%
\begin{pgfscope}%
\pgfsys@transformshift{3.766053in}{3.331710in}%
\pgfsys@useobject{currentmarker}{}%
\end{pgfscope}%
\end{pgfscope}%
\begin{pgfscope}%
\pgfpathrectangle{\pgfqpoint{1.920000in}{0.882640in}}{\pgfqpoint{11.904000in}{6.178480in}}%
\pgfusepath{clip}%
\pgfsetrectcap%
\pgfsetroundjoin%
\pgfsetlinewidth{1.003750pt}%
\definecolor{currentstroke}{rgb}{0.000000,0.000000,0.000000}%
\pgfsetstrokecolor{currentstroke}%
\pgfsetdash{}{0pt}%
\pgfpathmoveto{\pgfqpoint{4.332041in}{1.018757in}}%
\pgfpathlineto{\pgfqpoint{4.709366in}{1.018757in}}%
\pgfpathlineto{\pgfqpoint{4.709366in}{1.793371in}}%
\pgfpathlineto{\pgfqpoint{4.332041in}{1.793371in}}%
\pgfpathlineto{\pgfqpoint{4.332041in}{1.018757in}}%
\pgfusepath{stroke}%
\end{pgfscope}%
\begin{pgfscope}%
\pgfpathrectangle{\pgfqpoint{1.920000in}{0.882640in}}{\pgfqpoint{11.904000in}{6.178480in}}%
\pgfusepath{clip}%
\pgfsetrectcap%
\pgfsetroundjoin%
\pgfsetlinewidth{1.003750pt}%
\definecolor{currentstroke}{rgb}{0.000000,0.000000,0.000000}%
\pgfsetstrokecolor{currentstroke}%
\pgfsetdash{}{0pt}%
\pgfpathmoveto{\pgfqpoint{4.520703in}{1.018757in}}%
\pgfpathlineto{\pgfqpoint{4.520703in}{1.018757in}}%
\pgfusepath{stroke}%
\end{pgfscope}%
\begin{pgfscope}%
\pgfpathrectangle{\pgfqpoint{1.920000in}{0.882640in}}{\pgfqpoint{11.904000in}{6.178480in}}%
\pgfusepath{clip}%
\pgfsetrectcap%
\pgfsetroundjoin%
\pgfsetlinewidth{1.003750pt}%
\definecolor{currentstroke}{rgb}{0.000000,0.000000,0.000000}%
\pgfsetstrokecolor{currentstroke}%
\pgfsetdash{}{0pt}%
\pgfpathmoveto{\pgfqpoint{4.520703in}{1.793371in}}%
\pgfpathlineto{\pgfqpoint{4.520703in}{2.915291in}}%
\pgfusepath{stroke}%
\end{pgfscope}%
\begin{pgfscope}%
\pgfpathrectangle{\pgfqpoint{1.920000in}{0.882640in}}{\pgfqpoint{11.904000in}{6.178480in}}%
\pgfusepath{clip}%
\pgfsetrectcap%
\pgfsetroundjoin%
\pgfsetlinewidth{1.003750pt}%
\definecolor{currentstroke}{rgb}{0.000000,0.000000,0.000000}%
\pgfsetstrokecolor{currentstroke}%
\pgfsetdash{}{0pt}%
\pgfpathmoveto{\pgfqpoint{4.426372in}{1.018757in}}%
\pgfpathlineto{\pgfqpoint{4.615035in}{1.018757in}}%
\pgfusepath{stroke}%
\end{pgfscope}%
\begin{pgfscope}%
\pgfpathrectangle{\pgfqpoint{1.920000in}{0.882640in}}{\pgfqpoint{11.904000in}{6.178480in}}%
\pgfusepath{clip}%
\pgfsetrectcap%
\pgfsetroundjoin%
\pgfsetlinewidth{1.003750pt}%
\definecolor{currentstroke}{rgb}{0.000000,0.000000,0.000000}%
\pgfsetstrokecolor{currentstroke}%
\pgfsetdash{}{0pt}%
\pgfpathmoveto{\pgfqpoint{4.426372in}{2.915291in}}%
\pgfpathlineto{\pgfqpoint{4.615035in}{2.915291in}}%
\pgfusepath{stroke}%
\end{pgfscope}%
\begin{pgfscope}%
\pgfpathrectangle{\pgfqpoint{1.920000in}{0.882640in}}{\pgfqpoint{11.904000in}{6.178480in}}%
\pgfusepath{clip}%
\pgfsetbuttcap%
\pgfsetroundjoin%
\definecolor{currentfill}{rgb}{0.000000,0.000000,0.000000}%
\pgfsetfillcolor{currentfill}%
\pgfsetfillopacity{0.000000}%
\pgfsetlinewidth{1.003750pt}%
\definecolor{currentstroke}{rgb}{0.000000,0.000000,0.000000}%
\pgfsetstrokecolor{currentstroke}%
\pgfsetdash{}{0pt}%
\pgfsys@defobject{currentmarker}{\pgfqpoint{-0.041667in}{-0.041667in}}{\pgfqpoint{0.041667in}{0.041667in}}{%
\pgfpathmoveto{\pgfqpoint{0.000000in}{-0.041667in}}%
\pgfpathcurveto{\pgfqpoint{0.011050in}{-0.041667in}}{\pgfqpoint{0.021649in}{-0.037276in}}{\pgfqpoint{0.029463in}{-0.029463in}}%
\pgfpathcurveto{\pgfqpoint{0.037276in}{-0.021649in}}{\pgfqpoint{0.041667in}{-0.011050in}}{\pgfqpoint{0.041667in}{0.000000in}}%
\pgfpathcurveto{\pgfqpoint{0.041667in}{0.011050in}}{\pgfqpoint{0.037276in}{0.021649in}}{\pgfqpoint{0.029463in}{0.029463in}}%
\pgfpathcurveto{\pgfqpoint{0.021649in}{0.037276in}}{\pgfqpoint{0.011050in}{0.041667in}}{\pgfqpoint{0.000000in}{0.041667in}}%
\pgfpathcurveto{\pgfqpoint{-0.011050in}{0.041667in}}{\pgfqpoint{-0.021649in}{0.037276in}}{\pgfqpoint{-0.029463in}{0.029463in}}%
\pgfpathcurveto{\pgfqpoint{-0.037276in}{0.021649in}}{\pgfqpoint{-0.041667in}{0.011050in}}{\pgfqpoint{-0.041667in}{0.000000in}}%
\pgfpathcurveto{\pgfqpoint{-0.041667in}{-0.011050in}}{\pgfqpoint{-0.037276in}{-0.021649in}}{\pgfqpoint{-0.029463in}{-0.029463in}}%
\pgfpathcurveto{\pgfqpoint{-0.021649in}{-0.037276in}}{\pgfqpoint{-0.011050in}{-0.041667in}}{\pgfqpoint{0.000000in}{-0.041667in}}%
\pgfpathlineto{\pgfqpoint{0.000000in}{-0.041667in}}%
\pgfpathclose%
\pgfusepath{stroke,fill}%
}%
\begin{pgfscope}%
\pgfsys@transformshift{4.520703in}{3.643664in}%
\pgfsys@useobject{currentmarker}{}%
\end{pgfscope}%
\begin{pgfscope}%
\pgfsys@transformshift{4.520703in}{3.971816in}%
\pgfsys@useobject{currentmarker}{}%
\end{pgfscope}%
\begin{pgfscope}%
\pgfsys@transformshift{4.520703in}{3.333021in}%
\pgfsys@useobject{currentmarker}{}%
\end{pgfscope}%
\begin{pgfscope}%
\pgfsys@transformshift{4.520703in}{3.086862in}%
\pgfsys@useobject{currentmarker}{}%
\end{pgfscope}%
\begin{pgfscope}%
\pgfsys@transformshift{4.520703in}{2.957432in}%
\pgfsys@useobject{currentmarker}{}%
\end{pgfscope}%
\begin{pgfscope}%
\pgfsys@transformshift{4.520703in}{3.096401in}%
\pgfsys@useobject{currentmarker}{}%
\end{pgfscope}%
\begin{pgfscope}%
\pgfsys@transformshift{4.520703in}{6.833651in}%
\pgfsys@useobject{currentmarker}{}%
\end{pgfscope}%
\end{pgfscope}%
\begin{pgfscope}%
\pgfpathrectangle{\pgfqpoint{1.920000in}{0.882640in}}{\pgfqpoint{11.904000in}{6.178480in}}%
\pgfusepath{clip}%
\pgfsetrectcap%
\pgfsetroundjoin%
\pgfsetlinewidth{1.003750pt}%
\definecolor{currentstroke}{rgb}{0.000000,0.000000,0.000000}%
\pgfsetstrokecolor{currentstroke}%
\pgfsetdash{}{0pt}%
\pgfpathmoveto{\pgfqpoint{5.086691in}{1.018757in}}%
\pgfpathlineto{\pgfqpoint{5.464016in}{1.018757in}}%
\pgfpathlineto{\pgfqpoint{5.464016in}{1.721534in}}%
\pgfpathlineto{\pgfqpoint{5.086691in}{1.721534in}}%
\pgfpathlineto{\pgfqpoint{5.086691in}{1.018757in}}%
\pgfusepath{stroke}%
\end{pgfscope}%
\begin{pgfscope}%
\pgfpathrectangle{\pgfqpoint{1.920000in}{0.882640in}}{\pgfqpoint{11.904000in}{6.178480in}}%
\pgfusepath{clip}%
\pgfsetrectcap%
\pgfsetroundjoin%
\pgfsetlinewidth{1.003750pt}%
\definecolor{currentstroke}{rgb}{0.000000,0.000000,0.000000}%
\pgfsetstrokecolor{currentstroke}%
\pgfsetdash{}{0pt}%
\pgfpathmoveto{\pgfqpoint{5.275354in}{1.018757in}}%
\pgfpathlineto{\pgfqpoint{5.275354in}{1.018757in}}%
\pgfusepath{stroke}%
\end{pgfscope}%
\begin{pgfscope}%
\pgfpathrectangle{\pgfqpoint{1.920000in}{0.882640in}}{\pgfqpoint{11.904000in}{6.178480in}}%
\pgfusepath{clip}%
\pgfsetrectcap%
\pgfsetroundjoin%
\pgfsetlinewidth{1.003750pt}%
\definecolor{currentstroke}{rgb}{0.000000,0.000000,0.000000}%
\pgfsetstrokecolor{currentstroke}%
\pgfsetdash{}{0pt}%
\pgfpathmoveto{\pgfqpoint{5.275354in}{1.721534in}}%
\pgfpathlineto{\pgfqpoint{5.275354in}{2.720401in}}%
\pgfusepath{stroke}%
\end{pgfscope}%
\begin{pgfscope}%
\pgfpathrectangle{\pgfqpoint{1.920000in}{0.882640in}}{\pgfqpoint{11.904000in}{6.178480in}}%
\pgfusepath{clip}%
\pgfsetrectcap%
\pgfsetroundjoin%
\pgfsetlinewidth{1.003750pt}%
\definecolor{currentstroke}{rgb}{0.000000,0.000000,0.000000}%
\pgfsetstrokecolor{currentstroke}%
\pgfsetdash{}{0pt}%
\pgfpathmoveto{\pgfqpoint{5.181022in}{1.018757in}}%
\pgfpathlineto{\pgfqpoint{5.369685in}{1.018757in}}%
\pgfusepath{stroke}%
\end{pgfscope}%
\begin{pgfscope}%
\pgfpathrectangle{\pgfqpoint{1.920000in}{0.882640in}}{\pgfqpoint{11.904000in}{6.178480in}}%
\pgfusepath{clip}%
\pgfsetrectcap%
\pgfsetroundjoin%
\pgfsetlinewidth{1.003750pt}%
\definecolor{currentstroke}{rgb}{0.000000,0.000000,0.000000}%
\pgfsetstrokecolor{currentstroke}%
\pgfsetdash{}{0pt}%
\pgfpathmoveto{\pgfqpoint{5.181022in}{2.720401in}}%
\pgfpathlineto{\pgfqpoint{5.369685in}{2.720401in}}%
\pgfusepath{stroke}%
\end{pgfscope}%
\begin{pgfscope}%
\pgfpathrectangle{\pgfqpoint{1.920000in}{0.882640in}}{\pgfqpoint{11.904000in}{6.178480in}}%
\pgfusepath{clip}%
\pgfsetbuttcap%
\pgfsetroundjoin%
\definecolor{currentfill}{rgb}{0.000000,0.000000,0.000000}%
\pgfsetfillcolor{currentfill}%
\pgfsetfillopacity{0.000000}%
\pgfsetlinewidth{1.003750pt}%
\definecolor{currentstroke}{rgb}{0.000000,0.000000,0.000000}%
\pgfsetstrokecolor{currentstroke}%
\pgfsetdash{}{0pt}%
\pgfsys@defobject{currentmarker}{\pgfqpoint{-0.041667in}{-0.041667in}}{\pgfqpoint{0.041667in}{0.041667in}}{%
\pgfpathmoveto{\pgfqpoint{0.000000in}{-0.041667in}}%
\pgfpathcurveto{\pgfqpoint{0.011050in}{-0.041667in}}{\pgfqpoint{0.021649in}{-0.037276in}}{\pgfqpoint{0.029463in}{-0.029463in}}%
\pgfpathcurveto{\pgfqpoint{0.037276in}{-0.021649in}}{\pgfqpoint{0.041667in}{-0.011050in}}{\pgfqpoint{0.041667in}{0.000000in}}%
\pgfpathcurveto{\pgfqpoint{0.041667in}{0.011050in}}{\pgfqpoint{0.037276in}{0.021649in}}{\pgfqpoint{0.029463in}{0.029463in}}%
\pgfpathcurveto{\pgfqpoint{0.021649in}{0.037276in}}{\pgfqpoint{0.011050in}{0.041667in}}{\pgfqpoint{0.000000in}{0.041667in}}%
\pgfpathcurveto{\pgfqpoint{-0.011050in}{0.041667in}}{\pgfqpoint{-0.021649in}{0.037276in}}{\pgfqpoint{-0.029463in}{0.029463in}}%
\pgfpathcurveto{\pgfqpoint{-0.037276in}{0.021649in}}{\pgfqpoint{-0.041667in}{0.011050in}}{\pgfqpoint{-0.041667in}{0.000000in}}%
\pgfpathcurveto{\pgfqpoint{-0.041667in}{-0.011050in}}{\pgfqpoint{-0.037276in}{-0.021649in}}{\pgfqpoint{-0.029463in}{-0.029463in}}%
\pgfpathcurveto{\pgfqpoint{-0.021649in}{-0.037276in}}{\pgfqpoint{-0.011050in}{-0.041667in}}{\pgfqpoint{0.000000in}{-0.041667in}}%
\pgfpathlineto{\pgfqpoint{0.000000in}{-0.041667in}}%
\pgfpathclose%
\pgfusepath{stroke,fill}%
}%
\begin{pgfscope}%
\pgfsys@transformshift{5.275354in}{4.165781in}%
\pgfsys@useobject{currentmarker}{}%
\end{pgfscope}%
\begin{pgfscope}%
\pgfsys@transformshift{5.275354in}{3.279156in}%
\pgfsys@useobject{currentmarker}{}%
\end{pgfscope}%
\begin{pgfscope}%
\pgfsys@transformshift{5.275354in}{4.180590in}%
\pgfsys@useobject{currentmarker}{}%
\end{pgfscope}%
\begin{pgfscope}%
\pgfsys@transformshift{5.275354in}{3.011194in}%
\pgfsys@useobject{currentmarker}{}%
\end{pgfscope}%
\begin{pgfscope}%
\pgfsys@transformshift{5.275354in}{3.298362in}%
\pgfsys@useobject{currentmarker}{}%
\end{pgfscope}%
\begin{pgfscope}%
\pgfsys@transformshift{5.275354in}{5.769182in}%
\pgfsys@useobject{currentmarker}{}%
\end{pgfscope}%
\end{pgfscope}%
\begin{pgfscope}%
\pgfpathrectangle{\pgfqpoint{1.920000in}{0.882640in}}{\pgfqpoint{11.904000in}{6.178480in}}%
\pgfusepath{clip}%
\pgfsetrectcap%
\pgfsetroundjoin%
\pgfsetlinewidth{1.003750pt}%
\definecolor{currentstroke}{rgb}{0.000000,0.000000,0.000000}%
\pgfsetstrokecolor{currentstroke}%
\pgfsetdash{}{0pt}%
\pgfpathmoveto{\pgfqpoint{5.841342in}{1.018757in}}%
\pgfpathlineto{\pgfqpoint{6.218667in}{1.018757in}}%
\pgfpathlineto{\pgfqpoint{6.218667in}{1.697121in}}%
\pgfpathlineto{\pgfqpoint{5.841342in}{1.697121in}}%
\pgfpathlineto{\pgfqpoint{5.841342in}{1.018757in}}%
\pgfusepath{stroke}%
\end{pgfscope}%
\begin{pgfscope}%
\pgfpathrectangle{\pgfqpoint{1.920000in}{0.882640in}}{\pgfqpoint{11.904000in}{6.178480in}}%
\pgfusepath{clip}%
\pgfsetrectcap%
\pgfsetroundjoin%
\pgfsetlinewidth{1.003750pt}%
\definecolor{currentstroke}{rgb}{0.000000,0.000000,0.000000}%
\pgfsetstrokecolor{currentstroke}%
\pgfsetdash{}{0pt}%
\pgfpathmoveto{\pgfqpoint{6.030004in}{1.018757in}}%
\pgfpathlineto{\pgfqpoint{6.030004in}{1.018757in}}%
\pgfusepath{stroke}%
\end{pgfscope}%
\begin{pgfscope}%
\pgfpathrectangle{\pgfqpoint{1.920000in}{0.882640in}}{\pgfqpoint{11.904000in}{6.178480in}}%
\pgfusepath{clip}%
\pgfsetrectcap%
\pgfsetroundjoin%
\pgfsetlinewidth{1.003750pt}%
\definecolor{currentstroke}{rgb}{0.000000,0.000000,0.000000}%
\pgfsetstrokecolor{currentstroke}%
\pgfsetdash{}{0pt}%
\pgfpathmoveto{\pgfqpoint{6.030004in}{1.697121in}}%
\pgfpathlineto{\pgfqpoint{6.030004in}{2.628715in}}%
\pgfusepath{stroke}%
\end{pgfscope}%
\begin{pgfscope}%
\pgfpathrectangle{\pgfqpoint{1.920000in}{0.882640in}}{\pgfqpoint{11.904000in}{6.178480in}}%
\pgfusepath{clip}%
\pgfsetrectcap%
\pgfsetroundjoin%
\pgfsetlinewidth{1.003750pt}%
\definecolor{currentstroke}{rgb}{0.000000,0.000000,0.000000}%
\pgfsetstrokecolor{currentstroke}%
\pgfsetdash{}{0pt}%
\pgfpathmoveto{\pgfqpoint{5.935673in}{1.018757in}}%
\pgfpathlineto{\pgfqpoint{6.124335in}{1.018757in}}%
\pgfusepath{stroke}%
\end{pgfscope}%
\begin{pgfscope}%
\pgfpathrectangle{\pgfqpoint{1.920000in}{0.882640in}}{\pgfqpoint{11.904000in}{6.178480in}}%
\pgfusepath{clip}%
\pgfsetrectcap%
\pgfsetroundjoin%
\pgfsetlinewidth{1.003750pt}%
\definecolor{currentstroke}{rgb}{0.000000,0.000000,0.000000}%
\pgfsetstrokecolor{currentstroke}%
\pgfsetdash{}{0pt}%
\pgfpathmoveto{\pgfqpoint{5.935673in}{2.628715in}}%
\pgfpathlineto{\pgfqpoint{6.124335in}{2.628715in}}%
\pgfusepath{stroke}%
\end{pgfscope}%
\begin{pgfscope}%
\pgfpathrectangle{\pgfqpoint{1.920000in}{0.882640in}}{\pgfqpoint{11.904000in}{6.178480in}}%
\pgfusepath{clip}%
\pgfsetbuttcap%
\pgfsetroundjoin%
\definecolor{currentfill}{rgb}{0.000000,0.000000,0.000000}%
\pgfsetfillcolor{currentfill}%
\pgfsetfillopacity{0.000000}%
\pgfsetlinewidth{1.003750pt}%
\definecolor{currentstroke}{rgb}{0.000000,0.000000,0.000000}%
\pgfsetstrokecolor{currentstroke}%
\pgfsetdash{}{0pt}%
\pgfsys@defobject{currentmarker}{\pgfqpoint{-0.041667in}{-0.041667in}}{\pgfqpoint{0.041667in}{0.041667in}}{%
\pgfpathmoveto{\pgfqpoint{0.000000in}{-0.041667in}}%
\pgfpathcurveto{\pgfqpoint{0.011050in}{-0.041667in}}{\pgfqpoint{0.021649in}{-0.037276in}}{\pgfqpoint{0.029463in}{-0.029463in}}%
\pgfpathcurveto{\pgfqpoint{0.037276in}{-0.021649in}}{\pgfqpoint{0.041667in}{-0.011050in}}{\pgfqpoint{0.041667in}{0.000000in}}%
\pgfpathcurveto{\pgfqpoint{0.041667in}{0.011050in}}{\pgfqpoint{0.037276in}{0.021649in}}{\pgfqpoint{0.029463in}{0.029463in}}%
\pgfpathcurveto{\pgfqpoint{0.021649in}{0.037276in}}{\pgfqpoint{0.011050in}{0.041667in}}{\pgfqpoint{0.000000in}{0.041667in}}%
\pgfpathcurveto{\pgfqpoint{-0.011050in}{0.041667in}}{\pgfqpoint{-0.021649in}{0.037276in}}{\pgfqpoint{-0.029463in}{0.029463in}}%
\pgfpathcurveto{\pgfqpoint{-0.037276in}{0.021649in}}{\pgfqpoint{-0.041667in}{0.011050in}}{\pgfqpoint{-0.041667in}{0.000000in}}%
\pgfpathcurveto{\pgfqpoint{-0.041667in}{-0.011050in}}{\pgfqpoint{-0.037276in}{-0.021649in}}{\pgfqpoint{-0.029463in}{-0.029463in}}%
\pgfpathcurveto{\pgfqpoint{-0.021649in}{-0.037276in}}{\pgfqpoint{-0.011050in}{-0.041667in}}{\pgfqpoint{0.000000in}{-0.041667in}}%
\pgfpathlineto{\pgfqpoint{0.000000in}{-0.041667in}}%
\pgfpathclose%
\pgfusepath{stroke,fill}%
}%
\begin{pgfscope}%
\pgfsys@transformshift{6.030004in}{4.332029in}%
\pgfsys@useobject{currentmarker}{}%
\end{pgfscope}%
\begin{pgfscope}%
\pgfsys@transformshift{6.030004in}{2.930641in}%
\pgfsys@useobject{currentmarker}{}%
\end{pgfscope}%
\begin{pgfscope}%
\pgfsys@transformshift{6.030004in}{3.973821in}%
\pgfsys@useobject{currentmarker}{}%
\end{pgfscope}%
\begin{pgfscope}%
\pgfsys@transformshift{6.030004in}{2.719038in}%
\pgfsys@useobject{currentmarker}{}%
\end{pgfscope}%
\begin{pgfscope}%
\pgfsys@transformshift{6.030004in}{2.770872in}%
\pgfsys@useobject{currentmarker}{}%
\end{pgfscope}%
\begin{pgfscope}%
\pgfsys@transformshift{6.030004in}{2.731585in}%
\pgfsys@useobject{currentmarker}{}%
\end{pgfscope}%
\end{pgfscope}%
\begin{pgfscope}%
\pgfpathrectangle{\pgfqpoint{1.920000in}{0.882640in}}{\pgfqpoint{11.904000in}{6.178480in}}%
\pgfusepath{clip}%
\pgfsetrectcap%
\pgfsetroundjoin%
\pgfsetlinewidth{1.003750pt}%
\definecolor{currentstroke}{rgb}{0.000000,0.000000,0.000000}%
\pgfsetstrokecolor{currentstroke}%
\pgfsetdash{}{0pt}%
\pgfpathmoveto{\pgfqpoint{6.595992in}{1.031844in}}%
\pgfpathlineto{\pgfqpoint{6.973317in}{1.031844in}}%
\pgfpathlineto{\pgfqpoint{6.973317in}{1.703298in}}%
\pgfpathlineto{\pgfqpoint{6.595992in}{1.703298in}}%
\pgfpathlineto{\pgfqpoint{6.595992in}{1.031844in}}%
\pgfusepath{stroke}%
\end{pgfscope}%
\begin{pgfscope}%
\pgfpathrectangle{\pgfqpoint{1.920000in}{0.882640in}}{\pgfqpoint{11.904000in}{6.178480in}}%
\pgfusepath{clip}%
\pgfsetrectcap%
\pgfsetroundjoin%
\pgfsetlinewidth{1.003750pt}%
\definecolor{currentstroke}{rgb}{0.000000,0.000000,0.000000}%
\pgfsetstrokecolor{currentstroke}%
\pgfsetdash{}{0pt}%
\pgfpathmoveto{\pgfqpoint{6.784654in}{1.031844in}}%
\pgfpathlineto{\pgfqpoint{6.784654in}{1.018757in}}%
\pgfusepath{stroke}%
\end{pgfscope}%
\begin{pgfscope}%
\pgfpathrectangle{\pgfqpoint{1.920000in}{0.882640in}}{\pgfqpoint{11.904000in}{6.178480in}}%
\pgfusepath{clip}%
\pgfsetrectcap%
\pgfsetroundjoin%
\pgfsetlinewidth{1.003750pt}%
\definecolor{currentstroke}{rgb}{0.000000,0.000000,0.000000}%
\pgfsetstrokecolor{currentstroke}%
\pgfsetdash{}{0pt}%
\pgfpathmoveto{\pgfqpoint{6.784654in}{1.703298in}}%
\pgfpathlineto{\pgfqpoint{6.784654in}{2.704254in}}%
\pgfusepath{stroke}%
\end{pgfscope}%
\begin{pgfscope}%
\pgfpathrectangle{\pgfqpoint{1.920000in}{0.882640in}}{\pgfqpoint{11.904000in}{6.178480in}}%
\pgfusepath{clip}%
\pgfsetrectcap%
\pgfsetroundjoin%
\pgfsetlinewidth{1.003750pt}%
\definecolor{currentstroke}{rgb}{0.000000,0.000000,0.000000}%
\pgfsetstrokecolor{currentstroke}%
\pgfsetdash{}{0pt}%
\pgfpathmoveto{\pgfqpoint{6.690323in}{1.018757in}}%
\pgfpathlineto{\pgfqpoint{6.878986in}{1.018757in}}%
\pgfusepath{stroke}%
\end{pgfscope}%
\begin{pgfscope}%
\pgfpathrectangle{\pgfqpoint{1.920000in}{0.882640in}}{\pgfqpoint{11.904000in}{6.178480in}}%
\pgfusepath{clip}%
\pgfsetrectcap%
\pgfsetroundjoin%
\pgfsetlinewidth{1.003750pt}%
\definecolor{currentstroke}{rgb}{0.000000,0.000000,0.000000}%
\pgfsetstrokecolor{currentstroke}%
\pgfsetdash{}{0pt}%
\pgfpathmoveto{\pgfqpoint{6.690323in}{2.704254in}}%
\pgfpathlineto{\pgfqpoint{6.878986in}{2.704254in}}%
\pgfusepath{stroke}%
\end{pgfscope}%
\begin{pgfscope}%
\pgfpathrectangle{\pgfqpoint{1.920000in}{0.882640in}}{\pgfqpoint{11.904000in}{6.178480in}}%
\pgfusepath{clip}%
\pgfsetbuttcap%
\pgfsetroundjoin%
\definecolor{currentfill}{rgb}{0.000000,0.000000,0.000000}%
\pgfsetfillcolor{currentfill}%
\pgfsetfillopacity{0.000000}%
\pgfsetlinewidth{1.003750pt}%
\definecolor{currentstroke}{rgb}{0.000000,0.000000,0.000000}%
\pgfsetstrokecolor{currentstroke}%
\pgfsetdash{}{0pt}%
\pgfsys@defobject{currentmarker}{\pgfqpoint{-0.041667in}{-0.041667in}}{\pgfqpoint{0.041667in}{0.041667in}}{%
\pgfpathmoveto{\pgfqpoint{0.000000in}{-0.041667in}}%
\pgfpathcurveto{\pgfqpoint{0.011050in}{-0.041667in}}{\pgfqpoint{0.021649in}{-0.037276in}}{\pgfqpoint{0.029463in}{-0.029463in}}%
\pgfpathcurveto{\pgfqpoint{0.037276in}{-0.021649in}}{\pgfqpoint{0.041667in}{-0.011050in}}{\pgfqpoint{0.041667in}{0.000000in}}%
\pgfpathcurveto{\pgfqpoint{0.041667in}{0.011050in}}{\pgfqpoint{0.037276in}{0.021649in}}{\pgfqpoint{0.029463in}{0.029463in}}%
\pgfpathcurveto{\pgfqpoint{0.021649in}{0.037276in}}{\pgfqpoint{0.011050in}{0.041667in}}{\pgfqpoint{0.000000in}{0.041667in}}%
\pgfpathcurveto{\pgfqpoint{-0.011050in}{0.041667in}}{\pgfqpoint{-0.021649in}{0.037276in}}{\pgfqpoint{-0.029463in}{0.029463in}}%
\pgfpathcurveto{\pgfqpoint{-0.037276in}{0.021649in}}{\pgfqpoint{-0.041667in}{0.011050in}}{\pgfqpoint{-0.041667in}{0.000000in}}%
\pgfpathcurveto{\pgfqpoint{-0.041667in}{-0.011050in}}{\pgfqpoint{-0.037276in}{-0.021649in}}{\pgfqpoint{-0.029463in}{-0.029463in}}%
\pgfpathcurveto{\pgfqpoint{-0.021649in}{-0.037276in}}{\pgfqpoint{-0.011050in}{-0.041667in}}{\pgfqpoint{0.000000in}{-0.041667in}}%
\pgfpathlineto{\pgfqpoint{0.000000in}{-0.041667in}}%
\pgfpathclose%
\pgfusepath{stroke,fill}%
}%
\begin{pgfscope}%
\pgfsys@transformshift{6.784654in}{4.200799in}%
\pgfsys@useobject{currentmarker}{}%
\end{pgfscope}%
\begin{pgfscope}%
\pgfsys@transformshift{6.784654in}{2.885775in}%
\pgfsys@useobject{currentmarker}{}%
\end{pgfscope}%
\begin{pgfscope}%
\pgfsys@transformshift{6.784654in}{4.170949in}%
\pgfsys@useobject{currentmarker}{}%
\end{pgfscope}%
\begin{pgfscope}%
\pgfsys@transformshift{6.784654in}{5.636332in}%
\pgfsys@useobject{currentmarker}{}%
\end{pgfscope}%
\end{pgfscope}%
\begin{pgfscope}%
\pgfpathrectangle{\pgfqpoint{1.920000in}{0.882640in}}{\pgfqpoint{11.904000in}{6.178480in}}%
\pgfusepath{clip}%
\pgfsetrectcap%
\pgfsetroundjoin%
\pgfsetlinewidth{1.003750pt}%
\definecolor{currentstroke}{rgb}{0.000000,0.000000,0.000000}%
\pgfsetstrokecolor{currentstroke}%
\pgfsetdash{}{0pt}%
\pgfpathmoveto{\pgfqpoint{7.350642in}{1.036536in}}%
\pgfpathlineto{\pgfqpoint{7.727967in}{1.036536in}}%
\pgfpathlineto{\pgfqpoint{7.727967in}{1.692673in}}%
\pgfpathlineto{\pgfqpoint{7.350642in}{1.692673in}}%
\pgfpathlineto{\pgfqpoint{7.350642in}{1.036536in}}%
\pgfusepath{stroke}%
\end{pgfscope}%
\begin{pgfscope}%
\pgfpathrectangle{\pgfqpoint{1.920000in}{0.882640in}}{\pgfqpoint{11.904000in}{6.178480in}}%
\pgfusepath{clip}%
\pgfsetrectcap%
\pgfsetroundjoin%
\pgfsetlinewidth{1.003750pt}%
\definecolor{currentstroke}{rgb}{0.000000,0.000000,0.000000}%
\pgfsetstrokecolor{currentstroke}%
\pgfsetdash{}{0pt}%
\pgfpathmoveto{\pgfqpoint{7.539305in}{1.036536in}}%
\pgfpathlineto{\pgfqpoint{7.539305in}{1.018757in}}%
\pgfusepath{stroke}%
\end{pgfscope}%
\begin{pgfscope}%
\pgfpathrectangle{\pgfqpoint{1.920000in}{0.882640in}}{\pgfqpoint{11.904000in}{6.178480in}}%
\pgfusepath{clip}%
\pgfsetrectcap%
\pgfsetroundjoin%
\pgfsetlinewidth{1.003750pt}%
\definecolor{currentstroke}{rgb}{0.000000,0.000000,0.000000}%
\pgfsetstrokecolor{currentstroke}%
\pgfsetdash{}{0pt}%
\pgfpathmoveto{\pgfqpoint{7.539305in}{1.692673in}}%
\pgfpathlineto{\pgfqpoint{7.539305in}{2.579555in}}%
\pgfusepath{stroke}%
\end{pgfscope}%
\begin{pgfscope}%
\pgfpathrectangle{\pgfqpoint{1.920000in}{0.882640in}}{\pgfqpoint{11.904000in}{6.178480in}}%
\pgfusepath{clip}%
\pgfsetrectcap%
\pgfsetroundjoin%
\pgfsetlinewidth{1.003750pt}%
\definecolor{currentstroke}{rgb}{0.000000,0.000000,0.000000}%
\pgfsetstrokecolor{currentstroke}%
\pgfsetdash{}{0pt}%
\pgfpathmoveto{\pgfqpoint{7.444973in}{1.018757in}}%
\pgfpathlineto{\pgfqpoint{7.633636in}{1.018757in}}%
\pgfusepath{stroke}%
\end{pgfscope}%
\begin{pgfscope}%
\pgfpathrectangle{\pgfqpoint{1.920000in}{0.882640in}}{\pgfqpoint{11.904000in}{6.178480in}}%
\pgfusepath{clip}%
\pgfsetrectcap%
\pgfsetroundjoin%
\pgfsetlinewidth{1.003750pt}%
\definecolor{currentstroke}{rgb}{0.000000,0.000000,0.000000}%
\pgfsetstrokecolor{currentstroke}%
\pgfsetdash{}{0pt}%
\pgfpathmoveto{\pgfqpoint{7.444973in}{2.579555in}}%
\pgfpathlineto{\pgfqpoint{7.633636in}{2.579555in}}%
\pgfusepath{stroke}%
\end{pgfscope}%
\begin{pgfscope}%
\pgfpathrectangle{\pgfqpoint{1.920000in}{0.882640in}}{\pgfqpoint{11.904000in}{6.178480in}}%
\pgfusepath{clip}%
\pgfsetbuttcap%
\pgfsetroundjoin%
\definecolor{currentfill}{rgb}{0.000000,0.000000,0.000000}%
\pgfsetfillcolor{currentfill}%
\pgfsetfillopacity{0.000000}%
\pgfsetlinewidth{1.003750pt}%
\definecolor{currentstroke}{rgb}{0.000000,0.000000,0.000000}%
\pgfsetstrokecolor{currentstroke}%
\pgfsetdash{}{0pt}%
\pgfsys@defobject{currentmarker}{\pgfqpoint{-0.041667in}{-0.041667in}}{\pgfqpoint{0.041667in}{0.041667in}}{%
\pgfpathmoveto{\pgfqpoint{0.000000in}{-0.041667in}}%
\pgfpathcurveto{\pgfqpoint{0.011050in}{-0.041667in}}{\pgfqpoint{0.021649in}{-0.037276in}}{\pgfqpoint{0.029463in}{-0.029463in}}%
\pgfpathcurveto{\pgfqpoint{0.037276in}{-0.021649in}}{\pgfqpoint{0.041667in}{-0.011050in}}{\pgfqpoint{0.041667in}{0.000000in}}%
\pgfpathcurveto{\pgfqpoint{0.041667in}{0.011050in}}{\pgfqpoint{0.037276in}{0.021649in}}{\pgfqpoint{0.029463in}{0.029463in}}%
\pgfpathcurveto{\pgfqpoint{0.021649in}{0.037276in}}{\pgfqpoint{0.011050in}{0.041667in}}{\pgfqpoint{0.000000in}{0.041667in}}%
\pgfpathcurveto{\pgfqpoint{-0.011050in}{0.041667in}}{\pgfqpoint{-0.021649in}{0.037276in}}{\pgfqpoint{-0.029463in}{0.029463in}}%
\pgfpathcurveto{\pgfqpoint{-0.037276in}{0.021649in}}{\pgfqpoint{-0.041667in}{0.011050in}}{\pgfqpoint{-0.041667in}{0.000000in}}%
\pgfpathcurveto{\pgfqpoint{-0.041667in}{-0.011050in}}{\pgfqpoint{-0.037276in}{-0.021649in}}{\pgfqpoint{-0.029463in}{-0.029463in}}%
\pgfpathcurveto{\pgfqpoint{-0.021649in}{-0.037276in}}{\pgfqpoint{-0.011050in}{-0.041667in}}{\pgfqpoint{0.000000in}{-0.041667in}}%
\pgfpathlineto{\pgfqpoint{0.000000in}{-0.041667in}}%
\pgfpathclose%
\pgfusepath{stroke,fill}%
}%
\begin{pgfscope}%
\pgfsys@transformshift{7.539305in}{4.476192in}%
\pgfsys@useobject{currentmarker}{}%
\end{pgfscope}%
\begin{pgfscope}%
\pgfsys@transformshift{7.539305in}{3.401130in}%
\pgfsys@useobject{currentmarker}{}%
\end{pgfscope}%
\begin{pgfscope}%
\pgfsys@transformshift{7.539305in}{2.825996in}%
\pgfsys@useobject{currentmarker}{}%
\end{pgfscope}%
\begin{pgfscope}%
\pgfsys@transformshift{7.539305in}{3.576635in}%
\pgfsys@useobject{currentmarker}{}%
\end{pgfscope}%
\end{pgfscope}%
\begin{pgfscope}%
\pgfpathrectangle{\pgfqpoint{1.920000in}{0.882640in}}{\pgfqpoint{11.904000in}{6.178480in}}%
\pgfusepath{clip}%
\pgfsetrectcap%
\pgfsetroundjoin%
\pgfsetlinewidth{1.003750pt}%
\definecolor{currentstroke}{rgb}{0.000000,0.000000,0.000000}%
\pgfsetstrokecolor{currentstroke}%
\pgfsetdash{}{0pt}%
\pgfpathmoveto{\pgfqpoint{8.105292in}{1.125857in}}%
\pgfpathlineto{\pgfqpoint{8.482618in}{1.125857in}}%
\pgfpathlineto{\pgfqpoint{8.482618in}{1.645937in}}%
\pgfpathlineto{\pgfqpoint{8.105292in}{1.645937in}}%
\pgfpathlineto{\pgfqpoint{8.105292in}{1.125857in}}%
\pgfusepath{stroke}%
\end{pgfscope}%
\begin{pgfscope}%
\pgfpathrectangle{\pgfqpoint{1.920000in}{0.882640in}}{\pgfqpoint{11.904000in}{6.178480in}}%
\pgfusepath{clip}%
\pgfsetrectcap%
\pgfsetroundjoin%
\pgfsetlinewidth{1.003750pt}%
\definecolor{currentstroke}{rgb}{0.000000,0.000000,0.000000}%
\pgfsetstrokecolor{currentstroke}%
\pgfsetdash{}{0pt}%
\pgfpathmoveto{\pgfqpoint{8.293955in}{1.125857in}}%
\pgfpathlineto{\pgfqpoint{8.293955in}{1.018757in}}%
\pgfusepath{stroke}%
\end{pgfscope}%
\begin{pgfscope}%
\pgfpathrectangle{\pgfqpoint{1.920000in}{0.882640in}}{\pgfqpoint{11.904000in}{6.178480in}}%
\pgfusepath{clip}%
\pgfsetrectcap%
\pgfsetroundjoin%
\pgfsetlinewidth{1.003750pt}%
\definecolor{currentstroke}{rgb}{0.000000,0.000000,0.000000}%
\pgfsetstrokecolor{currentstroke}%
\pgfsetdash{}{0pt}%
\pgfpathmoveto{\pgfqpoint{8.293955in}{1.645937in}}%
\pgfpathlineto{\pgfqpoint{8.293955in}{2.419117in}}%
\pgfusepath{stroke}%
\end{pgfscope}%
\begin{pgfscope}%
\pgfpathrectangle{\pgfqpoint{1.920000in}{0.882640in}}{\pgfqpoint{11.904000in}{6.178480in}}%
\pgfusepath{clip}%
\pgfsetrectcap%
\pgfsetroundjoin%
\pgfsetlinewidth{1.003750pt}%
\definecolor{currentstroke}{rgb}{0.000000,0.000000,0.000000}%
\pgfsetstrokecolor{currentstroke}%
\pgfsetdash{}{0pt}%
\pgfpathmoveto{\pgfqpoint{8.199624in}{1.018757in}}%
\pgfpathlineto{\pgfqpoint{8.388286in}{1.018757in}}%
\pgfusepath{stroke}%
\end{pgfscope}%
\begin{pgfscope}%
\pgfpathrectangle{\pgfqpoint{1.920000in}{0.882640in}}{\pgfqpoint{11.904000in}{6.178480in}}%
\pgfusepath{clip}%
\pgfsetrectcap%
\pgfsetroundjoin%
\pgfsetlinewidth{1.003750pt}%
\definecolor{currentstroke}{rgb}{0.000000,0.000000,0.000000}%
\pgfsetstrokecolor{currentstroke}%
\pgfsetdash{}{0pt}%
\pgfpathmoveto{\pgfqpoint{8.199624in}{2.419117in}}%
\pgfpathlineto{\pgfqpoint{8.388286in}{2.419117in}}%
\pgfusepath{stroke}%
\end{pgfscope}%
\begin{pgfscope}%
\pgfpathrectangle{\pgfqpoint{1.920000in}{0.882640in}}{\pgfqpoint{11.904000in}{6.178480in}}%
\pgfusepath{clip}%
\pgfsetbuttcap%
\pgfsetroundjoin%
\definecolor{currentfill}{rgb}{0.000000,0.000000,0.000000}%
\pgfsetfillcolor{currentfill}%
\pgfsetfillopacity{0.000000}%
\pgfsetlinewidth{1.003750pt}%
\definecolor{currentstroke}{rgb}{0.000000,0.000000,0.000000}%
\pgfsetstrokecolor{currentstroke}%
\pgfsetdash{}{0pt}%
\pgfsys@defobject{currentmarker}{\pgfqpoint{-0.041667in}{-0.041667in}}{\pgfqpoint{0.041667in}{0.041667in}}{%
\pgfpathmoveto{\pgfqpoint{0.000000in}{-0.041667in}}%
\pgfpathcurveto{\pgfqpoint{0.011050in}{-0.041667in}}{\pgfqpoint{0.021649in}{-0.037276in}}{\pgfqpoint{0.029463in}{-0.029463in}}%
\pgfpathcurveto{\pgfqpoint{0.037276in}{-0.021649in}}{\pgfqpoint{0.041667in}{-0.011050in}}{\pgfqpoint{0.041667in}{0.000000in}}%
\pgfpathcurveto{\pgfqpoint{0.041667in}{0.011050in}}{\pgfqpoint{0.037276in}{0.021649in}}{\pgfqpoint{0.029463in}{0.029463in}}%
\pgfpathcurveto{\pgfqpoint{0.021649in}{0.037276in}}{\pgfqpoint{0.011050in}{0.041667in}}{\pgfqpoint{0.000000in}{0.041667in}}%
\pgfpathcurveto{\pgfqpoint{-0.011050in}{0.041667in}}{\pgfqpoint{-0.021649in}{0.037276in}}{\pgfqpoint{-0.029463in}{0.029463in}}%
\pgfpathcurveto{\pgfqpoint{-0.037276in}{0.021649in}}{\pgfqpoint{-0.041667in}{0.011050in}}{\pgfqpoint{-0.041667in}{0.000000in}}%
\pgfpathcurveto{\pgfqpoint{-0.041667in}{-0.011050in}}{\pgfqpoint{-0.037276in}{-0.021649in}}{\pgfqpoint{-0.029463in}{-0.029463in}}%
\pgfpathcurveto{\pgfqpoint{-0.021649in}{-0.037276in}}{\pgfqpoint{-0.011050in}{-0.041667in}}{\pgfqpoint{0.000000in}{-0.041667in}}%
\pgfpathlineto{\pgfqpoint{0.000000in}{-0.041667in}}%
\pgfpathclose%
\pgfusepath{stroke,fill}%
}%
\begin{pgfscope}%
\pgfsys@transformshift{8.293955in}{3.873470in}%
\pgfsys@useobject{currentmarker}{}%
\end{pgfscope}%
\begin{pgfscope}%
\pgfsys@transformshift{8.293955in}{3.376216in}%
\pgfsys@useobject{currentmarker}{}%
\end{pgfscope}%
\begin{pgfscope}%
\pgfsys@transformshift{8.293955in}{2.580609in}%
\pgfsys@useobject{currentmarker}{}%
\end{pgfscope}%
\begin{pgfscope}%
\pgfsys@transformshift{8.293955in}{2.586137in}%
\pgfsys@useobject{currentmarker}{}%
\end{pgfscope}%
\begin{pgfscope}%
\pgfsys@transformshift{8.293955in}{2.573718in}%
\pgfsys@useobject{currentmarker}{}%
\end{pgfscope}%
\end{pgfscope}%
\begin{pgfscope}%
\pgfpathrectangle{\pgfqpoint{1.920000in}{0.882640in}}{\pgfqpoint{11.904000in}{6.178480in}}%
\pgfusepath{clip}%
\pgfsetrectcap%
\pgfsetroundjoin%
\pgfsetlinewidth{1.003750pt}%
\definecolor{currentstroke}{rgb}{0.000000,0.000000,0.000000}%
\pgfsetstrokecolor{currentstroke}%
\pgfsetdash{}{0pt}%
\pgfpathmoveto{\pgfqpoint{8.859943in}{1.105590in}}%
\pgfpathlineto{\pgfqpoint{9.237268in}{1.105590in}}%
\pgfpathlineto{\pgfqpoint{9.237268in}{1.597214in}}%
\pgfpathlineto{\pgfqpoint{8.859943in}{1.597214in}}%
\pgfpathlineto{\pgfqpoint{8.859943in}{1.105590in}}%
\pgfusepath{stroke}%
\end{pgfscope}%
\begin{pgfscope}%
\pgfpathrectangle{\pgfqpoint{1.920000in}{0.882640in}}{\pgfqpoint{11.904000in}{6.178480in}}%
\pgfusepath{clip}%
\pgfsetrectcap%
\pgfsetroundjoin%
\pgfsetlinewidth{1.003750pt}%
\definecolor{currentstroke}{rgb}{0.000000,0.000000,0.000000}%
\pgfsetstrokecolor{currentstroke}%
\pgfsetdash{}{0pt}%
\pgfpathmoveto{\pgfqpoint{9.048605in}{1.105590in}}%
\pgfpathlineto{\pgfqpoint{9.048605in}{1.018757in}}%
\pgfusepath{stroke}%
\end{pgfscope}%
\begin{pgfscope}%
\pgfpathrectangle{\pgfqpoint{1.920000in}{0.882640in}}{\pgfqpoint{11.904000in}{6.178480in}}%
\pgfusepath{clip}%
\pgfsetrectcap%
\pgfsetroundjoin%
\pgfsetlinewidth{1.003750pt}%
\definecolor{currentstroke}{rgb}{0.000000,0.000000,0.000000}%
\pgfsetstrokecolor{currentstroke}%
\pgfsetdash{}{0pt}%
\pgfpathmoveto{\pgfqpoint{9.048605in}{1.597214in}}%
\pgfpathlineto{\pgfqpoint{9.048605in}{2.328871in}}%
\pgfusepath{stroke}%
\end{pgfscope}%
\begin{pgfscope}%
\pgfpathrectangle{\pgfqpoint{1.920000in}{0.882640in}}{\pgfqpoint{11.904000in}{6.178480in}}%
\pgfusepath{clip}%
\pgfsetrectcap%
\pgfsetroundjoin%
\pgfsetlinewidth{1.003750pt}%
\definecolor{currentstroke}{rgb}{0.000000,0.000000,0.000000}%
\pgfsetstrokecolor{currentstroke}%
\pgfsetdash{}{0pt}%
\pgfpathmoveto{\pgfqpoint{8.954274in}{1.018757in}}%
\pgfpathlineto{\pgfqpoint{9.142937in}{1.018757in}}%
\pgfusepath{stroke}%
\end{pgfscope}%
\begin{pgfscope}%
\pgfpathrectangle{\pgfqpoint{1.920000in}{0.882640in}}{\pgfqpoint{11.904000in}{6.178480in}}%
\pgfusepath{clip}%
\pgfsetrectcap%
\pgfsetroundjoin%
\pgfsetlinewidth{1.003750pt}%
\definecolor{currentstroke}{rgb}{0.000000,0.000000,0.000000}%
\pgfsetstrokecolor{currentstroke}%
\pgfsetdash{}{0pt}%
\pgfpathmoveto{\pgfqpoint{8.954274in}{2.328871in}}%
\pgfpathlineto{\pgfqpoint{9.142937in}{2.328871in}}%
\pgfusepath{stroke}%
\end{pgfscope}%
\begin{pgfscope}%
\pgfpathrectangle{\pgfqpoint{1.920000in}{0.882640in}}{\pgfqpoint{11.904000in}{6.178480in}}%
\pgfusepath{clip}%
\pgfsetbuttcap%
\pgfsetroundjoin%
\definecolor{currentfill}{rgb}{0.000000,0.000000,0.000000}%
\pgfsetfillcolor{currentfill}%
\pgfsetfillopacity{0.000000}%
\pgfsetlinewidth{1.003750pt}%
\definecolor{currentstroke}{rgb}{0.000000,0.000000,0.000000}%
\pgfsetstrokecolor{currentstroke}%
\pgfsetdash{}{0pt}%
\pgfsys@defobject{currentmarker}{\pgfqpoint{-0.041667in}{-0.041667in}}{\pgfqpoint{0.041667in}{0.041667in}}{%
\pgfpathmoveto{\pgfqpoint{0.000000in}{-0.041667in}}%
\pgfpathcurveto{\pgfqpoint{0.011050in}{-0.041667in}}{\pgfqpoint{0.021649in}{-0.037276in}}{\pgfqpoint{0.029463in}{-0.029463in}}%
\pgfpathcurveto{\pgfqpoint{0.037276in}{-0.021649in}}{\pgfqpoint{0.041667in}{-0.011050in}}{\pgfqpoint{0.041667in}{0.000000in}}%
\pgfpathcurveto{\pgfqpoint{0.041667in}{0.011050in}}{\pgfqpoint{0.037276in}{0.021649in}}{\pgfqpoint{0.029463in}{0.029463in}}%
\pgfpathcurveto{\pgfqpoint{0.021649in}{0.037276in}}{\pgfqpoint{0.011050in}{0.041667in}}{\pgfqpoint{0.000000in}{0.041667in}}%
\pgfpathcurveto{\pgfqpoint{-0.011050in}{0.041667in}}{\pgfqpoint{-0.021649in}{0.037276in}}{\pgfqpoint{-0.029463in}{0.029463in}}%
\pgfpathcurveto{\pgfqpoint{-0.037276in}{0.021649in}}{\pgfqpoint{-0.041667in}{0.011050in}}{\pgfqpoint{-0.041667in}{0.000000in}}%
\pgfpathcurveto{\pgfqpoint{-0.041667in}{-0.011050in}}{\pgfqpoint{-0.037276in}{-0.021649in}}{\pgfqpoint{-0.029463in}{-0.029463in}}%
\pgfpathcurveto{\pgfqpoint{-0.021649in}{-0.037276in}}{\pgfqpoint{-0.011050in}{-0.041667in}}{\pgfqpoint{0.000000in}{-0.041667in}}%
\pgfpathlineto{\pgfqpoint{0.000000in}{-0.041667in}}%
\pgfpathclose%
\pgfusepath{stroke,fill}%
}%
\begin{pgfscope}%
\pgfsys@transformshift{9.048605in}{3.135302in}%
\pgfsys@useobject{currentmarker}{}%
\end{pgfscope}%
\begin{pgfscope}%
\pgfsys@transformshift{9.048605in}{2.626323in}%
\pgfsys@useobject{currentmarker}{}%
\end{pgfscope}%
\begin{pgfscope}%
\pgfsys@transformshift{9.048605in}{5.333249in}%
\pgfsys@useobject{currentmarker}{}%
\end{pgfscope}%
\begin{pgfscope}%
\pgfsys@transformshift{9.048605in}{2.344452in}%
\pgfsys@useobject{currentmarker}{}%
\end{pgfscope}%
\begin{pgfscope}%
\pgfsys@transformshift{9.048605in}{2.523684in}%
\pgfsys@useobject{currentmarker}{}%
\end{pgfscope}%
\end{pgfscope}%
\begin{pgfscope}%
\pgfpathrectangle{\pgfqpoint{1.920000in}{0.882640in}}{\pgfqpoint{11.904000in}{6.178480in}}%
\pgfusepath{clip}%
\pgfsetrectcap%
\pgfsetroundjoin%
\pgfsetlinewidth{1.003750pt}%
\definecolor{currentstroke}{rgb}{0.000000,0.000000,0.000000}%
\pgfsetstrokecolor{currentstroke}%
\pgfsetdash{}{0pt}%
\pgfpathmoveto{\pgfqpoint{9.614593in}{1.116479in}}%
\pgfpathlineto{\pgfqpoint{9.991918in}{1.116479in}}%
\pgfpathlineto{\pgfqpoint{9.991918in}{1.550066in}}%
\pgfpathlineto{\pgfqpoint{9.614593in}{1.550066in}}%
\pgfpathlineto{\pgfqpoint{9.614593in}{1.116479in}}%
\pgfusepath{stroke}%
\end{pgfscope}%
\begin{pgfscope}%
\pgfpathrectangle{\pgfqpoint{1.920000in}{0.882640in}}{\pgfqpoint{11.904000in}{6.178480in}}%
\pgfusepath{clip}%
\pgfsetrectcap%
\pgfsetroundjoin%
\pgfsetlinewidth{1.003750pt}%
\definecolor{currentstroke}{rgb}{0.000000,0.000000,0.000000}%
\pgfsetstrokecolor{currentstroke}%
\pgfsetdash{}{0pt}%
\pgfpathmoveto{\pgfqpoint{9.803256in}{1.116479in}}%
\pgfpathlineto{\pgfqpoint{9.803256in}{1.018757in}}%
\pgfusepath{stroke}%
\end{pgfscope}%
\begin{pgfscope}%
\pgfpathrectangle{\pgfqpoint{1.920000in}{0.882640in}}{\pgfqpoint{11.904000in}{6.178480in}}%
\pgfusepath{clip}%
\pgfsetrectcap%
\pgfsetroundjoin%
\pgfsetlinewidth{1.003750pt}%
\definecolor{currentstroke}{rgb}{0.000000,0.000000,0.000000}%
\pgfsetstrokecolor{currentstroke}%
\pgfsetdash{}{0pt}%
\pgfpathmoveto{\pgfqpoint{9.803256in}{1.550066in}}%
\pgfpathlineto{\pgfqpoint{9.803256in}{2.154472in}}%
\pgfusepath{stroke}%
\end{pgfscope}%
\begin{pgfscope}%
\pgfpathrectangle{\pgfqpoint{1.920000in}{0.882640in}}{\pgfqpoint{11.904000in}{6.178480in}}%
\pgfusepath{clip}%
\pgfsetrectcap%
\pgfsetroundjoin%
\pgfsetlinewidth{1.003750pt}%
\definecolor{currentstroke}{rgb}{0.000000,0.000000,0.000000}%
\pgfsetstrokecolor{currentstroke}%
\pgfsetdash{}{0pt}%
\pgfpathmoveto{\pgfqpoint{9.708924in}{1.018757in}}%
\pgfpathlineto{\pgfqpoint{9.897587in}{1.018757in}}%
\pgfusepath{stroke}%
\end{pgfscope}%
\begin{pgfscope}%
\pgfpathrectangle{\pgfqpoint{1.920000in}{0.882640in}}{\pgfqpoint{11.904000in}{6.178480in}}%
\pgfusepath{clip}%
\pgfsetrectcap%
\pgfsetroundjoin%
\pgfsetlinewidth{1.003750pt}%
\definecolor{currentstroke}{rgb}{0.000000,0.000000,0.000000}%
\pgfsetstrokecolor{currentstroke}%
\pgfsetdash{}{0pt}%
\pgfpathmoveto{\pgfqpoint{9.708924in}{2.154472in}}%
\pgfpathlineto{\pgfqpoint{9.897587in}{2.154472in}}%
\pgfusepath{stroke}%
\end{pgfscope}%
\begin{pgfscope}%
\pgfpathrectangle{\pgfqpoint{1.920000in}{0.882640in}}{\pgfqpoint{11.904000in}{6.178480in}}%
\pgfusepath{clip}%
\pgfsetbuttcap%
\pgfsetroundjoin%
\definecolor{currentfill}{rgb}{0.000000,0.000000,0.000000}%
\pgfsetfillcolor{currentfill}%
\pgfsetfillopacity{0.000000}%
\pgfsetlinewidth{1.003750pt}%
\definecolor{currentstroke}{rgb}{0.000000,0.000000,0.000000}%
\pgfsetstrokecolor{currentstroke}%
\pgfsetdash{}{0pt}%
\pgfsys@defobject{currentmarker}{\pgfqpoint{-0.041667in}{-0.041667in}}{\pgfqpoint{0.041667in}{0.041667in}}{%
\pgfpathmoveto{\pgfqpoint{0.000000in}{-0.041667in}}%
\pgfpathcurveto{\pgfqpoint{0.011050in}{-0.041667in}}{\pgfqpoint{0.021649in}{-0.037276in}}{\pgfqpoint{0.029463in}{-0.029463in}}%
\pgfpathcurveto{\pgfqpoint{0.037276in}{-0.021649in}}{\pgfqpoint{0.041667in}{-0.011050in}}{\pgfqpoint{0.041667in}{0.000000in}}%
\pgfpathcurveto{\pgfqpoint{0.041667in}{0.011050in}}{\pgfqpoint{0.037276in}{0.021649in}}{\pgfqpoint{0.029463in}{0.029463in}}%
\pgfpathcurveto{\pgfqpoint{0.021649in}{0.037276in}}{\pgfqpoint{0.011050in}{0.041667in}}{\pgfqpoint{0.000000in}{0.041667in}}%
\pgfpathcurveto{\pgfqpoint{-0.011050in}{0.041667in}}{\pgfqpoint{-0.021649in}{0.037276in}}{\pgfqpoint{-0.029463in}{0.029463in}}%
\pgfpathcurveto{\pgfqpoint{-0.037276in}{0.021649in}}{\pgfqpoint{-0.041667in}{0.011050in}}{\pgfqpoint{-0.041667in}{0.000000in}}%
\pgfpathcurveto{\pgfqpoint{-0.041667in}{-0.011050in}}{\pgfqpoint{-0.037276in}{-0.021649in}}{\pgfqpoint{-0.029463in}{-0.029463in}}%
\pgfpathcurveto{\pgfqpoint{-0.021649in}{-0.037276in}}{\pgfqpoint{-0.011050in}{-0.041667in}}{\pgfqpoint{0.000000in}{-0.041667in}}%
\pgfpathlineto{\pgfqpoint{0.000000in}{-0.041667in}}%
\pgfpathclose%
\pgfusepath{stroke,fill}%
}%
\begin{pgfscope}%
\pgfsys@transformshift{9.803256in}{3.182251in}%
\pgfsys@useobject{currentmarker}{}%
\end{pgfscope}%
\begin{pgfscope}%
\pgfsys@transformshift{9.803256in}{2.341238in}%
\pgfsys@useobject{currentmarker}{}%
\end{pgfscope}%
\begin{pgfscope}%
\pgfsys@transformshift{9.803256in}{2.590996in}%
\pgfsys@useobject{currentmarker}{}%
\end{pgfscope}%
\begin{pgfscope}%
\pgfsys@transformshift{9.803256in}{5.072126in}%
\pgfsys@useobject{currentmarker}{}%
\end{pgfscope}%
\begin{pgfscope}%
\pgfsys@transformshift{9.803256in}{2.325143in}%
\pgfsys@useobject{currentmarker}{}%
\end{pgfscope}%
\begin{pgfscope}%
\pgfsys@transformshift{9.803256in}{2.298223in}%
\pgfsys@useobject{currentmarker}{}%
\end{pgfscope}%
\begin{pgfscope}%
\pgfsys@transformshift{9.803256in}{2.322546in}%
\pgfsys@useobject{currentmarker}{}%
\end{pgfscope}%
\begin{pgfscope}%
\pgfsys@transformshift{9.803256in}{2.240244in}%
\pgfsys@useobject{currentmarker}{}%
\end{pgfscope}%
\end{pgfscope}%
\begin{pgfscope}%
\pgfpathrectangle{\pgfqpoint{1.920000in}{0.882640in}}{\pgfqpoint{11.904000in}{6.178480in}}%
\pgfusepath{clip}%
\pgfsetrectcap%
\pgfsetroundjoin%
\pgfsetlinewidth{1.003750pt}%
\definecolor{currentstroke}{rgb}{0.000000,0.000000,0.000000}%
\pgfsetstrokecolor{currentstroke}%
\pgfsetdash{}{0pt}%
\pgfpathmoveto{\pgfqpoint{10.369243in}{1.142891in}}%
\pgfpathlineto{\pgfqpoint{10.746569in}{1.142891in}}%
\pgfpathlineto{\pgfqpoint{10.746569in}{1.507282in}}%
\pgfpathlineto{\pgfqpoint{10.369243in}{1.507282in}}%
\pgfpathlineto{\pgfqpoint{10.369243in}{1.142891in}}%
\pgfusepath{stroke}%
\end{pgfscope}%
\begin{pgfscope}%
\pgfpathrectangle{\pgfqpoint{1.920000in}{0.882640in}}{\pgfqpoint{11.904000in}{6.178480in}}%
\pgfusepath{clip}%
\pgfsetrectcap%
\pgfsetroundjoin%
\pgfsetlinewidth{1.003750pt}%
\definecolor{currentstroke}{rgb}{0.000000,0.000000,0.000000}%
\pgfsetstrokecolor{currentstroke}%
\pgfsetdash{}{0pt}%
\pgfpathmoveto{\pgfqpoint{10.557906in}{1.142891in}}%
\pgfpathlineto{\pgfqpoint{10.557906in}{1.018757in}}%
\pgfusepath{stroke}%
\end{pgfscope}%
\begin{pgfscope}%
\pgfpathrectangle{\pgfqpoint{1.920000in}{0.882640in}}{\pgfqpoint{11.904000in}{6.178480in}}%
\pgfusepath{clip}%
\pgfsetrectcap%
\pgfsetroundjoin%
\pgfsetlinewidth{1.003750pt}%
\definecolor{currentstroke}{rgb}{0.000000,0.000000,0.000000}%
\pgfsetstrokecolor{currentstroke}%
\pgfsetdash{}{0pt}%
\pgfpathmoveto{\pgfqpoint{10.557906in}{1.507282in}}%
\pgfpathlineto{\pgfqpoint{10.557906in}{1.973928in}}%
\pgfusepath{stroke}%
\end{pgfscope}%
\begin{pgfscope}%
\pgfpathrectangle{\pgfqpoint{1.920000in}{0.882640in}}{\pgfqpoint{11.904000in}{6.178480in}}%
\pgfusepath{clip}%
\pgfsetrectcap%
\pgfsetroundjoin%
\pgfsetlinewidth{1.003750pt}%
\definecolor{currentstroke}{rgb}{0.000000,0.000000,0.000000}%
\pgfsetstrokecolor{currentstroke}%
\pgfsetdash{}{0pt}%
\pgfpathmoveto{\pgfqpoint{10.463575in}{1.018757in}}%
\pgfpathlineto{\pgfqpoint{10.652237in}{1.018757in}}%
\pgfusepath{stroke}%
\end{pgfscope}%
\begin{pgfscope}%
\pgfpathrectangle{\pgfqpoint{1.920000in}{0.882640in}}{\pgfqpoint{11.904000in}{6.178480in}}%
\pgfusepath{clip}%
\pgfsetrectcap%
\pgfsetroundjoin%
\pgfsetlinewidth{1.003750pt}%
\definecolor{currentstroke}{rgb}{0.000000,0.000000,0.000000}%
\pgfsetstrokecolor{currentstroke}%
\pgfsetdash{}{0pt}%
\pgfpathmoveto{\pgfqpoint{10.463575in}{1.973928in}}%
\pgfpathlineto{\pgfqpoint{10.652237in}{1.973928in}}%
\pgfusepath{stroke}%
\end{pgfscope}%
\begin{pgfscope}%
\pgfpathrectangle{\pgfqpoint{1.920000in}{0.882640in}}{\pgfqpoint{11.904000in}{6.178480in}}%
\pgfusepath{clip}%
\pgfsetbuttcap%
\pgfsetroundjoin%
\definecolor{currentfill}{rgb}{0.000000,0.000000,0.000000}%
\pgfsetfillcolor{currentfill}%
\pgfsetfillopacity{0.000000}%
\pgfsetlinewidth{1.003750pt}%
\definecolor{currentstroke}{rgb}{0.000000,0.000000,0.000000}%
\pgfsetstrokecolor{currentstroke}%
\pgfsetdash{}{0pt}%
\pgfsys@defobject{currentmarker}{\pgfqpoint{-0.041667in}{-0.041667in}}{\pgfqpoint{0.041667in}{0.041667in}}{%
\pgfpathmoveto{\pgfqpoint{0.000000in}{-0.041667in}}%
\pgfpathcurveto{\pgfqpoint{0.011050in}{-0.041667in}}{\pgfqpoint{0.021649in}{-0.037276in}}{\pgfqpoint{0.029463in}{-0.029463in}}%
\pgfpathcurveto{\pgfqpoint{0.037276in}{-0.021649in}}{\pgfqpoint{0.041667in}{-0.011050in}}{\pgfqpoint{0.041667in}{0.000000in}}%
\pgfpathcurveto{\pgfqpoint{0.041667in}{0.011050in}}{\pgfqpoint{0.037276in}{0.021649in}}{\pgfqpoint{0.029463in}{0.029463in}}%
\pgfpathcurveto{\pgfqpoint{0.021649in}{0.037276in}}{\pgfqpoint{0.011050in}{0.041667in}}{\pgfqpoint{0.000000in}{0.041667in}}%
\pgfpathcurveto{\pgfqpoint{-0.011050in}{0.041667in}}{\pgfqpoint{-0.021649in}{0.037276in}}{\pgfqpoint{-0.029463in}{0.029463in}}%
\pgfpathcurveto{\pgfqpoint{-0.037276in}{0.021649in}}{\pgfqpoint{-0.041667in}{0.011050in}}{\pgfqpoint{-0.041667in}{0.000000in}}%
\pgfpathcurveto{\pgfqpoint{-0.041667in}{-0.011050in}}{\pgfqpoint{-0.037276in}{-0.021649in}}{\pgfqpoint{-0.029463in}{-0.029463in}}%
\pgfpathcurveto{\pgfqpoint{-0.021649in}{-0.037276in}}{\pgfqpoint{-0.011050in}{-0.041667in}}{\pgfqpoint{0.000000in}{-0.041667in}}%
\pgfpathlineto{\pgfqpoint{0.000000in}{-0.041667in}}%
\pgfpathclose%
\pgfusepath{stroke,fill}%
}%
\begin{pgfscope}%
\pgfsys@transformshift{10.557906in}{3.949524in}%
\pgfsys@useobject{currentmarker}{}%
\end{pgfscope}%
\begin{pgfscope}%
\pgfsys@transformshift{10.557906in}{2.148790in}%
\pgfsys@useobject{currentmarker}{}%
\end{pgfscope}%
\begin{pgfscope}%
\pgfsys@transformshift{10.557906in}{2.102972in}%
\pgfsys@useobject{currentmarker}{}%
\end{pgfscope}%
\begin{pgfscope}%
\pgfsys@transformshift{10.557906in}{2.928507in}%
\pgfsys@useobject{currentmarker}{}%
\end{pgfscope}%
\begin{pgfscope}%
\pgfsys@transformshift{10.557906in}{2.801365in}%
\pgfsys@useobject{currentmarker}{}%
\end{pgfscope}%
\begin{pgfscope}%
\pgfsys@transformshift{10.557906in}{2.067362in}%
\pgfsys@useobject{currentmarker}{}%
\end{pgfscope}%
\begin{pgfscope}%
\pgfsys@transformshift{10.557906in}{2.225974in}%
\pgfsys@useobject{currentmarker}{}%
\end{pgfscope}%
\begin{pgfscope}%
\pgfsys@transformshift{10.557906in}{2.082712in}%
\pgfsys@useobject{currentmarker}{}%
\end{pgfscope}%
\end{pgfscope}%
\begin{pgfscope}%
\pgfpathrectangle{\pgfqpoint{1.920000in}{0.882640in}}{\pgfqpoint{11.904000in}{6.178480in}}%
\pgfusepath{clip}%
\pgfsetrectcap%
\pgfsetroundjoin%
\pgfsetlinewidth{1.003750pt}%
\definecolor{currentstroke}{rgb}{0.000000,0.000000,0.000000}%
\pgfsetstrokecolor{currentstroke}%
\pgfsetdash{}{0pt}%
\pgfpathmoveto{\pgfqpoint{11.123894in}{1.147589in}}%
\pgfpathlineto{\pgfqpoint{11.501219in}{1.147589in}}%
\pgfpathlineto{\pgfqpoint{11.501219in}{1.514649in}}%
\pgfpathlineto{\pgfqpoint{11.123894in}{1.514649in}}%
\pgfpathlineto{\pgfqpoint{11.123894in}{1.147589in}}%
\pgfusepath{stroke}%
\end{pgfscope}%
\begin{pgfscope}%
\pgfpathrectangle{\pgfqpoint{1.920000in}{0.882640in}}{\pgfqpoint{11.904000in}{6.178480in}}%
\pgfusepath{clip}%
\pgfsetrectcap%
\pgfsetroundjoin%
\pgfsetlinewidth{1.003750pt}%
\definecolor{currentstroke}{rgb}{0.000000,0.000000,0.000000}%
\pgfsetstrokecolor{currentstroke}%
\pgfsetdash{}{0pt}%
\pgfpathmoveto{\pgfqpoint{11.312556in}{1.147589in}}%
\pgfpathlineto{\pgfqpoint{11.312556in}{1.018757in}}%
\pgfusepath{stroke}%
\end{pgfscope}%
\begin{pgfscope}%
\pgfpathrectangle{\pgfqpoint{1.920000in}{0.882640in}}{\pgfqpoint{11.904000in}{6.178480in}}%
\pgfusepath{clip}%
\pgfsetrectcap%
\pgfsetroundjoin%
\pgfsetlinewidth{1.003750pt}%
\definecolor{currentstroke}{rgb}{0.000000,0.000000,0.000000}%
\pgfsetstrokecolor{currentstroke}%
\pgfsetdash{}{0pt}%
\pgfpathmoveto{\pgfqpoint{11.312556in}{1.514649in}}%
\pgfpathlineto{\pgfqpoint{11.312556in}{2.046510in}}%
\pgfusepath{stroke}%
\end{pgfscope}%
\begin{pgfscope}%
\pgfpathrectangle{\pgfqpoint{1.920000in}{0.882640in}}{\pgfqpoint{11.904000in}{6.178480in}}%
\pgfusepath{clip}%
\pgfsetrectcap%
\pgfsetroundjoin%
\pgfsetlinewidth{1.003750pt}%
\definecolor{currentstroke}{rgb}{0.000000,0.000000,0.000000}%
\pgfsetstrokecolor{currentstroke}%
\pgfsetdash{}{0pt}%
\pgfpathmoveto{\pgfqpoint{11.218225in}{1.018757in}}%
\pgfpathlineto{\pgfqpoint{11.406888in}{1.018757in}}%
\pgfusepath{stroke}%
\end{pgfscope}%
\begin{pgfscope}%
\pgfpathrectangle{\pgfqpoint{1.920000in}{0.882640in}}{\pgfqpoint{11.904000in}{6.178480in}}%
\pgfusepath{clip}%
\pgfsetrectcap%
\pgfsetroundjoin%
\pgfsetlinewidth{1.003750pt}%
\definecolor{currentstroke}{rgb}{0.000000,0.000000,0.000000}%
\pgfsetstrokecolor{currentstroke}%
\pgfsetdash{}{0pt}%
\pgfpathmoveto{\pgfqpoint{11.218225in}{2.046510in}}%
\pgfpathlineto{\pgfqpoint{11.406888in}{2.046510in}}%
\pgfusepath{stroke}%
\end{pgfscope}%
\begin{pgfscope}%
\pgfpathrectangle{\pgfqpoint{1.920000in}{0.882640in}}{\pgfqpoint{11.904000in}{6.178480in}}%
\pgfusepath{clip}%
\pgfsetbuttcap%
\pgfsetroundjoin%
\definecolor{currentfill}{rgb}{0.000000,0.000000,0.000000}%
\pgfsetfillcolor{currentfill}%
\pgfsetfillopacity{0.000000}%
\pgfsetlinewidth{1.003750pt}%
\definecolor{currentstroke}{rgb}{0.000000,0.000000,0.000000}%
\pgfsetstrokecolor{currentstroke}%
\pgfsetdash{}{0pt}%
\pgfsys@defobject{currentmarker}{\pgfqpoint{-0.041667in}{-0.041667in}}{\pgfqpoint{0.041667in}{0.041667in}}{%
\pgfpathmoveto{\pgfqpoint{0.000000in}{-0.041667in}}%
\pgfpathcurveto{\pgfqpoint{0.011050in}{-0.041667in}}{\pgfqpoint{0.021649in}{-0.037276in}}{\pgfqpoint{0.029463in}{-0.029463in}}%
\pgfpathcurveto{\pgfqpoint{0.037276in}{-0.021649in}}{\pgfqpoint{0.041667in}{-0.011050in}}{\pgfqpoint{0.041667in}{0.000000in}}%
\pgfpathcurveto{\pgfqpoint{0.041667in}{0.011050in}}{\pgfqpoint{0.037276in}{0.021649in}}{\pgfqpoint{0.029463in}{0.029463in}}%
\pgfpathcurveto{\pgfqpoint{0.021649in}{0.037276in}}{\pgfqpoint{0.011050in}{0.041667in}}{\pgfqpoint{0.000000in}{0.041667in}}%
\pgfpathcurveto{\pgfqpoint{-0.011050in}{0.041667in}}{\pgfqpoint{-0.021649in}{0.037276in}}{\pgfqpoint{-0.029463in}{0.029463in}}%
\pgfpathcurveto{\pgfqpoint{-0.037276in}{0.021649in}}{\pgfqpoint{-0.041667in}{0.011050in}}{\pgfqpoint{-0.041667in}{0.000000in}}%
\pgfpathcurveto{\pgfqpoint{-0.041667in}{-0.011050in}}{\pgfqpoint{-0.037276in}{-0.021649in}}{\pgfqpoint{-0.029463in}{-0.029463in}}%
\pgfpathcurveto{\pgfqpoint{-0.021649in}{-0.037276in}}{\pgfqpoint{-0.011050in}{-0.041667in}}{\pgfqpoint{0.000000in}{-0.041667in}}%
\pgfpathlineto{\pgfqpoint{0.000000in}{-0.041667in}}%
\pgfpathclose%
\pgfusepath{stroke,fill}%
}%
\begin{pgfscope}%
\pgfsys@transformshift{11.312556in}{3.465100in}%
\pgfsys@useobject{currentmarker}{}%
\end{pgfscope}%
\begin{pgfscope}%
\pgfsys@transformshift{11.312556in}{2.190133in}%
\pgfsys@useobject{currentmarker}{}%
\end{pgfscope}%
\begin{pgfscope}%
\pgfsys@transformshift{11.312556in}{2.298583in}%
\pgfsys@useobject{currentmarker}{}%
\end{pgfscope}%
\begin{pgfscope}%
\pgfsys@transformshift{11.312556in}{2.819697in}%
\pgfsys@useobject{currentmarker}{}%
\end{pgfscope}%
\begin{pgfscope}%
\pgfsys@transformshift{11.312556in}{2.726237in}%
\pgfsys@useobject{currentmarker}{}%
\end{pgfscope}%
\end{pgfscope}%
\begin{pgfscope}%
\pgfpathrectangle{\pgfqpoint{1.920000in}{0.882640in}}{\pgfqpoint{11.904000in}{6.178480in}}%
\pgfusepath{clip}%
\pgfsetrectcap%
\pgfsetroundjoin%
\pgfsetlinewidth{1.003750pt}%
\definecolor{currentstroke}{rgb}{0.000000,0.000000,0.000000}%
\pgfsetstrokecolor{currentstroke}%
\pgfsetdash{}{0pt}%
\pgfpathmoveto{\pgfqpoint{11.878544in}{1.119738in}}%
\pgfpathlineto{\pgfqpoint{12.255869in}{1.119738in}}%
\pgfpathlineto{\pgfqpoint{12.255869in}{1.411091in}}%
\pgfpathlineto{\pgfqpoint{11.878544in}{1.411091in}}%
\pgfpathlineto{\pgfqpoint{11.878544in}{1.119738in}}%
\pgfusepath{stroke}%
\end{pgfscope}%
\begin{pgfscope}%
\pgfpathrectangle{\pgfqpoint{1.920000in}{0.882640in}}{\pgfqpoint{11.904000in}{6.178480in}}%
\pgfusepath{clip}%
\pgfsetrectcap%
\pgfsetroundjoin%
\pgfsetlinewidth{1.003750pt}%
\definecolor{currentstroke}{rgb}{0.000000,0.000000,0.000000}%
\pgfsetstrokecolor{currentstroke}%
\pgfsetdash{}{0pt}%
\pgfpathmoveto{\pgfqpoint{12.067207in}{1.119738in}}%
\pgfpathlineto{\pgfqpoint{12.067207in}{1.018757in}}%
\pgfusepath{stroke}%
\end{pgfscope}%
\begin{pgfscope}%
\pgfpathrectangle{\pgfqpoint{1.920000in}{0.882640in}}{\pgfqpoint{11.904000in}{6.178480in}}%
\pgfusepath{clip}%
\pgfsetrectcap%
\pgfsetroundjoin%
\pgfsetlinewidth{1.003750pt}%
\definecolor{currentstroke}{rgb}{0.000000,0.000000,0.000000}%
\pgfsetstrokecolor{currentstroke}%
\pgfsetdash{}{0pt}%
\pgfpathmoveto{\pgfqpoint{12.067207in}{1.411091in}}%
\pgfpathlineto{\pgfqpoint{12.067207in}{1.838018in}}%
\pgfusepath{stroke}%
\end{pgfscope}%
\begin{pgfscope}%
\pgfpathrectangle{\pgfqpoint{1.920000in}{0.882640in}}{\pgfqpoint{11.904000in}{6.178480in}}%
\pgfusepath{clip}%
\pgfsetrectcap%
\pgfsetroundjoin%
\pgfsetlinewidth{1.003750pt}%
\definecolor{currentstroke}{rgb}{0.000000,0.000000,0.000000}%
\pgfsetstrokecolor{currentstroke}%
\pgfsetdash{}{0pt}%
\pgfpathmoveto{\pgfqpoint{11.972875in}{1.018757in}}%
\pgfpathlineto{\pgfqpoint{12.161538in}{1.018757in}}%
\pgfusepath{stroke}%
\end{pgfscope}%
\begin{pgfscope}%
\pgfpathrectangle{\pgfqpoint{1.920000in}{0.882640in}}{\pgfqpoint{11.904000in}{6.178480in}}%
\pgfusepath{clip}%
\pgfsetrectcap%
\pgfsetroundjoin%
\pgfsetlinewidth{1.003750pt}%
\definecolor{currentstroke}{rgb}{0.000000,0.000000,0.000000}%
\pgfsetstrokecolor{currentstroke}%
\pgfsetdash{}{0pt}%
\pgfpathmoveto{\pgfqpoint{11.972875in}{1.838018in}}%
\pgfpathlineto{\pgfqpoint{12.161538in}{1.838018in}}%
\pgfusepath{stroke}%
\end{pgfscope}%
\begin{pgfscope}%
\pgfpathrectangle{\pgfqpoint{1.920000in}{0.882640in}}{\pgfqpoint{11.904000in}{6.178480in}}%
\pgfusepath{clip}%
\pgfsetbuttcap%
\pgfsetroundjoin%
\definecolor{currentfill}{rgb}{0.000000,0.000000,0.000000}%
\pgfsetfillcolor{currentfill}%
\pgfsetfillopacity{0.000000}%
\pgfsetlinewidth{1.003750pt}%
\definecolor{currentstroke}{rgb}{0.000000,0.000000,0.000000}%
\pgfsetstrokecolor{currentstroke}%
\pgfsetdash{}{0pt}%
\pgfsys@defobject{currentmarker}{\pgfqpoint{-0.041667in}{-0.041667in}}{\pgfqpoint{0.041667in}{0.041667in}}{%
\pgfpathmoveto{\pgfqpoint{0.000000in}{-0.041667in}}%
\pgfpathcurveto{\pgfqpoint{0.011050in}{-0.041667in}}{\pgfqpoint{0.021649in}{-0.037276in}}{\pgfqpoint{0.029463in}{-0.029463in}}%
\pgfpathcurveto{\pgfqpoint{0.037276in}{-0.021649in}}{\pgfqpoint{0.041667in}{-0.011050in}}{\pgfqpoint{0.041667in}{0.000000in}}%
\pgfpathcurveto{\pgfqpoint{0.041667in}{0.011050in}}{\pgfqpoint{0.037276in}{0.021649in}}{\pgfqpoint{0.029463in}{0.029463in}}%
\pgfpathcurveto{\pgfqpoint{0.021649in}{0.037276in}}{\pgfqpoint{0.011050in}{0.041667in}}{\pgfqpoint{0.000000in}{0.041667in}}%
\pgfpathcurveto{\pgfqpoint{-0.011050in}{0.041667in}}{\pgfqpoint{-0.021649in}{0.037276in}}{\pgfqpoint{-0.029463in}{0.029463in}}%
\pgfpathcurveto{\pgfqpoint{-0.037276in}{0.021649in}}{\pgfqpoint{-0.041667in}{0.011050in}}{\pgfqpoint{-0.041667in}{0.000000in}}%
\pgfpathcurveto{\pgfqpoint{-0.041667in}{-0.011050in}}{\pgfqpoint{-0.037276in}{-0.021649in}}{\pgfqpoint{-0.029463in}{-0.029463in}}%
\pgfpathcurveto{\pgfqpoint{-0.021649in}{-0.037276in}}{\pgfqpoint{-0.011050in}{-0.041667in}}{\pgfqpoint{0.000000in}{-0.041667in}}%
\pgfpathlineto{\pgfqpoint{0.000000in}{-0.041667in}}%
\pgfpathclose%
\pgfusepath{stroke,fill}%
}%
\begin{pgfscope}%
\pgfsys@transformshift{12.067207in}{1.953076in}%
\pgfsys@useobject{currentmarker}{}%
\end{pgfscope}%
\begin{pgfscope}%
\pgfsys@transformshift{12.067207in}{1.901499in}%
\pgfsys@useobject{currentmarker}{}%
\end{pgfscope}%
\begin{pgfscope}%
\pgfsys@transformshift{12.067207in}{1.928805in}%
\pgfsys@useobject{currentmarker}{}%
\end{pgfscope}%
\begin{pgfscope}%
\pgfsys@transformshift{12.067207in}{2.208208in}%
\pgfsys@useobject{currentmarker}{}%
\end{pgfscope}%
\begin{pgfscope}%
\pgfsys@transformshift{12.067207in}{1.932353in}%
\pgfsys@useobject{currentmarker}{}%
\end{pgfscope}%
\begin{pgfscope}%
\pgfsys@transformshift{12.067207in}{2.086414in}%
\pgfsys@useobject{currentmarker}{}%
\end{pgfscope}%
\begin{pgfscope}%
\pgfsys@transformshift{12.067207in}{2.179926in}%
\pgfsys@useobject{currentmarker}{}%
\end{pgfscope}%
\begin{pgfscope}%
\pgfsys@transformshift{12.067207in}{2.087365in}%
\pgfsys@useobject{currentmarker}{}%
\end{pgfscope}%
\end{pgfscope}%
\begin{pgfscope}%
\pgfpathrectangle{\pgfqpoint{1.920000in}{0.882640in}}{\pgfqpoint{11.904000in}{6.178480in}}%
\pgfusepath{clip}%
\pgfsetrectcap%
\pgfsetroundjoin%
\pgfsetlinewidth{1.003750pt}%
\definecolor{currentstroke}{rgb}{0.000000,0.000000,0.000000}%
\pgfsetstrokecolor{currentstroke}%
\pgfsetdash{}{0pt}%
\pgfpathmoveto{\pgfqpoint{12.633194in}{1.100654in}}%
\pgfpathlineto{\pgfqpoint{13.010519in}{1.100654in}}%
\pgfpathlineto{\pgfqpoint{13.010519in}{1.288975in}}%
\pgfpathlineto{\pgfqpoint{12.633194in}{1.288975in}}%
\pgfpathlineto{\pgfqpoint{12.633194in}{1.100654in}}%
\pgfusepath{stroke}%
\end{pgfscope}%
\begin{pgfscope}%
\pgfpathrectangle{\pgfqpoint{1.920000in}{0.882640in}}{\pgfqpoint{11.904000in}{6.178480in}}%
\pgfusepath{clip}%
\pgfsetrectcap%
\pgfsetroundjoin%
\pgfsetlinewidth{1.003750pt}%
\definecolor{currentstroke}{rgb}{0.000000,0.000000,0.000000}%
\pgfsetstrokecolor{currentstroke}%
\pgfsetdash{}{0pt}%
\pgfpathmoveto{\pgfqpoint{12.821857in}{1.100654in}}%
\pgfpathlineto{\pgfqpoint{12.821857in}{1.029761in}}%
\pgfusepath{stroke}%
\end{pgfscope}%
\begin{pgfscope}%
\pgfpathrectangle{\pgfqpoint{1.920000in}{0.882640in}}{\pgfqpoint{11.904000in}{6.178480in}}%
\pgfusepath{clip}%
\pgfsetrectcap%
\pgfsetroundjoin%
\pgfsetlinewidth{1.003750pt}%
\definecolor{currentstroke}{rgb}{0.000000,0.000000,0.000000}%
\pgfsetstrokecolor{currentstroke}%
\pgfsetdash{}{0pt}%
\pgfpathmoveto{\pgfqpoint{12.821857in}{1.288975in}}%
\pgfpathlineto{\pgfqpoint{12.821857in}{1.549924in}}%
\pgfusepath{stroke}%
\end{pgfscope}%
\begin{pgfscope}%
\pgfpathrectangle{\pgfqpoint{1.920000in}{0.882640in}}{\pgfqpoint{11.904000in}{6.178480in}}%
\pgfusepath{clip}%
\pgfsetrectcap%
\pgfsetroundjoin%
\pgfsetlinewidth{1.003750pt}%
\definecolor{currentstroke}{rgb}{0.000000,0.000000,0.000000}%
\pgfsetstrokecolor{currentstroke}%
\pgfsetdash{}{0pt}%
\pgfpathmoveto{\pgfqpoint{12.727526in}{1.029761in}}%
\pgfpathlineto{\pgfqpoint{12.916188in}{1.029761in}}%
\pgfusepath{stroke}%
\end{pgfscope}%
\begin{pgfscope}%
\pgfpathrectangle{\pgfqpoint{1.920000in}{0.882640in}}{\pgfqpoint{11.904000in}{6.178480in}}%
\pgfusepath{clip}%
\pgfsetrectcap%
\pgfsetroundjoin%
\pgfsetlinewidth{1.003750pt}%
\definecolor{currentstroke}{rgb}{0.000000,0.000000,0.000000}%
\pgfsetstrokecolor{currentstroke}%
\pgfsetdash{}{0pt}%
\pgfpathmoveto{\pgfqpoint{12.727526in}{1.549924in}}%
\pgfpathlineto{\pgfqpoint{12.916188in}{1.549924in}}%
\pgfusepath{stroke}%
\end{pgfscope}%
\begin{pgfscope}%
\pgfpathrectangle{\pgfqpoint{1.920000in}{0.882640in}}{\pgfqpoint{11.904000in}{6.178480in}}%
\pgfusepath{clip}%
\pgfsetbuttcap%
\pgfsetroundjoin%
\definecolor{currentfill}{rgb}{0.000000,0.000000,0.000000}%
\pgfsetfillcolor{currentfill}%
\pgfsetfillopacity{0.000000}%
\pgfsetlinewidth{1.003750pt}%
\definecolor{currentstroke}{rgb}{0.000000,0.000000,0.000000}%
\pgfsetstrokecolor{currentstroke}%
\pgfsetdash{}{0pt}%
\pgfsys@defobject{currentmarker}{\pgfqpoint{-0.041667in}{-0.041667in}}{\pgfqpoint{0.041667in}{0.041667in}}{%
\pgfpathmoveto{\pgfqpoint{0.000000in}{-0.041667in}}%
\pgfpathcurveto{\pgfqpoint{0.011050in}{-0.041667in}}{\pgfqpoint{0.021649in}{-0.037276in}}{\pgfqpoint{0.029463in}{-0.029463in}}%
\pgfpathcurveto{\pgfqpoint{0.037276in}{-0.021649in}}{\pgfqpoint{0.041667in}{-0.011050in}}{\pgfqpoint{0.041667in}{0.000000in}}%
\pgfpathcurveto{\pgfqpoint{0.041667in}{0.011050in}}{\pgfqpoint{0.037276in}{0.021649in}}{\pgfqpoint{0.029463in}{0.029463in}}%
\pgfpathcurveto{\pgfqpoint{0.021649in}{0.037276in}}{\pgfqpoint{0.011050in}{0.041667in}}{\pgfqpoint{0.000000in}{0.041667in}}%
\pgfpathcurveto{\pgfqpoint{-0.011050in}{0.041667in}}{\pgfqpoint{-0.021649in}{0.037276in}}{\pgfqpoint{-0.029463in}{0.029463in}}%
\pgfpathcurveto{\pgfqpoint{-0.037276in}{0.021649in}}{\pgfqpoint{-0.041667in}{0.011050in}}{\pgfqpoint{-0.041667in}{0.000000in}}%
\pgfpathcurveto{\pgfqpoint{-0.041667in}{-0.011050in}}{\pgfqpoint{-0.037276in}{-0.021649in}}{\pgfqpoint{-0.029463in}{-0.029463in}}%
\pgfpathcurveto{\pgfqpoint{-0.021649in}{-0.037276in}}{\pgfqpoint{-0.011050in}{-0.041667in}}{\pgfqpoint{0.000000in}{-0.041667in}}%
\pgfpathlineto{\pgfqpoint{0.000000in}{-0.041667in}}%
\pgfpathclose%
\pgfusepath{stroke,fill}%
}%
\begin{pgfscope}%
\pgfsys@transformshift{12.821857in}{1.787727in}%
\pgfsys@useobject{currentmarker}{}%
\end{pgfscope}%
\begin{pgfscope}%
\pgfsys@transformshift{12.821857in}{1.660945in}%
\pgfsys@useobject{currentmarker}{}%
\end{pgfscope}%
\begin{pgfscope}%
\pgfsys@transformshift{12.821857in}{1.814441in}%
\pgfsys@useobject{currentmarker}{}%
\end{pgfscope}%
\begin{pgfscope}%
\pgfsys@transformshift{12.821857in}{1.884813in}%
\pgfsys@useobject{currentmarker}{}%
\end{pgfscope}%
\begin{pgfscope}%
\pgfsys@transformshift{12.821857in}{1.605049in}%
\pgfsys@useobject{currentmarker}{}%
\end{pgfscope}%
\begin{pgfscope}%
\pgfsys@transformshift{12.821857in}{1.621684in}%
\pgfsys@useobject{currentmarker}{}%
\end{pgfscope}%
\end{pgfscope}%
\begin{pgfscope}%
\pgfpathrectangle{\pgfqpoint{1.920000in}{0.882640in}}{\pgfqpoint{11.904000in}{6.178480in}}%
\pgfusepath{clip}%
\pgfsetrectcap%
\pgfsetroundjoin%
\pgfsetlinewidth{1.003750pt}%
\definecolor{currentstroke}{rgb}{0.000000,0.000000,0.000000}%
\pgfsetstrokecolor{currentstroke}%
\pgfsetdash{}{0pt}%
\pgfpathmoveto{\pgfqpoint{13.387845in}{1.061020in}}%
\pgfpathlineto{\pgfqpoint{13.765170in}{1.061020in}}%
\pgfpathlineto{\pgfqpoint{13.765170in}{1.164719in}}%
\pgfpathlineto{\pgfqpoint{13.387845in}{1.164719in}}%
\pgfpathlineto{\pgfqpoint{13.387845in}{1.061020in}}%
\pgfusepath{stroke}%
\end{pgfscope}%
\begin{pgfscope}%
\pgfpathrectangle{\pgfqpoint{1.920000in}{0.882640in}}{\pgfqpoint{11.904000in}{6.178480in}}%
\pgfusepath{clip}%
\pgfsetrectcap%
\pgfsetroundjoin%
\pgfsetlinewidth{1.003750pt}%
\definecolor{currentstroke}{rgb}{0.000000,0.000000,0.000000}%
\pgfsetstrokecolor{currentstroke}%
\pgfsetdash{}{0pt}%
\pgfpathmoveto{\pgfqpoint{13.576507in}{1.061020in}}%
\pgfpathlineto{\pgfqpoint{13.576507in}{1.018757in}}%
\pgfusepath{stroke}%
\end{pgfscope}%
\begin{pgfscope}%
\pgfpathrectangle{\pgfqpoint{1.920000in}{0.882640in}}{\pgfqpoint{11.904000in}{6.178480in}}%
\pgfusepath{clip}%
\pgfsetrectcap%
\pgfsetroundjoin%
\pgfsetlinewidth{1.003750pt}%
\definecolor{currentstroke}{rgb}{0.000000,0.000000,0.000000}%
\pgfsetstrokecolor{currentstroke}%
\pgfsetdash{}{0pt}%
\pgfpathmoveto{\pgfqpoint{13.576507in}{1.164719in}}%
\pgfpathlineto{\pgfqpoint{13.576507in}{1.284302in}}%
\pgfusepath{stroke}%
\end{pgfscope}%
\begin{pgfscope}%
\pgfpathrectangle{\pgfqpoint{1.920000in}{0.882640in}}{\pgfqpoint{11.904000in}{6.178480in}}%
\pgfusepath{clip}%
\pgfsetrectcap%
\pgfsetroundjoin%
\pgfsetlinewidth{1.003750pt}%
\definecolor{currentstroke}{rgb}{0.000000,0.000000,0.000000}%
\pgfsetstrokecolor{currentstroke}%
\pgfsetdash{}{0pt}%
\pgfpathmoveto{\pgfqpoint{13.482176in}{1.018757in}}%
\pgfpathlineto{\pgfqpoint{13.670838in}{1.018757in}}%
\pgfusepath{stroke}%
\end{pgfscope}%
\begin{pgfscope}%
\pgfpathrectangle{\pgfqpoint{1.920000in}{0.882640in}}{\pgfqpoint{11.904000in}{6.178480in}}%
\pgfusepath{clip}%
\pgfsetrectcap%
\pgfsetroundjoin%
\pgfsetlinewidth{1.003750pt}%
\definecolor{currentstroke}{rgb}{0.000000,0.000000,0.000000}%
\pgfsetstrokecolor{currentstroke}%
\pgfsetdash{}{0pt}%
\pgfpathmoveto{\pgfqpoint{13.482176in}{1.284302in}}%
\pgfpathlineto{\pgfqpoint{13.670838in}{1.284302in}}%
\pgfusepath{stroke}%
\end{pgfscope}%
\begin{pgfscope}%
\pgfpathrectangle{\pgfqpoint{1.920000in}{0.882640in}}{\pgfqpoint{11.904000in}{6.178480in}}%
\pgfusepath{clip}%
\pgfsetbuttcap%
\pgfsetroundjoin%
\definecolor{currentfill}{rgb}{0.000000,0.000000,0.000000}%
\pgfsetfillcolor{currentfill}%
\pgfsetfillopacity{0.000000}%
\pgfsetlinewidth{1.003750pt}%
\definecolor{currentstroke}{rgb}{0.000000,0.000000,0.000000}%
\pgfsetstrokecolor{currentstroke}%
\pgfsetdash{}{0pt}%
\pgfsys@defobject{currentmarker}{\pgfqpoint{-0.041667in}{-0.041667in}}{\pgfqpoint{0.041667in}{0.041667in}}{%
\pgfpathmoveto{\pgfqpoint{0.000000in}{-0.041667in}}%
\pgfpathcurveto{\pgfqpoint{0.011050in}{-0.041667in}}{\pgfqpoint{0.021649in}{-0.037276in}}{\pgfqpoint{0.029463in}{-0.029463in}}%
\pgfpathcurveto{\pgfqpoint{0.037276in}{-0.021649in}}{\pgfqpoint{0.041667in}{-0.011050in}}{\pgfqpoint{0.041667in}{0.000000in}}%
\pgfpathcurveto{\pgfqpoint{0.041667in}{0.011050in}}{\pgfqpoint{0.037276in}{0.021649in}}{\pgfqpoint{0.029463in}{0.029463in}}%
\pgfpathcurveto{\pgfqpoint{0.021649in}{0.037276in}}{\pgfqpoint{0.011050in}{0.041667in}}{\pgfqpoint{0.000000in}{0.041667in}}%
\pgfpathcurveto{\pgfqpoint{-0.011050in}{0.041667in}}{\pgfqpoint{-0.021649in}{0.037276in}}{\pgfqpoint{-0.029463in}{0.029463in}}%
\pgfpathcurveto{\pgfqpoint{-0.037276in}{0.021649in}}{\pgfqpoint{-0.041667in}{0.011050in}}{\pgfqpoint{-0.041667in}{0.000000in}}%
\pgfpathcurveto{\pgfqpoint{-0.041667in}{-0.011050in}}{\pgfqpoint{-0.037276in}{-0.021649in}}{\pgfqpoint{-0.029463in}{-0.029463in}}%
\pgfpathcurveto{\pgfqpoint{-0.021649in}{-0.037276in}}{\pgfqpoint{-0.011050in}{-0.041667in}}{\pgfqpoint{0.000000in}{-0.041667in}}%
\pgfpathlineto{\pgfqpoint{0.000000in}{-0.041667in}}%
\pgfpathclose%
\pgfusepath{stroke,fill}%
}%
\begin{pgfscope}%
\pgfsys@transformshift{13.576507in}{1.520382in}%
\pgfsys@useobject{currentmarker}{}%
\end{pgfscope}%
\begin{pgfscope}%
\pgfsys@transformshift{13.576507in}{1.456310in}%
\pgfsys@useobject{currentmarker}{}%
\end{pgfscope}%
\begin{pgfscope}%
\pgfsys@transformshift{13.576507in}{1.417743in}%
\pgfsys@useobject{currentmarker}{}%
\end{pgfscope}%
\begin{pgfscope}%
\pgfsys@transformshift{13.576507in}{1.490609in}%
\pgfsys@useobject{currentmarker}{}%
\end{pgfscope}%
\begin{pgfscope}%
\pgfsys@transformshift{13.576507in}{1.349146in}%
\pgfsys@useobject{currentmarker}{}%
\end{pgfscope}%
\end{pgfscope}%
\begin{pgfscope}%
\pgfpathrectangle{\pgfqpoint{1.920000in}{0.882640in}}{\pgfqpoint{11.904000in}{6.178480in}}%
\pgfusepath{clip}%
\pgfsetrectcap%
\pgfsetroundjoin%
\pgfsetlinewidth{1.003750pt}%
\definecolor{currentstroke}{rgb}{1.000000,0.498039,0.054902}%
\pgfsetstrokecolor{currentstroke}%
\pgfsetdash{}{0pt}%
\pgfpathmoveto{\pgfqpoint{2.068090in}{1.148419in}}%
\pgfpathlineto{\pgfqpoint{2.445415in}{1.148419in}}%
\pgfusepath{stroke}%
\end{pgfscope}%
\begin{pgfscope}%
\pgfpathrectangle{\pgfqpoint{1.920000in}{0.882640in}}{\pgfqpoint{11.904000in}{6.178480in}}%
\pgfusepath{clip}%
\pgfsetrectcap%
\pgfsetroundjoin%
\pgfsetlinewidth{1.003750pt}%
\definecolor{currentstroke}{rgb}{1.000000,0.498039,0.054902}%
\pgfsetstrokecolor{currentstroke}%
\pgfsetdash{}{0pt}%
\pgfpathmoveto{\pgfqpoint{2.822740in}{1.276589in}}%
\pgfpathlineto{\pgfqpoint{3.200065in}{1.276589in}}%
\pgfusepath{stroke}%
\end{pgfscope}%
\begin{pgfscope}%
\pgfpathrectangle{\pgfqpoint{1.920000in}{0.882640in}}{\pgfqpoint{11.904000in}{6.178480in}}%
\pgfusepath{clip}%
\pgfsetrectcap%
\pgfsetroundjoin%
\pgfsetlinewidth{1.003750pt}%
\definecolor{currentstroke}{rgb}{1.000000,0.498039,0.054902}%
\pgfsetstrokecolor{currentstroke}%
\pgfsetdash{}{0pt}%
\pgfpathmoveto{\pgfqpoint{3.577391in}{1.322998in}}%
\pgfpathlineto{\pgfqpoint{3.954716in}{1.322998in}}%
\pgfusepath{stroke}%
\end{pgfscope}%
\begin{pgfscope}%
\pgfpathrectangle{\pgfqpoint{1.920000in}{0.882640in}}{\pgfqpoint{11.904000in}{6.178480in}}%
\pgfusepath{clip}%
\pgfsetrectcap%
\pgfsetroundjoin%
\pgfsetlinewidth{1.003750pt}%
\definecolor{currentstroke}{rgb}{1.000000,0.498039,0.054902}%
\pgfsetstrokecolor{currentstroke}%
\pgfsetdash{}{0pt}%
\pgfpathmoveto{\pgfqpoint{4.332041in}{1.343284in}}%
\pgfpathlineto{\pgfqpoint{4.709366in}{1.343284in}}%
\pgfusepath{stroke}%
\end{pgfscope}%
\begin{pgfscope}%
\pgfpathrectangle{\pgfqpoint{1.920000in}{0.882640in}}{\pgfqpoint{11.904000in}{6.178480in}}%
\pgfusepath{clip}%
\pgfsetrectcap%
\pgfsetroundjoin%
\pgfsetlinewidth{1.003750pt}%
\definecolor{currentstroke}{rgb}{1.000000,0.498039,0.054902}%
\pgfsetstrokecolor{currentstroke}%
\pgfsetdash{}{0pt}%
\pgfpathmoveto{\pgfqpoint{5.086691in}{1.300102in}}%
\pgfpathlineto{\pgfqpoint{5.464016in}{1.300102in}}%
\pgfusepath{stroke}%
\end{pgfscope}%
\begin{pgfscope}%
\pgfpathrectangle{\pgfqpoint{1.920000in}{0.882640in}}{\pgfqpoint{11.904000in}{6.178480in}}%
\pgfusepath{clip}%
\pgfsetrectcap%
\pgfsetroundjoin%
\pgfsetlinewidth{1.003750pt}%
\definecolor{currentstroke}{rgb}{1.000000,0.498039,0.054902}%
\pgfsetstrokecolor{currentstroke}%
\pgfsetdash{}{0pt}%
\pgfpathmoveto{\pgfqpoint{5.841342in}{1.295358in}}%
\pgfpathlineto{\pgfqpoint{6.218667in}{1.295358in}}%
\pgfusepath{stroke}%
\end{pgfscope}%
\begin{pgfscope}%
\pgfpathrectangle{\pgfqpoint{1.920000in}{0.882640in}}{\pgfqpoint{11.904000in}{6.178480in}}%
\pgfusepath{clip}%
\pgfsetrectcap%
\pgfsetroundjoin%
\pgfsetlinewidth{1.003750pt}%
\definecolor{currentstroke}{rgb}{1.000000,0.498039,0.054902}%
\pgfsetstrokecolor{currentstroke}%
\pgfsetdash{}{0pt}%
\pgfpathmoveto{\pgfqpoint{6.595992in}{1.310695in}}%
\pgfpathlineto{\pgfqpoint{6.973317in}{1.310695in}}%
\pgfusepath{stroke}%
\end{pgfscope}%
\begin{pgfscope}%
\pgfpathrectangle{\pgfqpoint{1.920000in}{0.882640in}}{\pgfqpoint{11.904000in}{6.178480in}}%
\pgfusepath{clip}%
\pgfsetrectcap%
\pgfsetroundjoin%
\pgfsetlinewidth{1.003750pt}%
\definecolor{currentstroke}{rgb}{1.000000,0.498039,0.054902}%
\pgfsetstrokecolor{currentstroke}%
\pgfsetdash{}{0pt}%
\pgfpathmoveto{\pgfqpoint{7.350642in}{1.359520in}}%
\pgfpathlineto{\pgfqpoint{7.727967in}{1.359520in}}%
\pgfusepath{stroke}%
\end{pgfscope}%
\begin{pgfscope}%
\pgfpathrectangle{\pgfqpoint{1.920000in}{0.882640in}}{\pgfqpoint{11.904000in}{6.178480in}}%
\pgfusepath{clip}%
\pgfsetrectcap%
\pgfsetroundjoin%
\pgfsetlinewidth{1.003750pt}%
\definecolor{currentstroke}{rgb}{1.000000,0.498039,0.054902}%
\pgfsetstrokecolor{currentstroke}%
\pgfsetdash{}{0pt}%
\pgfpathmoveto{\pgfqpoint{8.105292in}{1.360883in}}%
\pgfpathlineto{\pgfqpoint{8.482618in}{1.360883in}}%
\pgfusepath{stroke}%
\end{pgfscope}%
\begin{pgfscope}%
\pgfpathrectangle{\pgfqpoint{1.920000in}{0.882640in}}{\pgfqpoint{11.904000in}{6.178480in}}%
\pgfusepath{clip}%
\pgfsetrectcap%
\pgfsetroundjoin%
\pgfsetlinewidth{1.003750pt}%
\definecolor{currentstroke}{rgb}{1.000000,0.498039,0.054902}%
\pgfsetstrokecolor{currentstroke}%
\pgfsetdash{}{0pt}%
\pgfpathmoveto{\pgfqpoint{8.859943in}{1.337601in}}%
\pgfpathlineto{\pgfqpoint{9.237268in}{1.337601in}}%
\pgfusepath{stroke}%
\end{pgfscope}%
\begin{pgfscope}%
\pgfpathrectangle{\pgfqpoint{1.920000in}{0.882640in}}{\pgfqpoint{11.904000in}{6.178480in}}%
\pgfusepath{clip}%
\pgfsetrectcap%
\pgfsetroundjoin%
\pgfsetlinewidth{1.003750pt}%
\definecolor{currentstroke}{rgb}{1.000000,0.498039,0.054902}%
\pgfsetstrokecolor{currentstroke}%
\pgfsetdash{}{0pt}%
\pgfpathmoveto{\pgfqpoint{9.614593in}{1.310900in}}%
\pgfpathlineto{\pgfqpoint{9.991918in}{1.310900in}}%
\pgfusepath{stroke}%
\end{pgfscope}%
\begin{pgfscope}%
\pgfpathrectangle{\pgfqpoint{1.920000in}{0.882640in}}{\pgfqpoint{11.904000in}{6.178480in}}%
\pgfusepath{clip}%
\pgfsetrectcap%
\pgfsetroundjoin%
\pgfsetlinewidth{1.003750pt}%
\definecolor{currentstroke}{rgb}{1.000000,0.498039,0.054902}%
\pgfsetstrokecolor{currentstroke}%
\pgfsetdash{}{0pt}%
\pgfpathmoveto{\pgfqpoint{10.369243in}{1.297492in}}%
\pgfpathlineto{\pgfqpoint{10.746569in}{1.297492in}}%
\pgfusepath{stroke}%
\end{pgfscope}%
\begin{pgfscope}%
\pgfpathrectangle{\pgfqpoint{1.920000in}{0.882640in}}{\pgfqpoint{11.904000in}{6.178480in}}%
\pgfusepath{clip}%
\pgfsetrectcap%
\pgfsetroundjoin%
\pgfsetlinewidth{1.003750pt}%
\definecolor{currentstroke}{rgb}{1.000000,0.498039,0.054902}%
\pgfsetstrokecolor{currentstroke}%
\pgfsetdash{}{0pt}%
\pgfpathmoveto{\pgfqpoint{11.123894in}{1.276627in}}%
\pgfpathlineto{\pgfqpoint{11.501219in}{1.276627in}}%
\pgfusepath{stroke}%
\end{pgfscope}%
\begin{pgfscope}%
\pgfpathrectangle{\pgfqpoint{1.920000in}{0.882640in}}{\pgfqpoint{11.904000in}{6.178480in}}%
\pgfusepath{clip}%
\pgfsetrectcap%
\pgfsetroundjoin%
\pgfsetlinewidth{1.003750pt}%
\definecolor{currentstroke}{rgb}{1.000000,0.498039,0.054902}%
\pgfsetstrokecolor{currentstroke}%
\pgfsetdash{}{0pt}%
\pgfpathmoveto{\pgfqpoint{11.878544in}{1.220783in}}%
\pgfpathlineto{\pgfqpoint{12.255869in}{1.220783in}}%
\pgfusepath{stroke}%
\end{pgfscope}%
\begin{pgfscope}%
\pgfpathrectangle{\pgfqpoint{1.920000in}{0.882640in}}{\pgfqpoint{11.904000in}{6.178480in}}%
\pgfusepath{clip}%
\pgfsetrectcap%
\pgfsetroundjoin%
\pgfsetlinewidth{1.003750pt}%
\definecolor{currentstroke}{rgb}{1.000000,0.498039,0.054902}%
\pgfsetstrokecolor{currentstroke}%
\pgfsetdash{}{0pt}%
\pgfpathmoveto{\pgfqpoint{12.633194in}{1.180776in}}%
\pgfpathlineto{\pgfqpoint{13.010519in}{1.180776in}}%
\pgfusepath{stroke}%
\end{pgfscope}%
\begin{pgfscope}%
\pgfpathrectangle{\pgfqpoint{1.920000in}{0.882640in}}{\pgfqpoint{11.904000in}{6.178480in}}%
\pgfusepath{clip}%
\pgfsetrectcap%
\pgfsetroundjoin%
\pgfsetlinewidth{1.003750pt}%
\definecolor{currentstroke}{rgb}{1.000000,0.498039,0.054902}%
\pgfsetstrokecolor{currentstroke}%
\pgfsetdash{}{0pt}%
\pgfpathmoveto{\pgfqpoint{13.387845in}{1.110700in}}%
\pgfpathlineto{\pgfqpoint{13.765170in}{1.110700in}}%
\pgfusepath{stroke}%
\end{pgfscope}%
\begin{pgfscope}%
\pgfsetrectcap%
\pgfsetmiterjoin%
\pgfsetlinewidth{0.803000pt}%
\definecolor{currentstroke}{rgb}{0.000000,0.000000,0.000000}%
\pgfsetstrokecolor{currentstroke}%
\pgfsetdash{}{0pt}%
\pgfpathmoveto{\pgfqpoint{1.920000in}{0.882640in}}%
\pgfpathlineto{\pgfqpoint{1.920000in}{7.061120in}}%
\pgfusepath{stroke}%
\end{pgfscope}%
\begin{pgfscope}%
\pgfsetrectcap%
\pgfsetmiterjoin%
\pgfsetlinewidth{0.803000pt}%
\definecolor{currentstroke}{rgb}{0.000000,0.000000,0.000000}%
\pgfsetstrokecolor{currentstroke}%
\pgfsetdash{}{0pt}%
\pgfpathmoveto{\pgfqpoint{13.824000in}{0.882640in}}%
\pgfpathlineto{\pgfqpoint{13.824000in}{7.061120in}}%
\pgfusepath{stroke}%
\end{pgfscope}%
\begin{pgfscope}%
\pgfsetrectcap%
\pgfsetmiterjoin%
\pgfsetlinewidth{0.803000pt}%
\definecolor{currentstroke}{rgb}{0.000000,0.000000,0.000000}%
\pgfsetstrokecolor{currentstroke}%
\pgfsetdash{}{0pt}%
\pgfpathmoveto{\pgfqpoint{1.920000in}{0.882640in}}%
\pgfpathlineto{\pgfqpoint{13.824000in}{0.882640in}}%
\pgfusepath{stroke}%
\end{pgfscope}%
\begin{pgfscope}%
\pgfsetrectcap%
\pgfsetmiterjoin%
\pgfsetlinewidth{0.803000pt}%
\definecolor{currentstroke}{rgb}{0.000000,0.000000,0.000000}%
\pgfsetstrokecolor{currentstroke}%
\pgfsetdash{}{0pt}%
\pgfpathmoveto{\pgfqpoint{1.920000in}{7.061120in}}%
\pgfpathlineto{\pgfqpoint{13.824000in}{7.061120in}}%
\pgfusepath{stroke}%
\end{pgfscope}%
\begin{pgfscope}%
\definecolor{textcolor}{rgb}{0.000000,0.000000,0.000000}%
\pgfsetstrokecolor{textcolor}%
\pgfsetfillcolor{textcolor}%
\pgftext[x=7.872000in,y=7.144453in,,base]{\color{textcolor}\sffamily\fontsize{12.000000}{14.400000}\selectfont Diagrama de cajas de JCR}%
\end{pgfscope}%
\end{pgfpicture}%
\makeatother%
\endgroup%


\imagen{diagrama_cajas}{Dispersión entre las diferencias del JCR calculado y el real en función del año}{1}

Al analizar el gráfico, se observa claramente que los años más antiguos presentan un margen de error mucho más elevado debido a la falta de datos de Crossref en esos años. De esta manera, se puede asegurar que el módulo desarrollado es preciso y confiable, aunque se debe tener en cuenta la limitación de la falta de datos para los años más antiguos.

Dadas las circunstancias, se ha tomado la decisión de utilizar solamente los datos de los años más recientes para entrenar los modelos de aprendizaje automático. Esta medida se ha tomado debido a la limitación de datos disponibles para los años más antiguos, que resultaría en un margen de error demasiado elevado si se incluyeran en el entrenamiento de los modelos. Aunque esto pueda suponer una pérdida de información valiosa, se considera que es preferible tener resultados más precisos y confiables al utilizar datos más actualizados. Con esta estrategia, se espera obtener resultados más precisos y útiles.

\subsection{Modelos de predicción}

Se han probado varios modelos regresores utilizando la librería Scikit-learn, con el objetivo de encontrar el que mejor se ajuste a los datos disponibles y pueda hacer predicciones más exactas. Además, estos modelos se han evaluado mediante el error cuadrático medio\footnote{RMSE (Root Mean Square Error) es una métrica ampliamente utilizada para evaluar la precisión o el rendimiento de un modelo de regresión. Representa la raíz cuadrada de la media de los errores cuadrados entre los valores predichos por el modelo y los valores reales del conjunto de datos.}. A continuación, se enumeran los distintos modelos probados:
\begin{itemize}
    \item Linear Regression 
    \item Random Forest Regressor
    \item AdaBoost Regressor
    \item XGB Regressor
    \item Support Vector Machine Regressor (SVM)
    \item Multi Layer Perceptron (MLP)
    \item Stacking Regressor
\end{itemize}

Se han seleccionado estos modelos debido a su popularidad y a su variedad, que abarca una amplia gama de métodos disponibles.

Además se ha tenido en cuenta la relevancia y los hallazgos presentados en el artículo titulado <<Do We Need Hundreds of Classifiers to Solve Real World Classification Problems?>>~\cite{fer2014}. Este estudio evaluó 179 clasificadores provenientes de 17 familias diferentes, abarcando una amplia variedad de métodos, incluyendo análisis discriminante, Bayesianos, redes neuronales, máquinas de soporte vectorial, árboles de decisión, clasificadores basados en reglas, técnicas de \textit{boosting}, \textit{bagging}, \textit{stacking}, modelos lineales generalizados, vecinos más cercanos, regresión de múltiples splines de adaptación y muchos otros métodos. 

Los resultados de este estudio destacaron que los clasificadores de Random Forest fueron los más prometedores, alcanzando una precisión máxima del 94,1\% en los conjuntos de datos utilizados, superando el 90\% en el 84,3\% de los casos. Además, se encontró que los clasificadores SVM con núcleo gaussiano, también obtuvieron una alta precisión del 92,3\%. Estos hallazgos respaldan la decisión de considerar estos modelos en este estudio, ya que se ha demostrado que ofrecen un rendimiento destacado en comparación con otros clasificadores.

Por otro lado, hay otros artículos que sugieren la utilización de los modelos XGBoost y Stacking en proyectos de aprendizaje automático. El artículo titulado <<Getting Started with XGBoost in scikit-learn>>~\cite{Wade2020} destaca que XGBoost es un algoritmo de aprendizaje automático que ha ganado popularidad debido a su desempeño sobresaliente en competiciones de Kaggle y en la predicción de datos tabulares. XGBoost es un modelo de \textit{ensemble} que combina varios modelos de aprendizaje en uno solo, ofreciendo resultados superiores a los modelos individuales. Además, se destaca su capacidad de regularización y velocidad de procesamiento, lo que lo convierte en una opción atractiva para aplicaciones prácticas.

Por su parte, el artículo <<Stacked generalization: an introduction to super learning>>~\cite{naimi2018} presenta el método de \textit{Stacked Generalization}, también conocido como \textit{Super Learner}. Este enfoque permite combinar varios algoritmos de predicción en un único modelo. Utilizando la validación cruzada, se busca obtener una combinación óptima de las predicciones de una biblioteca de algoritmos candidatos. La optimización se realiza mediante una función objetivo especificada por el usuario, como minimizar el error cuadrático medio o maximizar el área bajo la curva característica de operación del receptor. Aunque la implementación de \textit{Super Learner} puede tener ciertas complejidades conceptuales y técnicas, su uso ha demostrado ser beneficioso en diversas aplicaciones y puede ofrecer mejoras significativas en la precisión de las predicciones.

En nuestro caso, tanto XGBoost como Stacking han demostrado ser enfoques efectivos en la predicción del JCR, y su inclusión se justifica por su rendimiento sobresaliente y su potencial para mejorar la precisión de las predicciones en este proyecto.



\subsubsection{Conjunto de datos}
Para el conjunto inicial de datos, se ha realizado una selección cuidadosa de entre toda la información extraída, eligiendo los siguientes atributos: el número de citas, el factor de impacto JCR y la diferencia entre los datos extraídos y los valores reales. Estos atributos han sido tomados exclusivamente de los años comprendidos entre el 2018 y el 2020. Además, se ha incluido el número de citas correspondiente al año 2021. La elección de estos años se basa en la observación de que presentan menor error en los datos extraídos, lo cual contribuye a mejorar la calidad de los atributos seleccionados. Finalmente, se realizará la predicción del Factor de Impacto JCR para el año 2021.

\imagen{pipeline}{Representación esquemática de los datos de entrenamiento (X,y)}{1}

Por otro lado, cuando se trata de valores nulos o vacíos (i.e.,\textit{missing values}), se ha utilizado la mediana como método imputación de valores.

\subsubsection{Entrenamiento de los modelos}
Para el entrenamiento, se ha llevado a cabo una técnica de validación cruzada anidada (\textit{nested cross-validation}) con el fin de evaluar y seleccionar el mejor modelo posible para predecir el Índice de Impacto de las revistas científicas. Esta técnica implica el uso de dos niveles de validación cruzada: en el nivel externo se evalúa el desempeño general del modelo, mientras que en el nivel interno se ajustan los parámetros del modelo mediante una búsqueda en rejilla (\texttt{grid search}) para encontrar la mejor combinación de hiperparámetros. La ventaja de esta técnica es que permite evaluar la capacidad de generalización del modelo de manera más realista, evitando la selección de modelos sobreajustados (\textit{overfitting}). En comparación con una validación cruzada al uso, la técnica de validación cruzada anidada puede resultar más costosa en términos de cómputo, pero proporciona resultados más precisos y fiables.

\imagen{cv}{Esquema de la validación cruzada anidada}{1}

\subsubsection{Evaluación de los modelos}
Finalmente, tras la evaluación de todos los modelos, se seleccionan aquellos que han obtenido mejores resultados. Como se puede observar en los siguientes gráficos, los modelos con mayor precisión son AdaBoost Regressor, Random Forest Regressor y XGB Regressor.

\imagen{mse}{Estimación del RMSE de los modelos evaluados en la experimentación}{0.6}
\imagen{desviacion.png}{Desviación de las estimaciones}{0.6}

Los motivos por los cuales se han obtenido estos resultados pueden ser diversos. Por un lado, podemos ver que los modelos de \textit{ensemble} como AdaBoost, Random Forest y XGBoost obtienen, en general, mejores resultados. Esto puede deberse a su capacidad para capturar relaciones no lineales en los datos. Así, pueden modelar relaciones complejas entre las variables de entrada y la variable de salida, lo cual es especialmente útil cuando existen patrones no lineales en los datos. 

Por otro lado, los algoritmos de \textit{ensemble} tienden a reducir el sobreajuste en comparación con modelos individuales como MLP y SVM (los algoritmos de \textit{ensemble} combinan múltiples modelos más simples, lo que ayuda a mitigar el sesgo y la variabilidad inherente a un solo modelo).

Estos algoritmos suelen ser, también, más robustos ante la presencia de ruido o valores atípicos en los datos (como ocurre en nuestro caso). Utilizan técnicas como el muestreo \textit{bootstrap} y la combinación de múltiples árboles de decisión, lo que les permite ser menos sensibles a observaciones atípicas y errores de medición.

En cambio, MLP y SVM a menudo requieren una cuidadosa normalización y escala de los datos de entrada para un rendimiento óptimo. Además, MLP puede ser más sensible a la selección de hiperparámetros y puede requerir una búsqueda más exhaustiva de la configuración adecuada, aumentando así el tiempo de entrenamiento.

\subsubsection{Resultados}
Tras analizar los resultados de esta etapa, para poder hacer uso de ellos más adelante, se guardan en los siguientes ficheros:
\begin{itemize}
    \item Fichero CSV con los resultados de la validación cruzada. Por cada iteración incluye el modelo que se estima, el valor de los parámetros con mejores resultados y el RMSE.
    \item Ficheros binarios \textit{pickle}, donde se almacenan cada uno de los modelos entrenados. Posteriormente, se almacenarán en la base de datos para poder realizar predicciones desde la aplicación web.
\end{itemize}


\section{Creación de la aplicación web}
...









\capitulo{6}{Trabajos relacionados}

En el campo de la evaluación de revistas científicas, existen varios trabajos y proyectos previos que han tratado de extraer datos y calcular el factor de impacto. A continuación se presenta una breve descripción de algunos de estos trabajos relacionados:

\begin{enumerate}
 %   \item \textbf{The Journal Impact Factor: A Valid and Reliable Indicator of Journal Quality?}: Este estudio realizado por Bo-Christer Björk y Torgny Roxå en el año 2007, analiza la validez y la fiabilidad del factor de impacto como indicador de la calidad de una revista. El estudio concluye que el factor de impacto es un indicador válido y fiable, pero que debe ser utilizado con precaución y en combinación con otros indicadores.

  \item \textbf{An index to quantify an individual's scientific research output}\cite{hirsch2005}:  Este estudio realizado por Jorge E. Hirsch en el año 2005, presenta el h-index como una nueva métrica para evaluar el impacto de la investigación científica. El estudio argumenta que el h-index es más preciso que el factor de impacto y es menos sujeto a manipulación.

  \item \textbf{Modeling and Prediction of the Impact Factor of Journals Using Open-Access Databases}\cite{templ2020}: Este estudio de 2020 se enfoca en el uso de bases de datos de acceso abierto para modelar y predecir el factor de impacto de las revistas científicas. El estudio sugiere que es posible utilizar bases de datos como Google Scholar, ResearchGate y Scopus para estimar el factor de impacto de revistas nuevas, pequeñas o independientes que no están incluidas en el Science Citation Index (SCI) y que no reciben un factor de impacto de la base de datos Web of Science (WoS). El estudio también señala que los resultados obtenidos con el modelo desarrollado sugieren que es posible predecir el factor de impacto WoS utilizando bases de datos de acceso abierto alternativas.
\end{enumerate}


\capitulo{7}{Conclusiones y Líneas de trabajo futuras}

%Todo proyecto debe incluir las conclusiones que se derivan de su desarrollo. Éstas pueden ser de diferente índole, dependiendo de la tipología del proyecto, pero normalmente van a estar presentes un conjunto de conclusiones relacionadas con los resultados del proyecto y un conjunto de conclusiones técnicas. 
%Además, resulta muy útil realizar un informe crítico indicando cómo se puede mejorar el proyecto, o cómo se puede continuar trabajando en la línea del proyecto realizado. 


\section{Líneas de trabajo futuras}
En un futuro, si se desea utilizar la aplicación para calcular el índice de impacto en un campo distinto al de \textit{computer science}, solo se requiere subministrar (con permisos de administrador de la API) un CSV con la lista de revistas del nuevo campo deseado. La red se volverá a entrenar y se reiniciará todo el proceso, permitiendo así la adaptación de la aplicación a diferentes áreas de investigación. Con esta flexibilidad, la aplicación puede ser empleada para el cálculo del índice de impacto en una amplia variedad de campos, ampliando así su alcance y aplicabilidad.
\capitulo{8}{Prototipo inicial}

Al revisar la viabilidad del proyecto se plantearon posibles dificultades que podían surgir. Tras comprender que las complicaciones eran numerosas, se decidió crear un \textbf{prototipo sencillo}, consistente en un \textit{script} en Python, que trataría de lanzar  mil peticiones de búsqueda. Esto nos permitiría establecer los límites de realizar ``web scrapping'' sobre Google Scholar .

Así pues, manos a la obra, se desarrolló un \textit{script} sencillo, que solicita acceso a la página principal de Google Scholar y, mediante métodos HTTP, realiza una búsqueda (a partir de parámetros solicitados por pantalla). Finalmente, extrae el título de la página resultante tras hacer la búsqueda. 

Todo esto se logra haciendo uso de la biblioteca de Python \textbf{BeautifulSoup} (ver documentación en el siguiente \href{https://beautiful-soup-4.readthedocs.io/en/latest/}{enlace}). Esta biblioteca contiene métodos centrados en la extracción de datos de archivos HTML y XML para su posterior análisis. Es fácil advertir cuán idónea resulta esta biblioteca para nuestro propósito.

Tras diseñar y programar el código mencionado, se procede a su prueba. La primera ejecución del \textit{script} resultó desaletadora, ya que, a partir de la solicitud número \textbf{726}, Google Scholar detecta que un \textbf{\textit{bot}} está realizando búsquedas. A partir de ese momento, nuestra ip queda bloqueada y las solicitudes fallan sin excepción. Google Schoolar nos redirecciona a una página (figura \ref{fig:error}) donde se solicita al usuario resolver un \textit{captcha}. 

\imagen{error}{Captura de reCAPTCHA de Google}{.70}

El número de solicitudes exitosas es demasiado bajo para cumplir su función en nuestro proyecto, por lo que se procede a buscar una solución alternativa. 

Tras distintas pruebas, se encontró una forma de superar la barrera del \textit{captcha}. A saber: añadiendo a la \textit{url} una sección extra que permite suprimir esta excepción durante un periodo concreto de tiempo. Así pues, logramos ejecutar con éxito el \textit{script} tantas veces como se desee. Ahora ya se puede decir que el proyecto es \textbf{viable}.


\include{./tex/9_API}


\bibliographystyle{plain}
\bibliography{bibliografia}

\end{document}
