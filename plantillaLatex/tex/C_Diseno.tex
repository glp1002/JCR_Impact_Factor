\apendice{Especificación de diseño}

\section{Introducción}

\section{Diseño de datos}

\section{Diseño procedimental}

\section{Diseño arquitectónico}

% usuarios de la aplicacion
\section{Aplicación}

Se pretende desarrollar una aplicación web sencilla que utilizará una arquitectura cliente-servidor de dos capas.

La funcionalidad principal de la aplicación será predecir el índice de impacto de una lista de revistas elegida por el usuario. Además, se proporcionarán gráficos e información adicional para ayudar a los usuarios a entender mejor los resultados. Por otro lado, se podrán distinguir dos perfiles de usuario: el usuario normal y el administrador. El administrador tendrá permisos para ``reentrenar'' la red que predice los índices y gestionar al resto de usuarios.

El \textit{backend} se programará en Python y el \textit{frontend} con HTML, CSS y JavaScript. Para ello, se utilizará Flask, que se trata de un \textit{framework} de Python para el desarrollo de aplicaciones web.

Para la conexión con la base de datos, se hará uso de \textit{psycopg2}, que es un módulo de Python que proporciona una interfaz para conectarse y interactuar con bases de datos PostgreSQL. 
Es una de las librerías más populares y ampliamente utilizadas para trabajar con PostgreSQL en Python. Además, es compatible con la mayoría de las versiones de Python y es muy fácil de utilizar.

Por otro lado, para la autenticación, se han evaluado varias opciones que ofrece Flask, a saber: Flask-Login, Flask-Security y Flask-JWT. 
Finalmente, se ha elegido Flask-Login debido a su sencillez y facilidad de uso. Esta opción permite al usuario iniciar sesión con un nombre de usuario y una contraseña y almacena la información de la sesión en las \textit{cookies} del navegador. Este paquete incluye la protección de rutas con decoradores.


\subsection{Interfaces}
La aplicación estará conformada por seis interfaces principales, que se detallan a continuación:

\begin{itemize}
    \item Interfaces iniciales con funciones de inicio de sesión y registro, verificando la validez de los datos y permitiendo recuperar contraseñas.
    \item Interfaz principal para seleccionar una revista y un año, con opción de calcular el JCR y elegir modelos de predicción.
    \item Interfaces con el histórico y la predicción del JCR, además de gráficos representativos.
    \item Interfaz con una lista de las revistas disponibles con barra de búsqueda.
    \item Área de usuario donde modificar las credenciales.
\end{itemize}


