\capitulo{3}{Conceptos teóricos}

Cuando un investigador termina una etapa de su investigación, genera un artículo científico en el que plasma el resultado de su trabajo. Esta documentación sirve de precedente para aquellos que posteriormente investiguen sobre temáticas relacionadas. Así pues, el artículo científico es el elemento principal en torno al que giran los estudios bibliométricos.

Antes de adentrarnos en más detalles, se ilustrarán algunos conceptos esenciales sobre este proyecto, para dotarlo de mayor comprensión y claridad. En esta sección, por tanto, se abarcarán los conceptos de bibliometría, cienciometría e índice de impacto, como conceptos relativos al avance de la ciencia y la producción de conocimientos a partir de la actividad de la investigación.


\section{Bibliometría}

Palabra que proviene del griego \textit{biblio} (libro) y \textit{-metría} (medición)

Ciencia que aplica métodos matemáticos para encontrar comportamientos estadísticos en la literatura científica. Estos estudios y análisis pretenden cuantificar toda la actividad científica escrita con el objetivo final de orientar sobre el impacto de una investigación \cite{Amat1989}.

Antiguamente, se evaluaba la producción científica por pares, que es un proceso en el que expertos en un campo revisan y evalúan la calidad y el mérito de una investigación antes de su publicación. 
Al final de la década de 1950, en un momento en que la producción científica aumentaba progresivamente a un ritmo sostenido, nació la idea de una evaluación basada en cantidades, ciertamente menos costosa que la evaluación de pares: desde ese momento, comenzó a hablarse de bibliometría \cite{Turbanti2017}.
Así, los estudios bibliométricos tienen su origen en la década de 1960 con la creación del Science Citation Index por Eugene Garfield y el análisis de redes de citas realizado por Derek John de Solla Price. Estos trabajos establecieron las bases sólidas de la bibliometría como disciplina.

El objetivo de los estudios bibliométricos pueden limitase solamente al análisis de la envergadura, el crecimiento y la distribución de la literatura científica, pero también son útiles a la hora de encontrar las revistas donde es más conveniente publicar un artículo o descubrir a los autores más importantes en cada ámbito, así como las nuevas tendencias.

Sin embargo, la información organizada de esta forma también tiene sus \textbf{limitaciones}. Ejemplo de ello es el patrón de citas usado en cada área de investigación; es mucho más común incluir citas en investigaciones tecnológicas y científicas. Es decir, cada temática deberá ser tratada de forma distinta. Para ello se normalizará el índice de impacto.
Por otro lado, también influye la base de datos utilizada, ya que cada una tiene un método de indexación distinta.
El idioma es otro ejemplo de limitación. Ya que el inglés es el idioma predominante, será más complicado encontrar citas de documentos 	escritos en otros idiomas.
También es preciso tener en cuenta problemas de dispersión debidos a perfiles duplicados o nombres similares. Para poder solucionar este tipo de confusiones, es recomendable crear identificadores (por ejemplo, el ORCID) para cada autor.
De esta forma, todos los comportamientos irregulares (como las citas a uno mismo) pueden tenerse en consideración.
Finalmente, el momento en que se realizan los estudios es también fundamental, ya que el nivel de conocimientos en torno a una materia determinada varía con el tiempo.

\subsection{Ejemplo}

Un ejemplo de análisis bibliométrico (tomado del libro "Documentación científica y nuevas tecnologías de la información" \cite{Amat1989}) podría involucrar el determinar los autores que abordan un tema específico en una revista en particular. Para ello, deberían registrarse los siguientes datos:
\begin{enumerate}
    \item Lista de identificadores de los autores (ordenados alfabéticamente) aparecidos en la revista en cuestión.
    \item Cantidad de artículos de cada uno de estos autores en la revista.
    \item Revistas en las que aparecen los artículos citados.
    \item Los artículos (cantidad) que aparecen en cada una de las revistas listadas.
\end{enumerate}


\section{Cienciometría}

Palabra que proviene del latín  \textit{scientia} (ciencia) y del griego \textit{-metría} (medición)

Se trata del estudio de los metadatos e indicadores sobre la bibliografía científica con el fin de medir y analizar toda la producción científica \cite{Amat1989}.

La cienciometría tiene como objetivo clasificar y organizar el conocimiento científico a través de la creación de sistemas de conceptos y facilitar su transferencia a través de la educación y formación. Además, promueve la comunicación de conocimientos de un idioma a otro mediante la utilización de símbolos lingüísticos, permite la síntesis y resumen de la información científica, y proporciona un medio para recuperar y almacenar información a través de la indexación y lenguajes documentales.

La cienciometría, puesta en práctica, trabaja relacionada con otras ciencias y disciplinas como las que se muestran en la siguiente imagen:
%~\ref{fig:diagrama_cienciomatria}
\imagen{diagrama_cienciometria}{Ambos campos se superponen}{.60}

Es decir, se trata de un concepto más amplio que engloba al anterior (bibliometría). Así, se podría decir que la bibliometría surge como resultado del contacto interdisciplinar de entre el conjunto de disciplinas que integran lo que se conoce como <<ciencia de la ciencia>>, cuya fuente es la propia cienciometría \cite{Amat1989}.

Sin embargo, hoy en día la frontera entre cienciometría y bibliometría ha desaparecido casi por completo y ambos términos se usan prácticamente de forma sinónima \cite{Vitanov2016}.

\section{Webmetría}

Palabra que proviene del inglés web y del griego -metría (medición)

Aparte de las citaciones tradicionales, existen también referencias generadas por los lectores en la web. Se trata de la <<métrica de la web>> (\textit{webmetrics} o \textit{cybermetrics}). Se define como el estudio de los aspectos cuantitativos de la construcción y uso de recursos de información, estructuras y tecnologías en la web, basándose en enfoques bibliométricos \cite{bjorneborn2004}. 


El objetivo es obtener información sobre el número y el tipo de hiperenlaces, la estructura del World Wide Web y los modelos de uso de los recursos. Así pues, se contabiliza el número de veces que un sitio web o un documento publicado en internet es accedido y se divide entre el número de páginas del sitio mismo \cite{Turbanti2017}. Esto nos permite calcular la frecuencia con la que una página web promedio ha sido enlazada en un momento dado; un alto factor de impacto de la web indica la popularidad y, probablemente, el prestigio de una página web. Debido al creciente número de documentos publicados y disponibles en la red, en especial los \textit{e-journals} de acceso abierto, se han desarrollado nuevas herramientas de medición, lo que ha llevado a algunos investigadores a considerar un posible paralelismo entre las citaciones tradicionales y las referencias en la web. \cite{Turbanti2017}.

 
\section{Índice de impacto}

Existe un gran abanico de índices distintos englobados dentro de la bibliometría. Sin embargo, en este caso, nos centraremos especialmente en los índices de citas o, en inglés, ``Citation Index''.

Se trata de índices de autor con características especiales, pues no solamente citan junto a cada autor la lista de los documentos por él publicados, sino que añade, en cada referencia, la lista de los documentos que ha citado esta referencia en su propia bibliografía. Permiten localizar otros autores que han tratado las mismas materias y buscar también documentos más recientes que éste ya conocido. \cite{Amat1989}. Contienen literatura científica de medicina, psicología, agricultura, tecnología y documentación científica en general. 

A continuación, se enumeran algunos de los índices de impacto más comunes.


%\subsection{Ejemplo}

%En un índice de citas, aparecen los artículos de un mismo autor agrupados por orden cronológico. Justo debajo, aparece la lista de autores que han citado este artículo y, a su lado, la referencia del documento (ver figura 2).

%	(Insertar imágen en plantilla del TFG)

%Es decir, bastará con saber el nombre de un solo autor que trabaje en el campo en el que se esté interesado, para obtener de forma inmediata (consultando simplemente el índice) el conjunto de documentos que traten del mismo tema.

%Generalmente se divide en dos subíndices: uno de autores (con referencias de documentos citados), y otro de fuentes (acompañadas de los títulos de los documentos).

%Otros intereses que se deducen de este índice de autores consiste en el valor que pueden tener los documentos citados debido a que el autor de un documento, especialista en un campo, ha tenido interés en dar a conocer a dichos autores.


\subsection{SJR}

SJR (SCImago Journal Rank) es un indicador de calidad de revistas científicas basado en la idea de que \textit{cuanto más citada es una revista, mayor importancia tiene}. 

Además, se entiende que \textit{no todas las citas son iguales}\cite{Svetla2022}. Es decir, el indicador se calcula utilizando un algoritmo que también tiene en cuenta la importancia de la revista que hace la cita, de tal manera que una cita hecha por una revista de alto impacto tendrá más peso que una cita de una revista de bajo impacto \cite{Svetla2022}. 

Así pues, se calculan las citas \textbf{ponderadas} de los últimos tres años para determinar el SJR de cada revista.

\subsection{Hindex}
Hindex es un indicador de productividad y visibilidad de un investigador. El Hindex se calcula a partir de la cantidad de artículos publicados por un autor y la cantidad de veces que estos artículos han sido citados. 

El Hindex se determina buscando el número de publicaciones con al menos ese número de citas. Por ejemplo, un investigador con un Hindex de 10 tiene al menos 10 publicaciones que han sido citadas al menos 10 veces cada una \cite{Svetla2022}. 

Por lo tanto, se podría decir que este indicador es una métrica a nivel de autor se utiliza para medir la importancia, la productividad y el impacto de un investigador en la comunidad científica \cite{Svetla2022}. Este índice también puede ser aplicado a un grupo de investigadores (por ejemplo, un departamento, universidad o país).

Es importante mencionar que Google Scholar utiliza métricas basadas en el Hindex.

\subsection{CiteScore}
Es un indicador desarrollado por Scopus que tiene en cuenta el número de citas recibidas por una revista en un período de tres años, el número de documentos en la base de datos de Scopus y el número de documentos publicados en esa revista en ese período.

De esta forma, el CiteScore se calcula dividiendo el número de citas que recibe una revista en un año por los documentos publicados en los tres años previos, y dividiendo este número por el número de documentos indexados en Scopus publicados en los mismos tres años \cite{Svetla2022}. 


\subsection{JCR}
El JCR (\textit{Journal Citation Reports}) es una base de datos desarrollada por Clarivate Analytics que proporciona información sobre revistas científicas.

Consiste en el análisis biométrico de las revistas del banco de datos ISI (\textit{Intellect Scientific Information}), con número de citas al año de cada una, revistas que citan a otras revistas, listado de abreviaturas de títulos y su desarrollo, etc. JCR proporciona información estadística sobre la frecuencia de citas de las revistas, incluyendo el \textit{Impact Factor} (IF) y el \textit{Impact Factor of a Year} (IFY) entre otros indicadores \cite{clarivate2019}. 

De entre los datos estadísticos que se obtienen, nos importa especialmente el ya mencionado \textbf{Factor de Impacto}, que permiten determinar de una manera sistemática y objetiva la importancia relativa de las principales revistas de investigación internacionales dentro de sus categorías temáticas. 

El Factor de Impacto se calcula para cada revista en cada categoría. Para ello, se suma el número de citas que, durante un año, han ``referenciado'' a los artículos publicados durante los dos años anteriores en dicha revista. El resultado de esa suma se divide entre el número total de artículos publicados en esos últimos dos años.
Por ejemplo, si quisiéramos calcular el Factor de Impacto de la revista Educación en 2016, se haría de la siguiente manera:

\imagen{CalculoJCR}{Cálculo del JCR}{.99}
 
El resultado de esta operación se interpretará en función de si es mayor o menor que 1:
\begin{itemize}
	\item Mayor que 1: La revista ha tenido un mejor impacto del esperado en esa categoría.
	\item Menor que 1: La revista ha tenido un peor impacto del esperado en esa categoría.
\end{itemize}
Es decir, para poder realizar los cálculos, necesitaremos extraer el número de citas de cada artículo, la revista a la que pertenece, el año de publicación y la categoría a la que se refiere el artículo.


JCR cubre más de 12\,000 revistas de más de 80 disciplinas diferentes, lo que permite comparar el impacto y la calidad de las revistas en diferentes campos. Los datos de JCR se utilizan a menudo como un indicador de la calidad y el impacto de una revista en la comunidad científica \cite{clarivate2019}.

La última actualización del JCR ofrece los datos del Factor de Impacto del 2022. Este índice se calcula con un cierto retraso respecto al final del año: suele aparecer alrededor de los meses de mayo o junio del año siguiente. Esto limita el acceso a la información actualizada y justifica la importancia del trabajo que está siendo realizado, de forma que se pueda brindar información actualizada y accesible sobre el impacto de los artículos y revistas a la comunidad científica, lo cual es esencial para la toma de decisiones y la evaluación del desempeño científico.