
\apendice{Plan de Proyecto Software}

\section{Introducción}

La planificación temporal es esencial para el éxito de cualquier proyecto, especialmente en el desarrollo de \textit{software}. En esta sección se detallará cómo se ha llevado a cabo el cronograma siguiendo una metodología ágil tipo Scrum, mediante el uso de \textit{sprints}. Se describirán los pasos necesarios para planificar y gestionar de manera eficiente el tiempo y los recursos disponibles, asegurando así el cumplimiento de los objetivos del proyecto en el plazo establecido.

\section{Metodología}
La metodología Scrum es un marco de trabajo ágil que se utiliza para la gestión de proyectos. Se basa en la colaboración entre el equipo de desarrollo y el cliente, así como en la entrega continua de entregables funcionales en ciclos cortos de tiempo conocidos como \textit{sprints}. Scrum se centra en la flexibilidad y la adaptabilidad, permitiendo más fácilmente la adaptación y los cambios de manera rápida y efectiva ante situaciones inesperadas. 

En el contexto del presente trabajo, el uso de la metodología Scrum será beneficioso puesto que nos permitirá tener un enfoque más dinámico y flexible, lo que nos permitirá adaptarnos a las necesidades del proyecto a medida que avanzamos. Además, Scrum nos proporcionará una mayor transparencia y comunicación, lo que nos permitirá tomar decisiones informadas y asegurar que el proyecto se entregue a tiempo y con éxito.

\section{Herramienta}
ZenHub es una herramienta de gestión de proyectos que se utiliza en muchos entornos empresariales para mejorar la eficiencia en el seguimiento y administración de tareas. Con ZenHub, es posible planificar y visualizar fácilmente las actividades de un proyecto, monitorear el progreso de cada tarea y hacer un seguimiento de los plazos de entrega. También ofrece funciones útiles como integración con GitHub, herramientas de informes y análisis de datos, y seguimiento de errores y problemas.

\section{Planificación temporal}

El presente proyecto comienza en septiembre y se extiende hasta junio. Durante este período de tiempo, es importante asegurarnos de que todas las tareas y hitos estén claramente definidos.

Para lograr esto, se han realizado reuniones periódicas para discutir el progreso del proyecto y asegurarnos de que estamos en el camino correcto. Además, hemos implementado \textit{sprints} regulares para asegurarnos de que estamos avanzando de manera constante y cumpliendo con nuestras metas a tiempo.

Con esta planificación temporal sólida, estamos seguros de que podremos completar el proyecto a tiempo y cumplir con los objetivos establecidos. Sin embargo, es importante ser flexibles y estar preparados para hacer ajustes según sea necesario a medida que avanzamos en el proyecto, puesto que se realiza al mismo tiempo que trascurre el curso académico.

En las siguientes tablas se recogen los distintos \textit{sprints} junto con su duración, objetivos y tareas. Además, se adjuntará el gráfico \textit{burndown}\footnote{Gráfico burndown: Se trata de una herramienta comúnmente utilizada en la gestión de proyectos ágiles para realizar un seguimiento de un \textit{sprint}. Este gráfico muestra la cantidad de trabajo que queda por hacer en relación con el tiempo disponible para hacerlo. A medida que se completa el trabajo, la línea del gráfico debería disminuir hasta alcanzar cero en el momento en que finaliza el \textit{sprint}.} de cada uno de los \textit{sprints}.


\tabla{Sprint 1}{|l|l|l|l|}{4}{sprint1}{\textbf{Sprt.} & \textbf{Duración} & \textbf{Objetivos} & \textbf{Tareas}\\}{
    1 & \parbox{55}{19/09/2022 03/10/2022} & \parbox{80}{Comenzar el desarrollo del proyecto} & \parbox{150}{\begin{itemize}\item Elegir un modelo de referencias bibliográficas. \item Definición de conceptos.\end{itemize}}\\
}

\imagen{sprint1}{Sprint 1: Burndown chart}{0.9}

\tabla{Sprint 2}{|l|l|l|l|}{4}{sprint2}{\textbf{Sprt.} & \textbf{Duración} & \textbf{Objetivos} & \textbf{Tareas}\\}{
    2 & \parbox{55}{03/10/2022 17/10/2022} & \parbox{80}{Tareas de investigación y documentación} & \parbox{150}{\begin{itemize}\item Búsqueda de antecedentes. \item Familiarizarse con la memoria en \LaTeX.\end{itemize}}\\
}

\imagen{sprint2}{Sprint 2: Burndown chart}{0.9}

\tabla{Sprint 3}{|l|l|l|l|}{4}{sprint3}{\textbf{Sprt.} & \textbf{Duración} & \textbf{Objetivos} & \textbf{Tareas}\\}{
    3 & \parbox{55}{17/10/2022 14/11/2022} & \parbox{80}{Investigación y creación de un prototipo inicial} & \parbox{150}{\begin{itemize}\item Búsqueda de información sobre MIAR. \item Documentación sobre el JCR y el índice de impacto. \item Prototipo que permita la búsqueda de artículos en GS. \item Extracción de datos: ¿qué datos se pueden extraer? \item Reflexión sobre el diseño de la BBDD. \item Prueba del prototipo.\end{itemize}}\\
}

\imagen{sprint3}{Sprint 3: Burndown chart}{0.9}

\tabla{Sprint 4}{|l|l|l|l|}{4}{sprint4}{\textbf{Sprt.} & \textbf{Duración} & \textbf{Objetivos} & \textbf{Tareas}\\}{
    4 & \parbox{55}{14/11/2022 28/11/2022} & \parbox{80}{Base de datos y prototipo} & \parbox{150}{\begin{itemize}\item Creación de la base de datos en MariaDB. \item Prueba de la BBDD. \item Mejoras del prototipo.\end{itemize}}\\
}

\imagen{sprint4}{Sprint 4: Burndown chart}{0.85}

\nota{Como se puede observar en el gráfico \textit{burndown} anterior, las tareas no fueron cerradas a tiempo. Sin embargo, es preciso aclarar que se trata de un error en la configuración de las tareas. Si bien todas las tareas fueron completadas a tiempo, fue preciso borrarlas en ZenHub y volverlas a crear más adelante cuando el \textit{sprint} ya había terminado debido a que estaban inicialmente mal configuradas.}

\tabla{Sprint 5}{|l|l|l|l|}{4}{sprint5}{\textbf{Sprt.} & \textbf{Duración} & \textbf{Objetivos} & \textbf{Tareas}\\}{
    5 & \parbox{55}{28/11/2022 12/12/2022} & \parbox{80}{Base de datos y prototipo} & \parbox{150}{\begin{itemize}\item Se muda la BBDD a Postgres. \item Pruebas e investigación para extraer el DOI.\end{itemize}}\\
}

\imagen{sprint5}{Sprint 5: Burndown chart}{0.9}

\tabla{Sprint 6}{|l|l|l|l|}{4}{sprint6}{\textbf{Sprt.} & \textbf{Duración} & \textbf{Objetivos} & \textbf{Tareas}\\}{
    6 & \parbox{55}{09/01/2023 23/01/2023} & \parbox{80}{Documentación y obtención del DOI} & \parbox{150}{\begin{itemize}\item Obtención del DOI a través de Crossref. \item Avance de la memoria y los anexos. \item Mejora de la BBDD.\end{itemize}}\\
}

\imagen{sprint6}{Sprint 6: Burndown chart}{0.9}

\tabla{Sprint 7}{|l|l|l|l|}{4}{sprint7}{\textbf{Sprt.} & \textbf{Duración} & \textbf{Objetivos} & \textbf{Tareas}\\}{
    6 & \parbox{55}{06/02/2023 20/02/2023} & \parbox{80}{Preparación para la conexión remota a los servidores de la universidad.} & \parbox{150}{\begin{itemize}\item Conectarse a la VPN de la universidad. \item Crear un servicio virtualizado en Miniconda. \item Mejorar la BBDD.\end{itemize}}\\
}

\imagen{sprint7}{Sprint 7: Burndown chart}{0.9}

\tabla{Sprint 8}{|l|l|l|l|}{4}{sprint8}{\textbf{Sprt.} & \textbf{Duración} & \textbf{Objetivos} & \textbf{Tareas}\\}{
    6 & \parbox{55}{20/02/2023 06/03/2023} & \parbox{80}{Conexión a los servidores y mejora de los prototipos. Avances en la API y documentación.} & \parbox{150}{\begin{itemize}\item Mejorar el prototipo. \item Prototipo con varios hilos de ejecución (usar servidores de la universidad). \item Nuevo prototipo con Selenium. \item Sistema inteligente de requests. \item Revisión de PoP. \item API: Autentificación y back-end.\end{itemize}}\\
}

\imagen{sprint8}{Sprint 8: Burndown chart}{0.9}

\tabla{Sprint 9}{|l|l|l|l|}{4}{sprint9}{\textbf{Sprt.} & \textbf{Duración} & \textbf{Objetivos} & \textbf{Tareas}\\}{
    6 & \parbox{55}{06/03/2023 20/03/2023} & \parbox{80}{Finalización del prototipo de extracción de Crossref.} & \parbox{150}{\begin{itemize}\item Plan de prueba para Crossref. \item Prototipo de Crossref. \item Comprobar límite de llamadas a Crossref. \item Crear entorno virtualizado. \item Fichero requirements. \item Licencia Crossref para Cited-by. \item Cálculo del IF.\end{itemize}}\\
}

\imagen{sprint9}{Sprint 9: Burndown chart}{0.9}

\tabla{Sprint 10}{|l|l|l|l|}{4}{sprint10}{\textbf{Sprt.} & \textbf{Duración} & \textbf{Objetivos} & \textbf{Tareas}\\}{
    6 & \parbox{55}{20/03/2023 03/04/2023} & \parbox{80}{Procesamiento de los datos.} & \parbox{150}{\begin{itemize}\item Normalizar listados JCR. \item Comparar JCR obtenidos con los valores reales.\end{itemize}}\\
}

Finalmente, se incluirá un gráfico \textit{Cumulative}\footnote{Gráfico Cumulative: En este gráfico, el eje vertical representa la cantidad total de trabajo completado, mientras que el eje horizontal representa el tiempo transcurrido durante el proyecto. A medida que se completa el trabajo, la línea del gráfico se eleva, lo que indica el progreso general del proyecto} que muestra la cantidad total de trabajo realizado de forma más generalizada.

\imagen{cumulative}{Gráfico Cumulative de todo el proyecto}{0.9}




\section{Estudio de viabilidad}

\subsection{Viabilidad económica}

\subsection{Viabilidad legal}


