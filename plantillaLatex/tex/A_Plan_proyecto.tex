
\apendice{Plan de Proyecto Software}

\section{Introducción}

La planificación temporal es esencial para el éxito de cualquier proyecto, especialmente en el desarrollo de \textit{software}. En esta sección se detallará cómo se llevará a cabo el cronograma siguiendo una metodología ágil tipo Scrum, mediante el uso de \textit{sprints}. Se describirán los pasos necesarios para planificar y gestionar de manera eficiente el tiempo y los recursos disponibles, asegurando así el cumplimiento de los objetivos del proyecto en el plazo establecido.


\section{Planificación temporal}

El presente proyecto comienza en septiembre y se extiende hasta junio. Durante este período de tiempo, es importante asegurarnos de que todas las tareas y hitos estén claramente definidos.

Para lograr esto, se han realizado reuniones periódicas para discutir el progreso del proyecto y asegurarnos de que estamos en el camino correcto. Además, hemos implementado \textit{sprints} regulares para asegurarnos de que estamos avanzando de manera constante y cumpliendo con nuestras metas a tiempo.

Con esta planificación temporal sólida, estamos seguros de que podremos completar el proyecto a tiempo y cumplir con los objetivos establecidos. Sin embargo, es importante ser flexibles y estar preparados para hacer ajustes según sea necesario a medida que avanzamos en el proyecto, puesto que se realiza al mismo tiempo que trascurre el curso académico.

En la siguiente tabla se recogen los distintos \textit{sprints} junto con su duración, objetivos y tareas.

\begin{table}[ht!]
  \caption{Planificación de Sprints}
  \centering
  \begin{tabular}{|c|m{2.2cm}|p{3cm}|p{5cm}|}
    \hline
    \textbf{Sprint} & \textbf{Duración} & \textbf{Objetivos} & \textbf{Tareas} \\
    \hline
    1 & 19/09/2022\newline 03/10/2022 & Comenzar el desarrollo del proyecto & Elegir un modelo de referencias bibliográficas.\newline Definición de conceptos. \\
    \hline
    2 & 03/10/2022\newline17/10/2022 & Tareas de investigación y documentación & Búsqueda de antecedentes.\newline Familiarizarse con la memoria en LaTeX. \\
    \hline
    3 & 17/10/2022\newline14/11/2022 & Investigación y creación de un prototipo inicial & Búsqueda de información sobre MIAR\newline Documentación sobre el JCR y el índice de impacto.\newline Prototipo que permita la búsqueda de artículos en GS.\newline Extracción de datos: ¿qué datos se pueden extraer?\newline Reflexión sobre el diseño de la BBDD.\newline Prueba del prototipo.\\
    \hline
    4 & 14/11/2022\newline28/11/2022 & Base de datos y prototipo & Creación de la base de datos en MariaDB.\newline Prueba de la BBDD. \newline Mejoras del prototipo. \\
    \hline
    5 & 28/11/2022\newline12/12/2022 & Base de datos y prototipo & Se muda la BBDD a Postgres.\newline Pruebas e investigación para extraer el DOI. \\
    \hline
    6 & 09/01/2023\newline23/01/2023 & Documentación y obtención del DOI & Obtención del DOI a través de Crossref.\newline Avence de la memoria y los anexos. \newline Mejora de la BBDD \\
    \hline
  \end{tabular}
\end{table}
 

\section{Estudio de viabilidad}

\subsection{Viabilidad económica}

\subsection{Viabilidad legal}


