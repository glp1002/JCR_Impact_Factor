\capitulo{1}{Introducción}

El factor de impacto de una publicación científica mide la frecuencia con la cual ha sido citado el artículo promedio de una revista en un año en particular. Específicamente, sirve para evaluar la importancia de una revista dentro de un determinado campo científico. Existen múltiples metodologías de cálculo y sus correspondientes métricas, siendo el JCR (Journal Citation Reports) y el SJR (Scimago Journal Rank) los dos índices de impacto más utilizados.
Además, cabe destacar que el factor de impacto es uno de los principales criterios empleados en los procesos de acreditación y promoción interna para evaluar la calidad del trabajo científico de millones de académicos en todo el mundo.
Por su propia naturaleza, el factor de impacto se calcula con carácter retrospectivo, i.e., sobre datos de años anteriores. Así pues, si a la hora de seleccionar la revista a la que mandarán un nuevo trabajo, los académicos quieren tener en cuenta el posible factor de impacto que tendrá la revista en el año de publicación del artículo, lo único que pueden hacer es fijarse en su índice de impacto de los años anteriores y hacer sus propias hipótesis/predicciones de futuro.
Dado que a día de hoy la herramienta Google Scholar recoge información extremadamente actualizada sobre la publicación y citación de artículos científicos (podría decirse que se actualiza prácticamente en tiempo real), creemos que puede ser muy útil para la comunidad científica utilizar los datos de Google Scholar como \textit{inputs} de modelos de aprendizaje automático para estimar el índice de impacto que tendrán las distintas revistas científicas en el año en curso.
El \textit{output} esperado del proyecto será una aplicación web de tipo open-access, la cual implementará algoritmos de aprendizaje supervisado que utilizarán los datos históricos disponibles en Google Scholar para predecir el valor del índice de impacto de todas las revistas científicas indexadas en el JCR. Dicha aplicación se dejará en un repositorio público, para así garantizar que pueda ser utilizada por toda la comunidad científica. 
