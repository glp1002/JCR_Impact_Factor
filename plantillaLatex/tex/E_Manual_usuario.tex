\apendice{Documentación de usuario}

\section{Introducción}

\section{Requisitos de usuarios}

\section{Instalación}

\section{Manual del usuario}


\subsection{Entorno virtualizado}

Se trata de un entorno aislado que te permite instalar y gestionar de forma independiente las bibliotecas y paquetes de Python que se necesitan para el proyecto, sin afectar a otros proyectos o al sistema operativo. Esto nos permite evitar  problemas de dependencias y compatibilidad entre diferentes versiones de bibliotecas, además de facilitar la gestión y el mantenimiento del proyectos.
 
Miniconda, por su parte, es una distribución de Python que incluye un gestor de paquetes llamado Conda, que te permite instalar y gestionar de forma sencilla bibliotecas y paquetes de Python.

Así pues, se procede a crear un entorno virtualizado de Python utilizando Miniconda. El entorno en cuestión se encuentra empaquetado
en la carpeta comprimida "environment.zip". Para poder hacer uso del
entorno bastará con ejecutar el siguiente comando:
\texttt{conda activate /ruta/al/environment.zip}


