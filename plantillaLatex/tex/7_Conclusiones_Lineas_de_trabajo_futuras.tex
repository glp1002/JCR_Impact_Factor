\capitulo{7}{Conclusiones y Líneas de trabajo futuras}

%Todo proyecto debe incluir las conclusiones que se derivan de su desarrollo. Éstas pueden ser de diferente índole, dependiendo de la tipología del proyecto, pero normalmente van a estar presentes un conjunto de conclusiones relacionadas con los resultados del proyecto y un conjunto de conclusiones técnicas. 
%Además, resulta muy útil realizar un informe crítico indicando cómo se puede mejorar el proyecto, o cómo se puede continuar trabajando en la línea del proyecto realizado. 


\section{Líneas de trabajo futuras}
En esta sección se proponen posibles direcciones para continuar avanzando en la comprensión del tema en cuestión y en su aplicación práctica.

Por un lado, si en el futuro se desea utilizar la aplicación para calcular el índice de impacto en un campo distinto al de \textit{computer science}, solo se requiere subministrar (con permisos de administrador de la API) un CSV con la lista de revistas del nuevo campo deseado. La red se volverá a entrenar y se reiniciará todo el proceso, permitiendo así la adaptación de la aplicación a diferentes áreas de investigación. Con esta flexibilidad, la aplicación puede ser empleada para el cálculo del índice de impacto en una amplia variedad de campos, ampliando así su alcance y aplicabilidad. 
En otras palabras, se sugiere ampliar el alcance de la aplicación incluyendo nuevas entradas a la base de datos.

%TO DO: Por otro lado... 

\section{Conclusiones}

% TO DO: A modo de reflexión final sobre el trabajo realizado...



