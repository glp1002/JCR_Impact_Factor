\capitulo{2}{Objetivos del proyecto}

El objetivo principal de este proyecto consiste en llevar a cabo una investigación que permita sobre cómo estimar el Índice de Impacto de una revista. Para lograrlo, se desarrollará un modelo de inteligencia artificial capaz de realizar predicciones precisas al respecto. Como último paso, se diseñará y programará una aplicación web que facilite el acceso a esta IA.

A continuación, se enumeran los principales objetivos de este proyecto:

\begin{enumerate}
\item Lectura de literatura científica sobre bibliometría para comprender bien el marco conceptual en el que se encuadra este proyecto.
\item Estudio de la metodología de cálculo de los distintos índices de impacto en general, y del JCR (Journal Citation Report) en particular. Asimismo, se estudiarán las sucesivas modificaciones/excepciones que se han ido introduciendo en el cálculo del JCR a lo largo de los años.
\item Estudio de la API de Google Scholar y otras APIs.
\item Diseño y creación de una base de datos en la que se almacenará la información bibliográfica descargada.
\item Implementación de las funciones de cálculo del índice JCR para aplicarlas sobre los datos extraídos. 
\item Experimentación y evaluación de distintos modelos de regresión (aprendizaje supervisado) para predecir el índice de impacto JCR a partir de las series temporales históricas disponibles. Selección del mejor algoritmo.
\item Diseño y creación de una aplicación web donde se incluirá el mejor modelo de regresión para ponerlo a disposición de la comunidad científica.
\end{enumerate}