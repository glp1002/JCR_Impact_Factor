\capitulo{6}{Trabajos relacionados}

En el campo de la evaluación de revistas científicas, existen varios trabajos y proyectos previos que han tratado de extraer datos y calcular el factor de impacto. A continuación se presenta una breve descripción de algunos de estos trabajos relacionados:

\begin{enumerate}
 %   \item \textbf{The Journal Impact Factor: A Valid and Reliable Indicator of Journal Quality?}: Este estudio realizado por Bo-Christer Björk y Torgny Roxå en el año 2007, analiza la validez y la fiabilidad del factor de impacto como indicador de la calidad de una revista. El estudio concluye que el factor de impacto es un indicador válido y fiable, pero que debe ser utilizado con precaución y en combinación con otros indicadores.

  \item \textbf{An index to quantify an individual's scientific research output}\cite{hirsch2005}:  Este estudio realizado por Jorge E. Hirsch en el año 2005, presenta el h-index como una nueva métrica para evaluar el impacto de la investigación científica. El estudio argumenta que el h-index es más preciso que el factor de impacto y es menos sujeto a manipulación.

  \item \textbf{Modeling and Prediction of the Impact Factor of Journals Using Open-Access Databases}\cite{templ2020}: Este estudio de 2020 se enfoca en el uso de bases de datos de acceso abierto para modelar y predecir el factor de impacto de las revistas científicas. El estudio sugiere que es posible utilizar bases de datos como Google Scholar, ResearchGate y Scopus para estimar el factor de impacto de revistas nuevas, pequeñas o independientes que no están incluidas en el Science Citation Index (SCI) y que no reciben un factor de impacto de la base de datos Web of Science (WoS). El estudio también señala que los resultados obtenidos con el modelo desarrollado sugieren que es posible predecir el factor de impacto WoS utilizando bases de datos de acceso abierto alternativas.
\end{enumerate}

