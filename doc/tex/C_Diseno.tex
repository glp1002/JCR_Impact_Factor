\apendice{Especificación de diseño}

\section{Introducción}

El objetivo principal de esta especificación de diseño es brindar una guía clara y completa para el desarrollo de la aplicación web, proporcionando una base sólida sobre la cual los desarrolladores puedan implementar el sistema de manera efectiva. Asimismo, este informe servirá como referencia y documentación para futuras etapas del proyecto y posibles mejoras o expansiones.

Así pues, en esta sección se describirán y analizarán en detalle los diferentes aspectos del diseño de la aplicación web, abarcando desde la estructura y organización de los datos hasta los procedimientos y algoritmos que permiten su funcionamiento adecuado. Además, se profundizará en el diseño arquitectónico, que define la distribución de los componentes y la interacción entre ellos.

\section{Diseño de los datos}

Comenzaremos detallando el diseño de los datos de la aplicación. En este caso, para el almacenamiento de la información, se ha utilizado una base de datos PostgreSQL, la cual es accedida a través de la aplicación haciendo uso de la librería de Python \textit{psycopg2}. 

\subsection{Diccionario de datos}
A continuación, se describirán las tablas que componen la base de datos y se explicará la estructura y los atributos de cada una de ellas.

\subsubsection{Tabla <<Users>>}

Esta tabla almacena la información de los usuarios registrados en la aplicación. Los atributos de la tabla son los siguientes:
\begin{itemize}
    \item \texttt{username}: campo de tipo \texttt{VARCHAR(255)} que guarda el nombre de usuario.
    \item \texttt{password}: campo de tipo \texttt{VARCHAR(255)} que almacena la contraseña del usuario.
    \item \texttt{email}: campo de tipo \texttt{VARCHAR(255)} que sirve como clave primaria y guarda la dirección de correo electrónico del usuario.
    \item \texttt{admin}: campo de tipo \texttt{BOOLEAN} que indica si el usuario tiene privilegios de administrador.
\end{itemize}

\subsubsection{Tabla <<Modelos>>}

En esta tabla se registran los modelos utilizados en la aplicación. Los atributos de la tabla son los siguientes:
\begin{itemize}
    \item \texttt{id}: campo de tipo \texttt{SERIAL}  que actúa como clave primaria y asigna un identificador único a cada modelo.
    \item \texttt{nombre}: campo de tipo \texttt{TEXT} que almacena el nombre del modelo.
    \item \texttt{rmse}: campo de tipo \texttt{FLOAT} que guarda el valor del error cuadrático medio asociado al modelo.
\end{itemize}


\subsubsection{Tabla <<Revista>>}

Esta tabla contiene la información de las revistas presentes en la aplicación. Los atributos de la tabla son los siguientes:
\begin{itemize}
    \item \texttt{nombre}: campo de tipo \texttt{VARCHAR(255)} que sirve como clave primaria y almacena el nombre de la revista.
    \item \texttt{ISSN}: campo de tipo \texttt{VARCHAR(9)} que garantiza la unicidad y no nulidad del International Standard Serial Number de la revista.
    \item \texttt{categoria}: campo de tipo \texttt{VARCHAR(255)} que guarda la categoría a la que pertenece la revista.
\end{itemize}

\subsubsection{Tabla <<Revista\_jcr>>}

Esta tabla registra las métricas JCR (Journal Citation Reports) asociadas a las revistas. Los atributos de la tabla son los siguientes:

\begin{itemize}
    \item \texttt{nombre}: campo de tipo \texttt{VARCHAR(255)} que guarda el nombre de la revista.
    \item \texttt{fecha}: campo de tipo \texttt{NUMERIC} que indica la fecha de la métrica JCR.
    \item \texttt{jcr}: campo de tipo \texttt{FLOAT} que almacena el valor de la métrica JCR para la revista en la fecha correspondiente.
    \item \texttt{citas}: campo de tipo \texttt{NUMERIC} que indica la cantidad de citas recibidas por la revista en la fecha dada.
    \item \texttt{diff}: campo de tipo \texttt{FLOAT} que representa la diferencia en la métrica JCR real en comparación con el JCR calculado por el alumno (en la fecha dada).
    \item \texttt{cuartil}: campo de tipo \texttt{VARCHAR(3)} que indica el cuartil al que ha pertenecido la revista en la fecha dada.
\end{itemize}
 
Estas tablas y sus respectivos atributos han sido diseñados de manera coherente para almacenar y organizar la información necesaria en la aplicación web de una forma sencilla.

\subsection{Optimización de la base de datos}

Para mejorar el rendimiento y la eficiencia de las consultas en la base de datos, se han aplicado diferentes técnicas de optimización. Una de estas técnicas es la creación de índices, que permiten acceder rápidamente a los registros de una tabla en función de los valores de una columna o conjunto de columnas. En el caso de la tabla <<Revista>>, se ha creado un índice llamado \texttt{nombre\_index} en la columna \texttt{nombre} mediante la instrucción \texttt{CREATE INDEX nombre\_index ON revista (nombre)}. Esto facilita la búsqueda y recuperación de registros en la tabla <<Revista>> basándose en el valor de la columna \texttt{nombre}.

Además de crear índices, se ha utilizado el comando \texttt{ANALYZE} para actualizar las estadísticas de las tablas. Esto permite al sistema de gestión de bases de datos (SGBD) recopilar información actualizada sobre la distribución de los datos y la estructura de las tablas. En el caso de la tabla <<Revista>>, se ha aplicado el comando \texttt{ANALYZE revista} para asegurar que las estadísticas se encuentren actualizadas y que el SGBD pueda tomar decisiones más acertadas al ejecutar consultas sobre esta tabla. Del mismo modo, se ha realizado el análisis del resto de tablas de la base de datos.

Estas técnicas permiten acelerar las consultas y mejorar la eficiencia en la recuperación de datos, lo que resulta en una aplicación web más rápida y con mejor respuesta para los usuarios.

\subsection{Diagramas de datos}
A continuación se mostrará un sencillo diagrama relacional (ver Figura \ref{fig:rel}).

% Diagrama relacional
\imagen{rel}{Diagrama Relacional}{1}


\section{Diseño procedimental}
Todo el diseño procedimental se detalla en los siguientes diagramas de secuencia: \ref{fig:diag_sec_1}, \ref{fig:diag_sec_2}, \ref{fig:diag_sec_3}, \ref{fig:diag_sec_4} y \ref{fig:diag_sec_5}. 

\imagen{diag_sec_1}{Diagrama de secuencia de inicio de sesión}{1}

\imagen{diag_sec_2}{Diagrama de secuencia de registro}{1}

\imagen{diag_sec_3}{Diagrama de secuencia de la consulta de revistas}{1}

\imagen{diag_sec_4}{Diagrama de secuencia de la consulta del histórico del JCR de una revista}{1}

\imagen{diag_sec_5}{Diagrama de secuencia de la predicción del JCR de una revista}{1}
\section{Diseño arquitectónico}

Se ha desarrollado una aplicación web sencilla que usa una arquitectura cliente-servidor de dos capas (ver Figura \ref{fig:arquitectura1}).

\imagen{arquitectura1}{Arquitectura Cliente-Servidor de dos capas}{0.8}

Una arquitectura Cliente-Servidor de dos capas para una aplicación web presenta varias ventajas significativas. En primer lugar, permite una clara separación de responsabilidades entre el cliente y el servidor, lo que facilita el mantenimiento y la escalabilidad del sistema. Además, esta arquitectura permite un rendimiento óptimo, ya que las tareas se distribuyen eficientemente entre el cliente y el servidor. 

Por otro lado, se ha programado el \textit{backend} en tres capas, usando una arquitectura Modelo-Vista-Controlador (MVC). Esta arquitectura promueve una clara separación de responsabilidades, lo que facilita la organización y mantenimiento del código. El modelo se encarga del acceso a datos, la vista se ocupa de la presentación de la interfaz de usuario, y el controlador de la lógica de negocio (además de coordinar las interacciones entre el modelo y la vista).

Además, la arquitectura MVC favorece la reutilización de código, ya que los componentes son independientes y pueden ser modificados o reemplazados sin afectar a los demás. Esto aumenta la flexibilidad y permite un desarrollo más eficiente.

\imagen{arquitectura2}{Arquitectura Modelo-Vista-Controlador de tres capas}{0.2}


\section{Diseño de Interfaces}
La aplicación estará conformada por seis interfaces principales, que se detallan a continuación:

\begin{itemize}
    \item Interfaces iniciales con funciones de inicio de sesión y registro, verificando la validez de los datos.
    \item Interfaz principal para seleccionar una revista y los modelos de predicción.
    \item Interfaces con el histórico y la predicción del JCR, además de gráficos representativos.
    \item Interfaz con una lista de las revistas disponibles con barra de búsqueda.
    \item Área de usuario donde modificar las credenciales.
\end{itemize}

En la etapa inicial del proyecto, se procedió al diseño de \textit{mockups} de las interfaces de la aplicación. Un \textit{mockups} se refiere a una representación visual estática de una interfaz de usuario, que muestra de manera aproximada la disposición de elementos, la estructura y la apariencia general de la aplicación. Los \textit{mockups} permiten visualizar y evaluar el diseño antes de implementarlo, lo que ayuda a identificar posibles mejoras y realizar ajustes necesarios de manera temprana.

A continuación, adjunto los \textit{mockups} diseñados al inicio del proyecto. Es preciso recordar que son una representación preliminar de las interfaces y sirven como punto de partida para el desarrollo de la aplicación. Estos diseños han evolucionado y se han ido refinando a medida que ha avanzado el proyecto.

\imagen{inicio_sesion}{\textit{Mockup} de inicio de sesión}{1}
\imagen{registro}{\textit{Mockup} de registro}{1}
\imagen{principal}{\textit{Mockup} de página principal}{1}
\imagen{prediccion}{\textit{Mockup} de predicción}{1}

Para contrastar con los \textit{mockups} iniciales, también se adjuntan las interfaces definitivas del proyecto. Estas interfaces representan la versión final del diseño de la aplicación y reflejan los cambios y mejoras implementados para lograr una experiencia de usuario óptima y satisfactoria. 

El cambio más relevante a destacar se encuentra en la interfaz principal. Se puede apreciar que el menú de selección del año ha desaparecido. Esta decisión es lógica puesto que los datos de los que disponemos en la base de datos son limitados. 

\imagen{inicio_sesion2}{Interfaz definitiva de inicio de sesión}{1}
\imagen{registro2}{Interfaz definitiva de registro}{1}
\imagen{principal2}{Interfaz definitiva de página principal}{1}
\imagen{prediccion2}{Interfaz definitiva de predicción}{1}

Finalmente, es importante destacar que el diseño de las interfaces con CSS ha sido creado desde cero, sin utilizar plantillas preexistentes. Esta decisión ha permitido una mayor flexibilidad y adaptación a las necesidades específicas del proyecto. Para asegurar la responsividad de la aplicación, se han empleado elementos de Bootstrap, garantizando así una experiencia de usuario consistente en diferentes dispositivos y tamaños de pantalla. Además, tanto el logo de la aplicación como la animación inicial han sido creados por la alumna utilizando Photoshop, lo que añade un toque personal y distintivo al aspecto visual de la aplicación.