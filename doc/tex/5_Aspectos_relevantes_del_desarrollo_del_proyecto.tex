\capitulo{5}{Aspectos relevantes del desarrollo del proyecto}

%Este apartado pretende recoger los aspectos más interesantes del desarrollo del proyecto, comentados por los autores del mismo.
%Debe incluir desde la exposición del ciclo de vida utilizado, hasta los detalles de mayor relevancia de las fases de análisis, diseño e implementación.
%Se busca que no sea una mera operación de copiar y pegar diagramas y extractos del código fuente, sino que realmente se justifiquen los caminos de solución que se han tomado, especialmente aquellos que no sean triviales.
%Puede ser el lugar más adecuado para documentar los aspectos más interesantes del diseño y de la implementación, con un mayor hincapié en aspectos tales como el tipo de arquitectura elegido, los índices de las tablas de la base de datos, normalización y desnormalización, distribución en ficheros3, reglas de negocio dentro de las bases de datos (EDVHV GH GDWRV DFWLYDV), aspectos de desarrollo relacionados con el WWW...
%Este apartado, debe convertirse en el resumen de la experiencia práctica del proyecto, y por sí mismo justifica que la memoria se convierta en un documento útil, fuente de referencia para los autores, los tutores y futuros alumnos.

En esta sección se presenta un resumen de los hallazgos más relevantes obtenidos a lo largo del proyecto, así como una descripción detallada de las distintas fases por las que se ha atravesado hasta lograr alcanzar una solución satisfactoria al problema planteado. A saber, la fase de extracción de datos, la fase de aprendizaje automático y la fase de creación de la aplicación web.

\imagen{ciclodevidaproyecto}{Fases del proyecto}{0.7}

\section{Extracción de la información}
La primera fase de este proyecto es la más extensa y consiste en la extracción de datos para alimentar la base de datos con la que entrenar los modelos. Para esto, es necesario tener en cuenta varias bases de datos bibliográficas que ofrecen diversos recursos para los investigadores. Entre las más importantes se encuentran Google Scholar, Scopus, WoS y Crossref.

\subsection{Prototipos para Google Scholar}

De todas las bases de datos bibliográficas mencionadas, la más destacada es Google Scholar, puesto que crece más rápido que cualquier base de datos tradicional en todos los campos científicos~\cite{harzing2010, lopez2017}.

Así pues, el primer prototipo diseñado consiste en la extracción de datos de Google Scholar. Sin embargo, si bien es cierto que Google Scholar es un motor de búsqueda gratuito, universal y rápido con una amplia cobertura, tiene muchas limitaciones~\cite{lopez2017}. 


Por ejemplo, no hay funcionalidades de exportación de datos o API disponibles debido a restricciones comerciales. Además, solo se pueden mostrar los primeros 1\,000 resultados de cada consulta. Aunque algunas técnicas como los retrasos temporales o el uso de \textit{proxies} pueden ayudar, no resuelven completamente estas limitaciones. Además, la extracción automatizada de estos datos va en contra de las políticas del archivo \texttt{robots.txt} de Google Scholar, lo que hace que los usuarios que realizan demasiadas consultas automatizadas sean bloqueados cada 200 solicitudes detectadas~\cite{lopez2017}.

Durante el avance del proyecto, nos dimos cuenta de muchas de las limitaciones mencionadas.

\subsubsection{Prototipo inicial}

Tras comprender que las complicaciones eran numerosas, se decidió crear un prototipo sencillo, consistente en un \textit{script} en Python, que trataría de lanzar mil peticiones de búsqueda. Esto nos permitiría establecer los límites de realizar \textit{web scrapping} sobre Google Scholar.

Así pues, manos a la obra, se desarrolló un \textit{script} sencillo, que solicita acceso a la página principal de Google Scholar y, mediante métodos HTTP, realiza una búsqueda (a partir de parámetros solicitados por pantalla). Finalmente, extrae el título de la página resultante tras hacer la búsqueda. Todo esto se logra haciendo uso de la biblioteca de Python BeautifulSoup.

% Filtro por revistas

El primer obstáculo encontrado es que Google Scholar no permite hacer búsquedas en función de la revista. Si se trata de escribir el nombre de una revista en el buscador, aparecerán artículos relacionados que mencionan esa revista, pero no siempre artículos de la revista en cuestión. En cambio, permite hacer búsquedas a partir de palabras clave.
Sin embargo, aunque no se puede buscar directamente por revista, sí se puede extraer de los resultados encontrados tras buscar una temática concreta.

\imagen{ejemploResultadoGS}{Ejemplo de búsqueda por palabra clave}{0.8}

Sin embargo, existe una problemática: en cada artículo se hace referencia a la revista que lo publica de forma distinta. Por ejemplo, como se puede apreciar en la figura~\ref{fig:etiquetaDeRevista}, la revista se encuentra ubicada en una <<etiqueta>> HTML diferente.

\imagen{etiquetaDeRevista}{Localización de la revista en Google Scholar}{0.8}

Por ello se buscó una solución alternativa.
Existe una etiqueta (\texttt{<b> … </b>}) que incluye aquellas palabras resaltadas en negrita en la página web. Google Scholar resalta en una búsqueda aquellas palabras que coinciden con alguno de los parámetros buscados. Así pues, si se realiza una búsqueda en Google Scholar pasando como parámetro el nombre de la revista que nos interesa, bastará con buscar en la clase  \texttt{gs\_a} (donde se almacenan los detalles del artículo) alguna etiqueta \texttt{<b>}.
Se ha escrito un pequeño código de prueba y, en principio, parece que funciona sin problemas. Se adjunta dicho código (\texttt{extraerRevistas.py}) en el repositorio de GitHub. %TODO: meter enñace
Este código hace dos búsquedas: una a la revista <<Alas Peruanas>> y otra a la revista <<The Lancet>> (elegidas de forma aleatoria).

%La salida obtenida (solo de la primera página de resultados) es la siguiente:

%\imagen{salidaObtenida}{Salida por consola}{1}

%La salida por pantalla mostrada en la imagen anterior sigue la siguiente estructura: [\texttt{‘nombre artículo’, ‘nombre de la revista’, número\_citas, fecha, id}]
La salida que se mostrará por pantalla sigue la siguiente estructura: [\texttt{‘nombre artículo’, ‘nombre de la revista’, número\_citas, fecha, id}]

De igual forma se recogen los resultados en los CSVs (codificados en UTF-8) con nombre: \texttt{BBDD1.csv} y \texttt{BBDD2.csv}.


% Paginación

Otra de las mejoras que se implementa en el prototipo es la capacidad de navegar a través de las distintas páginas de resultados de Google Scholar. Para ello, se incluye un nuevo campo en la consulta de búsqueda que irá aumentando de 10 en 10 por cada página de resultados nueva que se consulte.
Ahora ya se obtiene un número de resultados considerable (aproximadamente 100). Estos resultados se guardarán también en los CSVs (codificados en UTF-8) con los siguientes nombres: \texttt{BBDD3.csv} y \texttt{BBDD4.csv}.

% CAPTCHA y proxies

Tras diseñar y programar el código mencionado, se procede a su prueba. La primera ejecución del \textit{script} resultó desaletadora, ya que, a partir de la solicitud número 726, Google Scholar detecta que un \textit{bot} está realizando búsquedas. A partir de ese momento, nuestra dirección IP queda bloqueada y las solicitudes fallan sin excepción. Google Schoolar nos redirecciona a una página (figura \ref{fig:error}) donde se solicita al usuario resolver un \textit{captcha}. 

\imagen{error}{Captura de reCAPTCHA de Google}{0.8}

El número de solicitudes exitosas es demasiado bajo para cumplir su función en nuestro proyecto, por lo que se procede a buscar una solución alternativa. 

Tras distintas pruebas, se encontró una forma de superar la barrera del \textit{captcha}. A saber: añadiendo a la \textit{url} una sección de texto extra que permite suprimir esta excepción durante un periodo concreto de tiempo. Así pues, logramos ejecutar con éxito el \textit{script} tantas veces como fuese necesario. Sin embargo, esta solución tampoco es válida a largo plazo. Si se intenta ejecutar de nuevo el programa, en otra sesión, la dirección vuelve a ser inválida. Además, es un remedio poco práctico, ya que se debe concatenar distintos parámetros y cadenas de texto que, dependiendo del momento y el contexto en que se ejecute, pueden no servir.

Como la propuesta anterior no fue satisfactoria, se siguieron buscando opciones. La siguiente propuesta consistía en usar un agente de usuario distinto para cada solicitud\footnote{Un agente de usuario es cualquier software, que actúa en nombre de un usuario, que <<recupera, presenta y facilita la interacción del usuario final con el contenido web>>. Algunos ejemplos destacados de agentes de usuario son los navegadores web. La cadena User-Agent es uno de los criterios por los cuales los rastreadores web pueden ser excluidos del acceso a ciertas partes de un sitio web utilizando el Estándar de exclusión de robots (\texttt{archivo robots.txt}).}. De esta forma <<enmascaramos>> nuestra dirección IP. La biblioteca Request de Python ofrece métodos para lograrlo. Dicho esto, se implementó un método que extrae \textit{proxies} de listas públicas y gratuitas de Internet (e.g., \href{https://www.proxyscrape.com/free-proxy-list}{proxyscrape.com}). Se prueba la ejecución del prototipo una vez más y, finalmente, funciona sin inconvenientes. Ahora ya se puede decir que el proyecto es \textbf{viable}.


% DOI y Crossref
\subsubsection{Prototipo con extracción del DOI}

La siguiente meta de nuestro prototipo es la obtención del DOI, que es un campo fundamental que funcionará a modo de clave primaria en la base de datos. De esta forma se busca evitar futuros errores a la hora de identificar un artículo o a la hora de comparar artículos para eliminar duplicados. Sin embargo, se plantea una problemática al respecto: Google Scholar no comparte el DOI de los artículos en ninguna división de su página web (si bien es cierto que algunos artículos lo incluyen al final de su URL, pero no siempre ocurre esto). Puesto que por el momento no existe una forma de obtenerlo directamente, se han propuesto varias soluciones, a saber:
\begin{itemize}
	\item Comprobar el porcentaje de éxito de extraer el DOI de la URL de los artículos (en aquellos casos en los que aparezca).
	\item Acceder a la página del artículo en cuestión y extraerlo de dicha página.
         \item Introducir el nombre del artículo en la página de \href{https://www.crossref.org/}{Crossref} o similares, que te permiten obtener el DOI del artículo cuyo nombre pases como parámetro.
\end{itemize}
Como conclusión, se descarta la primera opción tras hacer una breve prueba, ya que falla enseguida. Se descarta también la segunda opción, ya que cada página sitúa el DOI en un sitio haciendo muy difícil la búsqueda del mismo a través de un algoritmo. Por lo tanto, la opción más adecuada en este caso es la tercera. Aunque supone una búsqueda extra en la complejidad algorítmica del prototipo, es la única forma segura de calcular el DOI sin equivocaciones. Por lo tanto, se incluye en el bucle del prototipo un nivel extra de profundidad de búsqueda.

Para hacer la búsqueda en Crossref es necesario indicar tanto el título como el año de publicación, a fin de asegurarnos de que se obtiene exactamente el artículo que deseamos y no otros similares. Además, es preciso tener en cuenta que Crossref no permite a los programas hacer búsquedas desde la misma interfaz que los usuarios (para evitar que los programas bloqueen las búsquedas de los usuarios). Por lo tanto, se realizarán las búsquedas en la dirección donde se ubica la API creada por Crossref especialmente para la extracción de datos por parte de programas.

% Multihilo
\subsubsection{Prototipo multihilo}

Con el prototipo listo, se prueba a extraer la información de los artículos de las primeras 20 páginas de resultados de Google Scholar de dos revistas. Los resultados no son muy esperanzadores: se obtienen 180 artículos de cada revista en 15 minutos. Estos resultados no son eficientes, por lo que se tratará de optimizar el prototipo siguiendo dos pautas. 

\begin{itemize}
    \item La primera pauta es tratar de extraer las llamadas a Crossref de forma que solo se tenga que realizar una única llamada una vez que se han extraído el resto de detalles sobre los artículos.
    \item La segunda pauta es emplear programación concurrente para ejecutar varios hilos al mismo tiempo.
\end{itemize}

Así, cada revista será procesada por un hilo distinto. Para ello, será necesario recurrir a la librería de Python \textit{multiprocessing}. Tras actualizar estos cambios, el prototipo obtiene resultados mucho más rápidos: obtiene los 180 artículos de ambas revistas en tan solo 8 minutos. El próximo objetivo es determinar el número máximo de hilos que se pueden lanzar sin impactar negativamente el rendimiento del prototipo.

Para ello, se solicitó permiso para acceder a los servidores de la universidad. A través de SSH, nos conectamos a una de las máquinas y, utilizando un entorno virtualizado de Miniconda, comenzamos a lanzar hilos. Sin embargo, el resultado fue desastroso, ya que al realizar 200 llamadas a Google Scholar, nos bloqueaban y teníamos que resolver un CAPTCHA. Intentamos insertar esperas de tiempo aleatorias y cambiar el user-agent en cada una de las llamadas. Incluso se probó a cambiar el término de búsqueda. Sin embargo, nada de esto funcionó. Matemáticamente, después de aproximadamente 200-250 llamadas, aparecía el CAPTCHA.


\subsubsection{Prototipo con Selenium}
La única solución posible frente a las restricciones de Google Scholar es tratar de <<humanizar>> las búsquedas para que parezca que las realiza un ser humano. Por lo tanto, se decidió emplear Selenium, que nos permite emular los pasos que realizaría un usuario normal al hacer una búsqueda. El nuevo prototipo entra directamente en el navegador con la ruta de la búsqueda, lo que ahorra la necesidad de realizar llamadas adicionales. Luego, hace clic en la sección citar de cada uno de los resultados de la búsqueda y aparece un cuadro emergente del cual se puede extraer la información completa de la cita. Esto resuelve otro de los problemas que teníamos con el prototipo anterior: Google Scholar acorta los nombres largos añadiendo puntos suspensivos, mientras que en la sección citar aparecen los nombres completos. 

Después de extraer esta información, se cierra el cuadro y se procede a realizar la misma acción en el siguiente artículo. Cuando se llega al décimo resultado (ya que solo hay 10 artículos por página de resultados), se emula un clic en el botón <<Siguiente>> y se procede a realizar lo mismo en la siguiente página de resultados, todo con pausas aleatorias de tiempo entre cada acción. Posteriormente, se procesan los datos de las citas usando expresiones regulares y transformando los diccionarios de datos en un CSV.

Como nota adicional, cabe mencionar que, si se incluye la etiqueta \texttt{source: <nombre de la revista>} en la búsqueda, es posible hacer una búsqueda solo por el nombre de la revista, algo que desconocíamos hasta ahora.

Aunque este método ha probado ser mejor que el anterior, ya que se obtuvieron casi 300 resultados, Google Scholar detecta que son consultas automatizadas y salta el CAPTCHA al final.

Para intentar resolver esta situación, se ha intentado superar el CAPTCHA emulando un clic sobre el botón correspondiente. No obstante, para lograrlo es necesario superar varias capas ocultas que se encuentran sobre dicho botón, lo cual resulta complejo. Una vez se logra acceder al botón, aparecen las imágenes que se deben reconocer y esto no se puede automatizar de manera sencilla. Se llega a la conclusión de que no es posible forzar más llamadas de las permitidas en Google Scholar, tal y como se ha mencionado en otros trabajos, como en el caso de \textit{Publish or Perish}~\cite{harzing2010}.

\subsection{Web of Science y Scopus}

Se consideraron también estas dos potentes fuentes bibliográficas. Sin embargo, para acceder a los datos a través de sus APIs correspondientes, se requiere una licencia que permita su uso. A pesar de haber solicitado dichas licencias, las limitaciones y restricciones de ambas APIs (tanto en el ámbito legal como en el económico) han impedido la realización de prototipos que cumplan con los objetivos requeridos para este proyecto. Finalmente, tras varios intentos, se terminan descartando ambas opciones.

\subsection{Prototipos para Crossref}

Vista la imposibilidad de obtener una cantidad aceptable de resultados de Google Scholar, Scopus y WoS, se decidió cambiar la fuente de datos. En este caso, Crossref (aunque no es una base de datos tan potente, tiene una API gratuita y sin tantas restricciones).

Este modelo nos permite la extracción de una gran cantidad de datos en un tiempo reducido. Para ilustrar la eficiencia de este modelo, se ha generado una gráfica donde se muestra el tiempo que se ha tardado en extraer los datos (de los últimos 20 años) de cada revista de la categoría de \textit{Computer Science}.

\imagen{grafica_tiempos}{Tiempo de extracción de datos de Crossref}{1}

Sin embargo, puesto que los datos que se extraen son menos completos y exactos, posteriormente se deberá tratar el margen de error al calcular el Índice de Impacto. Además, es preciso mencionar que, según retrocedemos en el tiempo, la escasez de datos disponibles en Crossref aumenta. Así, resulta evidente que la fiabilidad del cálculo del Índice de Impacto disminuye como consecuencia.

\section{Aprendizaje automático}
La segunda fase del proyecto consiste en predecir y estimar, a partir de los datos obtenidos, el valor del Índice de Impacto de cada revista seleccionada.

\subsection{Cálculo del JCR}
Antes de comenzar a predecir el JCR, se ha desarrollado un algoritmo para calcular el Índice de Impacto de las revistas científicas a partir de los datos extraídos en la fase anterior. Para comprobar la exactitud de los resultados obtenidos, se ha generado un gráfico de cajas en el que se contrastan los valores obtenidos con los datos reales del JCR. 

\imagen{diagrama_cajas}{Dispersión entre las diferencias del JCR calculado y el real en función del año}{1}

Al analizar el gráfico, se observa claramente que los años más antiguos presentan un margen de error mucho más elevado debido a la falta de datos de Crossref en esos años. De esta manera, se puede asegurar que el módulo desarrollado es preciso y confiable, aunque se debe tener en cuenta la limitación de la falta de datos para los años más antiguos.

Dadas las circunstancias, se ha tomado la decisión de utilizar solamente los datos de los años más recientes para entrenar los modelos de aprendizaje automático. Esta medida se ha tomado debido a la limitación de datos disponibles para los años más antiguos, que resultaría en un margen de error demasiado elevado si se incluyeran en el entrenamiento de los modelos. Aunque esto pueda suponer una pérdida de información valiosa, se considera que es preferible tener resultados más precisos y confiables al utilizar datos más actualizados. Con esta estrategia, se espera obtener resultados más precisos y útiles.

\subsection{Modelos de predicción}

Se han probado varios modelos regresores utilizando la librería Scikit-learn, con el objetivo de encontrar el que mejor se ajuste a los datos disponibles y pueda hacer predicciones más exactas. Además, estos modelos se han evaluado mediante el error cuadrático medio\footnote{RMSE (Root Mean Square Error) es una métrica ampliamente utilizada para evaluar la precisión o el rendimiento de un modelo de regresión. Representa la raíz cuadrada de la media de los errores cuadrados entre los valores predichos por el modelo y los valores reales del conjunto de datos.}. A continuación, se enumeran los distintos modelos probados:
\begin{itemize}
    \item Linear Regression 
    \item Random Forest Regressor
    \item AdaBoost Regressor
    \item XGB Regressor
    \item Support Vector Machine Regressor (SVM)
    \item Multi Layer Perceptron (MLP)
    \item Stacking Regressor
\end{itemize}

Se han seleccionado estos modelos debido a su popularidad y a su variedad, que abarca una amplia gama de métodos disponibles.

Además se ha tenido en cuenta la relevancia y los hallazgos presentados en el artículo titulado <<Do We Need Hundreds of Classifiers to Solve Real World Classification Problems?>>~\cite{fer2014}. Este estudio evaluó 179 clasificadores provenientes de 17 familias diferentes, abarcando una amplia variedad de métodos, incluyendo análisis discriminante, Bayesianos, redes neuronales, máquinas de soporte vectorial, árboles de decisión, clasificadores basados en reglas, técnicas de \textit{boosting}, \textit{bagging}, \textit{stacking}, modelos lineales generalizados, vecinos más cercanos, regresión de múltiples splines de adaptación y muchos otros métodos. 

Los resultados de este estudio destacaron que los clasificadores de Random Forest fueron los más prometedores, alcanzando una precisión máxima del 94,1\% en los conjuntos de datos utilizados, superando el 90\% en el 84,3\% de los casos. Además, se encontró que los clasificadores SVM con núcleo gaussiano, también obtuvieron una alta precisión del 92,3\%. Estos hallazgos respaldan la decisión de considerar estos modelos en este estudio, ya que se ha demostrado que ofrecen un rendimiento destacado en comparación con otros clasificadores.

Por otro lado, hay otros artículos que sugieren la utilización de los modelos XGBoost y Stacking en proyectos de aprendizaje automático. El artículo titulado <<Getting Started with XGBoost in scikit-learn>>~\cite{Wade2020} destaca que XGBoost es un algoritmo de aprendizaje automático que ha ganado popularidad debido a su desempeño sobresaliente en competiciones de Kaggle y en la predicción de datos tabulares. XGBoost es un modelo de \textit{ensemble} que combina varios modelos de aprendizaje en uno solo, ofreciendo resultados superiores a los modelos individuales. Además, se destaca su capacidad de regularización y velocidad de procesamiento, lo que lo convierte en una opción atractiva para aplicaciones prácticas.

Por su parte, el artículo <<Stacked generalization: an introduction to super learning>>~\cite{naimi2018} presenta el método de \textit{Stacked Generalization}, también conocido como \textit{Super Learner}. Este enfoque permite combinar varios algoritmos de predicción en un único modelo. Utilizando la validación cruzada, se busca obtener una combinación óptima de las predicciones de una biblioteca de algoritmos candidatos. La optimización se realiza mediante una función objetivo especificada por el usuario, como minimizar el error cuadrático medio o maximizar el área bajo la curva característica de operación del receptor. Aunque la implementación de \textit{Super Learner} puede tener ciertas complejidades conceptuales y técnicas, su uso ha demostrado ser beneficioso en diversas aplicaciones y puede ofrecer mejoras significativas en la precisión de las predicciones.

En nuestro caso, tanto XGBoost como Stacking han demostrado ser enfoques efectivos en la predicción del JCR, y su inclusión se justifica por su rendimiento sobresaliente y su potencial para mejorar la precisión de las predicciones en este proyecto.



\subsubsection{Conjunto de datos}
Para el conjunto inicial de datos, se ha realizado una selección cuidadosa de entre toda la información extraída, eligiendo los siguientes atributos: el número de citas, el factor de impacto JCR y la diferencia entre los datos extraídos y los valores reales. Estos atributos han sido tomados exclusivamente de los años comprendidos entre el 2018 y el 2020. Además, se ha incluido el número de citas correspondiente al año 2021. La elección de estos años se basa en la observación de que presentan menor error en los datos extraídos, lo cual contribuye a mejorar la calidad de los atributos seleccionados. Finalmente, se realizará la predicción del Factor de Impacto JCR para el año 2021.

\imagen{pipeline}{Representación esquemática de los datos de entrenamiento ($X$,$y$)}{1}

En la Figura \ref{fig:pipeline}, $X$ representa las variables de entrada al conjunto de entrenamiento e $y$ la clase a predecir.

Por otro lado, cuando se trata de valores nulos o vacíos (i.e.,\textit{missing values}), se ha utilizado la mediana como método imputación de valores.

\subsubsection{Entrenamiento de los modelos}
Para el entrenamiento, se ha llevado a cabo una técnica de validación cruzada anidada (\textit{nested cross-validation}) con el fin de evaluar y seleccionar el mejor modelo posible para predecir el Índice de Impacto de las revistas científicas. Esta técnica implica el uso de dos niveles de validación cruzada: en el nivel externo se evalúa el desempeño general del modelo, mientras que en el nivel interno se ajustan los parámetros del modelo mediante una búsqueda en rejilla (\texttt{grid search}) para encontrar la mejor combinación de hiperparámetros. La ventaja de esta técnica es que permite evaluar la capacidad de generalización del modelo de manera más realista, evitando la selección de modelos sobreajustados (\textit{overfitting}). En comparación con una validación cruzada al uso, la técnica de validación cruzada anidada puede resultar más costosa en términos de cómputo, pero proporciona resultados más precisos y fiables.

\imagen{cv}{Esquema de la validación cruzada anidada (5 repeticiones, 2 grupos)}{1}

\subsubsection{Evaluación de los modelos}
Finalmente, tras la evaluación de todos los modelos, se seleccionan aquellos que han obtenido mejores resultados. Como se puede observar en el siguiente gráfico (ver Figura \ref{fig:mse}), los modelos con mayor precisión son AdaBoost Regressor, Random Forest Regressor y XGB Regressor.

\imagen{mse}{Estimación del RMSE de los modelos evaluados en la experimentación}{0.7}

Los motivos por los cuales se han obtenido estos resultados pueden ser diversos. Por un lado, podemos ver que los modelos de \textit{ensemble} como AdaBoost, Random Forest y XGBoost obtienen, en general, mejores resultados. Esto puede deberse a su capacidad para capturar relaciones no lineales en los datos. Así, pueden modelar relaciones complejas entre las variables de entrada y la variable de salida, lo cual es especialmente útil cuando existen patrones no lineales en los datos. 

Por otro lado, los algoritmos de \textit{ensemble} tienden a reducir el sobreajuste en comparación con modelos individuales como MLP y SVM (los algoritmos de \textit{ensemble} combinan múltiples modelos más simples, lo que ayuda a mitigar el sesgo y la variabilidad inherente a un solo modelo).

Estos algoritmos suelen ser, también, más robustos ante la presencia de ruido o valores atípicos en los datos (como ocurre en nuestro caso). Utilizan técnicas como el muestreo \textit{bootstrap} y la combinación de múltiples árboles de decisión, lo que les permite ser menos sensibles a observaciones atípicas y errores de medición.

En cambio, MLP y SVM a menudo requieren una cuidadosa normalización y escala de los datos de entrada para un rendimiento óptimo. Además, MLP puede ser más sensible a la selección de hiperparámetros y puede requerir una búsqueda más exhaustiva de la configuración adecuada, aumentando así el tiempo de entrenamiento.

\subsubsection{Resultados}
Tras analizar los resultados de esta etapa, para poder hacer uso de ellos más adelante, se guardan en los siguientes ficheros:
\begin{itemize}
    \item Fichero CSV con los resultados de la validación cruzada. Por cada iteración incluye el modelo que se estima, el valor de los parámetros con mejores resultados y el RMSE.
    \item Ficheros binarios \textit{pickle}, donde se almacenan cada uno de los modelos entrenados. Posteriormente, se almacenarán en la base de datos para poder realizar predicciones desde la aplicación web.
\end{itemize}


\section{Creación de la aplicación web}

Se pueden destacar algunos aspectos del proceso de creación de la aplicación web, especialmente los relacionados con la base de datos y el servidor de aplicaciones en la nube Heroku. 

Uno de estos aspectos es la imposibilidad de realizar con Heroku Postgres\footnote{Heroku Postgres es una solución de alojamiento y gestión de bases de datos PostgreSQL proporcionada por Heroku.} operaciones para optimizar la base de datos (i.e.: sentencias como \texttt{ANALIZE}), probablemente por una cuestión de permisos de usuario.

Otro aspecto relevante en cuanto a Heroku es el comportamiento inconstante en la carga de las variables de sesión. En este caso, se están considerando variables de sesión tanto el idioma como el nombre del usuario que ha iniciado sesión. En algunas ocasiones, estas variables se cargan correctamente, mientras que en otras ocasiones no se cargan como se espera. Esto es particularmente desconcertante, ya que se espera que el comportamiento de las variables de sesión sea determinista y consistente. Se sospecha que este problema está relacionado con la velocidad de respuesta de la aplicación alojada en Heroku. Es posible que, debido a la naturaleza distribuida de la plataforma y a las variaciones en la carga del servidor, la respuesta del sistema para cargar las variables de sesión no siempre sea uniforme. Esto puede dar lugar a una experiencia inconsistente para los usuarios, ya que las variables de sesión pueden o no estar disponibles en diferentes momentos.

Por otro lado, en relación con la librería \textit{psycopg2} (que se ha usado para realizar la conexión a la base de datos), se ha descubierto (utilizando la función \texttt{copy\_from}) que no admite nombres de columnas en mayúsculas, lo cual es incompatible con la notación de PostgreSQL. En la documentación de psicopg2 se asegura que \texttt{copy\_from} funciona con la misma sintaxis de PostgreSQL, mientras que, en la documentación de PostgreSQL encontramos que <<\textit{Key words and unquoted identifiers are case insensitive}>>~\cite{bug}. Se ha notificado este error en la \textit{issue} \href{https://github.com/psycopg/psycopg2/issues/1581}{\#1581}.









