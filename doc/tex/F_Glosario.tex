\apendice{Glosario}

\textbf{AngularJS}: Entorno de trabajo de JavaScript para creación una página web simple, emulando una aplicación de escritorio.

\textbf{API}: Acrónimo de \textit{Application Programming Interface} (Interfaz de Programación de Aplicaciones). Es un conjunto de reglas y protocolos que permite la interacción entre diferentes aplicaciones de \textit{software}.

\textbf{API REST}: Una API basada en los principios del protocolo HTTP y diseñada para facilitar la comunicación y transferencia de datos entre sistemas. REST significa \textit{Representational State Transfer} (Transferencia de Estado Representacional).

\textbf{Babel}: Una herramienta de traducción y localización de software que permite adaptar una aplicación para soportar diferentes idiomas y regiones.

\textbf{Backend}: La parte de un sistema o aplicación que se encarga del procesamiento y almacenamiento de datos, así como de la lógica de negocio. Normalmente, se refiere a la parte del sistema que no es visible para los usuarios finales.

\textbf{BeautifulSoup4}: Una biblioteca de Python que facilita el análisis y extracción de datos de documentos HTML y XML.

\textbf{Bootstrap}: Un \textit{framework} de desarrollo web que proporciona estilos CSS y componentes JavaScript predefinidos, lo que facilita la creación de interfaces web responsivas y atractivas.

\textit{\textit{Breadcrumbs}}: En español migas de pan. Son un elemento de navegación utilizado en sitios web y aplicaciones para indicar la ubicación actual del usuario dentro de la estructura del sitio o la aplicación. Toman su nombre de la historia del cuento de Hansel y Gretel, donde los personajes dejaron un rastro de migas de pan para encontrar el camino de regreso a casa.

\textbf{Chartjs}: Una biblioteca JavaScript para la creación de gráficos interactivos y visualmente atractivos en páginas web.

\textbf{Crossref}: Una organización sin fines de lucro que proporciona servicios y datos relacionados con la identificación y el enlace de contenido académico.

\textbf{Crossvalidation}: Un método utilizado en aprendizaje automático (\textit{machine learning}) para evaluar y validar el rendimiento de un modelo utilizando diferentes subconjuntos de datos.

\textbf{Cryptography}: Una biblioteca de Python que proporciona herramientas para la implementación de algoritmos criptográficos, como el cifrado y la generación de firmas digitales.

\textbf{CSS}: Acrónimo de \textit{Cascading Style Sheets} (Hojas de Estilo en Cascada). Es un lenguaje utilizado para describir la presentación y el estilo de un documento HTML.

\textbf{Cuartil}: En estadística, un cuartil es un valor que divide un conjunto de datos ordenados en cuatro partes iguales. Los cuartiles se utilizan para analizar y describir la distribución de un conjunto de datos, especialmente en el contexto de medidas de tendencia central y dispersión.

\textbf{D3js}: Una biblioteca JavaScript para la creación de visualizaciones de datos interactivas y dinámicas en páginas web.

\textbf{DataTables}: Un complemento de JavaScript que proporciona funcionalidad avanzada de tablas interactivas y filtrado en páginas web.

\textbf{DOI}: Acrónimo de \textit{Digital Object Identifier} (Identificador de Objeto Digital). Es un identificador único utilizado para identificar de manera persistente un objeto digital, como un artículo científico.

\textbf{\textit{Dropdown}}: Un \textit{dropdown}, también conocido como menú desplegable o lista desplegable, es un elemento de interfaz gráfica que permite al usuario seleccionar una opción de una lista desplegable de opciones. 

\textbf{Enrutamiento}: El enrutamiento (en el contexto de los \textit{frameworks web}) se refiere a la forma en que una aplicación web maneja y responde a las solicitudes de los clientes para diferentes URL o rutas. En otras palabras, el enrutamiento determina cómo se mapean las URL entrantes a las funciones o controladores correspondientes en la aplicación.

\textbf{Flask}: Un \textit{framework} de desarrollo web minimalista y flexible para Python. Permite construir aplicaciones web rápidas y escalables.

\textbf{flask-babel}: Una extensión de Flask que simplifica la internacionalización y localización de aplicaciones web.

\textbf{Flask-Cors}: Una extensión de Flask que permite manejar la política de intercambio de recursos entre dominios (CORS) en aplicaciones web.

\textbf{Flask-Login}: Una extensión de Flask que proporciona funcionalidad de autenticación de usuarios y gestión de sesiones.

\textbf{Flask-Migrate}: Una extensión de Flask que facilita la gestión de migraciones de base de datos utilizando SQLAlchemy.

\textbf{Frontend}: La parte de un sistema o aplicación que interactúa directamente con los usuarios finales. Se refiere a la interfaz de usuario y la presentación visual.

\textbf{Frossrefapi}: Un paquete de Python que proporciona una interfaz para interactuar con la API de Crossref y acceder a metadatos de publicaciones académicas.

\textbf{Frozenlist}: Un paquete de Python que proporciona una lista inmutable (no modificable) que garantiza la integridad de los datos.

\textbf{Git}: El sistema de control de versiones más popular. Registra los cambios en los archivos de los programas, permitiendo la gestión de distintas versiones de código y la colaboración entre desarrolladores.

\textbf{Google Scholar}: Un motor de búsqueda especializado en literatura académica y científica. Proporciona acceso a artículos, tesis, resúmenes y otros recursos académicos.

\textbf{Gráfico \textit{burndown}}: Un gráfico utilizado en la metodología ágil para representar el progreso de un proyecto a lo largo del tiempo, mostrando el trabajo pendiente y la tendencia de finalización.

\textbf{Gunicorn}: Un servidor web HTTP para aplicaciones Python. Es comúnmente utilizado para implementar aplicaciones Flask en producción.

\textbf{Heroku}: Una plataforma en la nube que permite implementar, gestionar y escalar aplicaciones web.

\textbf{HTML}: Acrónimo de \textit{Hypertext Markup Language} (Lenguaje de Marcado de Hipertexto). Es el lenguaje estándar utilizado para crear páginas web.

\textbf{Httpcore}: Una biblioteca Python que proporciona una interfaz para realizar solicitudes HTTP de bajo nivel.

\textbf{Httpx}: Una biblioteca Python que proporciona una interfaz de alto nivel para realizar solicitudes HTTP.

\textbf{Idna}: Un módulo de Python que proporciona herramientas para la manipulación de nombres de dominio internacionalizados (IDN).

\textbf{IF}: Acrónimo de \textit{Impact Factor} (Factor de Impacto). Es una métrica utilizada para evaluar la importancia relativa de una revista científica dentro de su campo.

\textbf{Índice de Impacto}: Una medida que indica la influencia relativa de una revista científica. Se basa en la frecuencia con la que los artículos de la revista son citados por otros investigadores.

\textbf{Ipykernel}: Un kernel de IPython que permite ejecutar código Python en el entorno de Jupyter Notebook.

\textbf{Ipython}: Un entorno interactivo de programación y exploración de datos que proporciona características adicionales sobre el intérprete de Python estándar.

\textbf{JavaScript}: Un lenguaje de programación interpretado utilizado principalmente para agregar interactividad y dinamismo a las páginas web.

\textbf{JCR}: Acrónimo de \textit{Journal Citation Reports}. Es una base de datos que recopila información sobre las citas y el impacto de las revistas científicas.

\textbf{Jedi}: Una biblioteca de Python que proporciona funcionalidades avanzadas de autocompletado y análisis estático para el código Python.

\textbf{Jinja2}: Un motor de plantillas de Python utilizado en el framework Flask para generar contenido dinámico en las aplicaciones web.

\textbf{Joblib}: Una biblioteca de Python utilizada para la serialización y paralelización de tareas, especialmente en el contexto de machine learning.

\textbf{jQuery}: Una biblioteca de JavaScript rápida, pequeña y rica en características que simplifica la manipulación y el manejo de eventos en el código HTML.

\textbf{JSON}: JavaScript Object Notation, lenguaje de marcado que permite el intercambio de datos con texto sencillo.

\textbf{Jupyter\_client}: Un cliente de Jupyter Notebook que permite la comunicación con los kernels y el manejo de sesiones interactivas.

\textbf{Jupyter\_core}: Un conjunto de funcionalidades esenciales para el funcionamiento de los entornos Jupyter, incluyendo la gestión de notebooks y configuraciones.

\textbf{Kanban}: Una metodología ágil de gestión de proyectos que se centra en la visualización del flujo de trabajo y la optimización de la productividad.

\textbf{\LaTeX}: Un sistema de composición de documentos utilizado principalmente para la creación de documentos científicos y técnicos con alta calidad tipográfica.

\textbf{Logo}: Abreviatura de logotipo. Es un símbolo gráfico o una representación visual que se utiliza para identificar una marca, empresa, organización o producto. 

\textbf{Machine Learning}: Un campo de estudio de la inteligencia artificial que se centra en el desarrollo de algoritmos y modelos que permiten a las computadoras aprender y tomar decisiones basadas en datos.

\textbf{MariaDB}: Un sistema de gestión de bases de datos relacionales (RDBMS) derivado de MySQL. Ofrece compatibilidad con la mayoría de las características de MySQL y mejoras adicionales.

\textbf{Matplotlib}: Una biblioteca de Python ampliamente utilizada para la creación de gráficos estáticos, gráficos 2D y 3D, y visualizaciones de datos.

\textbf{Matplotlib-inline}: Una extensión de Jupyter Notebook que permite la visualización de gráficos Matplotlib de forma integrada en el notebook.

\textbf{Metadatos}: Datos que describen características o propiedades de un objeto. En el contexto de los documentos académicos, los metadatos pueden incluir información sobre los autores, título, resumen, fecha de publicación, etc.

\textbf{MIAR}: Acrónimo de \textit{Matriz de Información para el Análisis de Revistas}. Es una base de datos en línea que proporciona información sobre las revistas académicas y científicas.

\textbf{Miniconda}: Una versión minimalista de Anaconda, un sistema de gestión de entornos y paquetes de Python, que proporciona solo los componentes esenciales.

\textbf{Modularizado}: Práctica que consiste en dividir un sistema o programa en módulos o componentes más pequeños y autónomos. Cada módulo se ocupa de una funcionalidad específica y se puede desarrollar, mantener y probar de forma independiente.

\textbf{MySQL}: Gestor de base de datos relacional ampliamente utilizado para aplicaciones web.

\textbf{NumPy}: Una biblioteca de Python utilizada para realizar operaciones matemáticas y numéricas eficientes en matrices y arreglos multidimensionales.

\textbf{Open Source}: Un término que se refiere a programas o software cuyo código fuente es de acceso público y puede ser utilizado, modificado y distribuido por cualquier persona.

\textbf{Pandas}: Una biblioteca de Python utilizada para el análisis y manipulación de datos estructurados. Proporciona estructuras de datos y herramientas para el manejo eficiente de tablas y series temporales.

\textbf{Paquete}: Un conjunto de módulos y archivos relacionados que se agrupan y distribuyen juntos para facilitar su uso y reutilización en aplicaciones de software.

\textbf{PHP}: Lenguaje de programación interpretado utilizado para desarrollo web del lado del servidor.

\textbf{Pickleshare}: Un módulo de Python que proporciona una forma sencilla de almacenar y compartir objetos de Python mediante la serialización y deserialización utilizando el formato "pickle".

\textbf{Postgres}: Un sistema de gestión de bases de datos relacional de código abierto (RDBMS) conocido también como PostgreSQL.

\textbf{Product Owner}: Un rol en la metodología ágil Scrum que representa a los interesados y es responsable de gestionar el backlog del producto y priorizar las funcionalidades.

\textbf{Psycopg2}: Un adaptador de base de datos de PostgreSQL para Python que permite interactuar con bases de datos PostgreSQL utilizando Python.

\textbf{PyJWT}: Una biblioteca de Python que proporciona herramientas para la codificación y decodificación de JSON Web Tokens (JWT).

\textbf{Pytest}: Un marco de pruebas de Python que facilita la escritura y ejecución de pruebas unitarias, de integración y funcionales.

\textbf{Requests}: Una biblioteca de Python utilizada para realizar solicitudes HTTP de manera sencilla y eficiente.

\textbf{Scholarly}: Un paquete de Python que proporciona una interfaz para interactuar con la API de Google Scholar y obtener información sobre publicaciones académicas.

\textbf{Scikit-learn}: Una biblioteca de Python ampliamente utilizada para el aprendizaje automático (machine learning). Proporciona una amplia gama de algoritmos y herramientas para el análisis de datos y la construcción de modelos predictivos.

\textbf{Scipy}: Una biblioteca de Python utilizada para el cálculo científico y el análisis de datos. Proporciona funcionalidades para el álgebra lineal, estadísticas, optimización y más.

\textbf{Scopus}: Una base de datos bibliográfica y de citas que contiene información sobre publicaciones académicas, patentes y conferencias.

\textbf{Scrum Master}: Un rol en la metodología ágil Scrum responsable de facilitar el proceso y asegurar que el equipo siga las prácticas y principios de Scrum.

\textbf{Scrum}: Un marco de trabajo ágil para la gestión y desarrollo de proyectos. Se centra en la colaboración, la adaptabilidad y la entrega incremental.

\textbf{Selenium}: Una suite de herramientas utilizada para la automatización de pruebas en aplicaciones web. Permite controlar navegadores web y simular interacciones de los usuarios.

\textbf{Sprints}: Iteraciones cortas y enfocadas en las que se realiza el trabajo en un proyecto ágil. Cada sprint tiene una duración fija y al final de cada uno se entrega un incremento potencialmente entregable del producto.

\textbf{SQLAlchemy}: Una biblioteca de Python que proporciona una capa de abstracción sobre los motores de bases de datos relacionales. Permite interactuar con bases de datos utilizando código Python en lugar de SQL directamente.

\textbf{Story Points}: Una medida utilizada en la metodología ágil para estimar el esfuerzo o la complejidad de una tarea. Ayuda a determinar la capacidad de trabajo de un equipo en un \textit{sprint}.

\textbf{Web of Science}: Una base de datos bibliográfica y de citas ampliamente utilizada que proporciona información sobre publicaciones científicas, conferencias y patentes.

\textbf{Web Scraping}: La extracción de datos de páginas web de forma automatizada mediante el uso de programas o \textit{scripts}.

\textbf{XGBoost}: Una biblioteca de aprendizaje automático (\textit{machine learning}) que se enfoca en el algoritmo de Gradient Boosting.

\textbf{ZenHub}: Una herramienta de gestión de proyectos basada en GitHub que proporciona características adicionales como tableros Kanban, seguimiento de problemas y más.

