\apendice{Especificación de Requisitos}

\section{Introducción}

Para llevar a cabo el proyecto adecuadamente, se realizará un análisis exhaustivo de los requisitos funcionales y no funcionales de la aplicación. En la sección de Objetivos generales (\ref{sec:Objetivos generales}) se establecerán los resultados esperados y los criterios de éxito del proyecto. Seguidamente, se presentará el catálogo de requisitos (sección \ref{sec:Catálogo de requisitos}), detallando las funcionalidades y características que la aplicación debe cumplir. Finalmente, se procederá a la especificación de requisitos (sección \ref{sec:Especificación de requisitos}), describiendo cada uno de los identificados en el catálogo y estableciendo su prioridad, complejidad y dependencias.

A través de este informe, se pretende sentar las bases y establecer las directrices necesarias para el desarrollo exitoso de la aplicación web que proporcionará a los usuarios la información clave sobre el impacto de las revistas científicas, con el objetivo de facilitar la toma de decisiones informadas en el ámbito de la publicación académica.

\section{Objetivos generales}
\label{sec:Objetivos generales}

Los objetivos generales de este proyecto son los siguientes:

\begin{enumerate}
    \item Realizar el proceso de extracción de datos relevantes para el cálculo del Índice de Impacto de las revistas científicas. Esta tarea constituye la parte más desafiante del proyecto, ya que implica llevar a cabo una investigación exhaustiva y realizar pruebas utilizando diversas técnicas de \textit{web scraping} y simulación con Selenium. Se requerirá desarrollar métodos eficientes para recopilar los datos necesarios que no estén fácilmente accesibles y estructurados.
    
    \item Implementar modelos de predicción utilizando técnicas de \textit{machine learning)}, en particular utilizando la biblioteca Scikit-learn. Estos modelos permitirán predecir el Índice de Impacto de las revistas científicas con base en las características y métricas disponibles. Se explorarán diferentes algoritmos y técnicas para obtener predicciones precisas y confiables.
    
    \item Desarrollar una aplicación web utilizando el \textit{framework} Flask. Esta aplicación será la interfaz principal para que los usuarios puedan acceder y visualizar los resultados del proyecto. La aplicación web proporcionará una experiencia de usuario intuitiva y amigable, permitiendo la búsqueda y filtrado de revistas, así como la visualización de los índices de impacto calculados y las predicciones generadas por los modelos.
\end{enumerate}

El objetivo central del proyecto es proporcionar a los usuarios una herramienta integral que les permita evaluar y comparar el impacto de las revistas científicas de manera eficiente y precisa. Para lograr esto, se abordarán tres aspectos clave: la extracción de datos, los modelos de predicción y el desarrollo de la aplicación web. Cada uno de estos objetivos se complementa para lograr un sistema funcional y útil para la comunidad académica y científica.

Es importante reiterar que el proceso de extracción de datos es considerado como la parte más desafiante del proyecto debido a la necesidad de investigar y probar diferentes técnicas de \textit{web scraping}. El éxito en esta etapa es fundamental para garantizar la disponibilidad de los datos necesarios para el cálculo del Índice de Impacto y para el entrenamiento de los modelos de predicción.


\section{Catálogo de Historias de Usuario}
\label{sec:Catálogo de requisitos}

En este caso, dado que se ha elegido seguir una metodología ágil, se considera un catálogo de historias de usuario (usadas en ambos marcos, Scrum y Kanban). Para poder redactar una historia de usuario, hay tres elementos clave~\cite{Asana_2022}:
\begin{itemize}
    \item Perfil: El rol del usuario final.
    \item Necesidad: El objetivo que tiene la función de \textit{software} para el usuario final.
    \item Propósito: El objetivo de la experiencia del usuario final con la función de \textit{software}.
\end{itemize}

Habiendo identificado estos elementos, la historia de usuario se redactará siguiendo este formato: \textbf{<<Como [perfil], quiero [necesidad], para lograr [propósito]>>}.

Se ha elaborado el catálogo de historias de usuario con el objetivo de cubrir todos los objetivos mencionados en el apartado anterior sin entrar en detalles (los requisitos más específicos con respecto a la aplicación web están definidos en la sección \ref{sec:Requisitos de aplicación}). Para visualizar mejor el catálogo, se han clasificado las historias de usuario en tres categorías: extracción de datos (Tabla \ref{tab:extracción_de_datos}), modelos de predicción  (Tabla \ref{tab:Modelos de predicción}) y aplicación web (Tabla \ref{tab:Aplicación web}).

\begin{table}[h]
\centering
\begin{tabular}{|p{3cm}|p{8cm}|}
\hline
\textbf{Identificador} & \textbf{Descripción} \\
\hline
HU1 & Como desarrollador, quiero obtener información relevante para el cálculo del Índice de Impacto (JCR) de las revistas científicas a partir de fuentes externas. \\
\hline
HU2 & Como desarrollador, quiero aplicar técnicas de \textit{web scraping} y simulación con Selenium para extraer los datos de manera eficiente y precisa. \\
\hline
HU3 & Como desarrollador, quiero procesar y almacenar los datos extraídos en una base de datos para su posterior uso en el cálculo del Índice de Impacto y en los modelos de predicción. \\
\hline
\end{tabular}
\caption{HU Extracción de datos}
\label{tab:extracción_de_datos}
\end{table}
\newpage

\begin{table}[h]
\centering
\begin{tabular}{|p{3cm}|p{8cm}|}
\hline
\textbf{Identificador} & \textbf{Descripción} \\
\hline
HU4 & Como desarrollador, quiero implementar modelos de predicción utilizando técnicas de \textit{machine learning} (scikit-learn) para predecir el Índice de Impacto de las revistas científicas. \\
\hline
HU5 & Como desarrollador, quiero evaluar y comparar diferentes algoritmos de \textit{machine learning} para seleccionar los modelos más adecuados. \\
\hline
HU6 & Como desarrollador, quiero entrenar los modelos utilizando los datos almacenados y ajustar sus parámetros para obtener predicciones precisas. \\
\hline
\end{tabular}
\caption{HU Modelos de predicción}
\label{tab:Modelos de predicción}
\end{table}

\begin{table}[h]
\centering
\begin{tabular}{|p{3cm}|p{8cm}|}
\hline
\textbf{Identificador} & \textbf{Descripción} \\
\hline
HU7 & Como usuario, quiero acceder a una aplicación web donde pueda buscar y filtrar revistas científicas para evaluar su Índice de Impacto. \\
\hline
HU8 & Como usuario, quiero visualizar de forma clara y comprensible los índices de impacto calculados y las predicciones generadas por los modelos de predicción. \\
\hline
HU9 & Como usuario, quiero interactuar con la aplicación web, seleccionar revistas específicas y obtener información detallada sobre cada una de ellas. \\
\hline
\end{tabular}
\caption{HU Aplicación web}
\label{tab:Aplicación web}
\end{table}

Estas historias de usuario representan las necesidades y objetivos de los diferentes perfiles de usuarios en relación con la extracción de datos, los modelos de predicción y la aplicación web. Siguiendo una metodología ágil, estas historias de usuario servirán como guía para el desarrollo incremental y la entrega de funcionalidades de valor en cada iteración del proyecto.

\newpage
\section{Requisitos de la aplicación}
\label{sec:Requisitos de aplicación}

En esta sección se enumerarán, en mayor detalle, los requisitos que se esperan de la aplicación web en concreto.

\subsection{Requisitos funcionales}

\begin{itemize}
  \item \textbf{RF1}: Autenticación y Registro
  \begin{itemize}
    \item \textbf{RF1.1}: La aplicación debe permitir a los usuarios autenticarse mediante un nombre de usuario y contraseña válidos.
    \item \textbf{RF1.2}: La aplicación debe permitir a los usuarios registrarse proporcionando un nombre de usuario, una contraseña, una dirección de correo electrónico válida y confirmación de la contraseña.
    \item \textbf{RF1.3}: Las contraseñas deben tener un formato válido (i.e.: una longitud mínima de 8 caracteres, caracteres especiales requeridos, etc.).
    \item \textbf{RF1.4}: Los cuadros de texto de las contraseñas deben tener la opción de mostrar y ocultar el contenido por motivos de seguridad.
  \end{itemize}
  
  \item \textbf{RF2}: Búsqueda y selección de revistas
  \begin{itemize}
    \item \textbf{RF2.1}: En la interfaz principal debe mostrar un botón que redirija a otra interfaz con una lista de revistas y su información relevante.
    \item \textbf{RF2.2}: La interfaz con la lista de revistas debe incluir una barra de búsqueda por texto y paginación para filtrar las revistas.
    \item \textbf{RF2.3}: La interfaz principal debe permitir a los usuarios seleccionar la categoría y el nombre de la revista utilizando listas \textit{dropdowns} con búsqueda por texto (activándose el botón para consultar el histórico del JCR de la revista).
    \item \textbf{RF2.4}: La interfaz principal debe permitir a los usuarios seleccionar uno o más modelos de predicción (activándose el botón de predicción del JCR).
    \item \textbf{RF2.5}: En la interfaz principal se deben validar los campos obligatorios y mostrar un mensaje de error en caso de que no se completen.
  \end{itemize}

  \newpage
  \item \textbf{RF3}: Interfaz de Historial JCR
  \begin{itemize}
    \item \textbf{RF3.1}: La interfaz de Historial JCR debe mostrar un gráfico con el JCR de la revista seleccionada en los últimos 5 años.
    \item \textbf{RF3.2}: Debe mostrar un gráfico con el cuartil de la revista seleccionada en los últimos 5 años.
  \end{itemize}
  
  \item \textbf{RF4}: Interfaz de Predicción JCR
  \begin{itemize}
    \item \textbf{RF4.1}: La interfaz de Predicción JCR debe permitir a los usuarios seleccionar uno o más modelos de predicción.
    \item \textbf{RF4.2}: Debe mostrar la predicción del JCR de cada modelo para el año actual.
    \item \textbf{RF4.3}: Debe mostrar un gráfico con la línea temporal del JCR de la revista y la predicción de los dos últimos años en diferentes colores para cada modelo.
    \item \textbf{RF4.4}: Los usuarios deben poder ocultar cualquier línea del gráfico haciendo clic sobre su leyenda.
  \end{itemize}
  
  \item \textbf{RF5}: Barra inferior y superior de las interfaces
  \begin{itemize}
    \item \textbf{RF5.1}: La barra inferior debe incluir enlaces a la política de privacidad y a los términos de uso de la aplicación.
    \item \textbf{RF5.2}: La barra inferior debe incluir el logo de la Universidad de Burgos que redirecciona a su página web oficial.
    \item \textbf{RF5.3}: La barra superior debe mostrar el logo de la aplicación que redirecciona siempre a la página principal.
    \item \textbf{RF5.4}: La barra superior debe incluir un menú \textit{dropdown} con opciones de internacionalización en español, inglés, italiano y francés.
    \item \textbf{RF5.5}: La barra superior debe mostrar un enlace a la ayuda en línea.
    \item \textbf{RF5.6}: La barra superior debe mostrar el nombre e imagen del usuario que ha iniciado sesión.
    \item \textbf{RF5.7}: El menú \textit{dropdown} del nombre del usuario debe permitir cerrar sesión (eliminando las variables de sesión y redireccionando a la interfaz de inicio de sesión) o editar el perfil (redireccionando a una página donde se pueda consultar y modificar la información del usuario).
  \end{itemize}
  
  \item \textbf{RF6}: Navegación
  \begin{itemize}
    \item \textbf{RF6.1}: Todas las interfaces deben tener navegación en forma de \textit{breadcrumbs} para indicar la ubicación actual del usuario en la aplicación.
  \end{itemize}
  
\end{itemize}

\subsection{Requisitos No Funcionales}

\begin{itemize}
  \item \textbf{RNF1}: Internacionalización
  \begin{itemize}
    \item \textbf{RNF1.1}: La aplicación debe tomar el idioma configurado en el navegador o el idioma <<más cercano>> disponible por defecto.
  \end{itemize}
  
  \item \textbf{RNF2}: Seguridad
  \begin{itemize}
    \item \textbf{RNF2.1}: Las contraseñas deben ser almacenadas de forma segura utilizando técnicas de cifrado adecuadas.
    \item \textbf{RNF2.2}: La aplicación debe protegerse contra ataques de fuerza bruta en los formularios de autenticación y registro.
  \end{itemize}
  
  \item \textbf{RNF3}: Rendimiento
  \begin{itemize}
    \item \textbf{RNF3.1}: La aplicación debe ser capaz de manejar de manera eficiente la carga de usuarios concurrentes.
    \item \textbf{RNF3.2}: Las consultas a la base de datos deben optimizarse para proporcionar respuestas rápidas a los usuarios.
  \end{itemize}
  
  \item \textbf{RNF4}: Usabilidad
  \begin{itemize}
    \item \textbf{RNF4.1}: La interfaz de usuario debe ser intuitiva y fácil de usar para los usuarios.
    \item \textbf{RNF4.2}: Los mensajes de error deben ser claros y descriptivos, brindando orientación sobre cómo solucionar los problemas.
  \end{itemize}
  
  \item \textbf{RNF5}: Mantenibilidad
  \begin{itemize}
    \item \textbf{RNF5.1}: El código fuente de la aplicación debe estar bien estructurado, modularizado y documentado.
    \item \textbf{RNF5.2}: El código debe seguir las mejores prácticas de desarrollo de \textit{software} y ser fácilmente mantenible y escalable.
  \end{itemize}
\end{itemize}


\section{Especificación de requisitos}
\label{sec:Especificación de requisitos}

En esta sección se enumerarán los distintos casos de uso para la aplicación web desarrollada.

\subsection{Actores}
Antes de comenzar con los casos de uso, se identifican los distintos actores que pueden interactuar con la aplicación y se definen a continuación:
\begin{itemize}
    \item \textbf{Usuario no registrado}: Referido a una persona que no ha creado una cuenta o perfil para entrar en la aplicación.
    \item \textbf{Usuario registrado}: Referido a una persona que ha creado una cuenta o perfil en la aplicación. Tendrá la posibilidad de modificar sus credenciales.
    \item \textbf{Administrador}: Usuario registrado con privilegios de administrador.
\end{itemize}

\subsection{Casos de Uso}
En cuanto a los casos de uso en si, se han identificado los siguientes:

\begin{itemize}
\item \textbf{CU-1} Autenticación (ver Tabla \ref{tab:cu1}).
\item \textbf{CU-2} Registro (ver Tabla \ref{tab:cu2}).
\item \textbf{CU-3} Interfaz Principal (ver Tabla \ref{tab:cu3}).
\item \textbf{CU-4} Interfaz de Lista de Revistas (ver Tabla \ref{tab:cu4}).
\item \textbf{CU-5} Buscar revistas (ver Tabla \ref{tab:cu5}).
\item \textbf{CU-6} Consultar historial de la revista (ver Tabla \ref{tab:cu6}).
\item \textbf{CU-7} Consultar predicción (ver Tabla \ref{tab:cu7}).
\item \textbf{CU-8} Consultar política de privacidad y términos de uso (ver Tabla \ref{tab:cu8}).
\item \textbf{CU-9} Redireccionar a la página de la Universidad de Burgos (ver Tabla \ref{tab:cu9}).
\item \textbf{CU-10} Redireccionar a la página principal (ver Tabla \ref{tab:cu10}).
\item \textbf{CU-11} Navegación con breadcrumbs (ver Tabla \ref{tab:cu11}).
\end{itemize}

\begin{table}[p]
\centering
\begin{tabularx}{\linewidth}{ p{0.21\columnwidth} p{0.71\columnwidth} }
\toprule
\textbf{CU-1} & \textbf{Autenticación}\\
\toprule
\textbf{Versión} & 1.0 \\
\textbf{Autor} & Gadea Lucas Pérez \\
\textbf{Requisitos asociados} & RF1, RF1.1, RF1.4  \\
\textbf{Descripción} & Este caso de uso describe el proceso de autenticación de los usuarios en la aplicación.\\
\textbf{Precondición} & El usuario debe estar registrado en la aplicación.\\
\textbf{Acciones} &
\begin{enumerate}
\def\labelenumi{\arabic{enumi}.}
\tightlist
\item El usuario ingresa su nombre de usuario y contraseña (pudiendo mostrar y ocultar su contenido).
\item El sistema verifica la validez de las credenciales ingresadas.
\item El sistema otorga acceso al usuario si las credenciales son válidas.
\end{enumerate}\\
\textbf{Postcondición} & El usuario ha iniciado sesión en la aplicación.\\
\textbf{Excepciones} &
\begin{itemize}
\item Si las credenciales son inválidas, se muestra un mensaje de error y se solicita al usuario que vuelva a ingresar las credenciales.
\item Si el usuario no está registrado, se muestra un mensaje indicando que debe registrarse antes de iniciar sesión.
\end{itemize}\\
\textbf{Importancia} & Alta\\
\bottomrule
\end{tabularx}
\caption{CU-1 Autenticación.}
\label{tab:cu1}
\end{table}


\begin{table}[p]
\centering
\begin{tabularx}{\linewidth}{ p{0.21\columnwidth} p{0.71\columnwidth} }
\toprule
\textbf{CU-2} & \textbf{Registro}\\
\toprule
\textbf{Versión} & 1.0 \\
\textbf{Autor} & Gadea Lucas Pérez \\
\textbf{Requisitos asociados} & RF1, RF1.3, RF1.4  \\
\textbf{Descripción} & Este caso de uso describe el proceso de registro de nuevos usuarios en la aplicación.\\
\textbf{Precondición} & - \\
\textbf{Acciones} &
\begin{enumerate}
\def\labelenumi{\arabic{enumi}.}
\tightlist
\item El usuario selecciona la opción de registro en la interfaz de inicio de sesión.
\item El usuario proporciona un nombre de usuario, una contraseña (pudiendo mostrar y ocultar su contenido), una dirección de correo electrónico válida y confirma la contraseña.
\item El sistema verifica que la contraseña tenga un formato válido y que la dirección de correo electrónico sea válida.
\item El sistema verifica que la contraseña y la confirmación de la contraseña coincidan.
\item El sistema crea un nuevo usuario en la base de datos con la información proporcionada.
\end{enumerate}\\
\textbf{Postcondición} & El usuario se ha registrado exitosamente.\\
\textbf{Excepciones} &
\begin{itemize}
\item Si la contraseña no cumple con el formato válido (i.e.: no tiene una longitud mínima de 8 caracteres, no contiene caracteres especiales requeridos, etc.), se muestra un mensaje de error y se solicita al usuario que ingrese una contraseña válida.
\item Si la dirección de correo electrónico no es válida, se muestra un mensaje de error y se solicita al usuario que ingrese una dirección de correo electrónico válida.
\item Si la contraseña y la confirmación de la contraseña no coinciden, se muestra un mensaje de error y se solicita al usuario que vuelva a ingresar la contraseña y su confirmación.
\end{itemize}\\
\textbf{Importancia} & Alta\\
\bottomrule
\end{tabularx}
\caption{CU-2 Registro.}
\label{tab:cu2}
\end{table}


\begin{table}[p]
\centering
\begin{tabularx}{\linewidth}{ p{0.21\columnwidth} p{0.71\columnwidth} }
\toprule
\textbf{CU-3} & \textbf{Interfaz Principal}\\
\toprule
\textbf{Versión} & 1.0 \\
\textbf{Autor} & Gadea Lucas Pérez \\
\textbf{Requisitos asociados} & RF2, RF2.1, RF2.2, RF2.3, RF2.4 \\
\textbf{Descripción} & Este caso de uso describe la interfaz principal de la aplicación, desde donde los usuarios pueden acceder a la lista de revistas y realizar búsquedas y filtrados.\\
\textbf{Precondición} & El usuario ha iniciado sesión en la aplicación. \\
\textbf{Acciones} &
\begin{enumerate}
\def\labelenumi{\arabic{enumi}.}
\tightlist
\item El usuario accede a la interfaz principal después de iniciar sesión correctamente.
\item La interfaz principal muestra un botón que redirige a la interfaz de lista de revistas.
\item La interfaz principal muestra listas desplegables (\textit{dropdowns}) con opciones de categorías y nombres de revistas.
\item El usuario puede utilizar la función de búsqueda por texto para filtrar las revistas en base a palabras clave.
\item La interfaz principal valida los campos obligatorios y muestra un mensaje de error si no se completan.
\end{enumerate}\\
\textbf{Postcondición} & - \\
\textbf{Excepciones} & - \\
\textbf{Importancia} & Alta\\
\bottomrule
\end{tabularx}
\caption{CU-3 Interfaz Principal.}
\label{tab:cu3}
\end{table}

\begin{table}[p]
\centering
\begin{tabularx}{\linewidth}{ p{0.21\columnwidth} p{0.71\columnwidth} }
\toprule
\textbf{CU-4} & \textbf{Interfaz de Lista de Revistas}\\
\toprule
\textbf{Versión} & 1.0 \\
\textbf{Autor} & Gadea Lucas Pérez \\
\textbf{Requisitos asociados} & RF2, RF2.2 \\
\textbf{Descripción} & Este caso de uso describe la interfaz que muestra una lista de revistas y su información relevante, con funcionalidades de búsqueda y paginación.\\
\textbf{Precondición} & El usuario ha accedido a la interfaz de lista de revistas desde la interfaz principal. \\
\textbf{Acciones} &
\begin{enumerate}
\def\labelenumi{\arabic{enumi}.}
\tightlist
\item La interfaz muestra una lista de revistas con su información relevante.
\item La interfaz incluye una barra de búsqueda por texto para filtrar las revistas en base a palabras clave.
\item La interfaz muestra una paginación para navegar entre las diferentes páginas de resultados.
\end{enumerate}\\
\textbf{Postcondición} & - \\
\textbf{Excepciones} & - \\
\textbf{Importancia} & Alta\\
\bottomrule
\end{tabularx}
\caption{CU-4 Interfaz de Lista de Revistas.}
\label{tab:cu4}
\end{table}


\begin{table}[p]
\centering
\begin{tabularx}{\linewidth}{ p{0.21\columnwidth} p{0.71\columnwidth} }
\toprule
\textbf{CU-5} & \textbf{Buscar revistas}\\
\toprule
\textbf{Versión} & 1.0 \\
\textbf{Autor} & Gadea Lucas Pérez \\
\textbf{Requisitos asociados} & RF2, RF2.2 \\
\textbf{Descripción} & Este caso de uso describe la funcionalidad para buscar revistas en la interfaz de lista de revistas desde la interfaz principal. Permite al usuario utilizar la barra de búsqueda por texto y navegar por la paginación de la lista de revistas. Además, el usuario puede acceder a la información adicional de una revista haciendo clic sobre su nombre. \\
\textbf{Precondición} & El usuario ha iniciado sesión en la aplicación y está en la interfaz principal. \\
\textbf{Acciones} &
\begin{enumerate}
\def\labelenumi{\arabic{enumi}.}
\tightlist
\item El usuario accede a la interfaz de lista de revistas desde la interfaz principal.
\item El usuario introduce un texto de búsqueda en la barra de búsqueda.
\item El usuario navega por la paginación de la lista de revistas.
\item El usuario hace clic sobre el nombre de una revista para acceder a su información adicional.
\end{enumerate}\\
\textbf{Postcondición} & - \\
\textbf{Excepciones} & - \\
\textbf{Importancia} & Alta\
\bottomrule
\end{tabularx}
\caption{CU-5 Buscar revistas.}
\label{tab:cu5}
\end{table}


\begin{table}[p]
\centering
\begin{tabularx}{\linewidth}{ p{0.21\columnwidth} p{0.71\columnwidth} }
\toprule
\textbf{CU-6} & \textbf{Consultar historial de la revista}\\
\toprule
\textbf{Versión} & 1.0 \\
\textbf{Autor} & Gadea Lucas Pérez \\
\textbf{Requisitos asociados} & RF2.3, RF3.1, RF3.2 \\
\textbf{Descripción} & Este caso de uso permite a los usuarios consultar el historial del JCR de una revista seleccionada. La interfaz principal debe permitir a los usuarios seleccionar la categoría y el nombre de la revista utilizando listas \textit{dropdowns} con búsqueda por texto, lo que activará el botón para consultar el histórico del JCR de la revista. \\
\textbf{Precondición} & El usuario ha iniciado sesión en la aplicación y está en la interfaz principal.\\
\textbf{Acciones} &
\begin{enumerate}
\def\labelenumi{\arabic{enumi}.}
\tightlist
\item El usuario selecciona la categoría de la revista en la lista \textit{dropdown}.
\item El usuario selecciona el nombre de la revista en la lista \textit{dropdown} o realiza una búsqueda por texto.
\item El usuario hace clic en el botón para consultar el histórico del JCR de la revista.
\item Se muestra la interfaz de Historial JCR con un gráfico del JCR de la revista en los últimos 5 años.
\item Se muestra un gráfico con el cuartil de la revista en los últimos 5 años.
\end{enumerate}\\
\textbf{Postcondición} & - \\
\textbf{Excepciones} & - \\
\textbf{Importancia} & Alta\\
\bottomrule
\end{tabularx}
\caption{CU-6 Consultar historial de la revista.}
\label{tab:cu6}
\end{table}


\begin{table}[p]
\centering
\begin{tabularx}{\linewidth}{ p{0.21\columnwidth} p{0.71\columnwidth} }
\toprule
\textbf{CU-7} & \textbf{Consultar predicción}\\
\toprule
\textbf{Versión} & 1.0 \\
\textbf{Autor} & Gadea Lucas Pérez \\
\textbf{Requisitos asociados} & RF2.4, RF4.1, RF4.2, RF4.3, RF4.4 \\
\textbf{Descripción} & Este caso de uso permite a los usuarios consultar la predicción del JCR de una revista utilizando uno o más modelos de predicción. La interfaz principal debe permitir a los usuarios seleccionar uno o más modelos de predicción, activando así el botón de predicción del JCR.\\
\textbf{Precondición} & El usuario ha iniciado sesión en la aplicación y está en la interfaz principal.\\
\textbf{Acciones} &
\begin{enumerate}
\def\labelenumi{\arabic{enumi}.}
\tightlist
\item El usuario selecciona uno o más modelos de predicción en la interfaz principal.
\item Se activa el botón de predicción del JCR.
\item El usuario hace clic en el botón de predicción del JCR.
\item Se muestra la interfaz de Predicción JCR.
\item El usuario selecciona uno o más modelos de predicción en la interfaz de Predicción JCR.
\item Se muestra la predicción del JCR de cada modelo para el año actual.
\item Se muestra un gráfico con la línea temporal del JCR de la revista y la predicción de los dos últimos años en diferentes colores para cada modelo.
\item El usuario puede ocultar cualquier línea del gráfico haciendo clic sobre su leyenda.
\end{enumerate}\\
\textbf{Postcondición} & - \\
\textbf{Excepciones} & - \\
\textbf{Importancia} & Alta\\
\bottomrule
\end{tabularx}
\caption{CU-7 Consultar predicción.}
\label{tab:cu7}
\end{table}


\begin{table}[p]
\centering
\begin{tabularx}{\linewidth}{ p{0.21\columnwidth} p{0.71\columnwidth} }
\toprule
\textbf{CU-8} & \textbf{Consultar política de privacidad y términos de uso}\\
\toprule
\textbf{Versión} & 1.0 \\
\textbf{Autor} & Gadea Lucas Pérez \\
\textbf{Requisitos asociados} & RF5.1 \\
\textbf{Descripción} & Este caso de uso muestra enlaces a la política de privacidad y a los términos de uso de la aplicación en la barra inferior de las interfaces.\\
\textbf{Precondición} & El usuario ha iniciado sesión en la aplicación.\\
\textbf{Acciones} &
\begin{enumerate}
\def\labelenumi{\arabic{enumi}.}
\tightlist
\item El usuario accede a cualquier interfaz donde se muestra la barra inferior.
\item El usuario selecciona el hiperenlace de política de privacidad o términos de uso.
\item El usuario es redireccionado a la interfaz correspondiente donde se muestra la información.
\end{enumerate}\\ 
\textbf{Postcondición} & - \\
\textbf{Excepciones} & - \\
\textbf{Importancia} & Media \\
\bottomrule
\end{tabularx}
\caption{CU-8: Consultar política de privacidad y términos de uso.}
\label{tab:cu8}
\end{table}

\begin{table}[p]
\centering
\begin{tabularx}{\linewidth}{ p{0.21\columnwidth} p{0.71\columnwidth} }
\toprule
\textbf{CU-9} & \textbf{Redireccionar a la página web oficial de la Universidad de Burgos}\\
\toprule
\textbf{Versión} & 1.0 \\
\textbf{Autor} & Gadea Lucas Pérez \\
\textbf{Requisitos asociados} & RF5.2 \\
\textbf{Descripción} & Este caso de uso redirige al usuario hacia la página web oficial de la Universidad de Burgos al hacer clic en su logo ubicado en la barra inferior.\\
\textbf{Precondición} & El usuario ha iniciado sesión en la aplicación.\\
\textbf{Acciones} &
\begin{enumerate}
\def\labelenumi{\arabic{enumi}.}
\tightlist
\item El usuario accede a cualquier interfaz donde se muestra la barra inferior.
\item El usuario selecciona el logo de la Universidad de Burgos.
\item El usuario es redireccionado a la página oficial de la universidad.
\end{enumerate}\\ 
\textbf{Postcondición} & - \\
\textbf{Excepciones} & - \\
\textbf{Importancia} & Baja \\
\bottomrule
\end{tabularx}
\caption{CU-9: Redireccionar a la página web oficial de la Universidad de Burgos.}
\label{tab:cu9}
\end{table}


\begin{table}[p]
\centering
\begin{tabularx}{\linewidth}{ p{0.21\columnwidth} p{0.71\columnwidth} }
\toprule
\textbf{CU-10} & \textbf{Redireccionar a la página principal}\\
\toprule
\textbf{Versión} & 1.0 \\
\textbf{Autor} & Gadea Lucas Pérez \\
\textbf{Requisitos asociados} & RF5.3 \\
\textbf{Descripción} & Este caso de uso redirige al usuario hacia la página principal de la aplicación al hacer clic en el logo de la aplicación ubicado en la barra superior.\\
\textbf{Precondición} & El usuario ha iniciado sesión en la aplicación.\\
\textbf{Acciones} &
\begin{enumerate}
\def\labelenumi{\arabic{enumi}.}
\tightlist
\item El usuario accede a cualquier interfaz donde se muestra la barra superior.
\item El usuario selecciona el logo de la aplicación.
\item El usuario es redireccionado a la página principal de la aplicación.
\end{enumerate}\\ 
\textbf{Postcondición} & - \\
\textbf{Excepciones} & - \\
\textbf{Importancia} & Alta \\
\bottomrule
\end{tabularx}
\caption{CU-10: Redireccionar a la página principal.}
\label{tab:cu10}
\end{table}


\begin{table}[p]
\centering
\begin{tabularx}{\linewidth}{ p{0.21\columnwidth} p{0.71\columnwidth} }
\toprule
\textbf{CU-11} & \textbf{Navegación con breadcrumbs}\\
\toprule
\textbf{Versión} & 1.0 \\
\textbf{Autor} & Gadea Lucas Pérez \\
\textbf{Requisitos asociados} & RF6.1 \\
\textbf{Descripción} & Este caso de uso abarca la implementación de la navegación en forma de breadcrumbs en todas las interfaces de la aplicación. Los breadcrumbs permiten al usuario conocer la ubicación actual dentro de la aplicación y navegar hacia atrás de manera intuitiva.\\
\textbf{Precondición} & El usuario ha iniciado sesión en la aplicación.\\
\textbf{Acciones} &
\begin{enumerate}
\def\labelenumi{\arabic{enumi}.}
\tightlist
\item El usuario accede a cualquier interfaz donde se muestran las breadcrums.
\item El usuario pulsa sobre el hipervínculo del \textit{breadcrumb}.
\item El usuario es redireccionado a la página correspondiente.
\end{enumerate}\\ 
\textbf{Postcondición} & - \\
\textbf{Excepciones} & - \\
\textbf{Importancia} & Media \\
\bottomrule
\end{tabularx}
\caption{CU-11: Navegación y breadcrumbs.}
\label{tab:cu11}
\end{table}



\subsection{Diagrama de casos de usos}

Finalmente, para poder visualizar todos estos casos, se incluye un diagrama de casos de uso (ver Figura \ref{fig:cu}).

\imagen{cu}{Diagrama de Casos de Uso}{1.1}