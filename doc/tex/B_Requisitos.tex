\apendice{Especificación de Requisitos}

\section{Introducción}

Para llevar a cabo el proyecto adecuadamente, se realizará un análisis exhaustivo de los requisitos funcionales y no funcionales de la aplicación. En la sección de Objetivos generales (\ref{sec:Objetivos generales}) se establecerán los resultados esperados y los criterios de éxito del proyecto. Seguidamente, se presentará el catálogo de requisitos (sección \ref{sec:Catálogo de requisitos}), detallando las funcionalidades y características que la aplicación debe cumplir. Finalmente, se procederá a la especificación de requisitos (sección \ref{sec:Especificación de requisitos}), describiendo cada uno de los identificados en el catálogo y estableciendo su prioridad, complejidad y dependencias.

A través de este informe, se pretende sentar las bases y establecer las directrices necesarias para el desarrollo exitoso de la aplicación web que proporcionará a los usuarios la información clave sobre el impacto de las revistas científicas, con el objetivo de facilitar la toma de decisiones informadas en el ámbito de la publicación académica.

\section{Objetivos generales}
\label{sec:Objetivos generales}

Los objetivos generales de este proyecto son los siguientes:

\begin{enumerate}
    \item Realizar el proceso de extracción de datos relevantes para el cálculo del Índice de Impacto de las revistas científicas indexadas en el Journal Citation Reports (JCR) de Clarivate. Esta tarea constituye la parte más desafiante del proyecto, ya que implica llevar a cabo una investigación exhaustiva y realizar pruebas utilizando diversas técnicas de \textit{web scraping} y simulación con Selenium. Se requerirá desarrollar métodos eficientes para recopilar los datos necesarios que no estén fácilmente accesibles y estructurados.
    
    \item Implementar modelos de predicción utilizando técnicas de \textit{machine learning}, en particular utilizando la biblioteca \href{https://scikit-learn.org/stable/index.html}{Scikit-learn}. Estos modelos permitirán predecir el Índice de Impacto de las revistas científicas en base en las características y métricas disponibles. Se explorarán diferentes algoritmos y técnicas para elegir el que mejor se ajuste a los datos y así obtener predicciones precisas y confiables
    
    \item Desarrollar una aplicación web utilizando el \textit{framework} Flask. Esta aplicación será la interfaz principal para que los usuarios puedan acceder y visualizar los resultados del proyecto. La aplicación web proporcionará una experiencia de usuario intuitiva y amigable, permitiendo la búsqueda y filtrado de revistas, así como la visualización de los índices de impacto calculados y las predicciones generadas por los modelos.
\end{enumerate}

El objetivo central del proyecto es proporcionar a los usuarios una herramienta integral que les permita evaluar y comparar el impacto de las revistas científicas de manera eficiente y precisa. En particular, se pretende proporcionar valores del Índice de Impacto de años anteriores, y la predicción para el año en curso. Para lograr esto, se abordarán tres aspectos clave: la extracción de datos, los modelos de predicción y el desarrollo de la aplicación web. Cada uno de estos objetivos se complementa para lograr un sistema funcional y útil para la comunidad académica y científica.

Es importante reiterar que el proceso de extracción de datos es considerado como la parte más desafiante del proyecto debido a la necesidad de investigar y probar diferentes técnicas de \textit{web scraping}. El éxito en esta etapa es fundamental para garantizar la disponibilidad de los datos necesarios para el cálculo del Índice de Impacto y para el entrenamiento de los modelos de predicción.


\section{Catálogo de Historias de Usuario}
\label{sec:Catálogo de requisitos}

En este caso, dado que se ha elegido seguir una metodología ágil, se considera un catálogo de historias de usuario (usadas en ambos marcos, Scrum y Kanban). Para poder redactar una historia de usuario, hay tres elementos clave~\cite{Asana_2022}:
\begin{itemize}
    \item Perfil: El rol del usuario final.
    \item Necesidad: El objetivo que tiene la función de \textit{software} para el usuario final.
    \item Propósito: El objetivo de la experiencia del usuario final con la función de \textit{software}.
\end{itemize}

Habiendo identificado estos elementos, la historia de usuario se redactará siguiendo este formato: \textbf{<<Como [perfil], quiero [necesidad], para lograr [propósito]>>}.

Se ha elaborado el catálogo de historias de usuario (HU) con el objetivo de cubrir todos los objetivos mencionados en el apartado anterior sin entrar en detalles (los requisitos más específicos con respecto a la aplicación web están definidos en la sección \ref{sec:Requisitos de aplicación}). Para visualizar mejor el catálogo, se han clasificado las historias de usuario en tres categorías: extracción de datos (Tabla \ref{tab:extracción_de_datos}), modelos de predicción  (Tabla \ref{tab:Modelos de predicción}) y aplicación web (Tabla \ref{tab:Aplicación web}).

\begin{table}[h]
\centering
\begin{tabular}{p{3cm}p{8cm}}
\toprule
\textbf{Identificador} & \textbf{Descripción} \\
\toprule
HU1 & Como desarrollador, quiero obtener información relevante para el cálculo del Índice de Impacto (JCR) de las revistas científicas a partir de fuentes externas. \\
\midrule
HU2 & Como desarrollador, quiero aplicar técnicas de \textit{web scraping} y simulación con Selenium para extraer los datos de manera eficiente y precisa. \\
\midrule
HU3 & Como desarrollador, quiero procesar y almacenar los datos extraídos en una base de datos para su posterior uso en el cálculo del Índice de Impacto y en los modelos de predicción. \\
\bottomrule
\end{tabular}
\caption{HU Extracción de datos}
\label{tab:extracción_de_datos}
\end{table}
\newpage


\begin{table}[h]
\centering
\begin{tabular}{p{3cm}p{8cm}}
\toprule
\textbf{Identificador} & \textbf{Descripción} \\
\toprule
HU4 & Como desarrollador, quiero implementar modelos de predicción utilizando técnicas de \textit{machine learning} (scikit-learn) para predecir el Índice de Impacto de las revistas científicas. \\
\midrule
HU5 & Como desarrollador, quiero evaluar y comparar diferentes algoritmos de \textit{machine learning} para seleccionar los modelos más adecuados. \\
\midrule
HU6 & Como desarrollador, quiero entrenar los modelos utilizando los datos almacenados y ajustar sus parámetros para obtener predicciones precisas. \\
\bottomrule
\end{tabular}
\caption{HU Modelos de predicción}
\label{tab:Modelos de predicción}
\end{table}


\begin{table}[h]
\centering
\begin{tabular}{p{3cm}p{8cm}}
\toprule
\textbf{Identificador} & \textbf{Descripción} \\
\toprule
HU7 & Como usuario, quiero acceder a una aplicación web donde pueda buscar y filtrar revistas científicas para evaluar su Índice de Impacto. \\
\midrule
HU8 & Como usuario, quiero visualizar de forma clara y comprensible los índices de impacto calculados y las predicciones generadas por los modelos de predicción. \\
\midrule
HU9 & Como usuario, quiero interactuar con la aplicación web, seleccionar revistas específicas y obtener información detallada sobre cada una de ellas. \\
\bottomrule
\end{tabular}
\caption{HU Aplicación web}
\label{tab:Aplicación web}
\end{table}

Estas historias de usuario representan las necesidades y objetivos de los diferentes perfiles de usuarios en relación con la extracción de datos, los modelos de predicción y la aplicación web. Siguiendo una metodología ágil, estas historias de usuario servirán como guía para el desarrollo incremental y la entrega de funcionalidades de valor en cada iteración del proyecto.

\newpage
\section{Requisitos de la aplicación}
\label{sec:Requisitos de aplicación}

En esta sección se enumerarán, en mayor detalle, los requisitos que se esperan de la aplicación web en concreto.

\subsection{Requisitos funcionales}

\begin{itemize}
  \item \textbf{RF1}: Autenticación y Registro
  \begin{itemize}
    \item \textbf{RF1.1}: La aplicación debe permitir a los usuarios autenticarse mediante un nombre de usuario y contraseña válidos.
    \item \textbf{RF1.2}: La aplicación debe permitir a los usuarios registrarse proporcionando un nombre de usuario, una contraseña, una dirección de correo electrónico válida y confirmación de la contraseña.
    \item \textbf{RF1.3}: Las contraseñas deben tener un formato válido (i.e.: una longitud mínima de 8 caracteres, caracteres especiales requeridos, etc.).
    \item \textbf{RF1.4}: Los cuadros de texto de las contraseñas deben tener la opción de mostrar y ocultar el contenido por motivos de seguridad.
  \end{itemize}
  
  \item \textbf{RF2}: Búsqueda y selección de revistas
  \begin{itemize}
    \item \textbf{RF2.1}: En la interfaz principal debe mostrar un botón que redirija a otra interfaz con una lista de revistas y con un listado de revistas y la información más relevante sobre las mismas
    \item \textbf{RF2.2}: La interfaz con la lista de revistas debe incluir una barra de búsqueda por texto y paginación para filtrar las revistas.
    \item \textbf{RF2.3}: La interfaz principal debe permitir a los usuarios seleccionar la categoría y el nombre de la revista utilizando listas \textit{dropdowns} con búsqueda por texto (activándose el botón para consultar el histórico del JCR de la revista).
    \item \textbf{RF2.4}: La interfaz principal debe permitir a los usuarios seleccionar uno o más modelos de predicción (activándose el botón de predicción del JCR).
    \item \textbf{RF2.5}: En la interfaz principal se deben validar los campos obligatorios y mostrar un mensaje de error en caso de que no se completen.
  \end{itemize}

  \newpage
  \item \textbf{RF3}: Interfaz de Historial JCR
  \begin{itemize}
    \item \textbf{RF3.1}: La interfaz de Historial JCR debe mostrar un gráfico con el JCR de la revista seleccionada en los últimos 5 años.
    \item \textbf{RF3.2}: Asímismo, debe mostrar un gráfico con el cuartil de la revista seleccionada en los últimos 5 años.
  \end{itemize}
  
  \item \textbf{RF4}: Interfaz de Predicción JCR
  \begin{itemize}
    \item \textbf{RF4.1}: La interfaz de Predicción JCR debe permitir a los usuarios seleccionar uno o más modelos de predicción.
    \item \textbf{RF4.2}: Debe mostrar la predicción del JCR de cada modelo para el año actual.
    \item \textbf{RF4.3}: Debe mostrar un gráfico con la evolución temporal del JCR de la revista y la predicción de los dos últimos años en diferentes colores para cada modelo.
    \item \textbf{RF4.4}: Los usuarios deben poder ocultar cualquier línea del gráfico haciendo clic sobre su leyenda.
  \end{itemize}
  
  \item \textbf{RF5}: Barra inferior y superior de las interfaces
  \begin{itemize}
    \item \textbf{RF5.1}: La barra inferior debe incluir enlaces a la política de privacidad y a los términos de uso de la aplicación.
    \item \textbf{RF5.2}: La barra inferior debe incluir el logo de la Universidad de Burgos que redirecciona a su página web oficial.
    \item \textbf{RF5.3}: La barra superior debe mostrar el logo de la aplicación que redirecciona siempre a la página principal.
    \item \textbf{RF5.4}: La alicación debe tomar el idioma configurado en el navegador o el idioma <<más cercano>> disponible por defecto. Además, la barra superior debe incluir un menú \textit{dropdown} con opciones de internacionalización en español, inglés, italiano y francés. 
    \item \textbf{RF5.5}: La barra superior debe mostrar un enlace a la ayuda en línea.
    \item \textbf{RF5.6}: La barra superior debe mostrar el nombre e imagen del usuario que ha iniciado sesión.
    \item \textbf{RF5.7}: El menú \textit{dropdown} del nombre del usuario debe permitir cerrar sesión (eliminando las variables de sesión y redireccionando a la interfaz de inicio de sesión) o editar el perfil (redireccionando a una página donde se pueda consultar y modificar la información del usuario).
  \end{itemize}
  
  \item \textbf{RF6}: Navegación
  \begin{itemize}
    \item \textbf{RF6.1}: Todas las interfaces deben tener navegación en forma de \textit{breadcrumbs} para indicar la ubicación actual del usuario en la aplicación.
  \end{itemize}
  
\end{itemize}

\subsection{Requisitos No Funcionales}

\begin{itemize}

  \item \textbf{RNF1}: Seguridad
  \begin{itemize}
    \item \textbf{RNF1.1}: Las contraseñas deben ser almacenadas de forma segura utilizando técnicas de cifrado adecuadas.
    \item \textbf{RNF1.2}: La aplicación debe protegerse contra ataques de fuerza bruta en los formularios de autenticación y registro.
  \end{itemize}
  
  \item \textbf{RNF2}: Rendimiento
  \begin{itemize}
    \item \textbf{RNF2.1}: La aplicación debe ser capaz de manejar de manera eficiente la carga de usuarios concurrentes.
    \item \textbf{RNF2.2}: Las consultas a la base de datos deben optimizarse para proporcionar respuestas rápidas a los usuarios.
  \end{itemize}
  
  \item \textbf{RNF3}: Usabilidad
  \begin{itemize}
    \item \textbf{RNF3.1}: La interfaz de usuario debe ser intuitiva y fácil de usar para los usuarios.
    \item \textbf{RNF3.2}: Los mensajes de error deben ser claros y descriptivos, brindando orientación sobre cómo solucionar los problemas.
  \end{itemize}
  
  \item \textbf{RNF4}: Mantenibilidad
  \begin{itemize}
    \item \textbf{RNF4.1}: El código fuente de la aplicación debe estar bien estructurado, modularizado y documentado.
    \item \textbf{RNF4.2}: El código debe seguir las mejores prácticas de desarrollo de \textit{software} y ser fácilmente mantenible y escalable.
  \end{itemize}
\end{itemize}


\section{Especificación de requisitos}
\label{sec:Especificación de requisitos}

En esta sección se enumerarán los distintos casos de uso para la aplicación web desarrollada.

\subsection{Actores}
Antes de comenzar con los casos de uso, se identifican los distintos actores que pueden interactuar con la aplicación y se definen a continuación:
\begin{itemize}
    \item \textbf{Usuario registrado}: Referido a una persona que ha creado una cuenta o perfil en la aplicación. Tendrá la posibilidad de modificar sus credenciales.
    \item \textbf{Administrador}: Usuario registrado con privilegios de administrador. Actualmente no tiene ninguna funcionalidad adicional, pero se ha dejado con vistas a futuras ampliaciones del \textit{software}.
\end{itemize}

\subsection{Casos de Uso}
En cuanto a los casos de uso en sí mismos, se han identificado los siguientes:

\begin{itemize}
\item \textbf{CU-1} Inicio de sesión (ver Tabla \ref{tab:cu1}).
\item \textbf{CU-2} Registro (ver Tabla \ref{tab:cu2}).
\item \textbf{CU-3} Consultar revistas (ver Tabla \ref{tab:cu3}).
\item \textbf{CU-4} Consultar histórico (ver Tabla \ref{tab:cu4}).
\item \textbf{CU-5} Predecir JCR (ver Tabla \ref{tab:cu5}).
\item \textbf{CU-6} Cerrar sesión (ver Tabla \ref{tab:cu6}).

\begin{table}[p]
\centering
\begin{tabularx}{\linewidth}{ p{0.21\columnwidth} p{0.71\columnwidth} }
\toprule
\textbf{CU-1} & \textbf{Inicio de sesión}\\
\toprule
\textbf{Versión} & 1.0 \\
\textbf{Autor} & Gadea Lucas Pérez \\
\textbf{Requisitos asociados} & RF1, RF1.1, RF1.4  \\
\textbf{Descripción} & Este caso de uso describe el proceso de autenticación de los usuarios en la aplicación.\\
\textbf{Precondición} & El usuario debe estar registrado en la aplicación.\\
\textbf{Acciones} &
\begin{enumerate}
\def\labelenumi{\arabic{enumi}.}
\tightlist
\item El usuario introduce su nombre de usuario y contraseña (pudiendo mostrar y ocultar su contenido).
\item El sistema verifica la validez de las credenciales ingresadas.
\item El sistema otorga acceso al usuario si las credenciales son válidas.
\end{enumerate}\\
\textbf{Postcondición} & El usuario ha iniciado sesión en la aplicación.\\
\textbf{Excepciones} &
\begin{itemize}
\item Si las credenciales son inválidas, se muestra un mensaje de error y se solicita al usuario que vuelva a ingresar las credenciales.
\end{itemize}\\
\textbf{Importancia} & Alta\\
\bottomrule
\end{tabularx}
\caption{CU-1 Inicio de sesión}
\label{tab:cu1}
\end{table}


\begin{table}[p]
\centering
\begin{tabularx}{\linewidth}{ p{0.21\columnwidth} p{0.71\columnwidth} }
\toprule
\textbf{CU-2} & \textbf{Registro}\\
\toprule
\textbf{Versión} & 1.0 \\
\textbf{Autor} & Gadea Lucas Pérez \\
\textbf{Requisitos asociados} & RF1, RF1.3, RF1.4  \\
\textbf{Descripción} & Este caso de uso describe el proceso de registro de nuevos usuarios en la aplicación.\\
\textbf{Precondición} & - \\
\textbf{Acciones} &
\begin{enumerate}
\def\labelenumi{\arabic{enumi}.}
\tightlist
\item El usuario selecciona la opción de registro en la interfaz de inicio de sesión.
\item El usuario proporciona un nombre de usuario, una contraseña (pudiendo mostrar y ocultar su contenido), una dirección de correo electrónico válida y confirma la contraseña.
\item El sistema verifica que la contraseña tenga un formato válido y que la dirección de correo electrónico sea válida.
\item El sistema verifica que la contraseña y la confirmación de la contraseña coincidan.
\item El sistema crea un nuevo usuario en la base de datos con la información proporcionada.
\item El usuario es redireccionado a la interfaz de inicio tras recibir un mensaje de éxito.
\end{enumerate}\\
\textbf{Postcondición} & El usuario se ha registrado exitosamente.\\
\textbf{Excepciones} &
\begin{itemize}
\item Si la contraseña no cumple con el formato válido (i.e.: no tiene una longitud mínima de 8 caracteres, no contiene caracteres especiales requeridos, etc.), se muestra un mensaje de error y se solicita al usuario que ingrese una contraseña válida.
\item Si la dirección de correo electrónico no es válida, se muestra un mensaje de error y se solicita al usuario que ingrese una dirección válida.
\item Si la contraseña y la confirmación de la contraseña no coinciden, se muestra un mensaje de error y se solicita al usuario que vuelva a ingresar la contraseña y su confirmación.
\end{itemize}\\
\textbf{Importancia} & Alta\\
\bottomrule
\end{tabularx}
\caption{CU-2 Registro.}
\label{tab:cu2}
\end{table}

\begin{table}[p]
\centering
\begin{tabularx}{\linewidth}{ p{0.21\columnwidth} p{0.71\columnwidth} }
\toprule
\textbf{CU-3} & \textbf{Consultar revistas}\\
\toprule
\textbf{Versión} & 1.0 \\
\textbf{Autor} & Gadea Lucas Pérez \\
\textbf{Requisitos asociados} & RF2, RF2.2 \\
\textbf{Descripción} & Este caso de uso describe la interfaz que muestra una lista de revistas y su información relevante, con funcionalidades de búsqueda y paginación.\\
\textbf{Precondición} & El usuario ha iniciado sesión y ha accedido a la interfaz principal. \\
\textbf{Acciones} &
\begin{enumerate}
\def\labelenumi{\arabic{enumi}.}
\tightlist
\item El usuario accede a la interfaz de revistas pulsando sobre el icono correspondiente.
\item Se obtiene la lista completa de revistas de la base de datos.
\item La interfaz muestra una lista de revistas con su información relevante.
\item La interfaz incluye una barra de búsqueda por texto para filtrar las revistas en base a palabras clave. Muestra también una paginación para navegar entre las diferentes páginas de resultados.
\end{enumerate}\\
\textbf{Postcondición} & - \\
\textbf{Excepciones} &
\begin{itemize}
    \item Si la lista de revistas no se puede conseguir exitosamente de la base de datos, se redirecciona al usuario a una página de error.
\end{itemize} \\
\textbf{Importancia} & Alta\\
\bottomrule
\end{tabularx}
\caption{CU-3 Consultar revistas}
\label{tab:cu3}
\end{table}

\begin{table}[p]
\centering
\begin{tabularx}{\linewidth}{ p{0.21\columnwidth} p{0.71\columnwidth} }
\toprule
\textbf{CU-4} & \textbf{Consultar histórico}\\
\toprule
\textbf{Versión} & 1.0 \\
\textbf{Autor} & Gadea Lucas Pérez \\
\textbf{Requisitos asociados} & RF2.3, RF3.1, RF3.2 \\
\textbf{Descripción} & Este caso de uso describe el proceso para consultar el histórico del JCR de una revista en particular.\\
\textbf{Precondición} & El usuario ha iniciado sesión en la aplicación y está en la interfaz principal. \\
\textbf{Acciones} &
\begin{enumerate}
\def\labelenumi{\arabic{enumi}.}
\tightlist
\item La interfaz principal muestra listas desplegables (\textit{dropdowns}) con opciones de categorías y nombres de revistas. El usuario selecciona las opciones que desee.
\item El sistema valida los campos obligatorios y activa el botón. 
\item El usuario pulsa el botón y es redirigido a la interfaz de consulta del JCR.
\item Se carga la interfaz con la información solicitada (muestra dos gráficos, uno con los cuartiles y otro con los valores del JCR).
\end{enumerate}\\
\textbf{Postcondición} & - \\
\textbf{Excepciones} &
\begin{itemize}
    \item Si el usuario no selecciona ambos menús desplegables, el botón no se activa. Un \textit{tooltip} aconsejará completar la información.
\end{itemize} \\
\textbf{Importancia} & Alta\\
\bottomrule
\end{tabularx}
\caption{CU-4 Consultar histórico}
\label{tab:cu4}
\end{table}


\begin{table}[p]
\centering
\begin{tabularx}{\linewidth}{ p{0.21\columnwidth} p{0.71\columnwidth} }
\toprule
\textbf{CU-5} & \textbf{Consultar predicción}\\
\toprule
\textbf{Versión} & 1.0 \\
\textbf{Autor} & Gadea Lucas Pérez \\
\textbf{Requisitos asociados} & RF2.4, RF4.1, RF4.2, RF4.3, RF4.4 \\
\textbf{Descripción} & Este caso de uso permite a los usuarios consultar la predicción del JCR de una revista utilizando uno o más modelos de predicción.\\
\textbf{Precondición} & El usuario ha iniciado sesión en la aplicación y está en la interfaz principal. El usuario ya ha seleccionado la categoría y el nombre de la revista.\\
\textbf{Acciones} &
\begin{enumerate}
\def\labelenumi{\arabic{enumi}.}
\tightlist
\item El usuario selecciona uno o más modelos de predicción en la interfaz principal.
\item El sistema valida que al menos un modelo está seleccionado y activa el botón. 
\item El usuario hace clic sobre el botón.
\item Se muestra la interfaz de Predicción JCR con la predicción del JCR de cada modelo para el año actual.
\item El usuario puede ocultar cualquier línea del gráfico haciendo clic sobre su leyenda.
\end{enumerate}\\
\textbf{Postcondición} & - \\
\textbf{Excepciones} &
\begin{itemize}
    \item Si el usuario no selecciona alguno de los modelos, el botón no se activa. Un \textit{tooltip} aconsejará completar la información.
\end{itemize} \\
\textbf{Importancia} & Alta\\
\bottomrule
\end{tabularx}
\caption{CU-5 Consultar predicción.}
\label{tab:cu5}
\end{table}

\begin{table}[p]
\centering
\begin{tabularx}{\linewidth}{ p{0.21\columnwidth} p{0.71\columnwidth} }
\toprule
\textbf{CU-6} & \textbf{Cerrar sesión}\\
\toprule
\textbf{Versión} & 1.0 \\
\textbf{Autor} & Gadea Lucas Pérez \\
\textbf{Requisitos asociados} & RF1, RF2\\
\textbf{Descripción} & Este caso de abarca el proceso de cierre de sesión.\\
\textbf{Precondición} & El usuario ha iniciado sesión en la aplicación.\\
\textbf{Acciones} &
\begin{enumerate}
\def\labelenumi{\arabic{enumi}.}
\tightlist
\item El usuario selecciona (en el \textit{dropdown} de la esquina superior derecha) la opción <<cerra sesión>>.
\item Se eliminan las variables de sesión para el usuario actual. 
\item El usuario es redireccionado a la interfaz de inicio.
\end{enumerate}\\
\textbf{Postcondición} & El usuario ya no está \textit{logueado}\\
\textbf{Excepciones} & - \\
\textbf{Importancia} & Alta\\
\bottomrule
\end{tabularx}
\caption{CU-6 Cerrar sesión}
\label{tab:cu6}
\end{table}

\subsection{Diagrama de casos de usos}

Finalmente, para poder visualizar todos estos casos, se incluye un diagrama de casos de uso (ver Figura \ref{fig:cu}).

\imagen{cu}{Diagrama de Casos de Uso}{0.8}