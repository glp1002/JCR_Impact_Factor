
\apendice{Plan de Proyecto Software}

\section{Introducción}

La planificación temporal es esencial para el éxito de cualquier proyecto, especialmente en el desarrollo de \textit{software}. En esta sección se detallará cómo se ha llevado a cabo el cronograma siguiendo una metodología ágil tipo Scrum, mediante el uso de \textit{sprints}. Se describirán los pasos necesarios para planificar y gestionar de manera eficiente el tiempo y los recursos disponibles, asegurando así el cumplimiento de los objetivos del proyecto en el plazo establecido.

\section{Metodología Scrum y Kanban}
La metodología Scrum es un marco de trabajo ágil que se utiliza para la gestión de proyectos. Se basa en la colaboración entre el equipo de desarrollo y el cliente, así como en la entrega continua de productos funcionales en ciclos cortos de tiempo conocidos como \textit{sprints}~\cite{Palacio2022}. Scrum se centra en la flexibilidad y la adaptabilidad, permitiendo más fácilmente reajustes y los cambios de manera rápida y efectiva ante situaciones inesperadas. 

En el contexto del presente trabajo, el uso de la metodología Scrum será beneficioso puesto que nos permitirá tener un enfoque más dinámico y flexible, lo que nos posibilitará adaptarnos a las necesidades del proyecto a medida que avanzamos. Además, Scrum nos proporcionará una mayor transparencia y comunicación, lo que nos capacitará para tomar decisiones informadas y asegurar que el proyecto se entregue a tiempo y con éxito.

Por otro lado, se ha complementado la metodología Scrum con el modelo Kanban\footnote{Este modelo se refiere comúnmente a un lienzo o plantilla visual utilizada para estructurar y desarrollar ideas de negocio.} propio de la metodología Kanban~\cite{Martins_2022}. Esta combinación nos permitirá visualizar y organizar de manera efectiva las tareas, los hitos y los recursos disponibles, facilitando así la gestión y el seguimiento del proyecto.

Antes de continuar con la planificación del proyecto, es importante familiarizarnos con algunos conceptos clave que son fundamentales para comprender la metodología Scrum. A continuación, se proporcionará una breve descripción de los roles, artefactos y la medición de las tareas en \textit{story points}.

\subsection{Artefactos en Scrum}
En cuanto a los artefactos de Scrum, hay tres principales: el \textit{product backlog}, el \textit{sprint backlog} y el incremento. 
\begin{enumerate}
    \item El \textit{product backlog} es una lista priorizada de todas las funcionalidades, características o mejoras que se desean implementar en el producto final. 
    \item El \textit{sprint backlog} es una selección de elementos del \textit{backlog} del producto que se abordarán en un \textit{sprint} específico. 
    \item Y finalmente, el incremento es el resultado tangible y potencialmente entregable al final de cada \textit{sprint}, que incorpora nuevas funcionalidades o mejoras al producto.
\end{enumerate}

\subsection{Roles en Scrum}
En Scrum, existen tres roles principales: el Scrum Master, el Product Owner y el Equipo de Desarrollo. 
\begin{enumerate}
    \item El Scrum Master es responsable de garantizar que el equipo siga las prácticas y principios de Scrum, eliminando cualquier obstáculo que pueda afectar a la productividad. 
    \item El Product Owner es el encargado de definir y priorizar los elementos del \textit{product backlog}, asegurándose de que se cumplan las necesidades del cliente. 
    \item Por último, el Equipo de Desarrollo es responsable de la implementación de las tareas y la entrega del producto.
\end{enumerate}

Dado que en este caso el equipo está conformado solamente por los dos tutores y la alumna, es importante adaptar los roles y las prácticas de Scrum para que se ajusten a esta situación específica. Se pueden asignar los roles de Scrum Master y Product Owner a cada uno de los tutores y, el rol de Equipo de Desarrollo, correspondería a la alumna. 

\subsection{Medición de las tareas}
En cuanto a la medición de las tareas en \textit{story points}, Scrum utiliza esta técnica para estimar el esfuerzo relativo requerido para completar una tarea. Los \textit{story points} son una medida abstracta que no se relaciona directamente con el tiempo, sino con la complejidad y el esfuerzo necesario para completar una tarea en comparación con otras. Esta medida ayuda al equipo a planificar su capacidad de trabajo en cada \textit{sprint} y a realizar estimaciones más precisas en función de su experiencia previa.


\section{Herramienta}
Para llevar a cabo este proyecto, se han utilizado herramientas colaborativas como GitHub y ZenHub. GitHub proporciona un entorno seguro y eficiente para almacenar y gestionar el código fuente, permitiendo trabajar de manera transparente y controlada. Por otro lado, ZenHub ha sido fundamental en la organización de las tareas y el seguimiento de los avances, brindando una visión clara y estructurada del progreso del proyecto.  

\section{Planificación temporal}
El presente proyecto comienza en septiembre y se extiende hasta junio. Durante este período de tiempo, es importante asegurarnos de que todas las tareas e hitos estén claramente definidos.

Para lograr esto, se han realizado reuniones periódicas para discutir el progreso del proyecto y asegurarnos de que estamos en el camino correcto. Además, hemos implementado \textit{sprints} regulares (de entre una y dos semanas de duración) para asegurarnos de que estamos avanzando de manera constante y cumpliendo con nuestras metas a tiempo.

Con esta planificación temporal sólida, estamos seguros de que podremos completar el proyecto a tiempo y cumplir con los objetivos establecidos. Sin embargo, es importante ser flexibles y estar preparados para hacer ajustes según sea necesario a medida que avanzamos en el proyecto, puesto que se realiza al mismo tiempo que trascurre el curso académico.

En las siguientes tablas se recogen los distintos \textit{sprints} junto con su duración, objetivos y tareas. Además, se adjuntará el gráfico \textit{burndown}\footnote{Gráfico \textit{burndown}: se trata de una herramienta comúnmente utilizada en la gestión de proyectos ágiles para realizar un seguimiento de un \textit{sprint}. Este gráfico muestra la cantidad de trabajo que queda por hacer en relación con el tiempo disponible para hacerlo. A medida que se completa el trabajo, la línea del gráfico debería disminuir hasta alcanzar cero en el momento en que finaliza el \textit{sprint}.} de cada uno de los \textit{sprints}. Por otro lado, una descripción detallada de las tareas y su duración puede verse en GitHub/ZenHub.

\subsection{Sprint 1}

\begin{table}[h]
\centering
\begin{tabularx}{\textwidth}{llll}
\toprule
\textbf{Sprt.} & \textbf{Duración} & \textbf{Objetivos} & \textbf{Tareas}\\
\midrule
 1 & \parbox{55}{19/09/2022 03/10/2022} & \parbox{80}{Comenzar el desarrollo del proyecto} & \parbox{150}{\begin{itemize}\item Elegir un modelo de referencias bibliográficas. \item Definición de conceptos.\end{itemize}}\\
\bottomrule
\end{tabularx}
\caption{Tareas del sprint 1}
\label{tab:sprint1}
\end{table}


\imagen{sprint1}{Sprint 1: Burndown chart}{0.9}

\subsection{Sprint 2}
\begin{table}[h]
\centering
\begin{tabularx}{\textwidth}{llll}
\toprule
\textbf{Sprt.} & \textbf{Duración} & \textbf{Objetivos} & \textbf{Tareas}\\
\midrule
 2 & \parbox{55}{03/10/2022 17/10/2022} & \parbox{80}{Tareas de investigación y documentación} & \parbox{150}{\begin{itemize}\item Búsqueda de antecedentes. \item Familiarizarse con el procesador de textos \LaTeX.\end{itemize}}\\
\bottomrule
\end{tabularx}
\caption{Tareas del sprint 2}
\label{tab:sprint2}
\end{table}

\imagen{sprint2}{Sprint 2: Burndown chart}{0.9}

\subsection{Sprint 3}

\begin{table}[h]
\centering
\begin{tabularx}{\textwidth}{llll}
\toprule
\textbf{Sprt.} & \textbf{Duración} & \textbf{Objetivos} & \textbf{Tareas}\\
\midrule
3 & \parbox{55}{17/10/2022 14/11/2022} & \parbox{80}{Investigación y creación de un prototipo inicial} & \parbox{150}{\begin{itemize}\item Búsqueda de información sobre MIAR. \item Documentación sobre el JCR y el índice de impacto. \item Prototipo que permita la búsqueda de artículos en GS. \item Extracción de datos: ¿qué datos se pueden extraer? \item Reflexión sobre el diseño de la BBDD. \item Prueba del prototipo.\end{itemize}}\\
\bottomrule
\end{tabularx}
\caption{Tareas del sprint 3}
\label{tab:sprint3}
\end{table}

\imagen{sprint3}{Sprint 3: Burndown chart}{0.9}

\subsection{Sprint 4}
\begin{table}[h]
\centering
\begin{tabularx}{\textwidth}{llll}
\toprule
\textbf{Sprt.} & \textbf{Duración} & \textbf{Objetivos} & \textbf{Tareas}\\
\midrule
    4 & \parbox{55}{14/11/2022 28/11/2022} & \parbox{80}{Base de datos y prototipo} & \parbox{150}{\begin{itemize}\item Creación de la base de datos en MariaDB. \item Prueba de la BBDD. \item Mejoras del prototipo.\end{itemize}}\\
\bottomrule
\end{tabularx}
\caption{Tareas del sprint 4}
\label{tab:sprint4}
\end{table}

\nota{Como se puede observar no se ha añadido un gráfico \textit{burndown} en este \textit{sprint}. Esto se debe a que las tareas no fueron cerradas a tiempo. Sin embargo, es preciso aclarar que se trata de un error en la configuración de las tareas. Si bien todas las tareas fueron completadas a tiempo, fue preciso borrarlas en ZenHub y volverlas a crear más adelante cuando el \textit{sprint} ya había terminado debido a que estaban inicialmente mal configuradas.}

\subsection{Sprint 5}

\begin{table}[h]
\centering
\begin{tabularx}{\textwidth}{llll}
\toprule
\textbf{Sprt.} & \textbf{Duración} & \textbf{Objetivos} & \textbf{Tareas}\\
\midrule
    5 & \parbox{55}{28/11/2022 12/12/2022} & \parbox{80}{Base de datos y prototipo} & \parbox{150}{\begin{itemize}\item Se muda la BBDD a Postgres. \item Pruebas e investigación para extraer el DOI.\end{itemize}}\\
\bottomrule
\end{tabularx}
\caption{Tareas del sprint 5}
\label{tab:sprint5}
\end{table}

\imagen{sprint5}{Sprint 5: Burndown chart}{0.9}

\subsection{Sprint 6}
\begin{table}[h]
\centering
\begin{tabularx}{\textwidth}{llll}
\toprule
\textbf{Sprt.} & \textbf{Duración} & \textbf{Objetivos} & \textbf{Tareas}\\
\midrule
    6 & \parbox{55}{09/01/2023 23/01/2023} & \parbox{80}{Documentación y obtención del DOI} & \parbox{150}{\begin{itemize}\item Obtención del DOI a través de Crossref. \item Avance de la memoria y los anexos. \item Mejora de la BBDD.\end{itemize}}\\
\bottomrule
\end{tabularx}
\caption{Tareas del sprint 6}
\label{tab:sprint6}
\end{table}

\imagen{sprint6}{Sprint 6: Burndown chart}{0.9}

\subsection{Sprint 7}

\begin{table}[h]
\centering
\begin{tabularx}{\textwidth}{llll}
\toprule
\textbf{Sprt.} & \textbf{Duración} & \textbf{Objetivos} & \textbf{Tareas}\\
\midrule
    7 & \parbox{55}{06/02/2023 20/02/2023} & \parbox{80}{Preparación para la conexión remota a los servidores de la universidad.} & \parbox{150}{\begin{itemize}\item Conectarse a la VPN de la universidad. \item Crear un servicio virtualizado en Miniconda. \item Mejorar la BBDD.\end{itemize}}\\
\bottomrule
\end{tabularx}
\caption{Tareas del sprint 7}
\label{tab:sprint7}
\end{table}

\imagen{sprint7}{Sprint 7: Burndown chart}{0.9}

\subsection{Sprint 8}
\begin{table}[h]
\centering
\begin{tabularx}{\textwidth}{llll}
\toprule
\textbf{Sprt.} & \textbf{Duración} & \textbf{Objetivos} & \textbf{Tareas}\\
\midrule
    8 & \parbox{55}{20/02/2023 06/03/2023} & \parbox{80}{Conexión a los servidores y mejora de los prototipos. Avances en la API y documentación.} & \parbox{150}{\begin{itemize}\item Mejorar el prototipo. \item Prototipo con varios hilos de ejecución (usar servidores de la universidad). \item Nuevo prototipo con Selenium. \item Sistema inteligente de requests. \item Revisión de PoP. \item API: Autentificación y back-end.\end{itemize}}\\
\bottomrule
\end{tabularx}
\caption{Tareas del sprint 8}
\label{tab:sprint8}
\end{table}

\imagen{sprint8}{Sprint 8: Burndown chart}{0.9}

\subsection{Sprint 9}
\begin{table}[h]
\centering
\begin{tabularx}{\textwidth}{llll}
\toprule
\textbf{Sprt.} & \textbf{Duración} & \textbf{Objetivos} & \textbf{Tareas}\\
\midrule
     9 & \parbox{55}{06/03/2023 20/03/2023} & \parbox{80}{Finalización del prototipo de extracción de Crossref.} & \parbox{150}{\begin{itemize}\item Plan de prueba para Crossref. \item Prototipo de Crossref. \item Comprobar límite de llamadas a Crossref. \item Crear entorno virtualizado. \item Fichero requirements. \item Licencia Crossref para Cited-by. \item Cálculo del IF.\end{itemize}}\\
\bottomrule
\end{tabularx}
\caption{Tareas del sprint 9}
\label{tab:sprint9}
\end{table}

\imagen{sprint9}{Sprint 9: Burndown chart}{0.9}

\subsection{Sprint 10}
\begin{table}[h]
\centering
\begin{tabularx}{\textwidth}{llll}
\toprule
\textbf{Sprt.} & \textbf{Duración} & \textbf{Objetivos} & \textbf{Tareas}\\
\midrule
    10 & \parbox{55}{03/04/2023 17/04/2023} & \parbox{80}{Procesamiento de los datos, creación de los modelos de predicción y diseño de interfaces.} & \parbox{150}{\begin{itemize}\item Normalizar listados JCR. \item Comparar JCR obtenidos con los valores reales.\item Modelos de Crossvalidation. \item Diseño de las interfaces de la aplicación. \item Diagramas UML de la API.\end{itemize}}\\
\bottomrule
\end{tabularx}
\caption{Tareas del sprint 10}
\label{tab:sprint10}
\end{table}

\imagen{sprint10}{Sprint 10: Burndown chart}{0.9}

\subsection{Sprint 11}

\begin{table}[h]
\centering
\begin{tabularx}{\textwidth}{llll}
\toprule
\textbf{Sprt.} & \textbf{Duración} & \textbf{Objetivos} & \textbf{Tareas}\\
\midrule
    11 & \parbox{55}{24/04/2023 15/05/2023} & \parbox{80}{Creación de un prototipo de aplicación web y exportación de los modelos de predicción} & \parbox{150}{\begin{itemize}\item Prototipo de API \item API con manejo de errores \item API con conexión a BBDD \item API con interfaz de prueba \item API con internacionalización \item Generar ficheros binarios para los modelos de predicción.\end{itemize}}\\
\bottomrule
\end{tabularx}
\caption{Tareas del sprint 11}
\label{tab:sprint11}
\end{table}

\imagen{sprint11}{Sprint 11: Burndown chart}{0.9}

\subsection{Sprint 12}

\begin{table}[h]
\centering
\begin{tabularx}{\textwidth}{llll}
\toprule
\textbf{Sprt.} & \textbf{Duración} & \textbf{Objetivos} & \textbf{Tareas}\\
\midrule
    12 & \parbox{55}{15/05/2023 22/05/2023} & \parbox{80}{Aplicación con todas las funcionalidades básicas en base al prototipo y corrección de la carga de información en la BBDD.} & \parbox{150}{\begin{itemize}\item Funcionalidades básicas de la API \item Cargar datos correctos\end{itemize}}\\
\bottomrule
\end{tabularx}
\caption{Tareas del sprint 12}
\label{tab:sprint12}
\end{table}
 
\imagen{sprint12}{Sprint 12: Burndown chart}{0.9}

\subsection{Sprint 13}
\begin{table}[h]
\centering
\begin{tabularx}{\textwidth}{llll}
\toprule
\textbf{Sprt.} & \textbf{Duración} & \textbf{Objetivos} & \textbf{Tareas}\\
\midrule
    13 & \parbox{55}{15/05/2023 22/05/2023} & \parbox{80}{Lanzar la aplicación en Heroku y mejora de funcionalidades.} & \parbox{150}{\begin{itemize}\item Serie temporal \item Añadir el cuartil a las revistas \item  Configurar API en Heroku \item Configurar BBDD en Heroku\end{itemize}}\\
\bottomrule
\end{tabularx}
\caption{Tareas del sprint 13}
\label{tab:sprint13}
\end{table}

\imagen{sprint13}{Sprint 13: Burndown chart}{0.9}

\subsection{Sprint 14}
\begin{table}[h]
\centering
\begin{tabularx}{\textwidth}{llll}
\toprule
\textbf{Sprt.} & \textbf{Duración} & \textbf{Objetivos} & \textbf{Tareas}\\
\midrule
    14 & \parbox{55}{29/05/2023 22/05/2023} & \parbox{80}{Último sprint. Retoques finales y solución de \textit{bugs}} & \parbox{150}{\begin{itemize}\item Solucionar login \item Obtener cuartiles \item  Seguridad de contraseñas \item Documentación \item Bugs\end{itemize}}\\
\bottomrule
\end{tabularx}
\caption{Tareas del sprint 14}
\label{tab:sprint14}
\end{table}

\imagen{sprint14}{Sprint 14: Burndown chart}{0.9}

\subsection{Overview}
\definecolor{mypurple}{RGB}{128,0,128} 
\definecolor{mygreen}{RGB}{0,128,0} 
Finalmente, se incluye un gráfico acumulativo\footnote{Gráfico acumulativo: en este gráfico, el eje vertical representa la cantidad total de trabajo completado, mientras que el eje horizontal representa el tiempo transcurrido durante el proyecto. A medida que se completa el trabajo, la línea del gráfico se eleva, lo que indica el progreso general del proyecto} (ver figura \ref{fig:cumulative}) que muestra la cantidad total de trabajo realizado de forma más resumida.

El eje horizontal representa el período de tiempo, mientras que el eje vertical indica la cantidad acumulada de tareas completadas. A medida que avanza el tiempo, se puede observar cómo se han completado gradualmente las tareas y cómo ha progresado el proyecto. Además, las diferentes líneas de colores representan las diferentes etapas o estados de las tareas en un proyecto. Cada color corresponde a un estado específico, como \textcolor{mygreen}{<<Nuevas tareas>>}, \textcolor{mypurple}{<<En progreso>>}, \textcolor{orange}{<<Completado>>}, \textcolor{purple}{<<Cerradas>>}, etc. Estas líneas muestran cómo se distribuyen las tareas a lo largo del tiempo y cómo evolucionan desde su creación hasta su finalización. Al observar el gráfico, se puede identificar fácilmente la proporción de tareas en cada estado y cómo cambia esa proporción a medida que progresa el proyecto.

\imagen{cumulative}{Gráfico acumulativo de todo el proyecto}{0.9}



\section{Estudio de viabilidad}
El estudio de viabilidad es un componente esencial en el proceso de evaluación de cualquier proyecto o iniciativa. En el contexto de nuestra investigación, se ha desarrollado una aplicación \textit{open source} destinada a beneficiar y servir a la comunidad científica. 

Dado que la intención no es obtener beneficios económicos directos de este proyecto, se considerará un enfoque empresarial realista para asegurar la viabilidad y sostenibilidad a largo plazo de nuestro proyecto.

En las siguientes secciones, se analizarán en detalle los aspectos de viabilidad económica (sección \ref{sec:Viabilidad económica}) y legal (sección \ref{sec:Viabilidad legal}), con el objetivo de establecer un marco sólido que asegure la continuidad y éxito del proyecto.

\subsection{Viabilidad económica}
\label{sec:Viabilidad económica}
En primer lugar, la viabilidad económica se refiere a la evaluación de los recursos financieros necesarios para llevar a cabo el proyecto y asegurar su sostenibilidad a largo plazo. Aunque nuestra aplicación no tiene como objetivo obtener ganancias, es importante considerar los costos asociados al desarrollo, mantenimiento y mejora continua de la plataforma. Además, se deben analizar posibles fuentes de financiación, como subvenciones, donaciones o colaboraciones con instituciones interesadas en respaldar este tipo de iniciativas científicas abiertas.

Algunos aspectos relevantes para poder hacer estos cálculos son los siguientes: 
\begin{itemize}
    \item La empresa se situaría en España, lo que implica tener en cuenta las regulaciones y requisitos legales y fiscales del país.
    \item El proyecto tiene una duración estimada de 8 meses y una semana, desde octubre hasta la segunda semana de junio. Esto equivale aproximadamente a 34 semanas, considerando una duración promedio de 4 semanas por mes.
    \item El equipo encargado del proyecto está formado por una desarrolladora (la alumna), así como por el \textit{Product Owner} y el \textit{Scrum Master}, quienes actúan como tutores de la alumna en el desarrollo del proyecto.
\end{itemize}

Para todo esto, es fundamental considerar los costes y beneficios asociados al proyecto. Los \textbf{costes} se refieren a los desembolsos monetarios necesarios para llevar a cabo la iniciativa, incluyendo los gastos de inversión, los costes operativos y los costes de mantenimiento~\cite{costebeneficio2006}. Por otro lado, los \textbf{beneficios} se refieren a las ganancias económicas que se esperan obtener como resultado de la implementación del proyecto. Estos beneficios pueden incluir ingresos generados por la venta de productos o servicios, ahorros en costes operativos, incremento en la productividad, entre otros~\cite{costebeneficio2006}. 

\subsubsection{Costes}
Comenzaremos por calcular los costes totales de la supuesta empresa.

\begin{enumerate}
    \item \underline{Costes de los empleados}.
    
    Se procederá al cálculo del salario bruto de cada integrante del equipo de acuerdo con la normativa laboral vigente en España para un proyecto de desarrollo de \textit{software}. El salario bruto se determina considerando factores como la categoría profesional y la experiencia de cada empleado.

    \begin{enumerate}
        \item \underline{Desarrollador}: en este caso, la alumna es considerada una programadora \textit{junior}. En España, un programador \textit{junior} (con menos de 3 años de experiencia laboral) puede esperar un salario medio de alrededor de 19\,700 € brutos por año~\cite{Jobted}. Además, según el artículo 34 del Estatuto de los Trabajadores, la duración fijada en convenio colectivo de la jornada laboral es un máximo de 40 horas de trabajo efectivo~\cite{jornadalaboral}. Ahora, para calcular el salario bruto mensual, se puede utilizar la siguiente fórmula\footnote{Entendemos habitualmente 14 pagas (12 meses y 2 pagas extra).}:

        \[
        \text{{Salario Bruto Mensual}} = \frac{{\text{{Salario Bruto Anual}}}}{{14 \, \text{{pagas}}}}
        \]
        
        Sustituyendo el valor del salario bruto anual, obtenemos:
        
        \[
        \text{{Salario Bruto Mensual}} = \frac{{19\,700 \,\text{{€}}}}{{14  \,\text{{pagas}}}} \approx 1\,407,14 \, \text{{€}}
        \]

        \item \underline{Product Owner}: el salario bruto promedio anual para un Product Owner en España es de 41.866 €~\cite{PayScale2}. Utilizando la misma fórmula para calcular el salario bruto mensual, obtenemos:

        \[\text{{Salario Bruto Mensual}} = \frac{{41\,866 \, \text{{€}}}}{{14 \, \text{{pagas}}}} \approx 2\,990,43 \,\text{{€}}\]
        
        \item \underline{Scrum Master}: el salario bruto promedio anual para un Scrum Master en España es de 40.626 €~\cite{PayScale1}. Utilizando la misma fórmula para calcular el salario bruto mensual, obtenemos:
        
        \[\text{{Salario Bruto Mensual}} = \frac{{40\,626 \,\text{{€}}}}{{14 \, \text{{pagas}}}} \approx 2\,901,85  \,\text{{€}}\]
        
    \end{enumerate}
    
    Una vez obtenido el salario bruto, se deben aplicar los impuestos correspondientes de acuerdo con la legislación fiscal vigente. Los impuestos a tener en cuenta incluyen las contribuciones a la Seguridad Social~\cite{Cotización}, el impuesto sobre la renta\footnote{En cuanto al IRPF, el porcentaje aplicado puede variar según el nivel de ingresos y la situación fiscal personal. Según la Agencia Tributaria de España, se pueden encontrar las tablas y los porcentajes correspondientes en su documentación oficial~\cite{AEAT}.} (IRPF)~\cite{IRPF} y otros impuestos relacionados con la contratación laboral. La información detallada sobre los impuestos y las obligaciones legales se encuentra en los documentos oficiales y normativas correspondientes proporcionados por las autoridades fiscales y laborales citados anteriormente. 
    
    
    De forma resumida, Para el año 2023, se han establecido los siguientes porcentajes\footnote{El gobierno pone a disposición la plataforma ipyme~\cite{ipyme}, que proporciona información valiosa para facilitar el cálculo de los tramites empresariales.}, los cuales se detallan en la Tabla \ref{tab:seguridad-social} y Tabla \ref{tab:no-seguridad-social}.

\begin{table}[h]
\centering
\begin{tabularx}{320}{@{}XXr@{}}
\toprule
\textbf{Concepto} & \textbf{Descripción} & \textbf{Porc. (\%)} \\
\midrule
Cuota Contingencias Comunes & Gastos comunes en el ámbito laboral & 23,60 \\
Cuota Formación Profesional & Inversión en formación y capacitación & 0,60 \\
Desempleo & Prestaciones por desempleo & 5,50 \\
Accidentes de Trabajo & Prevención y compensación por accidentes laborales & 1,50 \\
FOGASA & Fondo de Garantía Salarial & 0,20 \\
\bottomrule
\end{tabularx}
\caption{Conceptos de Seguridad Social}
\label{tab:seguridad-social}
\end{table}

\begin{table}[h]
\centering
\begin{tabularx}{320}{@{}XXr@{}}
\toprule
\textbf{Concepto} & \textbf{Descripción} & \textbf{Porcentaje} \\
\midrule
IRPF & Impuesto sobre la Renta de las Personas Físicas & Variable\\
IVA & Impuesto sobre el Valor Añadido (tipo general) & 21,00\% \\
\bottomrule
\end{tabularx}
\caption{Otros conceptos}
\label{tab:no-seguridad-social}
\end{table}

    Con base en los cálculos previos de los salarios brutos y los impuestos aplicados, se puede determinar el coste total de la empresa en cuanto a los empleados. 
    
    \begin{enumerate}
        \item \underline{Desarrollador}: Dado que el salario bruto mensual de la desarrolladora es de aproximadamente 1\,641,67 €, se aplicarán los porcentajes correspondientes de los conceptos de la Seguridad Social según lo establecido en la Tabla \ref{tab:seguridad-social}. El total de los conceptos de Seguridad Social se calcula sumando los porcentajes aplicados al salario bruto mensual:

        \begin{align*}
        \text{Total SS} = 0.236 + 0.006 + 0.055 + 
         0.015 + 0.002 \approx 0.314
        \end{align*}
        
        El IRPF es un impuesto variable y su porcentaje de retención en nómina depende del nivel de ingresos y la situación fiscal personal. Por lo tanto, no se tendrá en cuenta para el resultado de este cálculo.
        
        Por último, el gasto de la empresa se calcula dividiendo el salario bruto mensual entre 1 menos el total de la Seguridad Social. Sustituyendo los valores correspondientes, obtenemos:
        
        \[
        \text{{Gasto mensual}} = \frac{{1\,407,14  \, \text{{€}}}}{{1 - 0,314 }} \approx 2\,051,22 \, \text{{€}}
        \]
        

        \item \underline{Product Owner}: Como el salario bruto anual para el Product Owner es de 3\,488,83 €, aplicando los porcentajes de la Tabla \ref{tab:seguridad-social}, el gasto total de la empresa para el Product Owner sería:

        \[
        \text{{Salario mensual}} = \frac{{2\,990,43  \text{{€}}}}{{1 - 0,314}} \approx 4\,359,22 \, \text{{€}}
        \]
        De las 154 horas de la jornada mensual (22 días laborables por 7 horas al día), hemos calculado una dedicación del 7\% de su tiempo. Por tanto, se va aplicar este 7\% al gasto mensual del Product Owner y del Scrum Master. Se ha estimado una dedicación de 2 horas de reuniones semanales más 3 horas repartidas a lo largo del mes.

        \[
        \text{{Gasto mensual}} = 4\,359,22 \, \text{{€}} \times 0,07 \approx 305,13 \,\text{{€}}
        \]

        \item \underline{Scrum Master}: De manera similar, para el Scrum Master, partimos de un salario bruto mensual aproximado de 3\,385,50 €. Aplicando los porcentajes de la Tabla \ref{tab:seguridad-social}, el gasto total de la empresa para el Scrum Master sería:

        \[
        \text{{Salario mensual}} = \frac{{2\,901,85 \, \text{{€}}}}{{1 - 0,314 }} \approx 4\,230,10 \, \text{{€}}
        \]

        Igualmente, al Scrum Master se le aplica un coeficiente reductor del 7\% de su salario, dado que ha invertido el mismo tiempo que el Product Owner.
        
        \[
        \text{{Gasto mensual}} = 4\,230,10 \, \text{{€}} \times 0,07 \approx 296,11 \,\text{{€}}
        \]

    \end{enumerate}

    A modo de resumen, se agrupa el resultado de todos los cálculos en la Tabla \ref{tab:costes-empleados}.
    
\begin{table}[h]
\centering
\begin{tabularx}{300}{@{}Xr@{}}
\toprule
\textbf{Empleado}  & \textbf{Gasto Mensual (€)}\\
\midrule
Desarrollador & 2\,051,22  \\
Product Owner & 305,13 \\
Scrum Master & 296,11  \\
\midrule
\textbf{Total Proyecto} & 2\,652,46 \\
\bottomrule
\end{tabularx}
\caption{Resumen de costes del personal}
\label{tab:costes-empleados}
\end{table}


    \item \underline{Costes de \textit{hardware}}.

    Para calcular los costes de \textit{hardware}, consideraremos únicamente el portátil de la alumna.
    Se trata de un MSI, modelo Prestige 15 A12UD-053XES con un procesador 12th Gen Intel(R) Core(TM) i7-1280P 2.00 GHz y una RAM de 32 GB. Este portatil se encuentra en el mercado a un precio de 1\,211,80 €.
    
    El coste de amortización del portátil de la alumna se basará en su vida útil estimada. Supongamos que el portátil tiene una vida útil de 4 años. Para calcular el coste anual de amortización, dividiremos el coste del portátil entre el número de años de vida útil:

        \[
        \text{{Coste anual de amortización}} = \frac{{1\,211,80  \, \text{{€}}}}{{4 \, \text{{años}}}} \approx 302,95 \, \text{{€/año}}
        \]

    Por tanto, como el proyecto ha durado 8 meses, los costes de \textit{hardware} suponen 302,95 por 66,66\% (proporción de los 8 meses sobre un año).

        \[
        \text{{Costes de \textit{hardware}}} = 302,95 \,\text{{€}} \, \times \, 66,66\% \approx 201,95 \, \text{{€}}
        \]
    \newline


    \item \underline{Costes de \textit{software}}.
    
    Comenzaremos por contemplar las licencias correspondientes a los sistemas operativos usados.

    \begin{itemize}
        \item \underline{Windows 10 Pro}: el valor aproximado de la licencia de Windows 10 Pro versión 22H2 es de 200 €. Al igual que en el caso del \textit{hardware}, consideraremos una vida útil de 4 años y aplicaremos la amortización correspondiente.

        \[
        \text{{Coste anual de amortización}} = \frac{{200 \, \text{{€}}}}{{4 \, \text{{años}}}} \approx 50 \, \text{{€/año}}
        \]

        Aplicando de nuevo el 66,66\% del porcentaje anual que ha supuesto nuestro proyecto (8 meses), tenemos lo siguiente:

        \[
        \text{{Costes de licencia Windows}} = 50 \,\text{{€}} \, \times \, 66,66\% \approx 33,33 \, \text{{€}}
        \]
        
        \item \underline{Linux Debian}: el portatil tiene una segunda partición con un sistema Debian que, debido a su filosofía de <<\textit{free software}>>~\cite{Debian}, es gratuito y no supone costes adicionales.
    \end{itemize}

    El resto del \textit{software} utilizado está disponible de forma gratuita o puede ser usado gracias a las licencias de estudiantes que ofrece la universidad, por lo que no ha generado costes adicionales. Suponiendo que finaliza el plazo de estas licencias gratuitas, debemos considerar los siguientes costes:

    \begin{itemize}
        \item \underline{Heroku}: se trata de un servidor de aplicaciones en la nube. Actualmente, se está utilizando una licencia gratuita de Heroku, gracias al \textit{Student Pack} de GitHub proporcionado por la universidad. Sin embargo, una vez que la licencia gratuita expire, se deberá considerar el coste mínimo de los planes de Heroku. A saber, el coste mínimo para el Dyno básico (web Gunicorn API.paperrank.app:app) es de 7,00 \$ (que equivale aproximadamente a 6,53 € mensuales). Además, será necesario tener en cuenta el coste adicional para disponer (al menos) de la versión mini de Heroku Postgres. Este coste es de 5,00 \$ (4,66 € mensuales).
        \item \underline{GitHub y extensión de ZenHub}: La licencia de GitHub (versión empresarial) tiene un coste de 17,95 € mensuales. Además, la extensión de ZenHub tiene un coste adicional de 12 €.  
    \end{itemize}

    Se adjunta una tabla resumen con los costes de \textit{software} estimados de forma mensual, considerando los elementos mencionados anteriormente (ver Tabla \ref{tab:costes-software}). 

\begin{table}[h]
\centering
\begin{tabularx}{320}{@{}lXr@{}}
\toprule
\textbf{Concepto} & \textbf{Descripción} & \textbf{Coste Mensual (€)} \\
\midrule
Windows 10 Pro & Licencia software & 4,17 \\
Heroku Dyno web & \textit{Cloud server} & 6,53 \\
Heroku Postgres & \textit{Cloud server} & 4,66 \\
GitHub Enterprise & Licencia software & 17,95 \\
ZenHub & Extensión GitHub & 12,00  \\
\midrule
\textbf{Total (mensual)} & & \textbf{45,31} \\
\bottomrule
\end{tabularx}
\caption{Costes de software mensuales}
\label{tab:costes-software}
\end{table}

    Para obtener los gastos totales del proyecto, simplemente se multiplica el costo mensual por el número de meses:

    \[
    \text{{Gastos totales}} = 45,31 \, \text{{€}} \times 8 \,\text{{ meses}} \approx 362,48 \, \text{{€}}
    \]

    Finalmente, aunque durante el desarrollo del proyecto se ha decidido no utilizar las licencias de las APIs de Google Scholar, Web of Science y Scopus, es importante tener en cuenta que en futuras investigaciones o aplicaciones más complejas, podría considerarse la necesidad de acceder a estas fuentes de información científica y académica. Estas APIs proporcionan un acceso especializado a una amplia gama de datos y recursos, lo cual podría requerir una inversión adicional en forma de licencias y sus correspondientes costos. Es recomendable evaluar detenidamente los requisitos del proyecto y el presupuesto disponible antes de tomar la decisión de adquirir estas licencias, en caso de que se considere necesario para la viabilidad y calidad de la aplicación.
\end{enumerate}

Para visualizar los costes de forma clara, se ha elaborado una última tabla resumen con todos los costes finales (ver Tabla \ref{tab:costes-totales}).

\begin{table}[h]
\centering
\begin{tabularx}{\textwidth}{@{}lXr@{}}
\toprule
\textbf{Concepto} & \textbf{Descripción} & \textbf{Coste Total (€)} \\
\midrule
Personal & Costes totales de empleados & 12\,416,43 \\
Hardware & Costes por el portátil & 302,70 \\
Software & Costes totales por licencias & 402,75 \\
\midrule
\textbf{Coste Total del Proyecto} & & 13\,121,88 \\
\bottomrule
\end{tabularx}
\caption{Resumen de costes del proyecto}
\label{tab:costes-totales}
\end{table}


\subsubsection{Beneficios}
Aunque la aplicación no tiene fines lucrativos y estará disponible de forma gratuita para la comunidad científica, es importante considerar una forma hipotética de generar ingresos para sostener el proyecto. Una opción sería incluir anuncios en la página web de la aplicación. 

La inclusión de anuncios publicitarios podría proporcionar una fuente potencial de ingresos. Sin embargo, es fundamental analizar las implicaciones y opciones disponibles antes de tomar una decisión. En primer lugar, se debe tener en cuenta que la incorporación de anuncios puede afectar la experiencia del usuario y la usabilidad de la aplicación. Es importante encontrar un equilibrio entre la generación de ingresos y la satisfacción de los usuarios.

Una opción es utilizar redes de publicidad en línea, como Google AdSense, que permiten mostrar anuncios relevantes y personalizados en la página web. Estas redes se encargan de gestionar los anuncios y los ingresos generados por ellos, ofreciendo a los propietarios de sitios web una parte de los ingresos publicitarios.

Otra posibilidad es buscar acuerdos directos con empresas o instituciones que deseen promocionar sus productos o servicios en la aplicación. Esto podría implicar la negociación de contratos publicitarios personalizados y la integración de anuncios específicos en la página web.

Es importante tener en cuenta que la inclusión de anuncios publicitarios requerirá un trabajo adicional en términos de diseño, implementación y gestión de la publicidad. Se deberá evaluar la viabilidad económica de los ingresos generados por los anuncios en comparación con los costes asociados. Es fundamental también considerar las implicaciones éticas y de privacidad asociadas con la publicidad en línea. Se deben seguir las políticas y regulaciones vigentes para garantizar la transparencia, el respeto de la privacidad de los usuarios y la integridad del proyecto.

\subsection{Viabilidad legal}
\label{sec:Viabilidad legal}
Por otro lado, la viabilidad legal implica evaluar las cuestiones jurídicas y normativas que puedan afectar la operación y distribución de la aplicación \textit{open source}. A pesar de que nuestro proyecto está destinado a ser utilizado por la comunidad científica en general, es necesario asegurarnos de cumplir con las regulaciones y requisitos legales relacionados con la privacidad, protección de datos y derechos de propiedad intelectual.

\subsubsection{Web Scraping}

La viabilidad legal de la aplicación es un aspecto crucial a considerar, especialmente cuando se utiliza una técnica como \textit{web scraping} para obtener información de fuentes externas. Es importante comprender los aspectos legales y los posibles problemas que pueden surgir en relación con los permisos y derechos de propiedad intelectual (especialmente en los sitios web que requieren registro de usuario).

En el caso específico de la aplicación, se realiza \textit{web scraping} utilizando la API de Crossref (\url{https://api.crossref.org}), por lo que el primer paso es revisar y comprender los términos de servicio y las políticas de esta API. Estos documentos pueden establecer las condiciones específicas para el uso de la API, incluyendo cualquier restricción o requisito legal. Es recomendable revisar las políticas de Crossref regularmente para asegurarse de cumplir con los términos actualizados. 

Según la página oficial, Crossref ofrece su API de REST públicamente~\cite{Rosa2020}, lo que permite a los usuarios acceder y utilizar los metadatos depositados por sus miembros. Crossref proporciona dichos metadatos sin restricciones de \textit{copyright} y los usuarios pueden utilizarlos para cualquier propósito. En este sentido, nuestro proyecto no supone ningún problema legal (lo cual no sucedía con los sitios de Google Scholar, Scopus y Web of Science).

Sin embargo, es importante tener en cuenta que algunos \textit{abstracts} contenidos en los metadatos pueden estar sujetos a derechos de autor por parte de los editores o autores originales. Por lo tanto, es esencial que al utilizar la API de Crossref se respeten los derechos de autor y se cumpla con los términos y condiciones establecidos tanto por Crossref como por sus editores y autores originales. Además, Crossref ha implementado un sistema de <<piscinas>> separadas para usuarios <<educados>> que incluyen información de contacto y respetan ciertas prácticas de acceso.

\newpage
\subsubsection{Licencias de software}

Dado que en el presente proyecto se han utilizado una variedad de herramientas y paquetes con diferentes licencias (ver Tabla \ref{tab:paquetes-licencias}), es fundamental tenerlas en cuenta al elegir una licencia para nuestro \textit{software}. A continuación, se enumeran y explican las opciones de licencia más relevantes:

\begin{itemize}
    \item \textbf{Licencia MIT}: esta licencia es ampliamente utilizada y permite el uso, copia, modificación y distribución del \textit{software}, tanto en proyectos comerciales como no comerciales, con la condición de mantener el aviso de derechos de autor y la exención de responsabilidad. Es una opción flexible y abierta que fomenta la colaboración y la adopción del \textit{software}.

    \item \textbf{Licencia BSD}: Las licencias BSD son flexibles y permiten la redistribución y el uso en proyectos comerciales y no comerciales. Estas licencias suelen requerir el mantenimiento del aviso de derechos de autor y la exención de responsabilidad. También pueden incluir cláusulas adicionales que limitan la utilización del nombre del titular de los derechos de autor en la promoción del \textit{software} derivado.

    \item \textbf{Licencia Apache}: La Licencia Apache es una licencia de código abierto que permite el uso, modificación y distribución del \textit{software} bajo ciertas condiciones. Estas condiciones incluyen el mantenimiento del aviso de derechos de autor, la exención de responsabilidad y la necesidad de proporcionar una copia de la licencia en cualquier distribución del \textit{software}.

    \item \textbf{Licencia GNU}: La Licencia Pública General de GNU (GPL) es una licencia de \textit{software} libre que garantiza a los usuarios la libertad de utilizar, estudiar, modificar y distribuir el \textit{software}. La GPL también exige que cualquier \textit{software} derivado se distribuya bajo los términos de la GPL, lo que garantiza que las mejoras y modificaciones sigan siendo de código abierto.

\end{itemize}

Después de considerar cuidadosamente las licencias de las herramientas utilizadas en nuestro proyecto y dada la naturaleza del mismo, se ha considerado que la más adecuada es la Licencia GNU. Esta licencia garantiza la libertad del \textit{software} y el mantenimiento de su código abierto. Esta opción es ampliamente utilizada y respaldada por la comunidad del \textit{software} libre y de código abierto.

\begin{table}[h]
\centering
\begin{tabular}{|l|l|l|}
\hline
\textbf{Paquete} & \textbf{Versión}& \textbf{Licencia} \\ \hline
Babel & 2.12.1 &  MIT License \\ \hline
beautifulsoup4 & 4.12.2 & MIT License \\ \hline
crossrefapi & 1.5.0 &  CC Attribution 4.0 International License \\ \hline
cryptography & 40.0.2 &  OSI Approved: Apache SW License, BSD License \\ \hline
Flask & 2.3.2 &  BSD License \\ \hline
flask-babel & 3.1.0 &  BSD 3 License \\ \hline
Flask-Cors & 3.0.10 &  MIT License \\ \hline
Flask-Login & 0.6.2 &  MIT License \\ \hline
Flask-Migrate & 4.0.4 &  MIT License \\ \hline
frozenlist & 1.3.3 &  Apache Software License (Apache 2) \\ \hline
gunicorn & 20.1.0 &  MIT License \\ \hline
habanero & 1.2.3 & Copyright (C) 2021 Scott Chamberlain \\ \hline
httpcore & 0.17.2 &  BSD License (BSD) \\ \hline
httpx & 0.24.1 &  BSD License \\ \hline
idna & 3.4 &  BSD License \\ \hline
imagesize & 1.4.1 &  MIT License \\ \hline
ipykernel & 6.23.1 &  BSD License (BSD 3-Clause License) \\ \hline
ipython & 8.13.2 &  BSD License (BSD-3-Clause) \\ \hline
jedi & 0.18.2 &  MIT License \\ \hline
Jinja2 & 3.1.2 &  BSD-3-Clause License \\ \hline
joblib & 1.2.0 &  BSD License (BSD) \\ \hline
jupyter\_client & 8.2.0 & BSD License (BSD 3-Clause License) \\ \hline
jupyter\_core & 5.3.0 & BSD License (BSD 3-Clause License) \\ \hline
matplotlib & 3.7.1 & PSF \\ \hline
matplotlib-inline & 0.1.6 & BSD 3-Clause License \\ \hline
numpy & 1.24.3 &  BSD License (BSD-3-Clause) \\ \hline
pandas & 2.0.1 &  BSD License (BSD 3-Clause License) \\ \hline
pickleshare & 0.7.5 &  MIT License \\ \hline
psycopg2 & 2.9.6 &  GNU Lesser General Public (LGPLv2) \\ \hline
requests & 2.31.0 &  Apache Software License (Apache 2.0) \\ \hline
scholarly & 1.7.11 &  Unlicensed \\ \hline
scikit-learn & 1.2.2 & BSD License (new BSD) \\ \hline
scipy & 1.10.1 &  BSD License (Copyright (c) 2001-2002 Enthought,...) \\ \hline
selenium & 4.9.1 &  Apache Software License (Apache 2.0) \\ \hline
SQLAlchemy & 2.0.15 &  MIT License \\ \hline
xgboost & 1.7.5 &  Apache Software License (Apache-2.0) \\ \hline
Bootstrap & 5.2.3 &  MIT License \\ \hline
D3js & 7.8.4 &  ISC \\ \hline
jQuery & 3.6.4 &  MIT License \\ \hline
DataTables & 1.13.4 &  MIT License \\ \hline
pytest & 7.2.0 &  MIT License \\ \hline
Chartjs & 4.0 & MIT License \\ \hline
\end{tabular}
\caption{Información de los paquetes y sus licencias}
\label{tab:paquetes-licencias}
\end{table}


