\capitulo{1}{Introducción}

El Factor de Impacto de una publicación científica mide la frecuencia con la cual ha sido citado el artículo promedio de una publicación en un año en particular. Específicamente, sirve para evaluar la importancia de una revista dentro de un determinado campo científico. Existen múltiples metodologías de cálculo y sus correspondientes métricas, siendo el JCR (Journal Citation Reports) y el SJR (Scimago Journal Rank) los dos índices de impacto más utilizados.
Además, cabe destacar que el Factor de Impacto es uno de los principales criterios empleados en los procesos de acreditación y promoción interna para evaluar la calidad del trabajo científico de millones de académicos en todo el mundo.

Por su propia naturaleza, el Factor de Impacto se calcula con carácter retrospectivo, i.e., sobre datos de años anteriores. Así pues, a la hora de seleccionar la revista a la que mandarán un nuevo trabajo, los académicos quieren tener en cuenta el posible Factor de Impacto que tendrá la revista en el año de publicación del artículo. Sin embargo, lo único que pueden hacer es fijarse en las métricas de los años anteriores y hacer sus propias hipótesis y predicciones de futuro.

Con esta problemática en mente, creemos que puede ser muy útil para la comunidad científica utilizar los datos bibliográficos que hay disponibles en la web como \textit{inputs} de modelos de aprendizaje automático para estimar el Índice de Impacto que tendrán las distintas revistas científicas en el año en curso (o en años futuros).

A día de hoy la herramienta Google Scholar recoge información extremadamente actualizada sobre la publicación y citación de artículos científicos (podría decirse que se actualiza prácticamente en tiempo real). Por ello, al inicio del proyecto, consideramos Google Scholar como primera opción de donde extraer los datos. Sin embargo, debido a la gran cantidad de limitaciones de esta herramienta (que fueron descubriéndose a lo largo de la creación de prototipos), se terminó descartando esta opción. Posteriormente, se probaron otras fuentes como Scopus o WoS (Web of Science) pero, finalmente, nos terminamos decantando por Crossref. Esta elección se justificará de forma detallada más adelante.

En resumen, el \textit{output} esperado del proyecto será una aplicación web de tipo open-access, la cual implementará algoritmos de aprendizaje supervisado que utilizarán los datos históricos extraídos para predecir el valor del Índice de Impacto de todas las revistas científicas indexadas en el JCR. Dicha aplicación se dejará en un repositorio público, para así garantizar que pueda ser utilizada por toda la comunidad científica. 

\section{Enlaces relevantes}

A continuación, se facilitan los hiperenlaces para acceder al resultado final del proyecto:

\begin{itemize}
    \item \textbf{Repositorio GitHub}: \url{https://github.com/glp1002/JCR_Impact_Factor}
    \item \textbf{Aplicación web}: \url{https://paperrank.herokuapp.com/}
\end{itemize}